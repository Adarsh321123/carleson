% This is the main point of entry to the blueprint.
% Add chapters of the blueprint here.
% This file is not meant to be built. Build src/web.tex or src/print.text instead.

% Note: still have to upstream section ~> chapter and subsection ~> section to Overleaf
% Replace spaces by dashes in labels using search&replace (leave out `, keep space)
% `((\\label|\\ref|\\eqref|\\uses|\\proves)\{([^ \n},]|,\n? *[a-zA-Z])*) `
% to
% `$1-`
%
\title{Carleson operators on doubling metric measure spaces}

\date{\today}

\maketitle

\begin{abstract}
    We prove a new generalization of a theorem of Carleson, namely bounds for a generalized Carleson operator on doubling metric measure spaces.
    Additionally, we explicitly reduce Carleson's classical result on pointwise convergence of Fourier series to this new theorem.
    Both proofs are presented in great detail, suitable as a blueprint for computer verification using the current capabilities of the software package Lean.
    Note that even Carleson's classical result has not yet been computer-verified.
\end{abstract}

\tableofcontents

\textbf{Acknowledgements}: We greatly appreciate the help by anyone proof reading this blueprint or helping with the formalization process.

\chapter{Introduction}

In \cite{carleson}, L. Carleson addressed a classical question regarding the convergence of Fourier series of continuous functions by proving their pointwise convergence almost everywhere. \Cref{classical-Carleson} represents a version of this result.

Let $f$ be a complex valued, $2\pi$-periodic bounded Borel measurable function on the real line, and for an integer $n$, define the Fourier coefficient as
\begin{equation}
    \widehat{f}_n:=\frac {1}{2\pi} \int_0^{2\pi} f(x) e^{- i nx} dx .
\end{equation}
Define the partial Fourier sum for $N\ge 0$ as
\begin{equation}
    S_Nf(x):=\sum_{n=-N}^N \widehat{f}_n e^{i nx}\ .
\end{equation}

\begin{theorem}[classical Carleson]
    \label{classical-Carleson}
    \leanok
    \lean{classical_carleson}
    \uses{smooth-approximation, convergence-for-smooth,
    control-approximation-effect}
    Let $f$ be a $2\pi$-periodic complex-valued uniformly continuous function on $\mathbb{R}$ satisfying the bound $|f(x)|\le 1$ for all $x\in \mathbb{R}$. For all $0<\epsilon<1$, there exists a Borel set $E\subset [0,2\pi]$ with Lebesgue measure $|E|\le \epsilon$ and a positive integer $N_0$ such that for all $x\in [0,2\pi]\setminus E$ and all integers $N>N_0$, we have
    \begin{equation}\label{aeconv}
    |f(x)-S_N f(x)|\le \epsilon.
    \end{equation}
\end{theorem}

Note that mere continuity implies uniform continuity in the setting of this theorem.
By applying this theorem with a sequence of $\epsilon_n:= 2^{-n}\delta$ for $n\ge 1$ and taking the union
of corresponding exceptional sets $E_n$, we see that outside a set of measure $\delta$, the partial Fourier
sums converge pointwise for $N\to \infty$. Applying this with a sequence of $\delta$ shrinking to zero and
taking the intersection of the corresponding exceptional sets, which has measure zero, we see that the Fourier
series converges outside a set of measure zero.

The purpose of this paper is twofold. On the one hand, it prepares computer verification of \Cref{classical-Carleson} by presenting a very detailed proof as a blueprint for coding in Lean. We pass through a bound for a generalization of the so-called Carleson operator to doubling metric measure spaces. This generalization is new, and proving these bounds constitutes the second purpose of this paper. This generalization incorporates several results from the recent literature, most prominently bounds for the polynomial Carleson operator of V. Lie \cite{lie-polynomial} as well as its generalization \cite{zk-polynomial}. A computer verification of our theorem will also entail a computer verification for the bulk of the work in these results.

We proceed to introduce the setup for our general theorem.
We carry a multi purpose parameter, a natural number
\begin{equation}
    a\ge 4
\end{equation} in our notation that as it gets larger will allow more general applications but will worsen the constants in the estimates.



A doubling metric measure space $(X,\rho,\mu, a)$ is a complete
and locally compact metric space $(X,\rho)$
equipped with a $\sigma$-finite non-zero Radon--Borel measure $\mu$ that satisfies the doubling condition that for all $x\in X$ and all $R>0$ we have
\begin{equation}\label{doublingx}
    \mu(B(x,2R))\le 2^a\mu(B(x,R))\,,
\end{equation}
where we have denoted by $B(x,R)$ the open ball of radius $R$ centred at $x$:
\begin{equation}\label{eq-define-ball}
 B(x,R):=\{y\in X: \rho(x,y)<R\}. \end{equation}

A collection $\Mf$ of real valued continuous functions on the doubling metric measure space $(X,\rho,\mu,a)$ is called compatible, if there is a point $o\in X$ where all the functions are equal to $0$, and if there exists for each ball $B \subset X$ a metric $d_B$ on $\Mf$, such that the following five properties \eqref{osccontrol}, \eqref{firstdb}, \eqref{monotonedb}, \eqref{seconddb}, and \eqref{thirddb} are satisfied. For every ball $B \subset X$
\begin{equation}\label{osccontrol}
    \sup_{x,y\in B}|\mfa(x)-{\mfa(y)}- \mfb(x)+{\mfb(y)}| \le d_{B}(\mfa,\mfb)\,.
\end{equation}
For any two balls $B_1=B(x_1,R)$, $B_2= B(x_2,2R)$ in $X$ with $x_1\in B_2$ and any $\mfa,\mfb\in \Mf$,
\begin{equation}\label{firstdb}
    d_{B_2}(\mfa,\mfb)\le 2^a d_{B_1}(\mfa,\mfb) .
\end{equation}
For any two balls $B_1, B_2$ in $X$ with $B_1 \subset B_2$ and any $\mfa, \mfb \in \Mf$
\begin{equation}\label{monotonedb}
    d_{B_1}(\mfa,\mfb) \le d_{B_2}(\mfa, \mfb)
\end{equation}
and for any two balls
$B_1=B(x_1,R)$, $B_2= B(x_2,2^aR)$
with $B_1\subset B_2$, and $\mfa,\mfb\in \Mf$,
\begin{equation}\label{seconddb}
    2d_{B_1}(\mfa,\mfb)
\le d_{B_2}(\mfa,\mfb) .
\end{equation}
For every ball $B$ in $X$ and every $d_B$-ball $\tilde B$ of radius $2R$ in $\Mf$, there is a collection $\mathcal{B}$ of
at most $2^a$ many $d_B$-balls of radius $R$ covering $\tilde B$, that is,
\begin{equation}\label{thirddb}
    \tilde B\subset \bigcup \mathcal{B}.
\end{equation}


Further, a compatible collection $\Mf$ is called cancellative, if
for any ball $B$ in $X$ of radius $R$, any Lipschitz function $\varphi: X\to \C$
supported on $B$, and any $\mfa,\mfb\in \Mf$ we have
\begin{equation}
    \label{eq-vdc-cond}
    |\int_B e(\mfa(x)-{\mfb(x)}) \varphi(x) d\mu(x)|\le 2^a \mu(B)\|\varphi\|_{\Lip(B)}
(1+d_B(\mfa,\mfb))^{-\frac{1}{a}},
\end{equation}
where $\|\cdot\|_{\Lip(B)}$ denotes the inhomogeneous Lipschitz norm on $B$:
$$
    \|\varphi\|_{\Lip(B)} = \sup_{x \in B} |\varphi(x)| + R \sup_{x,y \in B, x \neq y} \frac{|\varphi(x) - \varphi(y)|}{\rho(x,y)}\,.
$$

A one-sided Calder\'on--Zygmund kernel $K$ on the doubling metric measure space $(X, \rho, \mu, a)$ is a measurable function
 \begin{equation}\label{eqkernel0}
K:X\times X\to \mathbb{C}
 \end{equation}
such that for all $x,y',y\in X$ with $x\neq y$, we have
\begin{equation}\label{eqkernel-size}
    |K(x,y)| \leq \frac{2^{a^3}}{V(x,y)}
  \end{equation}
 and if $2\rho(y,y') \leq \rho(x,y)$, then
  \begin{equation}
  \label{eqkernel-y-smooth}
       |K(x,y) - K(x,y')| \leq \left(\frac{\rho(y,y')}{\rho(x,y)}\right)^{\frac{1}{a}}\frac{2^{a^3}}{V(x,y)},
\end{equation}
where \[V(x,y):=\mu(B(x,\rho(x,y))).\]
Define the maximally truncated non-tangential singular integral $T_{*}$ associated with $K$ by
\begin{equation}
    \label{def-tang-unm-op}
    T_{*}f(x):=\sup_{R_1 < R_2} \sup_{\rho(x,x')<R_1} \left|\int_{R_1< \rho(x',y) < R_2} K(x',y) f(y) \, \mathrm{d}\mu(y) \right|\,.
\end{equation}
We define the generalized Carleson operator $T$ by
 \begin{equation}
        \label{def-main-op}
        Tf(x):=\sup_{\mfa\in\Mf} \sup_{0 < R_1 < R_2}\left| \int_{R_1 < \rho(x,y) < R_2} K(x,y) f(y) e(\mfa(y)) \, \mathrm{d}\mu(y) \right|\, ,
\end{equation}
where $e(r)=e^{ir}$.

Our main result is the following restricted weak type estimate for $T$ in the range $1<q\le 2$, which by interpolation techniques recovers $L^q$ estimates for the open range
$1<q<2$.
\begin{theorem}[metric space Carleson]
\label{metric-space-Carleson}
\uses{R-truncation}
    For all integers $a \ge 4$ and real numbers $1<q\le 2$
    the following holds.
    Let $(X,\rho,\mu,a)$ be a doubling metric measure space. Let $\Mf$ be a
    cancellative compatible collection of functions and let $K$ be a one-sided Calder\'on--Zygmund kernel on $(X,\rho,\mu,a)$. Assume that for every bounded measurable function $g$ on $X$ supported on a set of finite measure we have
    \begin{equation}\label{nontanbound}
        \|T_{*}g\|_{2} \leq 2^{a^3} \|g\|_2\,,
    \end{equation}
    where $T_{*}$ is defined in
\eqref{def-tang-unm-op}.
    Then for all Borel sets $F$ and $G$ in $X$ and
    all Borel functions $f:X\to \C$ with
    $|f|\le \mathbf{1}_F$, we have, with $T$ defined in \eqref{def-main-op},
    \begin{equation}
    \label{resweak}
        \left|\int_{G} T f \, \mathrm{d}\mu\right| \leq \frac{2^{450a^3}}{(q-1)^5} \mu(G)^{1-\frac{1}{q}} \mu(F)^{\frac{1}{q}}\, .
        \end{equation}
\end{theorem}

In the one-dimensional Euclidean setting, with $K$ representing the Hilbert kernel:
\begin{equation*}
K(x,y)=(x-y)^{-1}
\end{equation*}
and $\Mf$ denoting the class of linear functions, the operator \eqref{def-main-op} is the classical Carleson operator, which plays a crucial role in proving the almost everywhere convergence of Fourier series \cite{carleson}, \cite{fefferman}, \cite{lacey-thiele}. The supremum in $R_1$ and $R_2$ is often omitted in classical treatments, but considering the maximal truncations can easily be reduced to the case without these truncations.

By replacing $\Mf$ with the class of polynomials vanishing at $0$ up to some fixed but arbitrary degree, we obtain the polynomial Carleson operator of Lie \cite{lie-quadratic} (quadratic case) and \cite{lie-polynomial}. The case of the class of polynomials with vanishing linear coefficient is simpler and was estimated in \cite{stein-wainger}. The polynomial Carleson operator was generalized to the high-dimensional Euclidean setting in \cite{zk-polynomial} for $K$ being a Calder\'on-Zygmund kernel with some H\"older regularity.

Doubling metric measure spaces are instances of spaces of homogeneous type. Indeed, by changing from a quasi-metric to an equivalent metric, every space of homogeneous type can be viewed as a doubling metric measure space (cf. \cite{MaciasSegovia}). Spaces of homogeneous type were introduced by \cite{MR0499948} as a natural setting for Calder\'on-Zygmund theory. We refer to the textbook \cite{stein-book} for an account of these spaces.

Our concept of a compatible collection $\Mf$ as a natural class of phase functions on a doubling metric measure space does not appear in \cite{stein-book} but is implicitly anticipated in \cite{zk-polynomial} and subsequent work of \cite{mnatsakanyan}, who proves a Carleson-type theorem for the Malmquist-Takenaka series, which leads to classes of phases related to Blaschke products. A generalization of \eqref{def-main-op} from the previously mentioned Euclidean setting into the anisotropic setting that was suggested in \cite{zk-polynomial} is included in our theory. The polynomial Carleson operator also plays a role in the study of maximally modulated singular Radon transforms along the parabola, see \cite{ramos} and \cite{becker2024maximal}.

For the proof of \Cref{metric-space-Carleson}, we largely follow \cite{zk-polynomial}, which in turn was inspired by \cite{lie-polynomial}. We make suitable modifications to adapt to our more general setting and have made a few technical improvements in the proof. In particular, in \Cref{overviewsection}, we explicitly divide the main work of the proof into mutually independent sections \ref{thmfromproplinear}, \ref{christsection}, \ref{proptopropprop}, \ref{antichainboundary}, \ref{treesection}, \ref{liphoel}, and \ref{sec-hlm}. Some of these sections follow a similar pattern, starting with a subsection dividing the proof into further mutually independent subsections. This modularization of our proof was strongly endorsed in personal communication by the author of \cite{zk-polynomial}.

\noindent \textit{Acknowledgement.}
L.B., F.v.D., R.S., and C.T. were funded by the Deutsche Forschungsgemeinschaft (DFG, German Research Foundation) under Germany's Excellence Strategy -- EXC-2047/1 -- 390685813.
L.B. , R.S., and C.T. were also supported by SFB 1060.
A.J. is funded by the T\"UBITAK (Scientific and Technological Research Council of T\"urkiye) under Grant Number 123F122.

\chapter{Proof of Metric Space Carleson, overview}
\label{overviewsection}

This section organizes the proof of \Cref{metric-space-Carleson} into sections \ref{thmfromproplinear}, \ref{christsection}, \ref{proptopropprop}, \ref{antichainboundary}, \ref{treesection}, \ref{liphoel}, and \ref{sec-hlm}. These sections are mutually independent except for referring to the statements formulated in the present section. \Cref{thmfromproplinear} proves the main \Cref{metric-space-Carleson}, while sections \ref{christsection}, \ref{proptopropprop}, \ref{antichainboundary}, \ref{treesection}, \ref{liphoel}, and \ref{sec-hlm} each prove one proposition that is stated in the present section. The present section also introduces all definitions used across these sections.

\Cref{global-auxiliary-lemmas} proves some auxiliary lemmas that are used in more than one of the sections 3-9.

Let $a, q$ be given as in \Cref{metric-space-Carleson}.




Define
\begin{equation}\label{defineD}
D:= 2^{100 a^2}\, ,
\end{equation}
\begin{equation}\label{definekappa}
\kappa:= 2^{-10a}\,,
\end{equation}
and
\begin{equation}
    \label{defineZ}
    Z := 2^{12a}\,.
\end{equation}
Let
 $\psi:\R \to \R$ be the unique compactly supported, piece-wise linear, continuous function with corners precisely at $\frac 1{4D}$, $\frac 1{2D}$, $\frac 14$ and $\frac 12$ which satisfies
 \begin{equation}
    \label{eq-psisum}
    \sum_{s\in \mathbb{Z}} \psi(D^{-s}x)=1
\end{equation}
for all $x>0$. This function vanishes outside $[\frac1{4D},\frac 12]$, is constant one on
$[\frac1{2D},\frac 14]$, and is Lipschitz
with constant $4D$.







Let a doubling metric measure space $(X,\rho,\mu, a)$ be given.
Let a cancellative compatible collection $\Mf$ of functions on $X$ be given.
Let $o\in X$ be a point such that $\mfa(o)=0$
for all $\mfa\in \Mf$.







Let a one-sided Calder\'on--Zygmund kernel $K$ on $X$ be given so that the operator $T_*$ defined in \eqref{def-tang-unm-op}
satisfies
\eqref{nontanbound}. Let $T$ be the corresponding operator as defined in \eqref{def-main-op}.


For $s\in\mathbb{Z}$, we define
\begin{equation}\label{defks}
    K_s(x,y):=K(x,y)\psi(D^{-s}\rho(x,y))\,,
\end{equation}
so that for each $x, y \in X$ with $x\neq y$ we have
$$K(x,y)=\sum_{s\in\mathbb{Z}}K_s(x,y).$$

In \Cref{thmfromproplinear}, we prove \Cref{metric-space-Carleson}
from its finitary version, \Cref{finitary-Carleson} below. Recall
that a function from a measure space to a finite set is measurable if the pre-image of each of the elements in the range is measurable.


\begin{proposition}[finitary Carleson]
\label{finitary-Carleson}
\leanok
\lean{finitary_carleson}
\uses{discrete-Carleson, grid-existence, tile-structure, tile-sum-operator}
Let ${\sigma_1},\sigma_2\colon X\to \mathbb{Z}$ be measurable functions with finite range and ${\sigma_1}\leq \sigma_2$. Let $\tQ\colon X\to \Mf$ be a measurable function with finite range. Let $F,G$ be bounded Borel sets in $X$. Then there is a Borel set $G'$ in $X$ with $2\mu(G')\leq \mu(G)$ such that
for all Borel functions $f:X\to \C$ with $|f|\le \mathbf{1}_F$.
\begin{equation*}
    \int_{G \setminus G'} \left|\sum_{s={\sigma_1}(x)}^{{\sigma_2}(x)} \int K_s(x,y) f(y) e(\tQ(x)(y)) \, \mathrm{d}\mu(y) \right| \mathrm{d}\mu(x)
\end{equation*}
\begin{equation}
    \label{eq-linearized}
    \le \frac{2^{440a^3}}{(q-1)^4} \mu(G)^{1-\frac{1}{q}}
     \mu(F)^{\frac 1 q}\,.
\end{equation}
\end{proposition}
Let measurable functions ${\sigma_1}\leq \sigma_2\colon X\to \mathbb{Z}$ with finite range be given. Let a measurable function
$\tQ\colon X\to \Mf$ with finite range
be given.
Let bounded Borel sets $F,G$ in $X$ be given.
Let $S$ be the smallest integer such that the ranges of
$\sigma_1$ and $\sigma_2$ are contained in $[-S,S]$ and $F$ and $G$ are contained
in the ball $B(o, D^S)$.

In \Cref{christsection}, we prove \Cref{finitary-Carleson} using a
bound for a dyadic model formulated in \Cref{discrete-Carleson} below.

% fixme (simplification):
% Georges Gonthier mentioned that it would simplify things if we have only 1 top cube.
% We should be able to do this by picking the center for level S at $o$
% Note that we don't actually have to cover all of $B(o,D^S)$, only the integration domain $G$
% Actually making this work requires modifying $G$ a little bit (it must lie within $B(o,D^S/2)$ or
% something).
% This would simplify some arguments in antichain bounds/estimates, where we currently have to deal with top cubes at multiple levels

A grid structure $(\mathcal{D}, c, s)$ on $X$ consists of a finite collection $\mathcal{D}$ of Borel
sets in $X$ called dyadic cubes, a surjective function $s\colon \mathcal{D}\to [-S, S]$
called scale function, and a function $c:\mathcal{D}\to X$
called center function such that the five properties
\eqref{coverdyadic}, \eqref{dyadicproperty}, \eqref{coverball},
\eqref{eq-vol-sp-cube}, and \eqref{eq-small-boundary} hold.

For each dyadic cube $I$ and each $-S\le k<s(I)$ we have
\begin{equation}\label{coverdyadic}
I\subset \bigcup_{J\in \mathcal {D}: s(J)=k}J\, .
\end{equation}
Any two non-disjoint dyadic cubes $I,J$ with $s(I)\le s(J)$ satisfy
\begin{equation}\label{dyadicproperty}
I\subset J.
\end{equation}
For any $x\in B(o,D^S)$, and every $k\in[-S,S]$, there
is a dyadic cube $I$ with $s(I)=k$ and
\begin{equation}\label{coverball}
x\in I.
\end{equation}
For any dyadic cube $I$,
 \begin{equation}
        \label{eq-vol-sp-cube}
        c(I)\in B(c(I), \frac{1}{4} D^{s(I)}) \subset I \subset B(c(I), 4 D^{s(I)})\,.
    \end{equation}
For any dyadic cube $I$ and any $t$ with $tD^{s(I)} \ge D^{-S}$,
\begin{equation}
        \label{eq-small-boundary}
        % fixme(Georges Gonthier): (maybe) The D in the RHS can be replaced by 2.
        % We can do the same in the statement and proof of {boundary-measure}.
        \mu(\{x \in I \ : \ \rho(x, X \setminus I) \leq t D^{s(I)}\}) \le D t^\kappa \mu(I)\,.
    \end{equation}





% fixme (simplification):
% We can make \mathcal(Q) take values Q(X), and still cover all of Q(X).
% We can do this by making \mathcal{Z} a subset of Q(X) satisfying 4.2.2.
% This would remove the need for Lemma 2.1.1 for the definition of \mathcal{Z} in Section 4.2, since it will be automatically finite, since Q(X) is finite.
% This condition is already implicitly used in the proof of Lemma 6.1.6 when using (2.0.13)

A tile structure $(\fP,\scI,\fc,\fcc,\pc,\ps)$
for a given grid structure $(\mathcal{D}, c, s)$
is a finite set $\fP$ of elements called tiles with five maps
\begin{align*}
\scI&\colon \fP\to {\mathcal{D}}\\
\fc&\colon \fP\to \mathcal{P}(\Mf) \\
\fcc &\colon \fP\to \Mf\\
\pc &\colon \fP\to X\\
\ps &\colon \fP\to \mathbb{Z}
\end{align*}
with $\scI$ surjective and $\mathcal{P}(\Mf)$ denoting the power set of $\Mf$ such that the five properties \eqref{eq-dis-freq-cover}, \eqref{eq-freq-dyadic},
\eqref{eq-freq-comp-ball}, \eqref{tilecenter}, and
\eqref{tilescale} hold.
For each dyadic cube $I$, the restriction of the map $\Omega$ to the set
\begin{equation}\label{injective}
    \fP(I)=\{\fp: \scI(\fp) =I\}
\end{equation}
is injective
and we have the disjoint covering property (we use the union symbol with dot on top to denote a disjoint union)
\begin{equation}\label{eq-dis-freq-cover}
\tQ(X)\subset \dot{\bigcup}_{\fp\in \fP(I)}\fc(\fp).
\end{equation}
For any tiles $\fp,\fq$ with $\scI(\fp)\subset \scI(\fq)$ and $\fc(\fp) \cap \fc(\fq) \neq \emptyset$ we have
\begin{equation} \label{eq-freq-dyadic}
\fc(\fq)\subset \fc(\fp) .
\end{equation}
For each tile $\fp$,
        \begin{equation}\label{eq-freq-comp-ball}
        \fcc(\fp)\in B_{\fp}(\fcc(\fp), 0.2) \subset \fc(\fp) \subset B_{\fp}(\fcc(\fp),1)\,,
        \end{equation}
        where
\begin{equation}
    B_{\fp} (\mfa, R) := \{\mfb \in \Mf \, : \, d_{\fp}(\mfa, \mfb) < R\,\} ,
\end{equation}
 and
\begin{equation}\label{defdp}
d_{\fp} := d_{B(\pc(\fp),\frac 14 D^{\ps(\fp)})}\, .
\end{equation}
We have for each tile $\fp$
\begin{equation}\label{tilecenter}
    \pc(\fp)=c(\scI(\fp)),
\end{equation}
\begin{equation}\label{tilescale}
    \ps(\fp)=s(\scI(\fp)).
\end{equation}


\begin{proposition}[discrete Carleson]
\label{discrete-Carleson}
\leanok
\lean{discrete_carleson}
\uses{exceptional-set, forest-union, forest-complement}
Let $(\mathcal{D}, c, s)$ be a grid structure and \begin{equation*}
    (\fP,\scI,\fc,\fcc,\pc,\ps)
\end{equation*}
a tile structure for this grid structure.
Define for $\fp\in \fP$
\begin{equation}\label{defineep}
    E(\fp)=\{x\in \scI(\fp): \tQ(x)\in \fc(\fp) , {\sigma_1}(x)\le \ps(\fp)\le {\sigma_2}(x)\}
\end{equation}
and
\begin{equation}\label{definetp}
    T_{\fp} f(x)= \mathbf{1}_{E(\fp)}(x) \int K_{\ps(\fp)}(x,y) f(y) e(\tQ(x)(y)-\tQ(x)(x))\, d\mu(y).
\end{equation}
Then there exists a Borel set $G'$ with $2\mu(G') \leq \mu(G)$ such that for all Borel functions $f:X\to \C$ with $|f|\le \mathbf{1}_F$
we have
\begin{equation}
    \label{disclesssim}
   \int_{G \setminus G'} \left| \sum_{\fp \in \fP} T_{\fp} f (x) \right| \, \mathrm{d}\mu(x) \le \frac{2^{440a^3}}{(q-1)^4} \mu(G)^{1-\frac{1}{q}} \mu(F)^{\frac{1}{q}}\,.
\end{equation}
\end{proposition}









The proof of \Cref{discrete-Carleson} is done in \Cref{proptopropprop}
by a reduction to two further propositions that we state below.


Fix a grid structure $(\mathcal{D}, c, s)$ and a tile structure $(\fP,\scI,\fc,\fcc,\pc,\ps)$
for this grid structure.

We define the relation
\begin{equation}\label{straightorder}
    \fp\le \fp'
\end{equation}
 on $\fP\times \fP$ meaning
$\scI(\fp)\subset \scI(\fp')$ and
$\Omega(\fp')\subset \Omega(\fp)$.
We further define for $\lambda,\lambda' >0$
the relation
\begin{equation}\label{wiggleorder}
    \lambda \fp \lesssim \lambda' \fp'
\end{equation}
on $\fP\times \fP$ meaning
$\scI(\fp)\subset \scI(\fp')$ and
\begin{equation}
    B_{\fp'}(\fcc(\fp'),\lambda' )
    \subset B_{\fp}(\fcc(\fp),\lambda )\, .
\end{equation}



Define for a tile $\fp$ and $\lambda>0$
\begin{equation}\label{definee1}
    E_1(\fp):=\{x\in \scI(\fp)\cap G: \tQ(x)\in \fc(\fp)\}\, ,
\end{equation}
\begin{equation}\label{definee2}
    E_2(\lambda, \fp):=\{x\in \scI(\fp)\cap G: \tQ(x)\in B_{\fp}(\fcc(\fp), \lambda)\}\, .
\end{equation}



Given a subset $\fP'$ of $\fP$, we define
$\fP(\fP')$ to be the set of
all $\fp \in \fP$ such that there exist $\fp' \in \fP'$ with $\scI(\fp)\subset \scI(\fp')$. Define the densities
\begin{equation}\label{definedens1}
    {\dens}_1(\fP') := \sup_{\fp'\in \fP'}\sup_{\lambda \geq 2} \lambda^{-a} \sup_{\fp \in \fP(\fP'), \lambda \fp' \lesssim \lambda \fp}
    \frac{\mu({E}_2(\lambda, \fp))}{\mu(\scI(\fp))}\, ,
\end{equation}
\begin{equation}\label{definedens2}
    {\dens}_2(\fP') := \sup_{\fp'\in \fP'}
    \sup_{r\ge 4D^{\ps(\fp)}}
    \frac{\mu(F\cap B(\pc(\fp),r))}{\mu(B(\pc(\fp),r))}\, .
\end{equation}





An antichain is a subset $\mathfrak{A}$
of $\fP$ such that for any distinct $\fp,\fq\in \mathfrak{A}$ we do not have have $\fp\le \fq$.

The following proposition is proved in \Cref{antichainboundary}.

\begin{proposition}[antichain operator]
\label{antichain-operator}
\leanok
\lean{antichain_operator}

\uses{dens2-antichain,dens1-antichain}
For any antichain $\mathfrak{A} $ and for all $f:X\to \C$ with $|f|\le \mathbf{1}_F$ and all $g:X\to\C$ with $|g| \le \mathbf{1}_G$
\begin{equation} \label{eq-antiprop}
    |\int \overline{g(x)} \sum_{\fp \in \mathfrak{A}} T_{\fp} f(x)\, d\mu(x)|
\end{equation}
\begin{equation}
    \le \frac{2^{150a^3}}{q-1} \dens_1(\mathfrak{A})^{\frac {q-1}{8a^4}}\dens_2(\mathfrak{A})^{\frac 1{q}-\frac 12} \|f\|_2 \|g\|_2\, .
\end{equation}
\end{proposition}

Let $n\ge 0$.
An $n$-forest is a pair $(\fU, \mathfrak{T})$
where $\fU$ is a subset of $\fP$
and $\mathfrak{T}$ is a map assigning to
each $\fu\in \fU$ a nonempty set $\fT (\fu)\subset \fP$ called tree
such that the following properties
\eqref{forest1}, \eqref{forest2},
\eqref{forest3},
\eqref{forest4},
\eqref{forest5}, and
\eqref{forest6}
hold.

For each $\fu\in \fU$ and each $\fp\in \fT(\fu)$
we have
\begin{equation}\label{forest1}
4\fp\lesssim 1\fu.
\end{equation}
For each $\fu \in \fU$ and each $\fp,\fp''\in \fT(\fu)$ and $\fp'\in \fP$
we have
\begin{equation}\label{forest2}
    \fp, \fp'' \in \mathfrak{T}(\fu), \fp \leq \fp' \leq \fp'' \implies \fp' \in \mathfrak{T}(\fu).
\end{equation}
We have
\begin{equation}\label{forest3}
   \|\sum_{\fu\in \fU} \mathbf{1}_{\scI(\fu)}\|_\infty \leq 2^n\,.
\end{equation}
We have for every $\fu\in \fU$
\begin{equation}\label{forest4}
\dens_1(\fT(\fu))\le 2^{4a + 1-n}\, .
\end{equation}
We have for $\fu, \fu'\in \fU$ with $\fu\neq \fu'$ and $\fp\in \fT(\fu')$ with $\scI(\fp)\subset \scI(\fu)$ that
\begin{equation}\label{forest5}
d_{\fp}(\fcc(\fp), \fcc(\fu))>2^{Z(n+1)}\, .
\end{equation}
We have for every $\fu\in \fU$ and $\fp\in \fT(\fu)$ that
\begin{equation}\label{forest6}
B(\pc(\fp), 8D^{\ps(\fp)})\subset \scI(\fu).
\end{equation}


The following proposition is proved in \Cref{treesection}.
\begin{proposition}[forest operator]
\label{forest-operator}
\leanok
\lean{forest_operator}
\uses{forest-row-decomposition,row-bound,row-correlation,disjoint-row-support}
For any $n\ge 0$ and any $n$-forest $(\fU,\fT)$ we have for all $f: X \to \mathbb{C}$ with $|f| \le \mathbf{1}_F$ and all bounded $g$ with bounded support
$$
    | \int \overline{g(x)} \sum_{\fu\in \fU} \sum_{\fp\in \fT(\fu)} T_{\fp} f(x) \, \mathrm{d}\mu(x)|
$$
$$
    \le
    2^{432a^3}2^{-\frac{q-1}{q} n} \dens_2\left(\bigcup_{\fu\in \fU}\fT(\fu)\right)^{\frac{1}{q}-\frac{1}{2}} \|f\|_2 \|g\|_2 \,.
$$
\end{proposition}

\Cref{metric-space-Carleson} is formulated at the level of generality
for general kernels satisfying the mere H\"older regularity condition \eqref{eqkernel-y-smooth}. On the other hand, the cancellative condition \eqref{eq-vdc-cond} is a testing condition against more regular,
namely Lipschitz functions. To bridge the gap, we follow \cite{zk-polynomial} to observe a variant of \eqref{eq-vdc-cond} that we formulate
in the following proposition proved in \Cref{liphoel}.


Define
\begin{equation}
    \tau:=\frac 1a\, .
\end{equation}
Define for any open ball $B$ of radius $R$ in $X$ the $L^\infty$-normalized $\tau$-H\"older norm by
\begin{equation}
    \label{eq-Holder-norm}
    \|\varphi\|_{C^\tau(B)} = \sup_{x \in B} |\varphi(x)| + R^\tau \sup_{x,y \in B, x \neq y} \frac{|\varphi(x) - \varphi(y)|}{\rho(x,y)^\tau}\,.
\end{equation}


\begin{proposition}[Holder van der Corput]
    \label{Holder-van-der-Corput}
    \leanok
    \lean{holder_van_der_corput}
    \uses{Lipschitz-Holder-approximation}
     Let $z\in X$ and $R>0$ and set $B=B(z,R)$.
     Let $\varphi: X \to \mathbb{C}$ by
     supported on $B$ and satisfy $\|{\varphi}\|_{C^\tau(B)}<\infty$.
     Let $\mfa, \mfb \in \Mf$. Then
    \begin{equation}
        \label{eq-vdc-cond-tau-2}
        |\int e(\mfa(x)-{\mfb(x)})\varphi(x) dx|\le
         2^{8a} \mu(B) \|{\varphi}\|_{C^\tau(B)}
       (1 + d_{B}(\mfa,\mfb))^{-\frac{1}{2a^2+a^3}}
    \,.
    \end{equation}
\end{proposition}

We further formulate a classical Vitali covering result
and maximal function estimate that we need throughout several sections.
This following proposition will typically be applied to the absolute value of a complex valued function and be proved in \Cref{sec-hlm}. By a ball $B$ we mean a set $B(x,r)$ with $x\in X$
and $r>0$ as defined in \eqref{eq-define-ball}.
For a finite collection $\mathcal{B}$ of balls in $X$
and $1\le p< \infty$ define the measurable function $M_{\mathcal{B},p}u$ on $X$ by
\begin{equation}\label{def-hlm}
M_{\mathcal{B},p}u(x):=\left(\sup_{B\in \mathcal{B}} \frac{\mathbf{1}_{B}(x)}{\mu(B)}\int _{B} |u(y)|^p\, d\mu(y)\right)^\frac 1p\, .
\end{equation}
Define further $M_{\mathcal{B}}:=M_{\mathcal{B},1}$.

\begin{proposition}[Hardy--Littlewood]
\label{Hardy-Littlewood}
\leanok
\lean{measure_biUnion_le_lintegral, maximalFunction_lt_top, snorm_maximalFunction, M_lt_top,
  laverage_le_M, snorm_M_le}
\uses{layer-cake-representation,covering-separable-space}
   Let $\mathcal{B}$ be a finite collection of balls in $X$.
If for some $\lambda>0$ and some measurable function $u:X\to [0,\infty)$ we have
\begin{equation}\label{eq-ball-assumption}
\int_{B} u(x)\, d\mu(x)\ge \lambda \mu(B)
\end{equation}
   for each $B\in \mathcal{B}$,
   then
   \begin{equation}\label{eq-besico}
\lambda \mu(\bigcup \mathcal{B}) \le 2^{2a}\int_X u(x)\, d\mu(x)\, .
\end{equation}
For every measurable function $v$
and $1\le p_1<p_2$ we have
\begin{equation}\label{eq-hlm}
    \|M_{\mathcal{B},p_1} v\|_{p_2}\le 2^{2a}\frac{p_2}{p_2-p_1} \|v\|_{p_2}\, .
\end{equation}
Moreover, given any measurable bounded function $w: X \to \C$ there exists a measurable function $Mw: X \to [0, \infty)$ such that the following \eqref{eq-ball-av} and \eqref{eq-hlm-2} hold. For each ball $B \subset X$ and each $x \in B$
\begin{equation}
    \label{eq-ball-av}
    \frac{1}{\mu(B)} \int_{B} |w(y)| \, \mathrm{d}\mu(y) \le Mw(x)
\end{equation}
and for all $1 \le p_1 < p_2 \le \infty$
\begin{equation}
    \label{eq-hlm-2}
    \|M(w^{p_1})^{\frac{1}{p_1}}\|_{p_2} \le 2^{4a} \frac{p_2}{p_2-p_1}\|w\|_{p_2}\,.
\end{equation}

\end{proposition}

This completes the overview of the proof of \Cref{metric-space-Carleson}.

\section{Auxiliary lemmas}
\label{global-auxiliary-lemmas}
We close this section by recording some auxiliary lemmas about the objects defined in \Cref{overviewsection}, which will be used in multiple sections to follow.

First, we record an estimate for the metrical entropy numbers of balls in the space $\Mf$ equipped with any of the metrics $d_B$, following from the doubling property \eqref{thirddb}.

\begin{lemma}[ball metric entropy]
    \label{ball-metric-entropy}
    \leanok
    \lean{Θ.mk_le_of_le_dist}
    Let $B' \subset X$ be a ball. Let $r > 0$, $\mfa \in \Mf$ and $k \in \mathbb{N}$. Suppose that $\mathcal{Z} \subset B_{B'}(\mfa, r2^k)$ satisfies that $\{B_{B'}(z,r)\mid z \in \mathcal{Z}\}$ is a collection of pairwise disjoint sets. Then
    $$ |\mathcal{Z}| \le 2^{ka}\,. $$
\end{lemma}
% remark(Floris): weakened the condition $d_{B'}(z,z') \ge r$ to $B_{B'}(z, r) \cap B_{B'}(z', r) = \emptyset$
% This also fixed an off-by-one error of $k$

\begin{proof}
    \leanok
    By applying property \eqref{thirddb} $k$ times, we obtain a collection $\mathcal{Z}' \subset \Mf$ with $|\mathcal{Z}'| = 2^{ka}$ and
    $$
        B_{B'}(\mfa,r2^k) \subset \bigcup_{z' \in \mathcal{Z}'} B_{B'}(z', \frac{r}{2})\,.
    $$
    Then each $z \in \mathcal{Z}$ is contained in one of the balls $B(z', \frac{r}{2})$, but by the separation assumption no such ball contains more than one element of $\mathcal{Z}$. Thus $|\mathcal{Z}| \le |\mathcal{Z}'| = 2^{ka}$.
\end{proof}

The next lemma concerns monotonicity of the metrics $d_{B(c(I), \frac 14 D^{s(I)})}$ with respect to inclusion of cubes $I$ in a grid.

\begin{lemma}[monotone cube metrics]
    \label{monotone-cube-metrics}
    \leanok
    \lean{𝓓.dist_mono, 𝓓.dist_strictMono}
    Let $(\mathcal{D}, c, s)$ be a grid structure. Denote for cubes $I \in \mathcal{D}$
    $$
        I^\circ := B(c(I), \frac{1}{4} D^{s(I)})\,.
    $$
    Let $I, J \in \mathcal{D}$ with $I \subset J$.
    Then for all $\mfa, \mfb \in\Mf$ we have
    $$
        d_{I^\circ}(\mfa, \mfb) \le d_{J^\circ}(\mfa, \mfb)\,,
    $$
    and if $I \ne J$ then we have
    $$
        d_{I^\circ}(\mfa, \mfb) \le 2^{-95a} d_{J^\circ}(\mfa, \mfb)\,.
    $$
\end{lemma}

\begin{proof}
    If $s(I) \ge s(J)$ then \eqref{dyadicproperty} and the assumption $I\subset J$ imply $I = J$. Then the lemma holds by reflexivity.

    If $s(J) \ge s(I)+1$, then using the monotonicity property \eqref{monotonedb}, \eqref{defineD} and \eqref{seconddb}, we get
    \begin{equation}
    \label{eq-dIJ-est}
        d_{I^\circ}(\mfa, \mfb) \le d_{B(c(I), 4 D^{s(I)})}(\mfa, \mfb) \le 2^{-100a} d_{B(c(I), 4D^{s(J)})}(\mfa, \mfb)\,.
    \end{equation}
    Using \eqref{eq-vol-sp-cube}, together with the inclusion $I \subset J$, we obtain
    $$
        c(I) \in I \subset J \subset B(c(J), 4 D^{s(J)})
    $$
    and consequently by the triangle inequality
    $$
        B(c(I), 4 D^{s(J)}) \subset B(c(J), 8 D^{s(J)})\,.
    $$
    Using this together with the monotonicity property \eqref{monotonedb} and \eqref{firstdb} in \eqref{eq-dIJ-est}, we obtain
    \begin{align*}
        d_{I^\circ}(\mfa, \mfb) &\le 2^{-100a} d_{B(c(J), 8D^{s(J)})}(\mfa, \mfb)\\
        &\le 2^{-100a + 5a} d_{B(c(J), \frac{1}{4}D^{s(J)})}(\mfa, \mfb)\\
        &= 2^{-95a}d_{J^\circ}(\mfa, \mfb)\,.
    \end{align*}
    This proves the second inequality claimed in the Lemma, from which the first follows since $a \ge 4$ and hence $2^{-95a} \le 1$.
\end{proof}

We also record the following basic estimates for the kernels $K_s$.

\begin{lemma}[kernel summand]
\label{kernel-summand}
\leanok
\lean{dist_mem_Icc_of_Ks_ne_zero, norm_Ks_le, norm_Ks_sub_Ks_le}
    Let $-S\le s\le S$ and $x,y,y'\in X$.
    If $K_s(x,y)\neq 0$, then we have
    \begin{equation}\label{supp-Ks}
      \frac{1}{4} D^{s-1} \leq \rho(x,y) \leq \frac{1}{2} D^s\, .
    \end{equation}
    We have
    \begin{equation}
       \label{eq-Ks-size}
        |K_s(x,y)|\le \frac{2^{102 a^3}}{\mu(B(x, D^{s}))}\,
    \end{equation}
    and
    \begin{equation}
        \label{eq-Ks-smooth}
        |K_s(x,y)-K_s(x, y')|\le \frac{2^{150a^3}}{\mu(B(x, D^{s}))}
        \left(\frac{ \rho(y,y')}{D^s}\right)^{\frac 1a}\,.
    \end{equation}
\end{lemma}

\begin{proof}
    By Definition \eqref{defks}, the function $K_s$ is the product of
    $K$ with a function which is supported in the set of all
    $x,y$ satisfying \eqref{supp-Ks}. This proves
    \eqref{supp-Ks}.

    Using \eqref{eqkernel-size} and the lower bound in \eqref{supp-Ks}
    we obtain
    \begin{equation}
        |K_s(x,y)|\le \frac{2^{a^3}}{\mu(B(x,\frac 14 D^{s-1}))}
    \end{equation}
    Using $D=2^{100a^2}$
    and the doubling property \eqref{doublingx} $2 +100a^2$ times estimates
    the last display by
    \begin{equation}
        \label{eq-Ks-aux}
        \le \frac{2^{2a+101a^3}}{\mu(B(x, D^{s}))}\, .
    \end{equation}
    Using $a\ge 4$ proves \eqref{eq-Ks-size}.

    If $2\rho(y,y') \le \rho(x,y)$, we obtain similarly with \eqref{eqkernel-y-smooth} and the lower bound in
    \eqref{supp-Ks}
    \begin{equation}
        |K_s(x,y)-K_s(x, y')|\le \frac{2^{a^3}}{\mu(B(x, \frac 14 D^{s-1}))}
        \left(\frac{ \rho(y,y')}{\frac 14 D^{s-1}}\right)^{\frac 1a}\,.
    \end{equation}
    As above, this is estimated by
    \begin{equation}
       \le \frac{4D 2^{2a+101a^3}}{\mu(B(x, D^{s}))}
        \left(\frac{ \rho(y,y')}{D^{s}}\right)^{\frac 1a}
         = \frac{2^{2+2a+100a^2+101a^3}}{\mu(B(x, D^{s}))}
        \left(\frac{ \rho(y,y')}{D^{s}}\right)^{\frac 1a}\,.
    \end{equation}
    Using $a\ge 4$, this proves \eqref{eq-Ks-smooth} in the case $2\rho(y,y') \le \rho(x,y)$. If $2\rho(y,y') > \rho(x,y)$, then by the lower bound in \eqref{supp-Ks} $2\rho(y,y') > \frac{1}{8}D^s$. Then \eqref{eq-Ks-smooth} follows from the triangle inequality, \eqref{eq-Ks-size} and $a \ge 4$.
\end{proof}


\chapter{Proof of Metric Space Carleson}
\label{thmfromproplinear}


Let Borel sets $F$, $G$ in $X$ be given.
We have that
\begin{equation}
    X=\bigcup_{R>0}B(o,R),
\end{equation} because every point of $X$
has finite distance from $o$.
\begin{lemma}[R truncation]
    \label{R-truncation}
    \uses{S-truncation}
    For all integers $R>0$
    $$
        \int \mathbf{1}_{G\cap B(o,R)}
        \sup_{1/R<R_1<R_2<R}\sup_{\mfa\in\Mf}
        \left| T_{R_1,R_2,\mfa} \mathbf{1}_{F}(x) \right|\, d\mu(x)
    $$
    \begin{equation} \label{Rcut}
        \leq \frac{2^{450a^3}}{(q-1)^5} \mu(G)^{1-\frac{1}{q}} \mu(F)^{\frac{1}{q}},
    \end{equation}
    where
    \begin{equation}\label{TRR}
        T_{R_1,R_2,\mfa} f(x)=
        \int_{R_1 < \rho(x,y) < R_2} K(x,y) f(y) e(\mfa(y)) \, \mathrm{d}\mu(y) .
    \end{equation}
\end{lemma}

We first show how \Cref{R-truncation} implies \Cref{metric-space-Carleson}. As $R$ tends to $\infty$, the integrand of the left-hand side of \eqref{Rcut} grows monotonically toward the integrand of the left-hand side of \eqref{resweak} for all $x$. By Lebesgue's monotone convergence theorem, the left-hand side of \eqref{Rcut} converges to the left-hand side of \eqref{resweak}. This verifies \Cref{metric-space-Carleson}.

It remains to prove \Cref{R-truncation}. Fix an integer $R>0$. By replacing $G$ with $G\cap B(o,R)$ if necessary, it suffices to show \eqref{Rcut} under the assumption that $G$ is contained in $B(o,R)$. We make this assumption. For every $x\in G$, the domain of integration in \eqref{TRR} is contained in $B(o,2R)$. By replacing $F$ with $F\cap B(o,2R)$ if necessary, it suffices to show \eqref{Rcut} under the assumption that $F$ is contained in $B(o,2R)$. We make this assumption.

Using the definition \eqref{defks} of $K_s$ and the partition of unity \eqref{eq-psisum}, we express \eqref{TRR} as the sum of
\begin{equation}\label{middles}
{T}_{1,s_1,s_2,\mfa}f(x):=\sum_{s_1 \le s\le s_2}
\int K_s(x,y) f(y) e(\mfa(y)) \, \mathrm{d}\mu(y)
\end{equation}
and
\begin{equation}\label{boundarys}
\sum_{s=s_1-2,s_1-1, s_2+1, s_2+2}
\int_{R_1 < \rho(x,y) < R_2} K_s(x,y) f(y) e(\mfa(y)) \,
 \mathrm{d}\mu(y),
\end{equation}
where $s_1$ is the smallest integer such that $D^{s_1-2}R_2>\frac 1{4D}$ and $s_2$ is the largest integer such that $D^{s_2+2}R_1<\frac 12$. We restrict the summation index $s$ by excluding summands with $s<s_1-2$ or $s>s_2+2$ because for these summands, the function $K_s$ vanishes on the domain of integration. We also omit the restriction in the integral for the summands in \eqref{middles} because in these summands, the support of $K_s$ is contained in the set described by this restriction.

We apply the triangle inequality and estimate the versions of \eqref{Rcut} separately with $T_{R_1,R_2,\mfa}$ replaced by \eqref{middles} and by each summand of \eqref{boundarys}. To handle the case \eqref{middles}, we employ the following lemma. Here, we utilize the fact that if $\frac 1R\le R_1\le R_2\le R$, then $s_1$ and $s_2$ as in \eqref{middles} are in an interval $[-S,S]$ for some sufficiently large $S$ depending on $R$.


\begin{lemma}[S truncation]
    \label{S-truncation}
    \uses{Hardy-Littlewood,kernel-summand,finitary-S-truncation}
    For all integers $S>0$
    $$
        \int \mathbf{1}_{G}(x)
        \max_{-S<s_1\le s_2<S}\sup_{\mfa\in\Mf}
        \left| T_{1, s_1,s_2,\mfa} \mathbf{1}_{F}(x) \right|\, d\mu(x)
    $$
    \begin{equation} \label{Scut}
        \leq \frac{2^{446a^3}}{(q-1)^5} \mu(G)^{1 - \frac{1}{q}} \mu(F)^{\frac{1}{q}},
    \end{equation}
    where $T_{1, s_1,s_2,\mfa}$ is defined in \eqref{middles}.
\end{lemma}

To reduce \Cref{R-truncation} to \Cref{S-truncation}, we need estimates for the summands in \eqref{boundarys}. Using \Cref{kernel-summand}, we obtain for arbitrary $s$ the inequality
\begin{multline}
\left|\int_{R_1 < \rho(x,y) < R_2} K_s(x,y) f(y) e(\mfa(y)) \, \mathrm{d}\mu(y)\right|\\
\leq \frac{2^{102 a^3}}{\mu(B(x, D^{s}))}
 \int_{B(x, D^s)} \mathbf{1}_F(y) \, \mathrm{d}\mu(y)
\leq 2^{102 a^3} M\mathbf{1}_F(x),
\end{multline}
where $M\mathbf{1}_F$ is as defined in \Cref{Hardy-Littlewood}.
Now, the left-hand side of \eqref{Rcut}, with $T_{R_1,R_2,\mfa}$ replaced by a summand of \eqref{boundarys}, can be estimated using H\"older's inequality and \Cref{Hardy-Littlewood} by
$$
    2^{102 a^3}\int \mathbf{1}_{G}(x) M\mathbf{1}_F(x)\, d\mu(x)
    \leq \frac{2^{102 a^3+4a}q}{q-1}\mu(G)^{1-\frac{1}{q}} \mu(F)^{\frac{1}{q}}\,.
$$
Applying the triangle inequality to estimate the left-hand side of \eqref{Rcut} by contributions from the summands in \eqref{middles} and \eqref{boundarys}, using \Cref{S-truncation} to control the first term, and the above to estimate the contribution from the four summands in \eqref{boundarys}, combined with $a\geq 4$ and $q < 2$, completes the reduction of \Cref{R-truncation} to \Cref{S-truncation}.

It remains to prove \Cref{S-truncation}. Fix $S>0$.
\begin{lemma}[finitary S truncation]
    \label{finitary-S-truncation}
    \uses{linearized-truncation}
    For all finite sets $\tilde{\Mf}\subset \Mf$
    $$
        \int \mathbf{1}_{G}(x)
        \max_{-S<s_1\le s_2<S}\sup_{\mfa\in\tilde{\Mf}}
        \left| T_{1, s_1,s_2,\mfa} \mathbf{1}_{F}(x) \right|\, d\mu(x)
    $$
    \begin{equation} \label{Sqcut}
    \leq \frac{2^{445a^3}}{(q-1)^5} \mu(G)^{1-\frac{1}{q}} \mu(F)^{\frac{1}{q}}\,.
    \end{equation}
\end{lemma}

We reduce \Cref{S-truncation} to \Cref{finitary-S-truncation}.
By the Lebesgue monotone convergence theorem,
applied to an increasing sequence of finite sets $\tilde{\Mf}$, inequality \eqref{Sqcut}
continues to hold for countable $\tilde{\Mf}$.

Let $\epsilon=\frac{1}{2S+1}$. Pick some $\mfa_0 \in \Mf$.
For $k \ge 0$, let the set $\tilde{\Mf}_k$ be a subset of $B_{B(o,2R)}(\mfa_0, k)$ of maximal size, such that for all $\mfa, \mfb \in \tilde{\Mf}_k$, it holds that $d_{B(o, 2R)}(\mfa, \mfb) \ge \epsilon$. Such a set exists, since by \Cref{ball-metric-entropy} there exists an upper bound for the size of such subsets in $B_{B(o, 2R)}(\mfa_0, k)$. Define
$$
    \tilde{\Mf} := \bigcup_{k \in \mathbb{N}} \tilde{\Mf}_k\,.
$$
Then the set $\tilde{\Mf}$ is at most countable, and it has the property that for any $\mfb \in \Mf$, there exists $\mfa \in \tilde{\Mf}$ with
$$
    d_{B(o, 2R)}(\mfb, \mfa) < \epsilon\,.
$$

For every $\mfa\in \Mf$, we have
\begin{equation}
\left| T_{1, s_1,s_2,\mfa} \mathbf{1}_{F}(x)\right|=
\left|\sum_{s_1 \le s\le s_2}
\int K_s(x,y) f(y) e(\mfa(y)-{\mfa(x)}) \, \mathrm{d}\mu(y)\right|
\end{equation}
Moreover, there is a $\tilde{\mfa}
\in \tilde{\Mf}$ with $d_{B(o,2R)}(\mfa,\tilde{\mfa})\le \epsilon$. Hence,
\begin{align*}
    &\left| T_{1, s_1,s_2,\mfa} \mathbf{1}_{F}(x)\right|-\left|T_{1, s_1,s_2,\tilde{\mfa}} \mathbf{1}_{F}(x) \right|\\
  &\leq\sum_{s_1 \le s\le s_2} \int |K_s(x,y)| \mathbf{1}_F(y) |e(\mfa(y)-{\mfa(x)})-e({\tilde{\mfa}(y)}-{\tilde{\mfa}(x)})| \, \mathrm{d}\mu(y)\\
    &\leq \sum_{s_1 \le s\le s_2} \epsilon \int |K_s(x,y)| \mathbf{1}_F(y) \, \mathrm{d}\mu(y)
\end{align*}
Using \Cref{kernel-summand}, we can estimate the above expression by
\begin{align*}
    &\sum_{s_1 \le s\le s_2} \frac{2^{102 a^3}}{\mu(B(x, D^{s}))}
\epsilon \int_{B(x, D^s)} \mathbf{1}_F(y) \, \mathrm{d}\mu(y)\\
    &\leq (2S+1)\epsilon 2^{102 a^3} M\mathbf{1}_F(x)\le 2^{102 a^3}M\mathbf{1}_F(x)
\end{align*}
We estimate the left-hand-side of \eqref{Scut} by
the sum of left-hand-side of \eqref{Sqcut} and
\begin{align*}
    &\int \mathbf{1}_{G}(x)
\max_{-S<s_1\le s_2<S}\sup_{\mfa\in {\Mf}}
\inf_{\tilde{\mfa}\in\tilde{\Mf}}
(| T_{1, s_1,s_2,\mfa}|-|T_{1, s_1,s_2,\tilde{\mfa}} |)\mathbf{1}_{F}(x)\, d\mu(x)\,,\\
\end{align*}
which, as we have just shown, is estimated by
\begin{align*}
     &2^{102 a^3}\int \mathbf{1}_{G}(x)
M\mathbf{1}_{F}(x)\, d\mu(x).
\end{align*}
By H\"older's inequality and \Cref{Hardy-Littlewood}
 (more precisely, \eqref{eq-hlm-2} with $p=q$), the above is no greater than
\begin{align*}
&\frac{2^{102 a^3+4a}q}{q-1} \mu(G)^{1-\frac{1}{q}} \mu(F)^{\frac{1}{q}}.
\end{align*}
Combining this with \Cref{finitary-S-truncation} and the fact that
\begin{align*}
    \frac{2^{102 a^3+4a}q}{q-1}\leq \frac{2^{445a^3}}{(q-1)^5}
\end{align*}
proves \Cref{S-truncation}.

It remains to prove \Cref{finitary-S-truncation}.
Fix a finite set
$\tilde{\Mf}$.
\begin{lemma}[linearized truncation]
    \label{linearized-truncation}
    \uses{finitary-Carleson}
    Let $\sigma_1,\sigma_2\colon X\to \mathbb{Z}$ be measurable functions with finite range
    $[-S,S]$ and $\sigma_1\leq \sigma_2$.
    Let $\tQ\colon X\to {\tilde{\Mf}}$ be a measurable function. Then we have
    \begin{equation} \label{Sqlin}
        \int \mathbf{1}_{G}(x)
        \left| {T}_{2,\sigma_1,\sigma_2,\tQ}\mathbf{1}_F(x)\right|\, d\mu(x)
        \le \frac{2^{445a^3}}{(q-1)^5} \mu(G)^{1-\frac{1}{q}} \mu(F)^{\frac{1}{q}},
    \end{equation}
    with
    \begin{equation}\label{middles1}
        {T}_{2,\sigma_1,\sigma_2,\tQ}f(x)=\sum_{\sigma_1(x) \le s\le \sigma_2(x)}
        \int K_s(x,y) f(y) e(\tQ(x)(y) - \tQ(x)(x)) \, \mathrm{d}\mu(y)\,.
    \end{equation}
\end{lemma}

We reduce \Cref{finitary-S-truncation} to
\Cref{linearized-truncation}.
For each $x$, let $\sigma_1(x)$ be the
minimal element $s'\in [-S,S]$ such that
\[\max_{s'\leq s_2<S}\max_{\mfa\in\tilde{\Mf}}
| T_{1, s',s_2,\mfa} \mathbf{1}_{F}(x)|
=
\max_{-S<s_1\le s_2<S}\max_{\mfa\in\tilde{\Mf}}
| {T}_{1, s_1,s_2,\mfa} \mathbf{1}_{F}(x)|:={T}_{1, x}.
\]
Similarly, let ${\sigma}_2(x)$ be the
minimal element $s''\in [-S,S]$ such that
\[\max_{\mfa\in\tilde{\Mf}}
| {T}_{1, {\sigma}_1(x), s'',\mfa} \mathbf{1}_{F}(x)|
=
{T}_{1,x}\,.
\]
Finally, choose a total order of the finite set $\tilde{\Mf}$
and let $\tQ(x)$ be the minimal element $\mfa$ with respect to this order such that
\[
| {T}_{1, {\sigma}_1(x),{\sigma}_2(x),\mfa} \mathbf{1}_{F}(x)| = {T}_{1,x}\,.
\]
With these choices, and noting that
\begin{equation*}
{T}_{1, {\sigma}_1(x),{\sigma}_2(x),\tQ(x)} \mathbf{1}_{F}(x)={T}_{2, {\sigma}_1,{\sigma}_2,\tQ} \mathbf{1}_{F}(x),
\end{equation*}
we conclude that the left-hand side of \eqref{Sqcut}
and \eqref{Sqlin} are equal. Thus, \Cref{finitary-S-truncation}
follows from \Cref{linearized-truncation}.

It remains to prove \Cref{linearized-truncation}.
Fix $\sigma_1$, $\sigma_2$,
and $\tQ$ as in the lemma.
Applying \Cref{finitary-Carleson} recursively, we obtain
a sequence of sets $G_n$ with $G_0=G$ and,
for each $n\ge 0$, $\mu(G_{n})\le 2^{-n} \mu(G)$ and
\begin{equation*}
    \int \mathbf{1}_{G_{n}\setminus G_{n+1}}(x)
    \left| {T}_{2,\sigma_1,\sigma_2, \tQ} \mathbf{1}_{F}(x) \right|\, d\mu(x)
\end{equation*}
\begin{equation}
    \le \frac{2^{440a^3}}{(q-1)^4} \mu(G_n)^{1 - \frac{1}{q}} \mu(F)^{\frac{1}{q}},
\end{equation}
Adding the first $n$ of these inequalities, we obtain by bounding a geometric series
    \begin{equation} \label{Sqcut2}
    \int \mathbf{1}_{G\setminus G_{n}}(x)
\left| {T}_{2,\sigma_1,\sigma_2, \tQ}\mathbf{1}_F(x) \right|\, d\mu(x)
\le \frac{2^{445a^3}}{(q-1)^5} \mu(G)^{1-\frac{1}{q}} \mu(F)^{\frac{1}{q}}.
\end{equation}
As the integrand is bounded by
\begin{equation}\mathbf{1}_{G\setminus G_{n}}(x)
\sum_{-S<s_1\le s_2<S}\sum_{\mfa\in\tilde{\Mf}}
\left| {T}_{1,s_1, s_2, \mfa} \mathbf{1}_{F}(x) \right|,
\end{equation}
which by interchange of summation and integration is seen to be integrable, we obtain by Lebesgue's dominated convergence theorem
 \begin{equation} \label{Sqcut3}
    \int \mathbf{1}_{G}(x)
\left| {T}_{2,\sigma_1,\sigma_2, \tQ}\mathbf{1}_F(x) \right|\, d\mu(x)
\le \frac{2^{445a^3}}{(q-1)^5} \mu(G)^{1-\frac{1}{q}} \mu(F)^{\frac{1}{q}}.
\end{equation}



This completes the proof of \Cref{linearized-truncation}
and thus \Cref{metric-space-Carleson}.

\chapter{Proof of Finitary Carleson}
\label{christsection}

To prove Proposition
\ref{finitary-Carleson}, we already fixed in \Cref{overviewsection}
measurable functions ${\sigma_1},\sigma_2, \tQ$ and Borel sets $F,G$. We have also
defined $S$ to be the smallest
integer such that the ranges of
$\sigma_1$ and $\sigma_2$ are contained in $[-S,S]$ and $F$ and $G$ are contained
in the ball $B(o, D^S)$.


The proof of the next lemma is done in \Cref{subsecdyadic},
following the construction of dyadic cubes in \cite[\S 3]{christ1990b}.

\begin{lemma}[grid existence]
    \label{grid-existence}
    \leanok
    \lean{grid_existence}
    \uses{counting-balls,boundary-measure} There exists a grid structure $(\mathcal{D}, c,s)$.
\end{lemma}




The next lemma, which we prove in \Cref{subsectiles}, should be compared
with the construction in \cite[Lemma 2.12]{zk-polynomial}.

\begin{lemma}[tile structure]
    \label{tile-structure}
    \leanok
    \lean{tile_existence}
    \uses{ball-metric-entropy,frequency-ball-cover,disjoint-frequency-cubes,frequency-cube-cover}
        For a given grid structure $(\mathcal{D}, c,s)$, there exists a tile structure
        $(\fP,\scI,\fc,\fcc,\pc,\ps)$.
\end{lemma}

Choose a grid structure $(\mathcal{D}, c,s)$ with \Cref{grid-existence} and a tile structure for this
grid structure $(\fP,\scI,\fc,\fcc,\pc,\ps)$ with \Cref{tile-structure}.
Applying \Cref{discrete-Carleson}, we obtain a Borel set $G'$ in $X$ with $2\mu(G')\leq \mu(G)$ such that for all Borel functions $f:X\to \C$ with $|f|\le \mathbf{1}_F$
we have \eqref{disclesssim}.

\begin{lemma}[tile sum operator]
    \label{tile-sum-operator}
    \leanok
    \lean{tile_sum_operator, integrable_tile_sum_operator}
    We have for all $x\in G\setminus G'$
    \begin{equation}\label{eq-sump}
        \sum_{\fp\in \fP}T_{\fp} f(x)= \sum_{s=\sigma_1(x)}^{\sigma_2(x)}
        \int K_{s}(x,y) f(y) e(\tQ(x)(y)-\tQ(x)(x))\, d\mu(y).
    \end{equation}
\end{lemma}
\begin{proof}
    Fix $x\in G\setminus G'$.
    Sorting the tiles $\fp$ on the left-hand-side of \eqref{eq-sump} by the value $\ps(\fp)\in [-S,S]$,
    it suffices to prove for every $-S\le s\le S$ that
    \begin{equation}\label{outsump}
        \sum_{\fp\in \fP: \ps(\fp)=s}T_{\fp} f(x)=0
    \end{equation}
    if $s\not\in [\sigma_1(x), \sigma_2(x)]$ and
    \begin{equation}\label{insump}
        \sum_{\fp\in \fP: \ps(\fp)=s}T_{\fp} f(x)=
        \int K_{s}(x,y) f(y) e(\tQ(x)(y) - \tQ(x)(x))\, d\mu(y).
    \end{equation}
    if $s\in [\sigma_1(x),\sigma_2(x)]$.
    If $s\not\in [\sigma_1(x), \sigma_2(x)]$, then by definition of $E(\fp)$ we have
    $x\not\in E(\fp)$ for any $\fp$ with $\ps(\fp)=s$ and thus $T_{\fp} f(x)=0$. This proves
    \eqref{outsump}.

    Now assume $s\in [\sigma_1(x),\sigma_2(x)]$.
    By \eqref{coverball} and $G\subset B(o,D^S)$, there is at least
    one $I\in \mathcal{D}$ with $s(I)=s$ and $x\in I$.
    By \eqref{dyadicproperty}, this $I$ is unique. By \eqref{eq-dis-freq-cover}, there is precisely one $\fp\in \fP(I)$ such that
    $\tQ(x)\in \fc(\fp)$. Hence there is precisely one $\fp\in \fP$ with $\ps(\fp)=s$ such that
    $x\in E(\fp)$. For this $\fp$, the value $T_{\fp}(x)$ by its definition in \eqref{definetp}
    equals the right-hand side of \eqref{insump}. This proves the lemma.
\end{proof}

We use this to prove \Cref{finitary-Carleson}.
\begin{proof}[Proof of \Cref{finitary-Carleson}]
\proves{finitary-Carleson}
We now estimate with \Cref{tile-sum-operator} and \Cref{discrete-Carleson}
\begin{equation}
 \int_{G \setminus G'} \left|\sum_{s={\sigma_1}(x)}^{{\sigma_2}(x)} \int K_s(x,y) f(y) e(\tQ(x)(y)) \, \mathrm{d}\mu(y)\right| \mathrm{d}\mu(x)
\end{equation}
\begin{equation}
 =\int_{G \setminus G'} \left|\sum_{s={\sigma_1}(x)}^{{\sigma_2}(x)} \int K_s(x,y) f(y) e(\tQ(x)(y) - \tQ(x)(x))\mathrm{d}\mu(y)\right| \mathrm{d}\mu(x)
\end{equation}
\begin{equation}
 =\int_{G \setminus G'} \left|\sum_{\fp\in \fP}T_{\fp} f(x)\right| \mathrm{d}\mu(x)
 \le \frac{2^{440a^3}}{(q-1)^4} \mu(G)^{1 - \frac{1}{q}} \mu(F)^{\frac{1}{q}} \,.
\end{equation}
This proves \eqref{eq-linearized} for the chosen set $G'$ and arbitrary $f$ and thus completes the proof of Proposition
\ref{finitary-Carleson}.
\end{proof}


\section{Proof of Grid Existence Lemma}
\label{subsecdyadic}

We begin with the construction of the centers of the dyadic cubes.
\begin{lemma}[counting balls]
    \label{counting-balls}
    \leanok
    \lean{counting_balls}
    Let $-S\le k\le S$. Consider $Y\subset X$ such that for any $y\in Y$, we have
    \begin{equation}\label{ybinb}
    y\in B(o,4D^S-D^k),
    \end{equation}
    furthermore, for any $y'\in Y$ with $y\neq y'$, we have
    \begin{equation} \label{eq-disj-yballs}
        B(y,D^k)\cap B(y',D^k)=\emptyset.
    \end{equation}
    Then the cardinality of $Y$ is bounded by
    \begin{equation}\label{boundY}
        |Y|\le 2^{3a + 200Sa^3}\, .
    \end{equation}
\end{lemma}


\begin{proof}
    \leanok
Let $k$ and $Y$ be given. By applying the doubling property \eqref{doublingx} inductively, we have for each integer $j\ge 0$
\begin{equation}\label{jballs}
    \mu(B(y,2^{j}D^k))\le 2^{aj} \mu(B(y,D^k))\, .
\end{equation}
Since $X$ is the union of the balls $B(y,2^{j}D^k)$ and $\mu$ is not zero, at least one of the balls $B(y,2^{j}D^k)$ has positive measure, thus $B(y,D^k)$ has positive measure.

Applying \eqref{jballs} for $j' = \ln_2(8D^{2S}) = 3 + 2S \cdot 100a^2$ by \eqref{defineD}, using $-S\le k\le S$, $y\in B(o,4D^S)$, and the triangle inequality, we have
\begin{equation}
    B(o, 4D^S) \subset B(y, 8D^S) \subset B(y,2^{j'}D^k) \, .
\end{equation}
Using the disjointedness of the balls in \eqref{eq-disj-yballs}, \eqref{ybinb}, and the triangle inequality for $\rho$, we obtain
\begin{equation}
|Y|\mu(B(o,4D^S))\le 2^{j'a}\sum_{y\in Y}\mu(B(y,D^k))
\end{equation}
\begin{equation}
\le
2^{j'a}\mu(\bigcup_{y\in Y}B(y,D^k))
\le 2^{j'a}\mu(o,4D^S)\, .
\end{equation}
As $\mu(o,4D^S)$ is not zero, the lemma follows.
\end{proof}

For each $-S\le k\le S$, let $Y_k$ be a set of
maximal cardinality in $X$ such that $Y=Y_k$ satisfies
the properties \eqref{ybinb} and \eqref{eq-disj-yballs}.
By the upper bound of \Cref{counting-balls}, such a set exists.

For each $-S\le k\le S$, choose an enumeration of the points in the finite set $Y_k$ and thus a total
order $<$ on $Y_{k}$.

\begin{lemma}[cover big ball]
    \label{cover-big-ball}
    \leanok
    \lean{cover_big_ball}
    For each $-S\le k\le S$, the ball
    $B(o, 4D^S-D^k)$ is contained
    in the union of the balls $B(y,2D^k)$ with $y\in Y_k$.
\end{lemma}

\begin{proof}
\leanok
Let $x$ be any point of $B(o, 4D^S-D^k)$. By maximality of $|Y_k|$, the ball
$B(x, D^k)$ intersects one of the balls
$B(y, D^k)$ with $y\in Y_k$. By the triangle
inequality, $x\in B(y,2D^k)$.
\end{proof}

Define the set
\begin{equation}
    \mathcal{C}:= \{(y,k): -S\le k\le S, y\in Y_k\}\,
\end{equation}
We totally order the set $\mathcal{C}$ lexicographically by setting
$(y,k)<(y',k')$ if $k< k'$ or both $k=k'$ and $y<y'$.
In what follows, we define recursively in the sense of this order a function
\begin{equation}
    (I_1,I_2,I_3): \mathcal{C}\to \mathcal{P}(X)\times \mathcal{P}(X)\times \mathcal{P}(X)\, .
\end{equation}


Assume the sets ${I}_j(y',k')$ have already been defined for $j=1,2,3$ if $k'<k$ and if $k=k'$ and $y'<y$.




If $k=-S$, define for $j\in \{1,2\}$ the set
${I}_j(y,k)$ to be $B(y,jD^{-S})$.
If $-S<k$, define for $j\in \{1,2\}$
and $y\in Y_k$ the set ${I}_j(y,k)$ to be
\begin{equation}\label{defineij}
\bigcup\{I_3(y',k-1):
y'\in Y_{k-1}\cap B(y,jD^k)\}.
\end{equation}
Define for {$-S\leq k\leq S$} and $y\in Y_k$
\begin{equation}\label{definei3}
I_3(y,k):={I_1}(y,k)\cup \left[{I_2}(y,k)\setminus \left[X_k\cup \bigcup\{I_3(y',k):y'\in Y_{k}, y'<y\})\right]\right]
\end{equation}
with
\begin{equation}
      X_{k}:=\bigcup\{I_1(y', k):y'\in Y_{k}\}.
\end{equation}


\begin{lemma}[basic grid structure]
    \label{basic-grid-structure}
    \uses{cover-big-ball}
    For each $-S\le k\le S$ and $1\le j\le 3$
    the following holds.

    If $j\neq 2$ and for some $x\in X$ and $y_1,y_2\in Y_k$ we have
    \begin{equation}\label{disji}
        x\in I_j(y_1,k)\cap I_j(y_2,k),
    \end{equation}
    then $y_1=y_2$.

    If $j\neq 1$, then
    \begin{equation}\label{unioni}
    B(o, 4D^S-2D^k)\subset \bigcup_{y\in Y_k} I_j(y,k)\, .
    \end{equation}
    We have for each $y\in Y_k$,
    \begin{equation}\label{squeezedyadic}
        B(y,\frac 12 D^k) \subset I_3(y,k)\subset
        B(y,4D^k).
    \end{equation}
\end{lemma}

\begin{proof}
We prove these statements simultaneously by induction on the ordered set of pairs $(y,k)$. Let $-S\le k\le S$.

We first consider \eqref{disji} for $j=1$. If $k=-S$, disjointedness of the sets $I_1(y,-S)$ follows by definition of $I_1$ and $Y_k$. If $k>-S$, assume $x$ is in $I_1(y_m,k)$ for $m=1,2$. Then, for $m=1,2$, there is $z_m\in Y_{k-1}\cap B(y_m,D^k)$ with $x\in I_3(z_m,k-1)$. Using \eqref{disji} inductively for $j=3$, we conclude $z_1=z_2$. This implies that the balls $B(y_1, D^k)$ and $B(y_2, D^k)$ intersect. By construction of $Y_k$, this implies $y_1=y_2$. This proves \eqref{disji} for $j=1$.

We next consider \eqref{disji} for $j=3$. Assume $x$ is in $I_3(y_m,k)$ for $m=1,2$ and $y_m\in Y_k$. If $x$ is in $X_k$, then by definition \eqref{definei3}, $x\in I_1(y_m,k)$ for $m=1,2$. As we have already shown \eqref{disji} for $j=1$, we conclude $y_1=y_2$. This completes the proof in case $x\in X_k$, and we may assume $x$ is not in $X_k$. By definition \eqref{definei3}, $x$ is not in $I_3(z,k)$ for any $z$ with $z<y_1$ or $z<y_2$. Hence, neither $y_1<y_2$ nor $y_2<y_1$, and by totality of the order of $Y_k$, we have $y_1=y_2$. This completes the proof of \eqref{disji} for $j=3$.

We show \eqref{unioni} for $j=2$. In case $k=-S$, this follows from \Cref{cover-big-ball}. Assume $k>-S$. Let $x$ be a point of $B(o, 4D^S-2D^k)$. By induction, there is $y'\in Y_{k-1}$ such that $x\in I_3(y',k-1)$. Using the inductive statement \eqref{squeezedyadic}, we obtain $x\in B(y',4D^{k-1})$. As $D>4$, by applying the triangle inequality with the points, $o$, $x$, and $y'$, we obtain that $y'\in B(o, 4D^S-D^k)$. By \Cref{cover-big-ball}, $y'$ is in $B(y,2D^k)$ for some $y\in Y_k$. It follows that $x\in I_2(y,k)$. This proves \eqref{unioni} for $j=2$.

We show \eqref{unioni} for $j=3$. Let $x\in B(o, 4D^S-2D^k)$. In case $x\in X_k$, then by definition of $X_k$ we have $x\in I_1(y,k)$ for some $y\in Y_k$ and thus $x\in I_3(y,k)$. We may thus assume $x\not\in X_k$. As we have already seen \eqref{unioni} for $j=2$, there is $y\in Y_k$ such that $x\in I_2(y,k)$. We may assume this $y$ is minimal with respect to the order in $Y_k$. Then $x\in I_3(y,k)$. This proves \eqref{unioni} for $j=3$.

Next, we show the first inclusion in \eqref{squeezedyadic}. Let $x\in B(y,\frac 12 D^k)$. As $I_1(y,k)\subset I_3(y,k)$, it suffices to show $x\in I_1(y,k)$. If $k=-S$, this follows immediately from the assumption on $x$ and the definition of $I_1$. Assume $k>-S$. By the inductive statement \eqref{unioni} and $D>4$, there is a $y'\in Y_{k-1}$ such that $x\in I_3(y',k-1)$. By the inductive statement \eqref{squeezedyadic}, we conclude $x\in B(y',4D^{k-1})$. By the triangle inequality with points $x$, $y$, $y'$, and $D>4$, we have $y'\in B(y,D^k)$. It follows by definition \eqref{defineij} that $I_3(y',k-1)\subset I_1(y,k)$, and thus $x\in I_3(y,k)$. This proves the first inclusion in \eqref{squeezedyadic}.

We show the second inclusion in \eqref{squeezedyadic}. Let $x\in I_3(y,k)$. As $I_1(y,k)\subset I_2(y,k)$ directly from the definition \eqref{defineij}, it follows by definition \eqref{definei3} that $x\in I_2(y,k)$. By definition \eqref{defineij}, there is $y'\in Y_{k-1}\cap B(y,2D^k)$ with $x\in I_3(y',k-1)$. By induction, $x\in B(y', 4D^{k-1})$. By the triangle inequality applied to the points $x,y',y$ and $D>4$, we conclude $x\in B(y,4D^k)$. This shows the second inclusion in \eqref{squeezedyadic} and completes the proof of the lemma.
\end{proof}

\begin{lemma}[cover by cubes]
    \label{cover-by-cubes}
    Let $-S\le l\le k\le S$ and
    $y\in Y_k$.
    We have
    \begin{equation}\label{3coverdyadic}
    I_3(y,k)\subset \bigcup_{y'\in Y_l} I_3(y',l)\, .
    \end{equation}
\end{lemma}
\begin{proof}
    Let $-S\le l\le k\le S$ and $y\in Y_k$.
    If $l=k$, the inclusion \eqref{3coverdyadic}
    is true from the definition of set union.
    We may then assume inductively that $k>l$ and the statement of the lemma is true if $k$ is replaced by $k-1$.
    Let $x\in I_3(y,k)$.
    By definition \eqref{definei3}, $x\in I_j(y,k)$
    for some $j\in \{1,2\}$. By \eqref{defineij},
    $x\in I_3(w,k-1)$ for some $w\in Y_{k-1}$.
    We conclude \eqref{3coverdyadic} by induction.
\end{proof}

\begin{lemma}[dyadic property]
    \label{dyadic-property}
    \uses{basic-grid-structure, cover-by-cubes}
    Let $-S\le l\le k\le S$ and $y\in Y_k$ and $y'\in Y_l$ with
    $I_3(y',l)\cap I_3(y,k)\neq \emptyset$. Then
    \begin{equation}
        \label{3dyadicproperty}
    I_3(y',l)\subset I_3(y,k).
    \end{equation}
\end{lemma}

\begin{proof}
Let $l,k,y,y'$ be as in the lemma. Pick $x\in I_3(y',l)\cap I_3(y,k)$. Assume first $l=k$. By \eqref{disji} of \Cref{basic-grid-structure}, we conclude $y'=y$, and thus \eqref{3dyadicproperty}. Now assume $l<k$. By induction, we may assume that the statement of the lemma is proven for $k-1$ in place of $k$.

By \Cref{cover-by-cubes}, there is a $y''\in Y_{k-1}$ such that $x\in I_3(y'',k-1)$. By induction, we have $I_3(y',l)\subset I_3(y'',k-1)$. If $l=k-1$, then by the disjointedness property of \Cref{basic-grid-structure}, we see that $y'=y''$, and hence \eqref{3dyadicproperty} follows. For $l<k-1$, it remains to prove
\begin{equation}\label{wyclaim}
I_3(y'',k-1)\subset I_3(y,k).
\end{equation}
We make a case distinction and assume first $x\in X_k$. By Definition \eqref{definei3}, we have $x\in I_1(y,k)$. By Definition \eqref{defineij}, there is a $v\in Y_{k-1}\cap B(y,D^k)$ with $x\in I_3(v,k-1)$. By \eqref{disji} of \Cref{basic-grid-structure}, we have $v=y''$. By Definition \eqref{defineij}, we then have $I_3(y'',k-1)\subset I_1(y,k)$. Then \eqref{wyclaim} follows by Definition \eqref{definei3} in the given case.

Assume now the case $x\notin X_k$. By \eqref{definei3}, we have $x\in I_2(y,k)$. Moreover, for any $u<y$ in $Y_k$, we have $x\not\in I_3(u,k)$. Let $u<y$. By transitivity of the order in $Y_k$, we conclude $x\not \in I_2(u,k)$. By \eqref{defineij} and the disjointedness property of \Cref{basic-grid-structure}, we have $I_3(y'',k-1)\cap I_2(u,k)= \emptyset$. Similarly, $I_3(y'',k-1)\cap I_1(u,k)= \emptyset$. Hence $I_3(y'',k-1)\cap I_3(u,k)=\emptyset$. As $u<y$ was arbitrary, we conclude with \eqref{definei3} the claim in the given case. This completes the proof of \eqref{wyclaim}, and thus also \eqref{3dyadicproperty}.
\end{proof}

For $-S\le k'\le k\le S$ and $y'\in Y_{k'}$, $y\in Y_k$
write $(y',k'|y,k)$ if $I_3(y',k')\subset I_3(y,k)$ and
\begin{equation}\label{bdcond}
    \inf_{x\in X\setminus I_3(y,k)}\rho(y',x)<6D^{k'}\, .
\end{equation}

\begin{lemma}[transitive boundary]
    \label{transitive-boundary}
    \uses{dyadic-property}
    Assume $-S\le k''< k'< k\le S$ and
    $y''\in Y_{k''}$, $y'\in Y_{k'}$, $y\in Y_k$.
    Assume there is $x\in X$ such that
    \begin{equation}
    x\in I_3(y'',k'')\cap I_3(y',k')\cap I_3(y,k)\, .
    \end{equation}
    If $(y'',k''|y,k)$, the also
    $(y'',k''|y',k')$ and $(y',k'|y,k)$
\end{lemma}

\begin{proof}
    As $x\in I_3(y'',k'')\cap I_3(y',k')$ and $k''< k'$, we have by \Cref{dyadic-property} that
    $I_3(y'',k'')\subset I_3(y',k')$. Similarly,
    $I_3(y',k')\subset I_3(y,k)$.
    Pick $x'\in X\setminus I_3(y,k)$ such that
    \begin{equation}\label{yppxp}
        \rho(y'',x')< 6D^{k''}\, ,
    \end{equation}
    which exists as $(y'',k''|y,k)$. As
    $x'\in X\setminus I_3(y',k')$ as well, we conclude
    $(y'',k''| y',k')$.
    By the triangle inequality, we have
    \begin{equation}
    \rho(y',x')\le \rho(y',x)+\rho(x,y'')+\rho(y'',x')
    \end{equation}
    Using the choice of $x$ and \eqref{squeezedyadic}
    as well as \eqref{yppxp}, we estimate this by
    \begin{equation}
    < 4D^{k'}+4D^{k''}+6D^{k''}\le 6D^{k'}\, ,
    \end{equation}
    where we have used $D>5$ and $k''<k'$.
    We conclude $(y',k'|y,k)$.
\end{proof}


\begin{lemma}[small boundary]
    \label{small-boundary}
    \uses{transitive-boundary}
    Let $K = 2^{4a+1}$. For each $-S+K\le k\le S$ and $y\in Y_k$ we have
        \begin{equation}
            \label{new-small-boundary}
            \sum_{z\in Y_{k-K}: (z,k-K|y,k)}\mu(I_3(z,k-K)) \le \frac 12 \mu(I_3(y,k))\,.
        \end{equation}
\end{lemma}

\begin{proof}
Let $K$ be as in the lemma. Let $-S+K\le k\le S$ and $y\in Y_k$.

Pick $k'$ so that $k-K\le k'\le k$.
For each $y''\in Y_{k-K}$ with $(y'',k-K| y,k)$,
by \Cref{cover-by-cubes} and \Cref{dyadic-property}, there is a unique $y'\in Y_{k'}$ such that
\begin{equation}\label{4.31}
    I_3(y'',k-K)\subset I_3(y',k')\subset I_3(y,k)\, .
\end{equation}
Using \Cref{transitive-boundary}, $(y',k'|y,k)$.

We conclude using the disjointedness property of
\Cref{basic-grid-structure} that
\begin{equation}\label{scalecompare}
    \sum_{y'': (y'',k-K|y,k)}\mu(I_3(y'',k-K))
   \le
\sum_{y': (y',k'|y,k)}
\mu(I_3(y',k')) \, .
   \end{equation}
Adding over $k-K<k'\le k$, and using
\[\mu(I_3(y',k'))\le 2^{4a} \mu(B(y', \frac 14 D^{k'}))\]
from the doubling property \eqref{doublingx} and
\eqref{squeezedyadic} gives
\begin{equation}\label{sumcompare}
    K\sum_{y'': (y'',k-K|y,k)}
    \mu(I_3(y'',k-K))
\end{equation}
\begin{equation}\label{sumcompare1}
    \le 2^{4a} \sum_{k-K<k'\le k}
   \left[ \sum_{y': (y',k'|y,k)}
\mu(B(y', \frac 14 D^{k'}))\right]
\end{equation}
Each ball $B(y', \frac 14 D^{k'})$ occurring in
\eqref{sumcompare1} is contained in $I_3(y',k')$
by \eqref{squeezedyadic} and in turn contained in
$I_3(y,k)$ by \eqref{4.31}. Assume for the moment all these balls are pairwise disjoint. Then
by additivity of the measure,
\begin{equation}
    K\sum_{y'': (y'',k-K|y,k)}
    \mu(I_3(y'',k-K))
    \le 2^{4a}
\mu(I_3(y,k))\,
\end{equation}
which by $K=2^{4a+1}$ implies \eqref{new-small-boundary}.

It thus remains to prove that the balls
occurring in
\eqref{sumcompare1} are pairwise disjoint.
Let $(u,l)$ and $(u',l')$ be two parameter pairs occurring
in the sum of \eqref{sumcompare1} and let
$ B(u, \frac 14 D^l)$ and $B({u'}, \frac 14 D^{l'})$
be the corresponding balls. If
$l=l'$, then the balls are equal or disjoint by
\eqref{squeezedyadic} and \eqref{disji} of \Cref{basic-grid-structure}. Assume then without loss of generality that $l'<l$. Towards a contradiction, assume that
\begin{equation}\label{bulbul}
    B(u, \frac 14 D^l)\cap B({u'}, \frac 14 D^{l'})\neq \emptyset
\end{equation}
As $(u',l'|y,k)$, there is a point $x$ in
$X\setminus I_3(y,k)$ with $\rho(x,u')<6D^{l'}$.
Using $D>25$, we conclude from the triangle inequality and \eqref{bulbul} that
$x\in B(u,\frac 12D^l)$. However, $B(u,\frac 12 D^l)\subset I_3(u,l)$,
and $I_3(u,l)\subset I_3(y,k)$, a contradiction to
$x\not\in I_3(y,k)$.
This proves the lemma.
\end{proof}

\begin{lemma}[smaller boundary]
    \label{smaller-boundary}
    \uses{small-boundary}
    Let $K = 2^{4a+1}$
    and let $n\ge 0$ be an integer. Then
    for each $-S+nK\le k\le S$ we have
    \begin{equation}
            \label{very-new-small}
            \sum_{y'\in Y_{k-nK}: (y',k-nK|y,k)}\mu(I_3(y',k-nK)) \le 2^{-n} \mu(I_3(y,k))\,.
        \end{equation}
\end{lemma}
\begin{proof}
    We prove this by induction on $n$. If $n=0$,
    both sides of \eqref{very-new-small} are equal to
    $\mu(I_3(y,k))$ by \eqref{disji}. If $n=1$, this follows from \Cref{small-boundary}.

    Assume $n>1$ and \eqref{very-new-small} has been proven for $n-1$.
    We write \eqref{very-new-small}
    \begin{equation}
        \sum_{y''\in Y_{k-nK}: (y'',k-nK|y,k)}\mu(I_3(y'',k-nK))
    \end{equation}
    \begin{equation}
        = \sum_{y'\in Y_{k-K}:(y',k-K|y,k)} \left[ \sum_{y''\in Y_{k-nK}: (y'',k-nK|y',k-K)}\mu(I_3(y'',k-nK)) \right]
    \end{equation}
    Applying the induction hypothesis, this is bounded by
    \begin{equation}
        = \sum_{y'\in Y_{k-K}:(y',k-K|y,k)} 2^{1-n}\mu(I_3(y',k-K))
    \end{equation}
    Applying \eqref{new-small-boundary} gives
    \eqref{very-new-small}, and proves the lemma.
\end{proof}

\begin{lemma}[boundary measure]
    \label{boundary-measure}
    \uses{smaller-boundary}
    For each $-S\le k\le S$ and $y\in Y_k$ and $0<t<1$
    with $tD^k\ge D^{-S}$ we have
    \begin{equation}
        \label{old-small-boundary}
        \mu(\{x \in I_3(y,k) \ : \ \rho(x, X \setminus I_3(y,k)) \leq t D^{k}\}) \le D t^\kappa \mu(I_3(y,k))\,.
    \end{equation}
\end{lemma}
\begin{proof}
Let $x\in I_3(y,k)$ with $\rho(x, X \setminus I_3(y,k)) \leq t D^{k}$. Let $K = 2^{4a+1}$ as in \Cref{smaller-boundary}.
Let $n$ be the largest integer such that
$D^{nK} \le \frac{1}{t}$, so that $tD^k \le D^{k-nK}$ and
\begin{equation}
\label{eq-n-size}
    D^{nK} > \frac{1}{tD^K}\,.
\end{equation}
Let $k' = k - nK$, by the assumption $tD^k \ge D^{-S}$, we have $k' \ge -S$. By \eqref{3coverdyadic}, there exists $y' \in Y_{k'}$ with $x \in I_3(y',k')$. By the squeezing property \eqref{squeezedyadic} and the assumption on $x$, we have
$$
    \rho(y', X \setminus I_3(y,k)) \le \rho(x,y') + \rho(x, X \setminus I_3(y,k)) \le 4 D^{k'} + t D^{k}\,.
$$
By the assumption on $n$ and the definition of $k'$, this is
$$
    \le 4D^{k'} + D^{k - nK} < 6D^{k'}\,.
$$
Together with \eqref{3dyadicproperty} thus $(y',k'|y,k)$. We have shown that
$$
    \{x \in I_3(y,k) \ : \ \rho(x, X \setminus I_3(y,k)) \leq t D^{k}\}
$$
$$
    \subset \bigcup_{y'\in Y_{k-nK}: (y',k-nK|y,k)}I_3(y',k-nK)\,.
$$
Using monotonicity and additivity of the measure and \Cref{smaller-boundary}, we obtain
$$
    \mu(\{x \in I_3(y,k) \ : \ \rho(x, X \setminus I_3(y,k)) \leq t D^{k}\}) \le 2^{-n} \mu(I_3(y,k))\,.
$$
By \eqref{eq-n-size} and the definition \eqref{defineD} of $D$, this is bounded by
$$
    t^{1/(100a^2 K)} D \mu(I_3(y,k))\,,
$$
which completes the proof by the definition \eqref{definekappa} of $\kappa$.
\end{proof}

Let $\mathcal{D}$ be the set of all $I_3(y,k)$ with $k\in [-S,S]$ and
$y\in Y_k$. Define
\begin{equation}
s(I_3(y,k)):=k
\end{equation}
\begin{equation}
c(I_3(y,k)):=y\,.
\end{equation}

\begin{proof}[Proof of \Cref{grid-existence}]
\proves{grid-existence}
We show that $(\mathcal{D},c,s)$ constitutes a grid structure. Property \eqref{eq-vol-sp-cube}
follows from \eqref{squeezedyadic}, while \eqref{eq-small-boundary} follows from \Cref{boundary-measure}.

Let $x\in B(o, D^S)$. We show properties
\eqref{coverdyadic},
\eqref{dyadicproperty} and
\eqref{coverball}
for $(\mathcal{D},c,s)$ and $x$.

We first show \eqref{coverdyadic} by contradiction. Then there is an $I$ violating the conclusion of
\eqref{coverdyadic}. Pick such $I=I_3(y,l)$ such that $l$ is minimal.
{By assumption, we have $-S\le k<l$; in particular $-S<l$}.
By definition, $I_3(y, l)$ is contained in $I_1(y, l)\cup I_2(y, l)$, which is contained in the union of $I_3(y',l-1)$ with $y'\in Y_{l-1}$.
By minimality of $l$, each such $I_3(y',l-1)$ is contained in the union of
all $I_3(z,k)$ with $z\in Y_k$. This proves \eqref{coverdyadic}.

We now show \eqref{dyadicproperty}. Assume to get a contradiction that
there are non-disjoint $I, J\in \mathcal{D}$ with $s(I)\le s(J)$
and $I \not \subset J$. We may assume the existence of such $I$ and $J$ with minimal
$s(J)-s(I)$. Let $k=s(I)$. Assume first $s(J)=k$. Let $I=I_3(y_1,k)$ and $J=I_3(y_2,k)$ with $y_1,y_2\in Y_k$.
If $y_1=y_2$, then $I=J$, a contradiction to $I\not \subset J$.
If $y_1\neq y_2$, then $I\cap J=\emptyset$ by \eqref{disji}, a contradiction to the non-disjointedness of $I,J$.
Assume now $s(J)>k$. Choose $y\in I\cap J$. By property \eqref{coverdyadic},
there is $K\in \mathcal{D}$ with $s(K)=s(J)-1$ and $y\in K$. By construction
of $J$, and pairwise disjointedness of all $I_3(w,s(J)-1)$ that we have already seen,
we have $K\subset J$. By minimality of $s(J)$, we have $I\subset K$.
This proves $I\subset J$ and thus \eqref{dyadicproperty}.

We next establish \eqref{coverball}. Let $-S\leq k\leq S$. Using \eqref{unioni} for $j=3$, we get
\begin{equation}
x\in B(o, D^S)\subset B(o, 4D^S-2D^k)\subset \bigcup_{y\in Y_k} I_3(y,k)\, .
\end{equation}
Thus, there exists a dyadic cube $I=I_3(y, k)$ with $s(I)=k$ and $x\in I$.
This proves \eqref{coverball}.
\end{proof}

\section{Proof of Tile Structure Lemma}
\label{subsectiles}
Choose a grid structure $(\mathcal{D}, c, s)$ with \Cref{grid-existence}
Let $I \in \mathcal{D}$. Suppose that
\begin{equation}
    \label{eq-tile-Z}
    \mathcal{Z} \subset \bigcup_{\mfa \in \tQ(X)} B_{I^\circ}(\mfa, 1)
\end{equation}
is such that for any $\mfa, \mfb \in \mathcal{Z}$ with $\mfa\ne \mfb$ we have
\begin{equation}
    \label{eq-tile-disjoint-Z}
    B_{I^\circ}(\mfa, 0.3) \cap B_{I^\circ}(\mfb, 0.3) = \emptyset\,.
\end{equation}
By \Cref{ball-metric-entropy} applied to each of the balls $B_{I^\circ}(\mfa, 1)$, $\mfa \in \tQ(X)$, we have
$$
    |\mathcal{Z}| \le 2^{2a} |\tQ(X)|\,.
$$
In particular, there exists a set $\mathcal{Z}$ satisfying both \eqref{eq-tile-Z} and \eqref{eq-tile-disjoint-Z} of maximal cardinality among all such sets. We pick for each $I \in \mathcal{D}$ such a set $\mathcal{Z}(I)$.

\begin{lemma}[frequency ball cover]
    \label{frequency-ball-cover}
    \leanok
    \lean{frequency_ball_cover}
    For each $I \in \mathcal{D}$, we have
    \begin{equation}
        \label{eq-tile-cover}
        \tQ(X) \subset \bigcup_{\mfa \in \tQ(X)} B_{I^\circ}(\mfa, 1) \subset \bigcup_{z \in \mathcal{Z}(I)} B_{I^\circ}(z, 0.7)\,.
    \end{equation}
\end{lemma}

\begin{proof}
% \leanok % the lemma is proven, but let's wait until \mathcal{Z} is defined to actually mark it.
    To show \eqref{eq-tile-cover} note that the first inclusion is obvious. For the second inclusion let $\mfb \in \bigcup_{\mfa \in \tQ(X)} B_{I^\circ}(\mfa, 1)$. By maximality of $\mathcal{Z}(I)$, there must be a point $z \in \mathcal{Z}(I)$ such that $B_{I^\circ}(z, 0.3) \cap B_{I^\circ}(\mfb, 0.3) \ne \emptyset$. Else, $\mathcal{Z}(I) \cup \{\mfb\}$ would be a set of larger cardinality than $\mathcal{Z}(I)$ satisfying \eqref{eq-tile-Z} and \eqref{eq-tile-disjoint-Z}. Fix such $z$, and fix a point $z_1 \in B_{I^\circ}(z, 0.3) \cap B_{I^\circ}(\mfb, 0.3)$. By the triangle inequality, we deduce that
    $$
        d_{I^\circ}(z,\mfb) \le d_{I^\circ}(z,z_1) + d_{I^\circ}(\mfb, z_1) < 0.3 + 0.3 = 0.6\,,
    $$
    and hence $\mfb \in B_{I^\circ}(z, 0.7)$.
\end{proof}

We define
$$
    \fP = \{(I, z) \ : \ I \in \mathcal{D}, z \in \mathcal{Z}(I)\}\,,
$$
$$\scI((I, z)) = I\qquad \text{and} \qquad \fcc((I, z)) = z.$$ We further set $$\ps(\fp) = s(\scI(\fp)),\qquad \qquad \pc(\fp) = c(\scI(\fp)).$$ Then \eqref{tilecenter}, \eqref{tilescale} hold by definition.

It remains to construct the map $\Omega$, and verify properties \eqref{eq-dis-freq-cover}, \eqref{eq-freq-dyadic} and
\eqref{eq-freq-comp-ball}. We first construct an auxiliary map $\Omega_1$. For each $I \in \mathcal{D}$, we pick an enumeration of the finite set $\mathcal{Z}(I)$
$$
    \mathcal{Z}(I) = \{z_1, \dotsc, z_M\}\,.
$$
We define {$\Omega_1:\fP \mapsto \mathcal{P}(\Mf) $ as below}. Set
$$
    \Omega_1((I, z_1)) = B_{I^\circ}(z_1, 0.7) \setminus \bigcup_{z \in \mathcal{Z}(I)\setminus \{z_1\}} B_{I^\circ}(z, 0.3)
$$
and then define iteratively
\begin{equation}
    \label{eq-def-omega1}
    \Omega_1((I, z_k)) = B_{I^\circ}(z_k, 0.7) \setminus \bigcup_{z \in \mathcal{Z}(I) \setminus \{z_k\}} B_{I^\circ}(z, 0.3) \setminus \bigcup_{i=1}^{k-1} \Omega_1((I, z_i))\,.
\end{equation}
\begin{lemma}[disjoint frequency cubes]
    \label{disjoint-frequency-cubes}
    \leanok
    \lean{Construction.disjoint_frequency_cubes}
    For each $I \in \mathcal{D}$, and $\fp_1, \fp_2\in \fP(I)$,
    if $$\Omega_1(\fp_1)\cap \Omega_1(\fp_2)\neq \emptyset,$$ then $\fp_1=\fp_2$.
\end{lemma}

\begin{proof}
    By the definition of the map $\scI$, we have
    $$
        \fP(I) = \{(I, z) \, : \, z \in \mathcal{Z}(I)\}\,.
    $$
    By \eqref{eq-def-omega1}, the set $\Omega_1((I, z_k))$ is disjoint from each $\Omega_1((I, z_i))$ with $i < k$. Thus the sets $\Omega_1(\fp)$, $\fp \in \fP(I)$ are pairwise disjoint.
\end{proof}

\begin{lemma}[frequency cube cover]
    \label{frequency-cube-cover}
    \leanok
    \lean{Construction.iUnion_ball_subset_iUnion_Ω₁, Construction.ball_subset_Ω₁, Construction.Ω₁_subset_ball}
    For each $I \in \mathcal{D}$, it holds that
    \begin{equation}
    \label{eq-omega1-cover}
            \bigcup_{z \in \mathcal{Z}(I)} B_{I^\circ}(z, 0.7)\subset \bigcup_{\fp \in \fP(I)} \Omega_1(\fp)\,.
    \end{equation}
    For every $\fp \in \fP$, it holds that
    \begin{equation}
        \label{eq-omega1-incl}
        B_{\fp}(\fcc(\fp), 0.3) \subset \Omega_1(\fp) \subset B_{\fp}(\fcc(\fp), 0.7)\,.
    \end{equation}
\end{lemma}

\begin{proof}
    For \eqref{eq-omega1-incl} let $\fp = (I, z)$.
    The second inclusion in \eqref{eq-omega1-incl} then follows from \eqref{eq-def-omega1} and the equality $B_{\fp}(\fcc(\fp), 0.7) = B_{I^\circ}(z, 0.7)$, which is true by definition.
    For the first inclusion in \eqref{eq-omega1-incl} let $\mfa \in B_{\fp}(\fcc(\fp),0.3)$. Let $k$ be such that $z = z_k$ in the enumeration we chose above. It follows immediately from \eqref{eq-def-omega1} and \eqref{eq-tile-disjoint-Z} that
    $\mfa \notin \Omega_1((I, z_i))$ for all $i < k$. Thus, again from \eqref{eq-def-omega1}, we have
    $\mfa \in \Omega_1((I,z_k))$.

    To show \eqref{eq-omega1-cover} let $\mfa \in \bigcup_{z \in \mathcal{Z}(I)} B_{I^\circ}(z,0.7)$.
    If there exists $z \in \mathcal{Z}(I)$ with $\mfa \in B_{I^\circ}(z,0.3)$, then
    $$
        z \in \Omega_1((I, z)) \subset \bigcup_{\fp \in \fP(I)} \Omega_1(\fp)
    $$
    by the first inclusion in \eqref{eq-omega1-incl}.

    Now suppose that there exists no $z \in \mathcal{Z}(I)$ with $\mfa \in B_{I^\circ}(z, 0.3)$. Let $k$ be minimal such that $\mfa \in B_{I^\circ}(z_k, 0.7)$. Since $\Omega_1((I, z_i)) \subset B_{I^\circ}(z_i, 0.7)$ for each $i$ by \eqref{eq-def-omega1}, we have that $\mfa \notin \Omega_1((I, z_i))$ for all $i < k$. Hence $\mfa \in \Omega_1((I, z_k))$, again by \eqref{eq-def-omega1}.
\end{proof}

Now we are ready to define the function $\Omega$. For all cubes $I \in \mathcal{D}$ such that there exists no $J \in \mathcal{D}$ with $I \subset J$ and $I \ne J$, we define for all $\fp \in \fP(I)$
\begin{equation}
    \label{eq-max-omega}
    \fc(\fp) = \Omega_1(\fp)\,.
\end{equation}
For cubes $I \in \mathcal{D}$ for which there exists $J \in \mathcal{D}$ with $I \subset J$ and $I \ne J$, we define $\Omega$ by recursion. We can pick an inclusion minimal $J \in \mathcal{D}$ among the finitely many cubes such that $I \subset J$ and $I \ne J$. This $J$ is unique: Suppose that $J'$ is another inclusion minimal cube with $I \subset J'$ and $I \ne J'$. Without loss of generality, we have that $s(J) \le s(J')$. By \eqref{dyadicproperty}, it follows that $J \subset J'$. Since $J'$ is minimal with respect to inclusion, it follows that $J = J'$. Then we define
\begin{equation}
    \label{eq-it-omega}
    \fc(\fp) = \bigcup_{z \in \mathcal{Z}(J) \cap \Omega_1(\fp)} \Omega((J, z)) \cup B_{\fp}(\fcc(\fp),0.2)\,.
\end{equation}

We now verify that $(\fP,\scI,\fc,\fcc,\pc,\ps)$ forms a tile structure.
\begin{proof}[Proof of \Cref{tile-structure}]
    \proves{tile-structure}
    First, we prove \eqref{eq-freq-comp-ball}. If $I \in \mathcal{D}$ is maximal in $\mathcal{D}$ with respect to set inclusion, then \eqref{eq-freq-comp-ball} holds for all $\fp \in \fP(I)$ by \eqref{eq-max-omega} and \eqref{eq-omega1-incl}. Now suppose that $I$ is not maximal in $\mathcal{D}$ with respect to set inclusion. Then we may assume by induction that for all $J \in \mathcal{D}$ with $I \subset J$ and all $\fp' \in \fP(J)$, \eqref{eq-freq-comp-ball} holds. Let $J$ be the unique minimal cube in $\mathcal{D}$ with $I \subsetneq J$.

    Suppose that $\mfa \in \Omega(\fp)$. If $\mfa\in B_{\fp}(\mathcal{Q}(\fp), 0.2)$, then since
    \begin{equation*}
        B_{\fp}(\mathcal{Q}(\fp), 0.2)\subset B_{\fp}(\mathcal{Q}(\fp), 1)\, ,
    \end{equation*}
    we conclude that $\mfa\in B_{\fp}(\mathcal{Q}(\fp), 0.7)$. If not, by \eqref{eq-it-omega}, there exists $z \in \mathcal{Z}(J) \cap \Omega_1(\fp)$ with $\mfa \in \Omega(J,z)$. Using the triangle inequality and \eqref{eq-omega1-incl}, we obtain
    $$
        d_{I^\circ}(\fcc(\fp),\mfa) \le d_{I^\circ}(\fcc(\fp), z) + d_{I^\circ}(z, \mfa) \le 0.7 + d_{I^\circ}(z, \mfa)\,.
    $$
    By \Cref{monotone-cube-metrics} and the induction hypothesis, this is estimated by
    $$
        \le 0.7 + 2^{-95a} d_{J^\circ}(z,\mfa) \le 0.7 + 2^{-95a}\cdot 1 < 1\,.
    $$
    This shows the second inclusion in \eqref{eq-freq-comp-ball}. The first inclusion is immediate from \eqref{eq-it-omega}.

    Next, we show \eqref{eq-dis-freq-cover}. Let $I \in \mathcal{D}$.

    If $I$ is maximal with respect to inclusion, then disjointedness of the sets $\fc(\fp)$ for $\fp \in \fP(I)$ follows from the definition \eqref{eq-max-omega} and \Cref{disjoint-frequency-cubes}. To obtain the inclusion in \eqref{eq-dis-freq-cover} one combines the inclusions \eqref{eq-tile-cover} and \eqref{eq-omega1-cover}
    of \Cref{frequency-cube-cover} with \eqref{eq-max-omega}.

    Now we turn to the case where there exists $J \in \mathcal{D}$ with $I \subset J$ and $I\ne J$. In this case we use induction: It suffices to show \eqref{eq-dis-freq-cover} under the assumption that it holds for all cubes $J \in \mathcal{D}$ with $I \subset J$. As shown before definition \eqref{eq-it-omega}, we may choose the unique inclusion minimal such $J$. To show disjointedness of the sets $\fc(\fp), \fp \in \fP(I)$ we pick two tiles $\fp, \fp' \in \fP(I)$ and $\mfa \in \fc(\fp) \cap \fc(\fp')$.
    Then we are by \eqref{eq-it-omega} in one of the following four cases.

    1. There exist $z \in \mathcal{Z}(J) \cap \Omega_1(\fp)$ such that $\mfa \in \Omega(J, z)$, and there exists $z' \in \mathcal{Z}(J) \cap \Omega_1(\fp')$ such that $\mfa \in \Omega(J, z')$. By the induction hypothesis, that \eqref{eq-dis-freq-cover} holds for $J$, we must have $z = z'$. By \Cref{disjoint-frequency-cubes}, we must then have $\fp = \fp'$.

    2. There exists $z \in \mathcal{Z}(J) \cap \Omega_1(\fp)$ such that $\mfa \in \Omega(J,z)$, and $\mfa \in B_{\fp'}(\fcc(\fp'), 0.2)$. Using the triangle inequality, \Cref{monotone-cube-metrics} and \eqref{eq-freq-comp-ball}, we obtain
    $$
        d_{\fp'}(\fcc(\fp'),z) \le d_{\fp'}(\fcc(\fp'), \mfa) + d_{\fp'}(z, \mfa) \le 0.2 + 2^{-95a} \cdot 1 < 0.3\,.
    $$
    Thus $z \in \Omega_1(\fp')$ by \eqref{eq-omega1-incl}. By \Cref{disjoint-frequency-cubes}, it follows that $\fp = \fp'$.

    3. There exists $z' \in \mathcal{Z}(J) \cap \Omega_1(\fp')$ such that $\mfa \in \Omega(J,z')$, and $\mfa \in B_{\fp}(\fcc(\fp), 0.2)$. This case is the same as case 2., after swapping $\fp$ and $\fp'$.

    4. We have $\mfa \in B_{\fp}(\fcc(\fp), 0.2) \cap B_{\fp'}(\fcc(\fp'), 0.2)$. In this case it follows that $\fp = \fp'$ since the sets $B_{\fp}(\fcc(\fp), 0.2)$ are pairwise disjoint by the inclusion \eqref{eq-omega1-incl} and \Cref{disjoint-frequency-cubes}.

    To show the inclusion in \eqref{eq-dis-freq-cover}, let $\mfa \in \tQ(X)$. By the induction hypothesis, there exists $\fp \in \fP(J)$ such that $\mfa \in \Omega(\fp)$. By definition of the set $\fP$, we have $\fp = (J, z)$ for some $z \in \mathcal{Z}(J)$. By \eqref{eq-tile-Z}, there exists $x \in X$ with $d_{J^\circ}(\tQ(x), z) \le 1$. By \Cref{monotone-cube-metrics}, it follows that $d_{I^\circ}(\tQ(x), z) \le 1$.
    Thus, by \eqref{eq-tile-cover}, there exists $z' \in \mathcal{Z}(I)$ with $z \in B_{I^\circ}(z', 0.7)$. Then by Lemma \eqref{frequency-cube-cover} there exists $\fp' \in \fP(I)$ with $z \in \mathcal{Z}(J) \cap \Omega_1(\fp')$. Consequently, by \eqref{eq-it-omega}, $\mfa \in \fc(\fp')$. This completes the proof of \eqref{eq-dis-freq-cover}.

    Finally, we show \eqref{eq-freq-dyadic}. Let $\fp, \fq \in \fP$ with $\scI(\fp) \subset \scI(\fp)$ and $\fc(\fp) \cap \fc(\fq) \ne \emptyset$. If we have $\ps(\fp) \ge \ps(\fq)$, then it follows from \eqref{dyadicproperty} that $I = J$, thus $\fp, \fq \in \fP(I)$. By \eqref{eq-dis-freq-cover} we have then either $\fc(\fp) \cap \fc(\fq) = \emptyset$ or $\fc(\fp) = \fc(\fq)$. By the assumption in \eqref{eq-freq-dyadic} we have $\fc(\fp) \cap \fc(\fq) \ne \emptyset$, so we must have $\fc(\fp) = \fc(\fq)$ and in particular $\fc(\fq) \subset \fc(\fp)$.

    So it remains to show \eqref{eq-freq-dyadic} under the additional assumption that $\ps(\fq) > \ps(\fp)$. In this case, we argue by induction on $\ps(\fq)-\ps(\fp)$. By \eqref{coverdyadic}, there exists a cube $J \in \mathcal{D}$ with $s(J) = \ps(\fq) - 1$ and $J \cap\scI(\fp) \ne \emptyset$. We pick one such $J$. By \eqref{dyadicproperty}, we have $\scI(\fp) \subset J \subset \scI(\fq)$.

    By \eqref{eq-tile-Z}, there exists $x \in X$ with $d_{\fq}(\tQ(x), \fcc(\fq)) \le 1$. By \Cref{monotone-cube-metrics}, it follows that $d_{J^\circ}(\tQ(x), \fcc(\fq)) \le 1$.
    Thus, by \eqref{eq-tile-cover}, there exists $z' \in \mathcal{Z}(J)$ with $\fcc(\fq) \in B_{J^\circ}(z', 0.7)$. Then by \Cref{frequency-cube-cover} there exists $\fq' \in \fP(J)$ with $\fcc(\fq) \in\Omega_1(\fq')$.
    By \eqref{eq-it-omega}, it follows that $\Omega(\fq) \subset \Omega(\fq')$. Note that then $\scI(\fp) \subset \scI(\fq')$ and $\fc(\fp) \cap \fc(\fq') \ne \emptyset$ and $\ps(\fq') - \ps(\fp) = \ps(\fq) - \ps(\fp) - 1$. Thus, we have by the induction hypothesis that $\Omega(\fq') \subset \Omega(\fp)$. This completes the proof.
\end{proof}

\chapter{Proof of discrete Carleson}
\label{proptopropprop}



Let a grid structure $(\mathcal{D}, c, s)$ and a tile structure $(\fP, \scI, \fc, \fcc)$ for this grid structure be given. In \Cref{subsectilesorg}, we decompose the set $\fP$ of tiles into subsets. Each subset will be controlled by one of three methods. The guiding principle of the decomposition is to be able to apply the forest estimate of \Cref{forest-operator} to the final subsets defined in \eqref{defc5}. This application is done in \Cref{subsecforest}. The miscellaneous subsets along the construction of the forests will either be thrown into exceptional sets, which are defined and controlled in \Cref{subsetexcset}, or will be controlled by the antichain estimate of \Cref{antichain-operator}, which is done in \Cref{subsecantichain}. \Cref{subsec-lessim-aux} contains some auxiliary lemmas needed for the proofs in Subsections \ref{subsecforest}-\ref{subsecantichain}.

\section{Organisation of the tiles}\label{subsectilesorg}

In the following definitions, $k, n$, and
$j$ will be nonnegative integers.
Define
$\mathcal{C}(G,k)$ to be the set of $I\in \mathcal{D}$
such that there exists a $J\in \mathcal{D}$ with $I\subset J$
and
\begin{equation}\label{muhj1}
   {\mu(G \cap J)} > 2^{-k-1}{\mu(J)}\, ,
\end{equation}
but there does not exist a $J\in \mathcal{D}$ with $I\subset J$ and
\begin{equation}\label{muhj2}
   {\mu(G \cap J)} > 2^{-k}{\mu(J)}\,.
\end{equation}
Let
\begin{equation}
    \label{eq-defPk}
    \fP(k)=\{\fp\in \fP \ : \ \scI(\fp)\in \mathcal{C}(G,k)\}
\end{equation}
Define $ {\mathfrak{M}}(k,n)$ to be the set of $\fp \in \fP(k)$ such that
 \begin{equation}\label{ebardense}
    \mu({E_1}(\fp)) > 2^{-n} \mu(\scI(\fp))
 \end{equation}
and there does not exist $\fp'\in \fP(k)$ with
$\fp'\neq \fp$ and $\fp\le \fp'$ such that
 \begin{equation}\label{mnkmax}
    \mu({E_1}(\fp')) > 2^{-n} \mu(\scI(\fp')).
 \end{equation}
Define for a collection $\fP'\subset \fP(k)$
\begin{equation}
    \label{eq-densdef}
   \dens_k' (\fP'):= \sup_{\fp'\in \fP'}\sup_{\lambda \geq 2} \lambda^{-a} \sup_{\fp \in \fP(k): \lambda \fp' \lesssim \lambda \fp}
    \frac{\mu({E}_2(\lambda, \fp))}{\mu(\scI(\fp))}\,.
\end{equation}
Sorting by density, we define
\begin{equation}
    \label{def-cnk}
    \fC(k,n):=\{\fp\in \fP(k) \ : \
    2^{4a}2^{-n}< \dens_k'(\{\fp\}) \le
    2^{4a}2^{-n+1}\}\,.
\end{equation}
Following Fefferman \cite{fefferman}, we
define for $\fp \in \fC(k,n)$
\begin{equation}\label{defbfp}
     \mathfrak{B}(\fp) := \{ \mathfrak{m} \in \mathfrak{M}(k,n) \ : \ 100 \fp \lesssim \mathfrak{m}\}
\end{equation}
and
\begin{equation}\label{defcnkj}
       \fC_1(k,n,j) := \{\fp \in \fC(k,n) \ : \ 2^{j} \leq |\mathfrak{B}(\fp)| < 2^{j+1}\}\,.
\end{equation}
and
\begin{equation}\label{defl0nk}
       \fL_0(k,n) := \{\fp \in \fC(k,n) \ : \ |\mathfrak{B}(\fp)| <1\}\,.
\end{equation}
Together with the following removal of minimal layers, the splitting into $\fC_1(k,n,j)$ will lead to a separation of trees.
Define recursively for $0\le l\le Z(n+1)$
\begin{equation}
    \label{eq-L1-def}
    \fL_1(k,n,j,l)
\end{equation}
to be the set of minimal elements with respect to $\le$ in
\begin{equation}
    \fC_1(k,n,j)\setminus \bigcup_{0\le l'<l}
\fL_1(k,n,j,l')\, .
\end{equation}
Define
\begin{equation}
    \label{eq-C2-def}
    \fC_2(k,n,j):= \fC_1(k,n,j)\setminus \bigcup_{0\le l'\le Z(n+1)}
\fL_1(k,n,j,l')\, .
\end{equation}

The remaining tile organization will be relative to
prospective tree tops, which we define now.
Define
\begin{equation}\label{defunkj}
     \fU_1(k,n,j)
\end{equation}
to be the set of all
$\fu \in \fC_1(k,n,j)$ such that
for all $\fp \in \fC_1(k,n,j)$
with $\scI(\fu)$ strictly contained in
$\scI(\fp)$ we have $B_{\fu}(\fcc(\fu), 100) \cap B_{\fp}(\fcc(\fp),100) = \emptyset$.

We first remove the pairs that are outside the immediate reach of any of the prospective tree tops.
Define
\begin{equation}
\label{eq-L2-def}
\fL_2(k,n,j)
\end{equation}
to be the set of all $\fp\in \fC_2(k,n,j)$ such that there
does not exist
$\fu\in \fU_1(k,n,j)$
with $\scI(\fp)\neq \scI(\fu)$ and $2\fp\lesssim \fu$.
Define
\begin{equation}
\label{eq-C3-def}
\fC_3(k,n,j):=\fC_2(k,n,j)
  \setminus \fL_2(k,n,j)\, .
\end{equation}


We next remove the maximal layers.
Define recursively for $0 \le l \le Z(n+1)$
\begin{equation}
    \label{eq-L3-def}
    \fL_3(k,n,j,l)
\end{equation}
to be the set of all maximal elements with respect to $\le$ in
\begin{equation}
    \fC_3(k,n,j) \setminus \bigcup_{0 \le l' < l} \fL_3(k,n,j,l')\,.
\end{equation}
Define
\begin{equation}
\label{eq-C4-def}
\fC_4(k,n,j):=\fC_3(k,n,j)
  \setminus \bigcup_{0 \le l \le Z(n+1)} \fL_3(k,n,j,l)\,.
\end{equation}

Finally, we remove the boundary pairs relative to the prospective tree tops. Define
\begin{equation}
    \label{eq-L-def}
    \mathcal{L}(\fu)
\end{equation}
to be the set of all $I \in \mathcal{D}$ with $I \subset \scI(\fu)$ and $s(I) = \ps(\fu) - Z(n+1) - 1$ and
\begin{equation}
    B(c(I), 8 D^{s(I)})\not \subset \scI(\fu)\, .
\end{equation}
Define
\begin{equation}
    \label{eq-L4-def}
    \fL_4(k,n,j)
\end{equation}
to be the set of all $\fp\in \fC_4(k,n,j)$ such that there exists
$\fu\in \fU_1(k,n,j)$
with $\scI(\fp) \subset \bigcup \mathcal{L}(\fu)$, and define
\begin{equation}\label{defc5}
\fC_5(k,n,j):=\fC_4(k,n,j)
  \setminus \fL_4(k,n,j)\, .
\end{equation}


We define three exceptional sets.
The first exceptional set $G_1$ takes into account the ratio of the measures of $F$ and $G$.
Define $\fP_{F,G}$ to be the set of all $\fp\in \fP$
with
\begin{equation}
    \dens_2(\{\fp\})\ge 2^{2a+5}\frac{\mu(F)}{\mu(G)}\,.
\end{equation}
Define
\begin{equation}\label{definegone}
    G_1:=\bigcup_{\fp\in \fP_{F,G} }\scI(\fp)\, .
\end{equation}
For an integer $\lambda\ge 0$, define $A(\lambda,k,n)$ to be the set
of all $x\in X$ such that
\begin{equation}
    \label{eq-Aoverlap-def}
    % remark(Georges Gonthier): removed 1+
    \sum_{\fp \in \mathfrak{M}(k,n)}\mathbf{1}_{\scI(\fp)}(x)>\lambda 2^{n+1}
\end{equation}
and define
\begin{equation}\label{definegone2}
    G_2:=
\bigcup_{k\ge 0}\bigcup_{k< n}
A(2n+6,k,n)\, .
\end{equation}
Define
    \begin{equation}\label{defineg3}
        G_3 :=
        \bigcup_{k\ge 0}\, \bigcup_{n \geq k}\,
        \bigcup_{0\le j\le 2n+3}
        \bigcup_{\fp \in \fL_4 (k,n,j)}
        \scI(\fp)\, .
     \end{equation}
Define $G'=G_1\cup G_2 \cup G_3$
The following bound of the measure of $G'$ will be proven in
\Cref{subsetexcset}.
\begin{lemma}[exceptional set]
    \label{exceptional-set}
    \uses{first-exception,second-exception, third-exception}
    We have
    \begin{equation}
        \mu(G')\le 2^{-2}\mu(G)\, .
    \end{equation}
\end{lemma}

In \Cref{subsecforest}, we identify each set $\fC_5(k,n,j)$ outside $G'$ as forest and use Proposition
\ref{forest-operator} to prove the following lemma.

\begin{lemma}[forest union]
    \label{forest-union}
    \uses{forest-operator,C-dens1,C6-forest, forest-geometry, forest-convex,forest-separation, forest-inner,forest-stacking}
    Let
    \begin{equation}
        \fP_1 =\bigcup_{k\ge 0}\bigcup_{n\ge k}
        \bigcup_{0\le j\le 2n+3}\fC_5(k,n,j)
    \end{equation}
    For all $f:X\to \C$ with $|f|\le \mathbf{1}_F$ we have
    \begin{equation}
        \label{disclesssim1}
        \int_{G \setminus G'} \left|\sum_{\fp \in \fP_1} T_{\fp} f \right|\, \mathrm{d}\mu \le \frac{2^{435a^3}}{(q-1)^3} \mu(G)^{1 - \frac{1}{q}} \mu(F)^{\frac{1}{q}}\,.
        % fixme(Georges Gonthier): denominator should be changed to (q-1)^4.
    \end{equation}
\end{lemma}

In \Cref{subsecantichain}, we decompose
the complement of the set of tiles in Lemma
\ref{forest-union} and apply the antichain estimate of
\Cref{antichain-operator} to prove the following lemma.

\begin{lemma}[forest complement]
    \label{forest-complement}
    \uses{antichain-operator,antichain-decomposition,L0-antichain,L2-antichain,L1-L3-antichain,C-dens1}
    Let
    \begin{equation}
        \fP_2 =\fP\setminus \fP_1\,.
    \end{equation}
    For all $f:X\to \C$ with $|f|\le \mathbf{1}_F$ we have
    \begin{equation}
        \label{disclesssim2}
        \int_{G \setminus G'} \left|\sum_{\fp \in \fP_2} T_{\fp} f\right| \, \mathrm{d}\mu \le \frac{2^{210a^3}}{(q-1)^4} \mu(G)^{1 - \frac{1}{q}} \mu(F)^{\frac{1}{q}}\,.
    \end{equation}
    % fixme: (q-1)^4 -> (q-1)^5
\end{lemma}
\Cref{discrete-Carleson} follows by applying
triangle inequality to \eqref{disclesssim}
according to the splitting in \Cref{forest-union}
and \Cref{forest-complement} and using both Lemmas as well
as the bound on the set $G'$ given by \Cref{exceptional-set}.


\section{Proof of the Exceptional Sets Lemma}
\label{subsetexcset}


We prove separate bounds for $G_1$, $G_2$, and $G_3$
in Lemmas \ref{first-exception},
\ref{second-exception}, and \ref{third-exception}. Adding up these bounds proves \Cref{exceptional-set}.

The bound for $G_1$ is follows from the Vitali covering lemma, \Cref{Hardy-Littlewood}.

\begin{lemma}[first exception]
    \label{first-exception}
    \uses{Hardy-Littlewood}
    We have
    \begin{equation}
        \mu(G_1)\le 2^{-4}\mu(G)\, .
    \end{equation}
\end{lemma}
\begin{proof}
    Let
    $$
        K = 2^{2a+5}\frac{\mu(F)}{\mu(G)}\,.
    $$
    For each $\fp\in \fP_{F,G}$ pick a
    $r(\fp)>4D^{\ps(\fp)}$ with
    $$
    {\mu(F\cap B(\pc(\fp),r(\fp)))}\ge K{\mu(B(\pc(\fp),r(\fp)))}\, .
    $$
    This ball exists by definition of $\fP_{F,G}$
    and $\dens_2$. By applying \Cref{Hardy-Littlewood}to the collection of balls
    $$
        \mathcal{B} = \{B(\pc(\fp),r(\fp)) \ : \ \fp \in \fP_{F,G}\}
    $$
    and the function $u = \mathbf{1}_F$, we obtain
    $$
        \mu(\bigcup \mathcal{B}) \le 2^{2a+1} K^{-1} \mu(F)\,.
    $$
    We conclude with \eqref{eq-vol-sp-cube} and $r(\fp)>4D^{\ps(\fp)}$
    $$
        \mu(G_1)= \mu(\bigcup_{\fp\in \fP_{F,G}} \scI(\fp))
        \le \mu(\bigcup \mathcal{B})\le 2^{2a+1} K^{-1} \mu (F) = 2^{-4}\mu(G)\,.
    $$
\end{proof}


We turn to the bound of $G_2$, which relies on the Dyadic Covering \Cref{dense-cover} and the
John-Nirenberg \Cref{John-Nirenberg} below.

\begin{lemma}[dense cover]
\label{dense-cover}

For each $k\ge 0$, the union of all dyadic cubes
in $\mathcal{C}(G,k)$ has measure at most $2^{k+1} \mu(G)$ .
\end{lemma}
\begin{proof}
The union of dyadic cubes in $\mathcal{C}(G,k)$
is contained the union of elements of the set $\mathcal{M}(k)$
of all dyadic cubes $J$ with
${\mu(G \cap J)} > 2^{-k-1}{\mu(J)}$.
The union of elements in the set $\mathcal{M}(k)$ is contained in the union of elements in
the set $\mathcal{M}^*(k)$ of maximal elements in
$\mathcal{M}(k)$ with respect to set inclusion. Hence
\begin{equation}\label{cbymstar}
\mu (\bigcup \mathcal{C}(G,k))\le \mu (\bigcup \mathcal{M}^*(k))\le
\sum_{J\in \mathcal{M}^*(k)}\mu(J)
\end{equation}
Using the definition of $\mathcal{M}(k)$ and then
the pairwise disjointedness of elements in
$\mathcal{M}^*(k)$,
we estimate \eqref{cbymstar} by
\begin{equation}
\le
2^{k+1}\sum_{J\in \mathcal{M}^*(k)}\mu(J\cap G)
\le 2^{k+1}\mu(G).
\end{equation}
This proves the lemma.
\end{proof}

\begin{lemma}[pairwise disjoint]
    \label{pairwise-disjoint}
    If $\fp, \fp' \in {\mathfrak{M}}(k,n)$ and
    \begin{equation}\label{eintersect}
        {E_1}(\fp)\cap {E_1}(\fp')\neq \emptyset,
    \end{equation}
    then $\fp=\fp'$.
\end{lemma}
\begin{proof}
Let $\fp,\fp'$ be as in the lemma. By definition of $E_1$,
we have
$E_1(\fp)\subset \scI(\fp)$ and analogously for $\fp'$, we conclude from \eqref{eintersect} that $\scI(\fp)\cap \scI(\fp')\neq \emptyset$. Let without loss of generality $\scI(\fp)$ be maximal in
$\{\scI(\fp),\scI(\fp')\}$, then $\scI(\fp')\subset \scI(\fp)$.
By \eqref{eintersect}, we conclude by definition of $E_1$ that $\fc(\fp)\cap \fc(\fp')\neq \emptyset$. By
\eqref{eq-freq-dyadic} we conclude $\fc(\fp)\subset \fc(\fp')$. It follows that $\fp'\le \fp$. By maximality
\eqref{mnkmax}
of $\fp'$, we have $\fp'=\fp$. This proves the lemma.
\end{proof}


\begin{lemma}[dyadic union]
\label{dyadic-union}

For each $x\in A(\lambda,k,n)$,
there is a dyadic cube $I$
that contains $x$ and is
a subset of
$A(\lambda,k,n)$.
\end{lemma}

\begin{proof}
Fix $k,n,\lambda,x$ as in the lemma such that $x\in A(\lambda,k,n)$.
Let $\mathcal{M}$ be the set of dyadic cubes $\scI(\fp)$ with $\fp$ in $\mathfrak{M}(k,n)$ and $x\in \scI(\fp)$.
By definition of $A(\lambda,k,n)$, the cardinality of $\mathcal{M}$ is at least $1+\lambda 2^{n+1}$.
Let $I$ be a cube of smallest scale in $\mathcal{M}$.
Then $I$ is contained in all cubes of $\mathcal{M}$.
It follows that $I\subset A(\lambda,k,n)$.
\end{proof}

\begin{lemma}[John Nirenberg]
\label{John-Nirenberg}
\uses{dense-cover, pairwise-disjoint,dyadic-union}
    For all integers $k,n,\lambda\ge 0$, we have
    \begin{equation}\label{alambdameasure}
        \mu(A(\lambda,k,n)) \le 2^{k+1-\lambda}\mu(G)\, .
    \end{equation}

\end{lemma}
\begin{proof}
Fix $k,n$ as in the lemma
and suppress notation to write
$A(\lambda)$ for $A(\lambda,k,n)$.
We prove the lemma by induction on $\lambda$.
For $\lambda=0$, we use that $A(\lambda)$ by definition of $\mathfrak{M}(k,n)$ is contained in the union of elements in $ \mathcal{C}(G,k)$. \Cref{dense-cover} then completes the base of the induction.

Now assume that the statement of \Cref{John-Nirenberg}
is proven for some integer $\lambda\ge 0$.
The set $A(\lambda+1)$ is contained in the set $A(\lambda)$.
Let $\mathcal{M}$ be the set of dyadic cubes which are a subset of $A(\lambda)$. By \Cref{dyadic-union}, the union of $\mathcal{M}$ is $A(\lambda)$.
Let $\mathcal{M}^*$ be the set of maximal dyadic cubes in $\mathcal{M}$.

Let $L\in \mathcal{M}^*$. For each $x\in L$, we have
\begin{equation}\label{suminout}
 \sum_{\fp \in {\mathfrak{M}}(k,n)} \mathbf{1}_{\scI(\fp)}(x)=
   \sum_{\fp \in {\mathfrak{M}}(k,n):\scI(\fp) \subset L} \mathbf{1}_{\scI(\fp)}(x)+
  \sum_{\fp \in {\mathfrak{M}}(k,n):\scI(\fp) \not \subset L} \mathbf{1}_{\scI(\fp)}(x)\, .
\end{equation}
If the second sum on the right-hand-side is not zero, there is
an element of $\mathcal{D}$ strictly containing $L$.
Let $\hat{L}$ be such a dyadic cube with minimal $s(L)$. Then $\hat{L}$ is contained in $\scI(\fp)$ for all $\fp$
contributing to the second sum in
\eqref{suminout}.
Hence the second sum in \eqref{suminout} is constant on
$\hat{L}$.
By maximality of $L$, the second sum is less than $1+\lambda 2^{n+1}$ somewhere on $\hat{L}$, thus on all of $\hat{L}$ and consequently also
at $x$.
If $x$ is in addition in $A(\lambda+1)$, then
the left-hand-side of \eqref{suminout} is at least
$1+(\lambda+1) 2^{n+1}$, so we have by the triangle inequality for the first sum on the right-hand side
\begin{equation}\label{mnkonl}
\sum_{\fp \in {\mathfrak{M}}(k,n):\scI(\fp) \subset L} \mathbf{1}_{\scI(\fp)}(x)\ge 2^{n+1}\, .\end{equation}
By \Cref{pairwise-disjoint}, we have
\begin{equation}
\sum_{\fp \in {\mathfrak{M}}(k,n):\scI(\fp) \subset L} \mu({E_1}(\fp)) \leq \mu(L)\, .
\end{equation}
Multiplying by $2^n$ and applying \eqref{ebardense}, we obtain
\begin{equation}\label{mnkintl}
    \sum_{\fp \in {\mathfrak{M}}(k,n):\scI(\fp) \subset L} \mu(\scI(\fp)) \leq 2^n \mu(L)\, .
\end{equation}
We then have with \eqref{mnkonl} and \eqref{mnkintl}
\begin{equation}
2^{n+1}\mu(A(\lambda+1)\cap L) =
 \int_{A(\lambda+1)\cap L} 2^{n+1} d\mu
\end{equation}
\begin{equation}
\le
    \int \sum_{\fp \in {\mathfrak{M}}(k,n):\scI(\fp) \subset L} \mathbf{1}_{\scI(\fp)} d\mu
\le 2^n \mu(L)\, .
\end{equation}
Hence
\begin{equation}
    2\mu(A(\lambda+1))=2\sum_{L\in \mathcal{M}^*}
\mu(A(\lambda+1)\cap L)\le
\sum_{L\in \mathcal{M}^*}\mu( L)= \mu(A(\lambda))\, .
\end{equation}
Using the induction hypothesis, this proves
\eqref{alambdameasure} for $\lambda+1$ and completes the proof of the lemma.
\end{proof}

\begin{lemma}[second exception]
\label{second-exception}
\uses{John-Nirenberg}
We have
\begin{equation}
    \mu(G_2)\le 2^{-4} \mu(G)\, .
\end{equation}
\end{lemma}
\begin{proof}

We use \Cref{John-Nirenberg} and sum twice a geometric series
to obtain
\begin{equation}
    \sum_{0\le k}\sum_{k< n}
\mu(A(2n+6,k,n))\le \sum_{0\le k}\sum_{k< n} 2^{k-5-2n}\mu(G)
\end{equation}
\begin{equation}
   \le \sum_{0\le k} 2^{-k-5}\mu(G)\le 2^{-4}\mu(G)\, .
\end{equation}
This proves the lemma.
\end{proof}


We turn to the set $G_3$.

\begin{lemma}[top tiles]
\label{top-tiles}
 \uses{John-Nirenberg}
    We have
    \begin{equation}\label{eq-musum}
        % Note(Georges Gonthier): suggested new factor
        \sum_{\mathfrak{m} \in \mathfrak{M}(k,n)} \mu(\scI(\mathfrak{m}))\le 2^{n+k+3}\mu(G).
    \end{equation}
\end{lemma}
\begin{proof}
% Note(Georges Gonthier): sum should start at 0, not 1
We write the left-hand side of \eqref{eq-musum}
\begin{equation}
    \int \sum_{\mathfrak{m} \in \mathfrak{M}(k,n)} \mathbf{0}_{\scI(\mathfrak{m})}(x) \, d\mu(x) \le
2^{n+1} \sum_{\lambda=0}^{|\mathfrak{M}|}\mu(A(\lambda, k,n))\,.
\end{equation}
Using \Cref{John-Nirenberg}
and then summing a geometric series, we estimate this by
\begin{equation}
    \le
2^{n+1}\sum_{\lambda=0}^{|\mathfrak{M}|}
2^{k+1-\lambda}\mu(G)
\le
2^{n+1}2^{k+2}\mu(G)\, .
\end{equation}
This proves the lemma.
\end{proof}


\begin{lemma}[tree count]
\label{tree-count}

Let $k,n,j\ge 0$. We have for every $x\in X$
\begin{equation}
    \sum_{\fu\in \fU_1(k,n,j)} \mathbf{1}_{\scI(\fu)}(x)
    \le 2^{-j}
    2^{9a} \sum_{\mathfrak{m}\in \mathfrak{M}(k,n)}
     \mathbf{1}_{\scI(\mathfrak{m})}(x)
\end{equation}
\end{lemma}

\begin{proof}
Let $x\in X$. For each
$\fu\in \fU_1(k,n,j)$ with $x\in \scI(\fu)$, as $\fu \in \fC_1(k,n,j)$,
there are at least $2^{j}$ elements $\mathfrak{m}\in \mathfrak{M}(k,n)$
with $100\fu \lesssim \mathfrak{m}$ and in particular
$x\in \scI(\mathfrak{m})$. Hence
\begin{equation}\label{ubymsum}
     \mathbf{1}_{\scI(\fu)}(x)
    \le 2^{-j}\sum_{\mathfrak{m} \in \mathfrak{M}(k,n): 100\fu\lesssim \mathfrak{m}} \mathbf{1}_{\scI(\mathfrak{m})}(x)\, .
\end{equation}
Conversely, for each $\mathfrak{m}\in \mathfrak{M}(k,n)$
with $x\in \scI(\mathfrak{m})$,
let $\fU(\mathfrak{m})$ be the set of
$\fu\in \fU_1(k,n,j)$ with $x\in \scI(\fu)$
and $100\fu \lesssim \mathfrak{m}$.
Summing \eqref{ubymsum} over $\fu$ and counting the pairs
$(\fu,\mathfrak{m})$ with $100\fu \lesssim \mathfrak{m}$
differently gives
\begin{equation}\label{usumbymsum}
     \sum_{\fu\in \fU_1(k,n,j)} \mathbf{1}_{\scI(\fu)}(x)
    \le 2^{-j}\sum_{\mathfrak{m} \in \mathfrak{M}(k,n)}
    \sum_{\fu \in \fU(\mathfrak{m})} \mathbf{1}_{\scI(\mathfrak{m})}(x)\, .
\end{equation}
We estimate the number of elements in $\fU(\mathfrak{m})$.
Let $\fu \in \fU(\mathfrak{m})$.
Then by definition of
$\fU(\mathfrak{m})$
\begin{equation}\label{dby2}
     d_{\fu}(\fcc(\fu),\fcc(\mathfrak{m}))\le 100\, .
\end{equation}
If $\fu'$ is a further element in $\fU(\mathfrak{m})$ with $\fu\neq \fu'$, then
\begin{equation}
    \fcc(\mathfrak{m})
    \in B_{\fu}(\fcc(\fu),100)\cap B_{\fu'}(\fcc(\fu'),100)\ .
\end{equation}
By the last display and definition of $\fU_1(k,n,j)$, none of $\scI(\fu)$, $\scI(\fu')$ is strictly contained in the other. As both contain $x$, we have $\scI(\fu)=\scI(\fu')$.
We then have $d_{\fu}=d_{\fu'}$.

By \eqref{eq-freq-comp-ball}, the balls
$B_{\fu}(\fcc(\fu),0.2)$ and
$B_{\fu}(\fcc(\fu'),0.2)$ are
contained respectively in $\fc(\fu)$
and $\fc(\fu')$ and thus are disjoint by \eqref{eq-dis-freq-cover}.
By \eqref{dby2} and the triangle inequality, both balls are contained in $B_{\fu}(\fcc(\mathfrak{m}), 100.2)$.

By \eqref{thirddb} applied nine times, there is a collection of at most
$2^{9a}$ balls of radius $0.2$ with respect to the metric $d_{\fu}$ which cover the ball $B_{\fu}(\fcc(\mathfrak{m}),100.2)$.
Let $B'$ be a ball in this cover.
As the center of $B'$ can be in at most one of the disjoint balls
$B_{\fu}(\fcc(\fu),0.2)$ and
$B_{\fu}(\fcc(\fu'),0.2)$,
the ball $B'$ can contain at most
one of the points $\fu$, $\fu'$.

Hence the set $\fU(\mathfrak{m})$ has at most
$2^{9a}$ many elements.
Inserting this into \eqref{usumbymsum} proves the lemma.
\end{proof}

\begin{lemma}[boundary exception]
\label{boundary-exception}

Let $\mathcal{L}(\fu)$ be as defined in \eqref{eq-L-def}. We have for each $\fu\in \fU_1(k,n,l)$,
\begin{equation}
\mu(\bigcup_{I\in \mathcal{L}(\fu)} I)
\le D^{1-\kappa Z(n+1)}
        \mu(\scI(\mathfrak{u})).
\end{equation}

\end{lemma}


\begin{proof}
  Let $\fu\in \fU_1(k,n,l)$.
Let $I \in \mathcal{L}(\fu)$. Then we have $s(I) = \ps(\fu) - Z(n+1) - 1$ and $I \subset \scI(\fu)$ and $B(c(I), 8D^{s(I)}) \not \subset \scI(\fu)$.
By \eqref{eq-vol-sp-cube}, the set $I$
is contained in $B(c(I), 4D^{s(I)})$.
By the triangle inequality, the set $I$
is contained in
\begin{equation}
    X(\fu):=\{x \in \scI(\fu) \, : \, \rho(x, X \setminus \scI(\fu)) \leq 12 D^{\ps(\fu) - Z(n+1)-1}\}\,.
\end{equation}
 By the small boundary property \eqref{eq-small-boundary}, noting that
 \begin{equation*}
     12D^{\ps(\fu) - Z(n+1) - 1} = 12D^{s(I)} > D^{-S}\ ,
 \end{equation*} we have
   $$
        \mu(X(\fu)) \le
        D\cdot(12 D^{-Z(n+1)-1})^\kappa
        \mu(\scI(\mathfrak{u})).
    $$
Using $\kappa<1$ and $D \ge 12$, this proves the lemma.
\end{proof}














    \begin{lemma}[third exception]
    \label{third-exception}
\uses{tree-count,boundary-exception, top-tiles}
       We have
\begin{equation}
    \mu(G_3)\le 2^{-4} \mu(G)\, .
\end{equation}
    \end{lemma}



    \begin{proof}
As each $\fp\in \fL_4(k,n,j)$
is contained in $\cup\mathcal{L}(\fu)$ for some
$\fu\in \fU_1(k,n,l)$, we have
\begin{equation}
\mu(\bigcup_{\fp \in \fL_4 (k,n,j)}\scI(\fp))
\le \sum_{\fu\in \fU_1(k,n,j)}
\mu(\bigcup_{I \in \mathcal{L} (\fu)}I).
\end{equation}
Using \Cref{boundary-exception} and then \Cref{tree-count}, we estimate this further
 by
\begin{equation}
    \le \sum_{\fu\in \fU_1(k,n,j)}
    D^{1-\kappa Z(n+1)}
        \mu(\scI(\mathfrak{u}))
\end{equation}
\begin{equation}
    \le 2^{100a^2+9a+1-j} \sum_{\mathfrak{m}\in \mathfrak{M}(k,n)}
     D^{-\kappa Z(n+1)}
    \mu(\scI(\mathfrak{m}))\,.
\end{equation}
Using \Cref{top-tiles}, we estimate this by
  \begin{equation}
     \le
2^{100a^2 + 9a + 1-j} D^{-\kappa Z(n+1)}
     2^{n+k+3}\mu(G)\, .
\end{equation}
Now we estimate $G_3$ defined in \eqref{defineg3} by
\begin{equation}
    \mu(G_3)\le \sum_{k\ge 0}\, \sum_{n \geq k}\,
    \sum_{0\le j\le 2n+3}
    \mu(\bigcup_{\fp \in \fL_4 (k,n,j)}
    \scI(\fp))
\end{equation}
\begin{equation}
    \le \sum_{k\ge 0}\, \sum_{n \geq k}\,
    \sum_{0\le j\le 2n+3}
    2^{100a^2 + 9a + 4 + n + k -j} D^{-\kappa Z(n+1)}\mu(G)
\end{equation}
Summing geometric series, using that $D^{\kappa Z}\ge 8$ by \eqref{defineD}, \eqref{definekappa} and \eqref{defineZ}, we estimate this by
\begin{equation}
    \le \sum_{k\ge 0}\, \sum_{n \geq k}\,
    2^{100a^2 + 9a + 5 + n + k} D^{-\kappa Z(n+1)}\mu(G)
\end{equation}
\begin{equation}
    = \sum_{k\ge 0} 2^{100a^2 + 9a + 5 + 2k} D^{-\kappa Z(k+1)} \sum_{n \geq k}\,
    2^{n - k} D^{-\kappa Z(n-k)}\mu(G)
\end{equation}
\begin{equation}
    \le \sum_{k\ge 0} 2^{100a^2 + 9a + 6 + 2k} D^{-\kappa Z(k+1)}\mu(G)
\end{equation}
\begin{equation}
   \le 2^{100a^2 + 9a + 7} D^{-\kappa Z}\mu(G)
\end{equation}
Using $D = 2^{100a^2}$ and $a \ge 4$ and $\kappa Z \ge 2$ by \eqref{defineD} and \eqref{definekappa} proves the lemma.
\end{proof}

\section{Auxiliary lemmas}
\label{subsec-lessim-aux}
Before proving \Cref{forest-union} and \Cref{forest-complement}, we collect some useful properties of $\lesssim$.

\begin{lemma}[wiggle order 1]
    \label{wiggle-order-1}

    If $n\fp \lesssim m\fp'$ and
    $n' \ge n$ and $m \ge m'$ then $n'\fp \lesssim m'\fp'$.
\end{lemma}

\begin{proof}
    This follows immediately from the definition \eqref{wiggleorder} of $\lesssim$ and the two inclusions $B_{\fp}(\fcc(\fp), n) \subset B_{\fp}(\fcc(\fp), n')$ and $B_{\fp'}(\fcc(\fp'), m') \subset B_{\fp'}(\fcc(\fp'), m)$.
\end{proof}

\begin{lemma}[wiggle order 2]
    \label{wiggle-order-2}
    \uses{monotone-cube-metrics}

    Let $n, m \ge 1$.
    If $\fp, \fp' \in \fP$ with $\scI(\fp) \ne \scI(\fp')$ and
    \begin{equation}
        \label{eq-wiggle1}
        n \fp \lesssim \fp'
    \end{equation}
    then
    \begin{equation}
        \label{eq-wiggle2}
        (n + 2^{-95 a} m) \fp \lesssim m\fp'\,.
    \end{equation}
\end{lemma}

\begin{proof}
    The assumption \eqref{eq-wiggle1} together with the definition \eqref{wiggleorder} of $\lesssim$ implies that $\scI(\fp) \subsetneq \scI(\fp')$. Let $\mfa \in B_{\fp'}(\fcc(\fp'), m)$. Then we have by the triangle inequality
    $$
        d_{\fp}(\fcc(\fp), \mfa) \le d_{\fp}(\fcc(\fp), \fcc(\fp')) + d_{\fp}(\fcc(\fp'), \mfa)
    $$
    Using \eqref{eq-wiggle1} and \eqref{wiggleorder} for the first summand, and \Cref{monotone-cube-metrics} for the second summand, this is estimated by
    $$
        n + 2^{-95a} d_{\fp'}(\fcc(\fp'), \mfa) < n + 2^{-95a} m\,.
    $$
    Thus $B_{\fp'}(\fcc(\fp'),m) \subset B_{\fp}(\fcc(\fp),n + 2^{-95a}m)$. Combined with $\scI(\fp) \subset \scI(\fp')$, this yields \eqref{eq-wiggle2}.
\end{proof}

\begin{lemma}[wiggle order 3]
\label{wiggle-order-3}
\uses{wiggle-order-1, wiggle-order-2}
    The following implications hold for all $\fq, \fq' \in \fP$:
    % Note: some fixes were suggested by Georges Gonthier
    % Note: we basically use {eq-sc1} in chapter 6. Mention this.
    \begin{equation}
        \label{eq-sc1}
        \fq \le \fq' \ \text{and} \ \lambda \ge 1.1 \implies \lambda \fq \lesssim \lambda \fq'\,,
    \end{equation}
    \begin{equation}
        \label{eq-sc2}
        10\fq \lesssim \fq' \ \text{and} \ \scI(\fq) \ne \scI(\fq') \implies 100 \fq \lesssim 100 \fq'\,,
    \end{equation}
    \begin{equation}
        \label{eq-sc3}
        2\fq \lesssim \fq' \ \text{and} \ \scI(\fq) \ne \scI(\fq') \implies 4 \fq \lesssim 500 \fq'\,.
    \end{equation}
\end{lemma}

\begin{proof}
    All three implications are easy consequences of \Cref{wiggle-order-1}, \Cref{wiggle-order-2} and the fact that $a \ge 4$.
\end{proof}



We call a collection $\mathfrak{A}$ of tiles convex if
\begin{equation}
    \label{eq-convexity}
    \fp \le \fp' \le \fp'' \ \text{and} \ \fp, \fp'' \in \mathfrak{A} \implies \fp' \in \mathfrak{A}\,.
\end{equation}

\begin{lemma}[P convex]
    \label{P-convex}

    For each $k$, the collection $\fP(k)$ is convex.
\end{lemma}

\begin{proof}
    Suppose that $\fp \le \fp' \le \fp''$ and $\fp, \fp'' \in \fP(k)$. By \eqref{eq-defPk} we have $\scI(\fp), \scI(\fp'') \in \mathcal{C}(G,k)$, so there exists by \eqref{muhj1} some $J \in \mathcal{D}$ with
    $$
        \scI(\fp') \subset \scI(\fp'') \subset J
    $$
    and $\mu(G \cap J) > 2^{-k-1} \mu(J)$. Thus \eqref{muhj1} holds for $\scI(\fp')$. On the other hand, by \eqref{muhj2}, there exists no $J \in \mathcal{D}$ with $\scI(\fp) \subset J$ and $\mu(G \cap J) > 2^{-k} \mu(J)$. Since $\scI(\fp) \subset \scI(\fp')$, this implies that \eqref{muhj2} holds for $\scI(\fp')$. Hence $\scI(\fp') \in \mathcal{C}(G,k)$, and therefore by \eqref{eq-defPk} $\fp' \in \fP(k)$.
\end{proof}

\begin{lemma}[C convex]
    \label{C-convex}
    \uses{P-convex}
    For each $k,n$, the collection $\fC(k,n)$ is convex.
\end{lemma}

\begin{proof}
    Let $\fp \le \fp' \le \fp''$ with $\fp, \fp'' \in \fC(k,n)$. Then, in particular, $\fp, \fp'' \in \fP(k)$, so, by \Cref{P-convex}, $\fp' \in \fP(k)$. Next, we show that if $\fq \le \fq' \in \fP(k)$ then $\dens'_k(\{\fq\}) \ge \dens_k'(\{\fq'\})$. If $\fp \in \fP(k)$ and $\lambda \ge 2$ with $\lambda \fq' \lesssim \lambda \fp$, then it follows from $\fq \le \fq'$, \eqref{eq-sc1} of \Cref{wiggle-order-3} and transitivity of $\lesssim$ that $\lambda \fq \lesssim \lambda \fp$. Thus the supremum in the definition \eqref{eq-densdef} of $\dens_k'(\{\fq\})$ is over a superset of the set the supremum in the definition of $\dens_k'(\{\fq'\})$ is taken over, which shows $\dens'_k(\{\fq\}) \ge \dens_k'(\{\fq'\})$. From $\fp' \le \fp''$, $\fp'' \in \fC(k,n)$ and \eqref{def-cnk} it then follows that
    $$
        2^{4a}2^{-n} < \dens_k'(\{\fp''\}) \le \dens_k'(\{\fp'\})\,.
    $$
    Similarly, it follows from $\fp \le \fp'$, $\fp \in \fC(k,n)$ and \eqref{def-cnk} that
    $$
        \dens_k'(\{\fp'\}) \le \dens_k'(\{\fp\}) \le 2^{4a}2^{-n+1}\,.
    $$
    Thus $\fp' \in \fC(k,n)$.
\end{proof}

\begin{lemma}[C1 convex]
    \label{C1-convex}
    \uses{C-convex}
    For each $k,n,j$, the collection $\fC_1(k,n,j)$ is convex.
\end{lemma}

\begin{proof}
    Let $\fp\le\fp'\le\fp''$ with $\fp, \fp'' \in \fC_1(k,n,j)$. By \Cref{C-convex} and the inclusion $\fC_1(k,n,j) \subset \fC(k,n)$, which holds by definition \eqref{defcnkj}, we have $\fp' \in \fC(k,n)$. By \eqref{eq-sc1} and transitivity of $\lesssim$ we have that $\fq \le \fq'$ and $100 \fq' \lesssim \mathfrak{m}$ imply $100 \fq \lesssim \mathfrak{m}$. So, by \eqref{defbfp}, $\mathfrak{B}(\fp'') \subset \mathfrak{B}(\fp') \subset \mathfrak{B}(\fp)$. Consequently, by \eqref{defcnkj}
    $$
        2^j \le |\mathfrak{B}(\fp'')|\le |\mathfrak{B}(\fp')| \le |\mathfrak{B}(\fp)| < 2^{j+1}\,,
    $$
    thus $\fp' \in \fC_1(k,n,j)$.
\end{proof}

\begin{lemma}[C2 convex]
    \label{C2-convex}
    \uses{C1-convex}
    For each $k,n,j$, the collection $\fC_2(k,n,j)$ is convex.
\end{lemma}

\begin{proof}
     Let $\fp \le \fp' \le \fp''$ with $\fp, \fp'' \in \fC_2(k,n,j)$. By \eqref{eq-C2-def}, we have
     \begin{equation*}
         \fC_2(k,n,j) \subset \fC_1(k,n,j)\ .
     \end{equation*} Combined with \Cref{C1-convex}, it follows that $\fp' \in \fC_1(k,n,j)$. Suppose that $\fp' \notin \fC_2(k,n,j)$. By \eqref{eq-C2-def}, this implies that there exists $0 \le l' \le Z(n+1)$ with $\fp' \in \fL_1(k,n,j,l')$. By the definition \eqref{eq-L1-def} of $\fL_1(k,n,j,l')$, this implies that $\fp$ is minimal with respect to $\le$ in $\fC_1(k,n,j) \setminus \bigcup_{l < l'} \fL_1(k,n,j,l)$. Since $\fp \in \fC_2(k,n,j)$ we must have $\fp \ne \fp'$. Thus $\fp \le \fp'$ and $\fp \ne \fp'$. By minimality of $\fp'$ it follows that $\fp \notin \fC_1(k,n,j) \setminus \bigcup_{l < l'} \fL_1(k,n,j,l)$. But by \eqref{eq-C2-def} this implies $\fp \notin \fC_2(k,n,j)$, a contradiction.
\end{proof}

\begin{lemma}[C3 convex]
    \label{C3-convex}
    \uses{C2-convex}
    For each $k,n,j$, the collection $\fC_3(k,n,j)$ is convex.
\end{lemma}

\begin{proof}
    Let $\fp \le \fp' \le \fp''$ with $\fp, \fp'' \in \fC_3(k,n,j)$. By \eqref{eq-C3-def} and \Cref{C2-convex} it follows that $\fp' \in \fC_2(k,n,j)$. Suppose that $\fp' \notin \fC_3(k,n,j)$. Then, by \eqref{eq-C3-def} and \eqref{eq-L2-def}, there exists $\fu \in \fU_1(k,n,j)$ with $2\fp' \lesssim \fu$ and $\scI(\fp') \ne \scI(\fu)$. Together this gives $\scI(\fp') \subsetneq \scI(\fu)$. From $\fp' \le \fp$, \eqref{eq-sc1} and transitivity of $\lesssim$ we then have $2\fp \lesssim \fu$. Also, $\scI(\fp) \subset \scI(\fp') \subsetneq \scI(\fu)$, so $\scI(\fp) \ne \scI(\fu)$. But then $\fp \in \fL_2(k,n,j)$, contradicting by \eqref{eq-C3-def} the assumption $\fp \in \fC_3(k,n,j)$.
\end{proof}

\begin{lemma}[C4 convex]
    \label{C4-convex}
    \uses{C3-convex}
    For each $k,n,j$, the collection $\fC_4(k,n,j)$ is convex.
\end{lemma}

\begin{proof}
    Let $\fp \le \fp' \le\fp''$ with $\fp, \fp'' \in \fC_4(k,n,j)$. As before we obtain from the inclusion $\fC_4(k,n,j) \subset \fC_3(k,n,j)$ and \Cref{C3-convex} that $\fp' \in \fC_3(k,n,j)$. Thus, if $\fp' \notin \fC_4(k,n,j)$ then by \eqref{eq-L3-def} there exists $l$ such that $\fp' \in \fL_3(k,n,j,l)$. Thus $\fp'$ is maximal with respect to $\le$ in $\fC_3(k,n,j) \setminus \bigcup_{0 \le l' < l} \fL_3(k,n,j,l')$. Since $\fp'' \in \fC_4(k,n,j)$ we must have $\fp' \ne \fp''$. Thus $\fp' \le \fp''$ and $\fp'\ne \fp''$. By minimality of $\fp'$ it follows that $\fp'' \notin \fC_3(k,n,j) \setminus \bigcup_{l < l'} \fL_3(k,n,j,l)$. But by \eqref{eq-C4-def} this implies $\fp'' \notin \fC_4(k,n,j)$, a contradiction.
\end{proof}

\begin{lemma}[C5 convex]
    \label{C5-convex}
    \uses{C4-convex}
    For each $k,n,j$, the collection $\fC_5(k,n,j)$ is convex.
\end{lemma}

\begin{proof}
    Let $\fp \le \fp' \le\fp''$ with $\fp, \fp'' \in \fC_5(k,n,j)$. Then $\fp, \fp'' \in \fC_4(k,n,j)$ by \eqref{defc5}, and thus by \Cref{C4-convex} also $\fp' \in \fC_4(k,n,j)$. Suppose that $\fp' \notin \fC_5(k,n,j)$. By \eqref{defc5}, it follows that $\fp' \in \fL_4(k,n,j)$.
    By \eqref{eq-L4-def}, there exists $\fu \in \fU_1(k,n,j)$ with $\scI(\fp') \subset \bigcup \mathcal{L}(\fu)$. Then also $\scI(\fp) \subset \bigcup \mathcal{L}(\fu)$, a contradiction.
\end{proof}

\begin{lemma}[dens compare]
    \label{dens-compare}

    We have for every $k\ge 0$ and $\fP'\subset \fP(k)$
\begin{equation}
    \dens_1(\fP')\le \dens_k'(\fP')\, .
\end{equation}
\end{lemma}
\begin{proof}
It suffices to show that for all $\fp'\in \fP'$
and $\lambda\ge 2$ and $\fp\in \fP(\fP')$ with $\lambda \fp' \lesssim \lambda \fp$ we have
\begin{equation}
    \frac{\mu({E}_2(\lambda, \fp))}{\mu(\scI(\fp))}
    \le \sup_{\fp'' \in \fP(k): \lambda \fp' \lesssim \lambda \fp''}
    \frac{\mu({E}_2(\lambda, \fp''))}{\mu(\scI(\fp''))}.
\end{equation}
    Let such $\fp'$, $\lambda$, $\fp$ be given.
    It suffices to show that $\fp\in \fP(k)$,
    that is, it satisfies \eqref{muhj1}
    and \eqref{muhj2}.

We show \eqref{muhj1}.
 As $\fp\in \fP(\fP')$, there exists
$\fp''\in \fP'$ with $\scI(\fp')\subset \scI(\fp'')$. By assumption on $\fP'$, we have $\fp''\in \fP(k)$ and there exists
$J\in \mathcal{D}$ with
   $\scI(\fp'')\subset J$ and
   \begin{equation}
       \mu(G\cap J)>2^{-k-1} \mu(J).
   \end{equation}
Then also $\scI(\fp')\subset J$, which proves
\eqref{muhj1} for $\fp$.

We show \eqref{muhj2}. Assume to get a contradiction that
there exists $J\in \mathcal{D}$ with
   $\scI(\fp)\subset J$ and
   \begin{equation}\label{mugj}
       \mu(G\cap J)>2^{-k} \mu(J).
   \end{equation}
   As $\lambda\fp'\lesssim \lambda\fp$, we have $\scI(\fp')\subset \scI(\fp)$, and therefore
    $\scI(\fp')\subset J$. This contradicts
   $\fp'\in \fP'\subset \fP(k)$. This proves
\eqref{muhj2} for $\fp$.
\end{proof}

\begin{lemma}[C dens1]
    \label{C-dens1}
    \uses{dens-compare}
    For each set $\mathfrak{A} \subset \mathfrak{C}(k,n)$, we have
    $$
        \dens_1(\mathfrak{A}) \le 2^{4a}2^{-n+1}\,.
    $$
\end{lemma}

\begin{proof}
    We have by \Cref{dens-compare} that
    $\dens_1(\mathfrak{A}) \le \dens_k'(\mathfrak{A})$. Since $\mathfrak{A} \subset \fC(k,n)$, it follows from monotonicity of suprema and the definition \eqref{eq-densdef} that
    $
        \dens_k'(\mathfrak{A}) \le \dens_k'(\fC(k,n))\,.
    $
    By \eqref{eq-densdef} and \eqref{def-cnk}, we have
    $$
        \dens_k'(\fC(k,n)) = \sup_{\fp \in \fC(k,n)} \dens_k'(\{\fp\}) \le 2^{4a}2^{-n+1}\,.
    $$
\end{proof}

\section{Proof of the Forest Union Lemma}
\label{subsecforest}

Fix $k,n,j\ge 0$.
Define
$$
    \fC_6(k,n,j)
$$
to be the set of all tiles $\fp \in \fC_5(k,n,j)$ such that $\scI(\fp) \not\subset G'$. The following chain of lemmas
establishes that the set $\fC_6(k,n,j)$ can be written as a union of a small number of $n$-forests.

For $\fu\in \fU_1(k,n,j)$, define
\begin{equation}
    \label{eq-T1-def}
    \mathfrak{T}_1(\fu):= \{\fp \in \fC_1(k,n,j) \ : \scI(\fp)\neq \scI(\fu), \ 2\fp \lesssim \fu\}\,.
\end{equation}
Define
\begin{equation}
    \label{eq-U2-def}
    \fU_2(k,n,j) := \{ \fu \in \fU_1(k,n,j) \, : \, \mathfrak{T}_1(\fu) \cap \fC_6(k,n,j) \ne \emptyset\}\,.
\end{equation}

Define a relation $\sim$ on $\fU_2(k,n,j)$
by setting $\fu\sim \fu'$
for $\fu,\fu'\in \fU_2(k,n,j)$
if $\fu=\fu'$ or there exists $\fp$ in $\mathfrak{T}_1(\fu)$
with $10 \fp\lesssim \fu'$.

\begin{lemma}[relation geometry]
    \label{relation-geometry}
    \uses{wiggle-order-3}
    If $\fu \sim \fu'$, then $\scI(u) = \scI(u')$ and
    \begin{equation*}
        B_{\fu}(\fcc(\fu), 100) \cap B_{\fu'}(\fcc(\fu'), 100) \neq \emptyset\ .
    \end{equation*}
\end{lemma}

\begin{proof}
    Let $\fu, \fu' \in \fU_2(k,n,j)$ with $\fu \sim \fu'$. If $\fu = \fu'$ then the conclusion of the Lemma clearly holds. Else, there exists $\fp \in \fC_1(k,n,j)$ such that $\scI(\fp) \ne \scI(\fu)$ and $2 \fp \lesssim \fu$ and $10 \fp \lesssim \fu'$.
    Using \Cref{wiggle-order-1} and \eqref{eq-sc2} of \Cref{wiggle-order-3}, we deduce that
    \begin{equation}
        \label{eq-Fefferman-trick0}
        100 \fp\lesssim 100 \fu\,, \qquad 100 \fp \lesssim 100\fu'\,.
    \end{equation}
    Now suppose that $B_{\fu}(\fcc(\fu), 100) \cap B_{\fu'}(\fcc(\fu'), 100) = \emptyset$. Then we have $\mathfrak{B}(\fu) \cap \mathfrak{B}(\fu') = \emptyset$, by the definition \eqref{defbfp} of $\mathfrak{B}$ and the definition \eqref{wiggleorder} of $\lesssim$, but also $\mathfrak{B}(\fu) \subset \mathfrak{B}(\fp)$ and $\mathfrak{B}(\fu') \subset \mathfrak{B}(\fp)$, by \eqref{defbfp}, \eqref{wiggleorder} and \eqref{eq-Fefferman-trick0}
    Hence,
    $$
        |\mathfrak{B}(\fp)| \geq |\mathfrak{B}(\fu)| + |\mathfrak{B}(\fu')| \geq 2^{j} + 2^j = 2^{j+1}\,,
    $$
    which contradicts $\fp \in \fC_1(k,n,j)$. Therefore we must have
    \begin{equation*}
        B_{\fu}(\fcc(\fu), 100) \cap B_{\fu'}(\fcc(\fu'), 100) \ne \emptyset\, .
    \end{equation*}

    It follows from $2\fp \lesssim \fu$ and $10\fp \lesssim \fu'$ that $\scI(\fp) \subset \scI(\fu)$ and $\scI(\fp) \subset \scI(\fu')$. By \eqref{dyadicproperty}, it follows that $\scI(\fu)$ and $\scI(\fu')$ are nested.
    Combining this with the conclusion of the last paragraph and definition \eqref{defunkj} of $\fU_1(k,n,j)$, we obtain that $\scI(\fu) = \scI(\fu')$.
\end{proof}


\begin{lemma}[equivalence relation]
\label{equivalence-relation}
\uses{relation-geometry}
For each $k,n,j$, the relation $\sim$ on
$\fU_2(k,n,j)$ is an equivalence relation.
\end{lemma}

\begin{proof}
    Reflexivity holds by definition.
    For transitivity, suppose that
    \begin{equation*}
        \fu, \fu', \fu'' \in \fU_1(k,n,j)
    \end{equation*}
    and $\fu \sim \fu'$, $\fu' \sim \fu''$.
    By \Cref{relation-geometry}, it follows that $\scI(\fu) =\scI(\fu') = \scI(\fu'')$, that there exists
    \begin{equation*}
        \mfa \in B_{\fu}(\fcc(\fu), 100) \cap B_{\fu'}(\fcc(\fu'), 100)
    \end{equation*}
    and that there exists
    \begin{equation*}
        \mfb \in B_{\fu'}(\fcc(\fu'), 100) \cap B_{\fu''}(\fcc(\fu''), 100)\, .
    \end{equation*}
    If $\fu = \fu'$, then $\fu \sim \fu''$ holds by assumption. Else, there exists by the definition of $\sim$ some $\fp \in \mathfrak{T}_1(\fu)$ with $10\fp\lesssim \fu'$.
    Then we have $2\fp \lesssim \fu$ and $\fp \ne \fu$ by definition of $\mathfrak{T}_1(\fu)$, so $4 \fp \lesssim 500 \fu$ by \eqref{eq-sc3}. For $q \in B_{\fu''}(\fcc(\fu''), 1)$ it follows by the triangle inequality that
    \begin{align*}
        d_{\fu}(\fcc(\fu), q) &\le d_{\fu}(\fcc(\fu), \mfa) + d_{\fu}(\mfa, \fcc(\fu'))\\
        &\quad+ d_{\fu}(\fcc(\fu'), \mfb) + d_{\fu}(\mfb, \fcc(\fu'')) +
        d_{\fu}(\fcc(\fu''), q)\,.
    \end{align*}
    Using \eqref{defdp} and the fact that $\scI(\fu) = \scI(\fu') = \scI(\fu'')$ this equals
    \begin{align*}
        &\quad d_{\fu}(\fcc(\fu), \mfa) + d_{\fu'}(\mfa, \fcc(\fu'))\\
        &\quad+ d_{\fu'}(\fcc(\fu'), \mfb) + d_{\fu''}(\mfb, \fcc(\fu'')) +
        d_{\fu''}(\fcc(\fu''), q)\\
        &< 100 + 100 + 100 + 100 + 1 < 500\,.
    \end{align*}
    Since $4\fp \lesssim 500 \fu$, it follows that $d_{\fp}(\fcc(\fp), q) < 4 < 10$. We have shown that $B_{\fu''}(\fcc(\fu''), 1) \subset B_{\fp}(\fcc(\fp), 10)$, combining this with $\scI(\fu'') = \scI(\fu)$ gives $\fu \sim \fu''$.

    For symmetry suppose that $\fu \sim \fu'$. By Lemma \eqref{relation-geometry}, it follows that $\scI(\fu) = \scI(\fu')$ and that there exists $\mfa \in B_{\fu}(\fcc(\fu), 100) \cap B_{\fu'}(\fcc(\fu'), 100)$. Again, for $\fu = \fu'$ symmetry is obvious. If $\fu \ne \fu'$, then there exists $\fp \in \mathfrak{T}_1(\fu')$ with $10\fp\lesssim \fu$. By definition of $\mathfrak{T}_1(\fu')$, \Cref{wiggle-order-1} and \eqref{eq-sc3}, it follows that
    \begin{equation}
        \label{eq-rel1}
        10\fp \lesssim 4\fp \lesssim 500 \fu'\,.
    \end{equation}
    If $q \in B_{\fu}(\fcc(\fu),1)$ then we have from the triangle inequality and the fact that $\scI(\fu) = \scI(\fu')$:
    \begin{align*}
        d_{\fu'}(\fcc(\fu'), q) &\le d_{\fu'}(\fcc(\fu'), \mfa) + d_{\fu'}(\mfa, \fcc(\fu)) + d_{\fu'}(\fcc(\fu), q)\\
        &= d_{\fu'}(\fcc(\fu'), \mfa) + d_{\fu}(\mfa, \fcc(\fu)) + d_{\fu}(\fcc(\fu), q)\\
        &< 100 + 100 + 1 < 500\,.
    \end{align*}
    Combining this with \eqref{eq-rel1} and \eqref{wiggleorder}, we get
    \begin{equation*}
     B_{\fu}(\fcc(\fu), 1) \subset B_{\fp}(\fcc(\fp), 10)\, .
    \end{equation*}
    Since $2\fp \lesssim \fu'$, we have $\scI(\fp) \subset \scI(\fu') = \scI(\fu)$. Thus, $10\fp \lesssim \fu$ which completes the proof of $\fu' \sim \fu$.
\end{proof}

Choose a set $\fU_3(k,n,j)$ of representatives for the equivalence
classes of $\sim$ in $\fU_2(k,n,j)$.
Define for each $\fu\in \fU_3(k,n,j)$
\begin{equation}\label{definesv}
\fT_2(\fu):=
   \bigcup_{\fu\sim \fu'}\mathfrak{T}_1(\fu')\cap \fC_6(k,n,j)\, .
\end{equation}

\begin{lemma}[C6 forest]
\label{C6-forest}
\uses{equivalence-relation}
We have
\begin{equation}
    \fC_6(k,n,j)=\bigcup_{\fu\in \fU_3(k,n,j)}\mathfrak{T}_2(\fu)\, .
\end{equation}
\end{lemma}
\begin{proof}
    Let $\fp \in \fC_6(k,n,j)$.
    By \eqref{eq-C4-def} and \eqref{defc5}, we have $\fp \in \fC_3(k,n,j)$. By \eqref{eq-L2-def} and \eqref{eq-C3-def}, there exists $\fu \in \fU_1(k,n,j)$ with $2\fp \lesssim \fu$ and $\scI(\fp) \ne \scI(\fu)$, that is, with $\fp \in \mathfrak{T}_1(\fu)$. Then $\mathfrak{T}_1(\fu)$ is clearly nonempty, so $\fu \in \fU_2(k,n,j)$. By the definition of $\fU_3(k,n,j)$, there exists $\fu' \in \fU_3(k,n,j)$ with $\fu \sim \fu'$. By \eqref{definesv}, we have $\fp \in \mathfrak{T}_2(\fu')$.
\end{proof}

\begin{lemma}[C6 convex]
    \label{C6-convex}
    \uses{C5-convex}
    Let $\fu \in \fU_3(k,n,j)$. If $\fp \le \fp' \le \fp''$ and $\fp, \fp'' \in \mathfrak{T}_2(\fu)$, then $\fp' \in \mathfrak{T}_2(\fu)$.
\end{lemma}

\begin{proof}
    Suppose that $\fp, \fp'' \in \mathfrak{T}_2(\fu)$. Then by \Cref{C5-convex}, we have
$\fp' \in \fC_5(k,n,j)$. Since $\fp \in \fC_6(k,n,j)$ we have $\scI(\fp) \not\subset G'$, hence $\scI(\fp') \not \subset G'$. This implies $\fp' \in \fC_6(k,n,j)$. Since $\fp'' \in \mathfrak{T}_2(\fu)$, we have $2\fp'' \lesssim \fu'$ and $\scI(\fp'')\ne\scI(\fu')$ for some $\fu' \sim \fp''$. By \eqref{eq-sc1}, we have $2\fp' \lesssim 2\fp''$, so by transitivity of $\lesssim$ we have $2\fp' \lesssim \fu'$. Finally, $\scI(\fp') \subset \scI(\fp'')$ implies $\scI(\fp') \ne \scI(\fu')$, thus $\fp' \in \mathfrak{T}_1(\fu') \subset \mathfrak{T}_2(\fu)$.
\end{proof}


\begin{lemma}[forest geometry]
    \label{forest-geometry}
    \uses{relation-geometry}
    For each $\fu\in \fU_3(k,n,j)$,
    the set $\mathfrak{T}_2(\fu)$
    satisfies \eqref{forest1}.
\end{lemma}
\begin{proof}
    Let $\fp \in \mathfrak{T}_2(\fu)$. By \eqref{definesv}, there exists $\fu' \sim \fu$ with $\fp \in \mathfrak{T}_1(\fu')$. Then we have $2\fp \lesssim \fu'$ and $\scI(\fp) \ne \scI(\fu')$, so by \eqref{eq-sc3} $4\fp \lesssim 500\fu'$.
    Further, by \Cref{relation-geometry}, we have that $\scI(\fu') = \scI(\fu)$ and there exists $\mfa \in B_{\fu'}(\fcc(\fu'),100) \cap B_{\fu}(\fcc(\fu),100)$.
    Let $\mfb \in B_{\fu}(\fcc(\fu), 1)$.
    Using the triangle inequality and the fact that $\scI(\fu') =\scI(\fu)$, we obtain
    \begin{align*}
        d_{\fu'}(\fcc(\fu'), \mfb) &\le d_{\fu'}(\fcc(\fu'), \mfa) + d_{\fu'}(\fcc(\fu), \mfa) + d_{\fu'}(\fcc(\fu), \mfb)\\
        &= d_{\fu'}(\fcc(\fu'), \mfa) + d_{\fu}(\fcc(\fu), \mfa) + d_{\fu}(\fcc(\fu), \mfb)\\
        &< 100 + 100 + 1 < 500\,.
    \end{align*}
    Combining this with $4\fp \lesssim 500\fu'$, we obtain
    $$
        B_{\fu}(\fcc(\fu), 1) \subset B_{\fu'}(\fcc(\fu'), 500) \subset B_{\fp}(\fcc(\fp), 4)\,.
    $$
    Together with $\scI(\fp) \subset \scI(\fu') = \scI(\fu)$, this gives $4\fp \lesssim 1\fu$, which is \eqref{forest1}.
\end{proof}

\begin{lemma}[forest convex]
    \label{forest-convex}
    \uses{C6-convex}
    For each $\fu\in \fU_3(k,n,j)$,
    the set $\mathfrak{T}_2(\fu)$
    satisfies the convexity condition \eqref{forest2}.
\end{lemma}

\begin{proof}
    Let $\fp, \fp'' \in \mathfrak{T}_2(\fu)$ and $\fp' \in \fP$ with $\fp \le \fp' \le \fp''$. By \eqref{definesv} we have $\fp, \fp'' \in \fC_6(k,n,j) \subset \fC_5(k,n,j)$. By \Cref{C5-convex}, we have $\fp' \in \fC_5(k,n,j)$. Since $\fp \in \fC_6(k,n,j)$ we have $\scI(\fp) \not \subset G'$, so $\scI(\fp') \not \subset G'$ and therefore also $\fp' \in \fC_6(k,n,j)$.

    By \eqref{definesv} there exists $\fu' \in \fU_2(k,n,j)$ with $\fp'' \in \mathfrak{T}_1(\fu')$ and hence $2\fp'' \lesssim \fu'$ and $\scI(\fp'') \ne \scI(\fu')$. Together this implies $\scI(\fp'') \subsetneq \scI(\fu')$. With the inclusion $\scI(\fp') \subset \scI(\fp'')$ from $\fp' \le \fp''$, it follows that $\scI(\fp') \subsetneq \scI(\fu')$ and hence $\scI(\fp') \ne \scI(\fu')$.
    By \eqref{eq-sc1} and transitivity of $\lesssim$ we further have $2\fp' \lesssim \fu'$, so $\fp' \in \mathfrak{T}_1(\fu')$.
    It follows that $\fp' \in \mathfrak{T}_2(\fu)$, which shows \eqref{forest2}.
\end{proof}

\begin{lemma}[forest separation]
    \label{forest-separation}
    \uses{monotone-cube-metrics}
    For each $\fu,\fu'\in \fU_3(k,n,j)$ with $\fu\neq \fu'$ and each $\fp \in \fT_2(\fu)$
    with $\scI(\fp)\subset \scI(\fu')$ we have
    \begin{equation}
    d_{\fp}(\fcc(\fp), \fcc(\fu')) > 2^{Z(n+1)}\,.
    \end{equation}
\end{lemma}

\begin{proof}
    By the definition \eqref{eq-C2-def} of $\fC_2(k,n,j)$, there exists a tile $\fp' \in \fC_1(k,n,j)$ with $\fp' \le \fp$ and $\ps(\fp') = \ps(\fp)- Z(n+1)$.
    By \Cref{monotone-cube-metrics} we have
    $$
        d_{\fp}(\fcc(\fp), \fcc(\fu')) \ge 2^{95a Z(n+1)} d_{\fp'}(\fcc(\fp), \fcc(\fu'))\,.
    $$
    By \eqref{eq-sc1} we have $2\fp' \lesssim 2\fp$, so by transitivity of $\lesssim$ there exists $\mathfrak{v} \sim \fu$ with $2\fp' \lesssim \mathfrak{v}$ and $\scI(\fp') \ne \scI(\mathfrak{v})$. Since $\fu, \fu'$ are not equivalent under $\sim$, we have $\mathfrak{v} \not \sim \fu'$, thus $10\fp' \not\lesssim \fu'$. This implies that there exists $q \in B_{\fu'}(\fcc(\fu'), 1) \setminus B_{\fp'}(\fcc(\fp'), 10)$.

    From $\fp' \le \fp$, $\scI(\fp') \subset \scI(\fp) \subset \scI(\fu')$ and \Cref{monotone-cube-metrics} it then follows that
    \begin{align*}
        &\quad d_{\fp'}(\fcc(\fp), \fcc(\fu'))\\
        &\ge -d_{\fp'}(\fcc(\fp), \fcc(\fp')) + d_{\fp'}(\fcc(\fp'), q) - d_{\fp'}(q, \fcc(\fu'))\\
        &\ge -d_{\fp'}(\fcc(\fp), \fcc(\fp')) + d_{\fp'}(\fcc(\fp'), q) - d_{\fu'}(q, \fcc(\fu'))\\
        &> -1 + 10 - 1 = 8\,.
    \end{align*}
    The lemma follows by combining the two displays with the fact that $95 a \ge 1$.
\end{proof}

\begin{lemma}[forest inner]
    \label{forest-inner}
    \uses{relation-geometry}
    For each $\fu\in \fU_3(k,n,j)$
    and each $\fp \in \mathfrak{T}_2(\fu)$
    we have
    \begin{equation}
        B(\pc(\fp), 8 D^{\ps(\fp)}) \subset \scI(\fu).
    \end{equation}
\end{lemma}

\begin{proof}
    Let $\fp \in \mathfrak{T}_2(\fu)$. Let
    \begin{equation}
        \label{eq-c3-tree}
        \fq \in \bigcup_{\fu \sim \fu'} \mathfrak{T}_1(\fu') \cap \fC_3(k,n,j)
    \end{equation}
    be a maximal element of this set with respect to $\le$ such that $\fp \le \fq$.
    We show that there is no $\fq' \in \fC_3(k,n,j)$ with $\fq \le \fq'$ and $\fq \ne \fq'$. Indeed, suppose $\fq'$ was such a tile.
    By \eqref{eq-C3-def} there exists $\fu'' \in \fU_1(k,n,j)$ with $2\fq' \lesssim \fu''$. Then we have in particular by \Cref{wiggle-order-1} that $10 \fp \lesssim \fu''$. Let $\fu' \sim \fu$ be such that $\fp \in \mathfrak{T}_1(\fu')$. By definition of $\sim$, we have $\fu' \sim \fu''$, hence $\fu \sim \fu''$. This implies that $\fq'$ is in the set in \eqref{eq-c3-tree}, contradicting maximality of $\fq$.

    Let $\fu' \sim \fu$ with $\fq \in \mathfrak{T}_1(\fu')$. By the definition \eqref{eq-T1-def} of $\mathfrak{T}_1$, we have $\ps(\fp) < \ps(\fu')$. By \Cref{relation-geometry}, we have $\ps(\fu) = \ps(\fu')$, hence $\ps(\fq) < \ps(\fu)$. By definition of $\fC_4(k,n,j)$, $\fp$ is not in any of the maximal $Z(n+1)$ layers of tiles in $\fC_3(k,n,j)$, and hence $\ps(\fp) \le \ps(\fq) - Z(n+1) \le \ps(\fu) - Z(n+1) - 1$.

    Thus, there exists some cube $I \in \mathcal{D}$ with $s(I) = \ps(\fu) - Z(n+1) - 1$ and $I \subset \scI(\fu)$ and $\scI(\fp) \subset I$. Since $\fp \in \fC_5(k,n,j)$, we have that $I \notin \mathcal{L}(\fu)$, so $B(c(I), 8D^{s(I)}) \subset \scI(\fu)$. By the triangle inequality, \eqref{defineD} and $a \ge 4$, the same then holds for the subcube $\scI(\fp) \subset I$.
\end{proof}


\begin{lemma}[forest stacking]
    \label{forest-stacking}
    It holds that
    \begin{equation}
        \sum_{\fu \in \fU_3(k,n,j)} \mathbf{1}_{\scI(\fu)} \le 1 + (4n+12)2^{n}\,.
    \end{equation}
    % Note(Georges Gonthier): Because we removed the `1 +` from \eqref{eq-Aoverlap-def} we can also remove the 1+ here.
\end{lemma}

\begin{proof}
    Suppose that a point $x$ is contained in more than $1 + (4n + 12)2^n$ cubes $\scI(\fu)$ with $\fu \in \fU_3(k,n,j)$. Since $\fU_3(k,n,j) \subset \fC_1(k,n,j)$ for each such $\fu$, there exists $\mathfrak{m} \in \mathfrak{M}(k,n)$ such that $100\fu \lesssim \mathfrak{m}$. We fix such an $\mathfrak{m}(\fu) := \mathfrak{m}$ for each $\fu$, and claim that the map $\fu \mapsto\mathfrak{m}(\fu)$ is injective. Indeed, assume for $\fu\neq \fu'$ there is $\mathfrak{m} \in \mathfrak{M}(k,n)$ such that $100\fu \lesssim \mathfrak{m}$ and $100\fu' \lesssim \mathfrak{m}$. By \eqref{dyadicproperty}, either $\scI(\fu) \subset \scI(\fu')$ or $\scI(\fu') \subset \scI(\fu)$. By \eqref{defunkj}, $B_{\fu}(\fcc(\fu),100) \cap B_{\fu'}(\fcc(\fu'), 100) = \emptyset$. This contradicts $\Omega(\mathfrak{m})$ being contained in both sets by \eqref{eq-freq-comp-ball}. Thus $x$ is contained in more than $1 + (4n + 12)2^n$ cubes $\scI(\mathfrak{m})$, $\mathfrak{m} \in \mathfrak{M}(k,n)$. Consequently, we have by \eqref{eq-Aoverlap-def} that $x \in A(2n + 6, k,n) \subset G_2$. Let $\scI(\fu)$ be an inclusion minimal cube among the $\scI(\fu'), \fu' \in \fU_3(k,n,j)$ with $x \in \scI(\fu)$. By the dyadic property \eqref{dyadicproperty}, we have $\scI(\fu) \subset \scI(\fu')$ for all cubes $\scI(\fu')$ containing $x$. Thus
    $$
        \scI(\fu) \subset \{y \ : \ \sum_{\fu \in \fU_3(k,n,j)} \mathbf{1}_{\scI(\fu)}(y) > 1 + (4n+12)2^{n}\} \subset G_2\,.
    $$
    Thus $\mathfrak{T}_1(\fu) \cap \fC_6(k,n,j) = \emptyset$.
    This contradicts $\fu \in \fU_2(k,n,j)$.
\end{proof}
We now turn to the proof of \Cref{forest-union}.
\begin{proof}[Proof of \Cref{forest-union}]
    \proves{forest-union}

    We first fix $k,n, j$.
    By \eqref{definetp} and \eqref{defineep}, we have that
    $\mathbf{1}_{\scI(\fp)} T_{\fp}f(x) = T_{\fp}f(x)$ and hence $\mathbf{1}_{G \setminus G'} T_{\fp}f(x)= 0$ for all $\fp \in \fC_5(k,n,j) \setminus \fC_6(k,n,j)$.
    Thus it suffices to estimate the contribution of the sets $\fC_6(k,n,j)$. By \Cref{forest-stacking}, we can decompose $\fU_3(k,n,j)$ as a disjoint union of at most $4n + 13$ collections $\fU_4(k,n,j,l)$, $1 \le l \le 4n+13$, each satisfying
    $$
        \sum_{\fu \in \fU_4(k,n,j,l)} \mathbf{1}_{\scI(\fu)} \le 2^n\,.
    $$
    By Lemmas \ref{forest-geometry}, \ref{forest-convex}, \ref{forest-separation}, \ref{forest-inner} and \ref{C-dens1}, the pairs
    $$
        (\fU_4(k,n,j,l), \mathfrak{T}_2|_{\fU_4(k,n,j,l)})
    $$
    are $n$-forests for each $k,n,j,l$, and by \Cref{C6-forest}, we have
    $$
        \fC_6(k,n,j) = \bigcup_{l = 1}^{4n + 13} \bigcup_{\fu \in \fU_4(k,n,j,l)} \mathfrak{T}_2(\fu)\,.
    $$
    Since $\scI(\fp) \not\subset G_1$ for all $\fp \in \fC_6(k,n,j)$, we have $\fC_6(k,n,j) \cap \fP_{F,G} = \emptyset$ and hence
    $$
        \dens_2(\bigcup_{\fu \in \fU_4(k,n,j,l)} \mathfrak{T}_2(\fu)) \le 2^{2a + 5} \frac{\mu(F)}{\mu(G)}\,.
    $$
    Using the triangle inequality according to the splitting by $k,n,j$ and $l$ in \eqref{disclesssim1} and applying \Cref{forest-operator} to each term, we obtain the estimate
    $$
        \sum_{k \ge 0}\sum_{n \ge k} (2n+3)(4n+13) 2^{432a^3}2^{-(1-\frac{1}{q})n}(2^{2a+5} \frac{\mu(F)}{\mu(G)})^{\frac{1}{q} - \frac{1}{2}} \|f\|_2 \|\mathbf{1}_{G\setminus G'}\|_2
    $$
    for the left hand side of \eqref{disclesssim1}. Since $|f| \le \mathbf{1}_F$, we have $\|f\|_2 \le \mu(F)^{1/2}$, and we have $\|\mathbf{1}_{G\setminus G'}\|_2 \le \mu(G)^{1/2}$. Combining this with $a \ge 4$, we estimate by
    $$
        2^{433a^3} \mu(F)^{\frac{1}{q}} \mu(G)^{1
        -\frac{1}{q}} \sum_{k \ge 0}\sum_{n \ge k}n^2 2^{-(1-\frac{1}{q})n}\,.
    $$ % n^2 -> (n+1)^2
    Interchanging the order of summation, the sum equals
    $$
        \sum_{n \ge 0} n^2(n+1) 2^{-\frac{q-1}{q}n} \le \frac{2^{2a^3}}{(q-1)^3}\,,
    $$
    which completes the proof of the lemma.
\end{proof}

\section{Proof of the Forest Complement Lemma}
\label{subsecantichain}

Define $\fP_{X \setminus G'}$ to be the set of all $\fp \in \fP$ such that $\scI(\fp) \not \subset G'$.
\begin{lemma}[antichain decomposition]
\label{antichain-decomposition}
    We have that
    \begin{align}
        \label{eq-fp'-decomposition}
        &\quad \fP_2 \cap \fP_{X \setminus G'}\\
        &= \bigcup_{k \ge 0} \bigcup_{n \ge k} \fL_0(k,n) \cap \fP_{X \setminus G'} \\
        &\quad\cup \bigcup_{k \ge 0} \bigcup_{n \ge k}\bigcup_{0 \le j \le 2n+3} \fL_2(k,n,j) \cap \fP_{X \setminus G'}\\
        &\quad\cup \bigcup_{k \ge 0} \bigcup_{n \ge k}\bigcup_{0 \le j \le 2n+3} \bigcup_{0 \le l \le Z(n+1)} \fL_1(k,n,j,l) \cap \fP_{X \setminus G'}\\
        &\quad\cup \bigcup_{k \ge 0} \bigcup_{n \ge k}\bigcup_{0 \le j \le 2n+3} \bigcup_{0 \le l \le Z(n+1)} \fL_3(k,n,j,l)\cap \fP_{X \setminus G'}\,.
    \end{align}
\end{lemma}
% fixme(Georges Gonthier): the second sentence of this proof is not technically true
% Can we restrict ourselves to a grid with only 1 top cube?
% Or alternatively, redefine \MfP_{X \ G'} to be the p s.t. I(p) \cap (G \ G') has nonzero measure.

\begin{proof}
    Let $\fp \in \fP_2 \cap \fP_{X \setminus G'}$. Clearly, for every cube $J \in \mathcal{D}$ there exists some $k \ge 0$ such that \eqref{muhj1} holds, and for no cube $J \in \mathcal{D}$ and no $k < 0$ does \eqref{muhj2} hold. Thus $\fp \in \fP(k)$ for some $k \ge 0$.

    Next, since $E_2(\lambda, \fp') \subset \scI(\fp')\cap G$ for every $\lambda \ge 2$ and every tile $\fp' \in \fP(k)$ with $\lambda\fp \lesssim \lambda \fp'$, it follows from \eqref{muhj2} that $\mu(E_2(\lambda, \fp')) \le 2^{-k} \mu(\scI(\fp'))$ for every such $\fp'$, so $\dens_k'(\{\fp\}) \le 2^{-k}$. Combining this with $a \ge 0$, it follows from \eqref{def-cnk} that there exists $n\ge k$ with $\fp \in \fC(k,n)$.

    Since $\fp \in \fP_{X \setminus G'}$, we have in particular $\scI(\fp) \not \subset A(2n + 6, k, n)$, so there exist at most $1 + (4n + 12)2^n < 2^{2n+4}$ tiles $\mathfrak{m} \in \mathfrak{M}(k,n)$ with $\fp \le \mathfrak{m}$. It follows that $\fp \in \fL_0(k,n)$ or $\fp \in \fC_1(k,n,j)$ for some $1 \le j \le 2n + 3$. In the former case we are done, in the latter case the inclusion to be shown follows immediately from the definitions of the collections $\fC_i$ and $\fL_i$.
\end{proof}

\begin{lemma}[L0 antichain]
\label{L0-antichain}
\uses{monotone-cube-metrics}
    We have that
    $$
        \fL_0(k,n) = \dot{\bigcup_{1 \le l \le n}} \fL_0(k,n,l)\,,
    $$
    where each $\fL_0(k,n,l)$ is an antichain.
\end{lemma}

\begin{proof}
    It suffices to show that $\fL_0(k,n)$ contains no chain of length $n + 1$. Suppose that we had such a chain $\fp_0 \le \fp_1 \le \dotsb \le \fp_{n}$ with $\fp_i \ne \fp_{i+1}$ for $i =0, \dotsc, n-1$. By \eqref{def-cnk}, we have that $\dens_k'(\{\fp_n\}) > 2^{-n}$. Thus, by \eqref{eq-densdef}, there exists $\fp' \in \fP(k)$ and $\lambda \ge 2$ with $\lambda \fp_n \le \lambda \fp'$ and
    \begin{equation}
        \label{eq-p'}
        \frac{\mu(E_2(\lambda, \fp'))}{\mu(\scI(\fp'))} > \lambda^{a} 2^{4a} 2^{-n}\,.
    \end{equation}
    Let $\mathfrak{O}$ be the set of all $\fp'' \in \fP(k)$ such that we have $ \scI(\fp'') = \scI(\fp')$ and $B_{\fp'}(\fcc(\fp'), \lambda) \cap \Omega(\fp'') \neq \emptyset$.
    We now show that
    \begin{equation}
        \label{eq-O-bound}
        |\mathfrak{O}| \le 2^{4a}\lambda^a\,.
    \end{equation}
    The balls $B_{\fp'}(\fcc(\fp''), 0.2)$, $\fp'' \in \mathfrak{O}$ are disjoint by \eqref{eq-freq-comp-ball},
    and by the triangle inequality contained in $B_{\fp'}(\fcc(\fp'), \lambda+1)$.
    % fixme(Georges Gonthier): \lambda+1 should be \lambda+1.2
    By assumption \eqref{thirddb} on $\Theta$, this ball can be covered with
    $$
        2^{a(\lceil \log_2(\lambda+1)\rceil + 2)} \le 2^{a(\log_2(\lambda) + 4)} = 2^{4a}\lambda^a
    $$  % Note(Georges Gonthier): this approximation still barely holds
    many $d_{\fp'}$-balls of radius $1/4$.
    % fixme(Georges Gonthier): 1/4 should be 1/5 (or something in between)
    By the triangle inequality, each such ball contains at most one $\fcc(\fp'')$, and each $\fcc(\fp'')$ is contained in one of the balls. Thus we get \eqref{eq-O-bound}.

    By \eqref{definee1} and \eqref{definee2} we have $E_2(\lambda, \fp') \subset \bigcup_{\fp'' \in \mathfrak{O}} E_1(\fp'')$, thus
    $$
        2^{4a}\lambda^a 2^{-n} < \sum_{\fp'' \in \mathfrak{O}} \frac{\mu(E_1(\fp''))}{\mu(\scI(\fp''))}\,.
    $$
    Hence there exists a tile $\fp'' \in \mathfrak{O}$ with
    \begin{equation*}
        \mu(E_1(\fp'')) \ge 2^{-n} \mu(\scI(\fp'))\,.
    \end{equation*}
    By the definition \eqref{mnkmax} of $\mathfrak{M}(k,n)$, there exists a tile $\mathfrak{m} \in \mathfrak{M}(k,n)$ with $\fp' \leq \mathfrak{m}$. From \eqref{eq-p'}, the inclusion $E_2(\lambda, \fp') \subset \scI(\fp')$ and $a\ge 1$ we obtain
    $$
        2^n \geq 2^{4a} \lambda^{a} \geq \lambda\,.
    $$
    From the triangle inequality, \Cref{monotone-cube-metrics} and $a \ge 1$, we now obtain for all $\mfa \in B_{\mathfrak{m}}(\fcc(\mathfrak{m}), 1)$ that
    \begin{align*}
        &\quad d_{\fp_0}(\fcc(\fp_0), \mfa)\\
        &\leq d_{\fp_0}(\fcc(\fp_0), \fcc(\fp_{n})) + d_{\fp_0}(\fcc(\fp_{n}), \fcc(\fp')) + d_{\fp_0}(\fcc(\fp'), \fcc(\fp''))\\
        &\quad+ d_{\fp_0}(\fcc(\fp''), \fcc(\mathfrak{m})) +
        d_{\fp_0}(\fcc(\mathfrak{m}), \mfa)\\
        &\leq 1 + 2^{-95an} (d_{\fp_{n}}(\fcc(\fp_n), \fcc(\fp')) + d_{\fp'}(\fcc(\fp'), \fcc(\fp''))\\
        &\quad+ d_{\fp''}(\fcc(\fp''), \fcc(\mathfrak{m})) +
        d_{\mathfrak{m}}(\fcc(\mathfrak{m}), \mfa))\\
        &\leq 1 + 2^{-95an}(\lambda + (\lambda + 1) + 1 + 1) \leq 100\,.
    \end{align*}
    Thus, by \eqref{straightorder}, $100\fp_0 \lesssim \mathfrak{m}$, a contradiction to $\fp_0 \notin \fC(k,n)$.
\end{proof}

\begin{lemma}[L2 antichain]
\label{L2-antichain}
\uses{monotone-cube-metrics}
    Each of the sets $\fL_2(k,n,j)$ is an antichain.
\end{lemma}

\begin{proof}
    Suppose that there are $\fp_0, \fp_1 \in \fL_2(k,n,j)$ with $\fp_0 \ne \fp_1$ and $\fp_0 \le \fp_1$. By \Cref{wiggle-order-1} and \Cref{wiggle-order-2}, it follows that $2\fp_0 \lesssim 200\fp_1$. Since $\fL_2(k,n,j)$ is finite, there exists a maximal $l \ge 1$ such that there exists a chain $2\fp_0 \lesssim 200 \fp_1 \lesssim \dotsb \lesssim 200 \fp_l$ with $\fp_i \ne \fp_{i+1}$ for $i = 0, \dotsc, l-1$.
    If we have $\fp_l \in \fU_1(k,n,j)$, then it follows from $2\fp_0 \lesssim 200 \fp_l \lesssim \fp_l$ and \eqref{eq-L2-def} that $\fp_0 \not\in \fL_2(k,n,j)$, a contradiction. Thus, by the definition \eqref{defunkj} of $\fU_1(k,n,j)$, there exists $\fp_{l+1} \in \fC_1(k,n,j)$ with $\scI(\fp_l) \subsetneq \scI(\fp_{l+1}) $ and $\mfa \in B_{\fp_l}(\fcc(\fp_l), 100) \cap B_{\fp_{l+1}}(\fcc(\fp_{l+1}), 100)$. Using the triangle inequality and \Cref{monotone-cube-metrics}, one deduces that $200 \fp_l \lesssim 200\fp_{l+1}$. This contradicts maximality of $l$.
\end{proof}

\begin{lemma}[L1 L3 antichain]
\label{L1-L3-antichain}
    Each of the sets $\fL_1(k,n,j,l)$ and $\fL_3(k,n,j,l)$ is an antichain.
\end{lemma}

\begin{proof}
    By its definition \eqref{eq-L1-def}, each set $\fL_1(k,n,j,l)$ is a set of minimal elements in some set of tiles with respect to $\le$. If there were distinct $\fp, \fq \in \fL_1(k,n,j,l)$ with $\fp \le \fq$, then $\fq$ would not be minimal. Hence such $\fp, \fq$ do not exist. Similarly, by \eqref{eq-L3-def}, each set $\fL_3(k,n,j,l)$ is a set of maximal elements in some set of tiles with respect to $\le$. If there were distinct $\fp, \fq \in \fL_3(k,n,j,l)$ with $\fp \le \fq$, then $\fp$ would not be maximal.
\end{proof}

We now turn to the proof of \Cref{forest-complement}.
\begin{proof}[Proof of \Cref{forest-complement}]
    \proves{forest-complement}
    If $\fp \not\in \fP_{X \setminus G'}$, then $\scI(\fp) \subset G'$. By \eqref{definetp} and \eqref{definee1}, it follows that
    $\mathbf{1}_{G \setminus G'} T_{\fp}f(x) = 0$. We thus have
    $$
        \mathbf{1}_{G\setminus G'} \sum_{\fp \in \fP_2} T_{\fp}f(x) = \mathbf{1}_{G\setminus G'} \sum_{\fp \in \fP_2 \cap \fP_{X \setminus G'}} T_{\fp}f(x)\,.
    $$
    Let $\fL(k,n)$ denote any of the terms $\fL_i(k,n,j,l) \cap \fP_{X \setminus \fP_2}$ on the right hand side of \eqref{eq-fp'-decomposition}, where the indices $j, l$ may be void. Then $\fL(k,n)$ is an antichain, by Lemmas \ref{L0-antichain},\ref{L2-antichain}, \ref{L1-L3-antichain}. Further, we have
    \begin{equation*}
    \dens_1(\fL(k,n)) \le 2^{4a+1 - n}
    \end{equation*}
    by \Cref{C-dens1}, and we have
    \begin{equation*}
     \dens_2(\fL(k,n)) \le 2^{2a+5} \frac{\mu(F)}{\mu(G)},
     \end{equation*}
     since
     \begin{equation*}
     \fL(k,n) \cap \fP_{F,G} \subset \fP_{X \setminus \fP_2} \cap \fP_{F, G} = \emptyset.
     \end{equation*}

    Applying now the triangle inequality according to the decomposition in \Cref{antichain-decomposition}, and then applying \Cref{antichain-operator} to each term, we obtain the estimate
    \begin{multline*}
        \le \sum_{k \ge 0} \sum_{n \ge k} (n + (2n+4) + 2(2n+4) Z(n+1)) \\
        \times 2^{201a^3}(q-1)^{-1} (2^{4a+1-n})^{\frac{q-1}{8a^4}} (2^{2a+5} \frac{\mu(F)}{\mu(G)})^{\frac{1}{q} - \frac{1}{2}} \|f\|_2\|\mathbf{1}_{G\setminus G'}\|_2\,.
    \end{multline*}
    Because $|f| \le \mathbf{1}_F$, we have $\|f\|_2 \le \mu(F)^{1/2}$, and we have $\|\mathbf{1}_{G\setminus G'}\|_2 \le \mu(G)^{1/2}$. Using this and \eqref{defineZ}, we bound
    $$
        \le 2^{202a^3} (q - 1)^{-1} \mu(F)^{\frac{1}{q}} \mu(G)^{\frac{1}{q'}} \sum_{k \ge 0} \sum_{n \ge k} n^2 2^{-n\frac{q-1}{8a^4}}\,.
    $$
    The last sum equals, by changing the order of summation,
    $$
        \sum_{n \ge 0} n^2(n+1) 2^{-n\frac{q-1}{8a^4}} \le \frac{2^{8a^3}}{(q-1)^3}\,.
    $$
    This completes the proof.
\end{proof}

\chapter{Proof of the Antichain Operator Proposition}

\label{antichainboundary}

Let an antichain $\mathfrak{A}$
and functions $f$, $g$ as in \Cref{antichain-operator} be given.
We prove \eqref{eq-antiprop}
in \Cref{sec-TT*-T*T}
as the geometric mean of two inequalities,
each involving one of the two densities.
One of these two inequalities will need a careful estimate formulated in
\Cref{tile-correlation} of
the $TT^*$ correlation between two tile operators.
\Cref{tile-correlation} will be proven in
\Cref{sec-tile-operator}.

The summation of the contributions of these individual correlations will require a
geometric \Cref{antichain-tile-count} counting the relevant tile pairs.
\Cref{antichain-tile-count} will be proven in Subsection
\ref{subsec-geolem}.






\section{The density arguments}\label{sec-TT*-T*T}

We begin with the following crucial disjointedness property of the sets $E(\fp)$ with $\fp \in \mathfrak{A}$.
\begin{lemma}[tile disjointness]
\label{tile-disjointness}
\lean{E_disjoint}
\leanok
Let $\fp,\fp'\in \mathfrak{A}$.
If there exists an $x\in X$ with $x\in E(\fp)\cap E(\fp')$,
then $\fp= \fp'$.
\end{lemma}
\begin{proof}\leanok
Let $\fp,\fp'$ and $x$ be given.
Assume without loss of generality that $\ps(\fp)\le \ps(\fp')$.
As we have $x\in E(\fp)\subset \scI(\fp)$ and $x\in E(\fp')\subset \scI(\fp')$ by Definition \eqref{defineep}, we conclude
for $i=1,2$ that
$\tQ(x)\in\fc(\fp)$ and $\tQ(x)\in\fc(\fp')$. By \eqref{eq-freq-dyadic} we have $\fc(\fp')\subset \fc(\fp)$. By Definition
\eqref{straightorder}, we conclude $\fp\le \fp'$. As $\mathfrak{A}$ is an antichain, we conclude $\fp=\fp'$.
This proves the lemma.
\end{proof}



Let $\mathcal{B}$ be the collection of balls
\begin{equation}
    B(\pc(\fp), 8D^{\ps(\fp)})
\end{equation}
with $\fp\in \mathfrak{A}$ and recall the definition of
$M_{\mathcal{B}}$ from Definition \ref{def-hlm}.

\begin{lemma}[maximal bound antichain]
    \label{maximal-bound-antichain}
    \uses{tile-disjointness}
    \lean{MaximalBoundAntichain}
    \leanok
    Let $x\in X$.
    Then
    \begin{equation}\label{hlmbound}
    | \sum_{\fp \in \mathfrak{A}}T_{\fp} f(x)|\le 2^{107 a^3} M_{\mathcal{B}} f (x) \, .
    \end{equation}
\end{lemma}

\begin{proof}
Fix $x\in X$. By \Cref{tile-disjointness}, there is at most one $\fp \in \mathfrak{A}$
such that
 $T_{\fp} f(x)$ is not zero.
 If there is no such $\fp$, the estimate \eqref{hlmbound} follows.

 Assume there is such a $\fp$.
 By definition of $T_{\fp}$ we have $x\in E(\fp)\subset \scI(\fp)$ and by the squeezing property \eqref{eq-vol-sp-cube}
\begin{equation}\label{eqtttt0}
    \rho(x, \pc(\fp))\le 4D^{\ps(\fp)}\, .
\end{equation}

Let $y\in X$ with $K_{\ps(\fp)}(x,y)\neq 0$. By Definition \eqref{defks} of $K_{\ps(\fp)}$
we have
\begin{equation}\label{supp-Ks1}
   \frac{1}{4} D^{\ps(\fp)-1}
   \leq \rho(x,y) \leq \frac{1}{2} D^{\ps(\fp)}\, .
\end{equation}
The triangle inequality with \eqref{eqtttt0} and \eqref{supp-Ks1} implies
\begin{equation}
    \rho(\pc(\fp),y)\le 8D^{\ps(\fp)}\, .
\end{equation}
Using the kernel bound \eqref{eqkernel-size} and the lower bound in \eqref{supp-Ks}
we obtain
\begin{equation}
|K_{\ps(\fp)}(x,y)|\le \frac{2^{a^3}}{\mu(B(x,\frac 14 D^{{\ps(\fp)}-1}))}\, .
\end{equation}
Using $D=2^{100a^2}$
and the doubling property \eqref{doublingx} $5 +100a^2$ times estimates
the last display by
\begin{equation}
\le \frac{2^{5a+101a^3}}{\mu(B(x, 8D^{\ps(\fp)}))}\, .
\end{equation}
 Using that {$|e(\mfa)|$} is bounded by $1$
for every $\mfa\in \Mf$, we estimate with the triangle inequality and the above information
 \begin{equation}
  | T_{\fp} f(x)|
    \le \frac{2^{5a+101 a^3}}{\mu(B(x, 8D^{\ps(\fp)}))} \int _{\mu(B(x, 8D^{\ps(\fp)}))} |f(y)|\, dy
  \end{equation}
This together with $a\ge 1$ proves the Lemma.
\end{proof}

Set
\begin{equation}
    \tilde{q}=\frac {2q}{1+q}\,.
\end{equation}
Since $1< q\le 2$, we have $1<\tilde{q}<q\le 2$.
\begin{lemma}[dens2 antichain]
\label{dens2-antichain}
\uses{Hardy-Littlewood,maximal-bound-antichain}
\lean{Dens2Antichain}
\leanok
We have that
\begin{equation}\label{eqttt9}
  \left|\int \overline{g(x)} \sum_{\fp \in \mathfrak{A}} T_{\fp} f(x)\, d\mu(x)\right|\le
  2^{111a^2}({q}-1)^{-1} \dens_2(\mathfrak{A})^{\frac 1{\tilde{q}}-\frac 12} \|f\|_2\|g\|_2\, .
\end{equation}
\end{lemma}
\begin{proof}
We have $f=\mathbf{1}_Ff$. Using H\"older's inequality, we obtain for
each $x\in B'$ and each $B'\in \mathcal{B}$ using $1<\tilde{q}\le 2$
\begin{equation}
    \frac 1{\mu(B')}\int_{B'} |f(y)|\, d\mu(y)
\end{equation}
\begin{equation}
    \le
    \left(\frac 1{\mu(B')}\int_{B'} |f(y)|^{\frac {2{\tilde{q}}}{3\tilde{q}-2}}\, d\mu(y)\right)^{\frac 32-\frac 1{\tilde{q}}}
    \left(\frac 1{\mu(B')}\int_{B'} \mathbf{1}_F(y)\, d\mu(y)\right)^{\frac 1{\tilde{q}}-\frac 12}
\end{equation}
\begin{equation}
    \le \left(M_{\mathcal{B}} (|f|^{\frac {2{\tilde{q}}}{3{\tilde{q}}-2}})(x)\right)^{\frac 32-\frac 1{\tilde{q}}}
\dens_2(\mathfrak{A})^{\frac 1{\tilde{q}}-\frac 12}\, .
\end{equation}
Taking the maximum over all $B'$ containing $x$, we obtain
\begin{equation} \label{eqttt1}
    M_{\mathcal{B}}|f|\le
    M_{\mathcal{B},\frac {2{\tilde{q}}}{3{\tilde{q}}-2} } |f|
    \dens_2(\mathfrak{A})^{\frac 1{\tilde{q}}-\frac 12}\, .
\end{equation}
We have with \Cref{Hardy-Littlewood}
\begin{equation}
\left\|M_{\mathcal{B}, \frac {2q}{3q-2}} f\right\|_2\le 2^{2a}(3\tilde{q}-2)(2\tilde{q}-2)^{-1}\|f\|_2\, .
\end{equation}
Using $1<\tilde{q}\le 2$ estimates the last display by
\begin{equation}\label{eqttt2}
 2^{2a+2} (\tilde{q}-1)^{-1} \|f\|_2\, .
\end{equation}
We obtain with Cauchy-Schwarz
and then \Cref{maximal-bound-antichain}
 \begin{equation}
     |\int \overline{g(x)} \sum_{\fp \in \mathfrak{A}} T_{\fp} f(x)\, d\mu(x)|
\end{equation}
 \begin{equation}
     \le \|g\|_2 \Big\| \sum_{\fp \in \mathfrak{A}} T_{\fp} f \Big\|_2
\end{equation}
 \begin{equation}
     \le 2^{107a^2}\|g\|_2 \| M_{\mathcal{B}}f \|_2
\end{equation}
With \eqref{eqttt1} and
\eqref{eqttt2} we can estimate the last display by
\begin{equation}
    \le 2^{107a^2+2a+2}(\tilde{q}-1)^{-1} \|g\|_2 \|f\|_2\dens_2(\mathfrak{A})^{\frac 1{\tilde{q}}-\frac 12}
\end{equation}
Using $a\ge 4$ and
$(\tilde q - 1)^{-1} = (q+1)/(q-1) \le 3(q-1)^{-1}$
proves the lemma.
\end{proof}


\begin{lemma}[dens1 antichain]\label{dens1-antichain}
\uses{Hardy-Littlewood, tile-correlation,antichain-tile-count}
Set $p:=4a^4$. We have
    \begin{equation}\label{eqttt3}
  \left|\int \overline{g(x)} \sum_{\fp \in \mathfrak{A}} T_{\fp} f(x)\, d\mu(x)\right|\le
   2^{150a^3}\dens_1(\mathfrak{A})^{\frac 1{2p}} \|f\|_2\|g\|_2\,.
\end{equation}
\end{lemma}



\begin{proof}

 We write for the expression inside the absolute values on the left-hand side of \eqref{eqttt3}
\begin{equation}
  \sum_{\fp \in \mathfrak{A}}\iint \overline{g(x)} \mathbf{1}_{E(\fp)}(x)
  {K_{\ps(\fp)}(x,y)}e(\tQ(x)(y) -
   \tQ(x)(x))
   f(y)\, d\mu(y)\,d\mu(x)
\end{equation}
\begin{equation}
  =\int \sum_{\fp \in \mathfrak{A}} \overline{T_{\fp} ^*g(y)} f(y)\, d\mu(y)
\end{equation}
with the adjoint operator
\begin{equation}\label{eq-tstarwritten}
    T_{\fp}^*g(y)=\int_{E(\fp)} \overline{K_{\ps(\fp)}(x,y)}e(-\tQ(x)(y)+
    \tQ(x)(x))g(x)\, d\mu(x)\, .
\end{equation}





 We have by expanding the square
\begin{equation}
    \int \Big|\sum_{\fp\in \mathfrak{A}}T^*_{\fp}g(y)\Big|^2\, d\mu(y)=
    \int \left(\sum_{\fp\in \mathfrak{A}} T^*_{\fp}g(y)\right)
    \left(\sum_{\fp'\in \mathfrak{A}}\overline{T^*_{\fp'}g(y)}\right)\, d\mu(y)
\end{equation}
\begin{equation}\label{eqtts1}
    \le \sum_{\fp\in \mathfrak{A}} \sum_{\fp'\in \mathfrak{A}}
    \Big|\int T^*_{\fp}g(y)\overline{T^*_{\fp'}g(y)}\, d\mu(y)\Big|\,.
\end{equation}
We split the sum into the terms with $\ps(\fp')\le \ps(\fp)$
and $\ps(\fp)< \ps(\fp')$. Using the symmetry of each summand,
we may switch $\fp$ and $\fp'$ in the second sum. Using further positivity
of each summand to replace the condition $\ps(\fp')< \ps(\fp)$
by $\ps(\fp')\le \ps(\fp)$ in the second sum, we estimate \eqref{eqtts1} by
\begin{equation}\label{eqtts2}
    \le2 \sum_{\fp\in \mathfrak{A}} \sum_{\fp'\in \mathfrak{A}: \ps(\fp')\le \ps(\fp)}
    \Big|\int T^*_{\fp}g(y)\overline{T^*_{\fp'}g(y)}\, d\mu(y)\Big|\,.
\end{equation}
The following basic $TT^*$ estimate will be proved in \Cref{sec-tile-operator}.
\begin{lemma}[tile correlation] % fixme: ideally we don't have a lemma inside a proof environment
    \label{tile-correlation}
    \uses{Holder-van-der-Corput,correlation-kernel-bound,tile-uncertainty,tile-range-support}
    Let $\fp, \fp'\in \fP$ with
    $\ps({\fp'})\leq \ps({\fp})$.
    Then
    \begin{equation}
        \label{eq-basic-TT*-est}
        \left|\int T^*_{\fp'}g\overline{T^*_{\fp}g}\right|
    \end{equation}
    \begin{equation}
        \le 2^{255a^3}\frac{(1+d_{\fp'}(\fcc(\fp'), \fcc(\fp))^{-1/(2a^2+a^3)}}{\mu(\scI(\fp))}\int_{E(\fp')}|g|\int_{E(\fp)}|g|\,.
    \end{equation}
    Moreover, the term \eqref{eq-basic-TT*-est} vanishes unless
    \begin{equation}
        \scI(\fp') \subset B(\pc(\fp), 15D^{\ps(\fp)})\, .
    \end{equation}
\end{lemma}
Define for $\fp\in \fP$
\begin{equation}
    B(\fp):=B(\pc(\fp), 15D^{\ps(\fp)})
\end{equation}
and define
\begin{equation}
    \label{eq-Dp-definition}
    \mathfrak{A}(\fp):=\{\fp'\in\mathfrak{A}: \ps(\fp')\leq \ps(\fp) \land \scI(\fp') \subset B(\fp)\}.
\end{equation}
Note that by the squeezing property \eqref{eq-vol-sp-cube}
and the doubling property \eqref{doublingx} applied
$6$ times we have
\begin{equation}\label{eqttt4}
    \mu(B(\fp))\le 2^{6a} \mu(\scI(\fp))\, .
\end{equation}
Using \Cref{tile-correlation} and \eqref{eqttt4}, we estimate \eqref{eqtts2} by
\begin{equation}\label{eqtts3}
     \le 2^{255a^3+6a+1} \sum_{\fp\in \mathfrak{A}}
    \int_{E(\fp)}|g|(y) h(\fp)\, d\mu(y)
\end{equation}
with $h(\fp)$ defined as
\begin{equation}\label{def-hp}
    \frac 1{\mu(B(\fp))}\int \sum_{\fp'\in \mathfrak{A}(\fp)}
    {(1+d_{\fp'}(\fcc(\fp'), \fcc(\fp))^{-1/(2a^2+a^3)}}(\mathbf{1}_{E(\fp')}|g|)(y')\, d\mu(y')\,.
\end{equation}
The following lemma will be proved in \Cref{subsec-geolem}.
\begin{lemma}[antichain tile count] % fixme: ideally we don't have a lemma inside a proof environment
    \label{antichain-tile-count}
    \uses{global-antichain-density}
    Set $p:=4a^4$ and let $p'$ be the dual exponent of $p$, that is $1/p+1/p'=1$.
    For every $\mfa\in\Mf$ and every subset $\mathfrak{A}'$ of $\mathfrak{A}$ we have
    \begin{equation}
        \label{eq-antichain-Lp}
        \Big\|\sum_{\fp\in\mathfrak{A}'}(1+d_{\fp}(\fcc(\fp), \mfa))^{-1/(2a^2+a^3)}\mathbf{1}_{E(\fp)}\mathbf{1}_G\Big\|_{p}
    \end{equation}
    \begin{equation}
        \le
        2^{104a}\dens_1(\mathfrak{A})^{\frac 1p}\mu\left(\cup_{\fp\in\mathfrak{A}'}I_{\fp}\right)^{\frac 1p}\, .
    \end{equation}
\end{lemma}

Note that $p\geq 4$ since $a>4$.
We estimate $h(\fp)$ as defined in \eqref{def-hp} with H\"older using $|g|\le \mathbf{1}_G$ and $E(\fp')\subset B(\fp)$ by

\begin{equation}
    \frac{\|g\mathbf{1}_{B(\fp)}\|_{p'}}{\mu(B(\fp))}
    \Big\|\sum_{\fp\in\mathfrak{A}(\fp)}(1+d_{\fp}(\fcc(\fp), \fcc(\fp'))^{-1/(2a^2+a^3)}\mathbf{1}_{E(\fp)}\mathbf{1}_G\Big\|_{p}\, .
\end{equation}
Then we apply \Cref{antichain-tile-count} to estimate this by
\begin{equation}\label{eqttt5}
    \le 2^{104a}
    \frac{\|g\mathbf{1}_{B(\fp)}\|_{p'}}{\mu(B(\fp))}
    \dens_1(\mathfrak{A})^{\frac 1p}\mu(B(\fp))^{\frac 1p}\,.
\end{equation}
Let $\mathcal{B}'$ be the collection of all balls
$B(\fp)$ with $\fp\in \mathfrak{A}$. Then
for each $\fp\in \mathfrak{A}$ and $x\in B(\fp)$ we have by
definition \eqref{def-hlm} of $M_{\mathcal{B}',p'}$
\begin{equation}
    \|g\mathbf{1}_{B(\fp)}\|_{p'}\le
    \mu(B(\fp))^{\frac 1{p'}} M_{\mathcal{B}',p'}g(x) \, .
\end{equation}
Hence we can estimate \eqref{eqttt5} by
\begin{equation}
\label{eqttt5b}
    \le
    2^{104a}
    (M_{\mathcal{B}', p'}g(x))
   \dens_1(\mathfrak{A})^{\frac 1p}\, .
\end{equation}
With this estimate of $h(\fp)$,
using $E(\fp)\subset B(\fp)$ by construction of $B(\fp)$, we estimate
\eqref{eqtts3} by
 \begin{equation}\label{eqtts4}
 \le 2^{255a^3+110a + 1} { \dens_1(\mathfrak{A})^{\frac 1p}}\sum_{\fp\in \mathfrak{A}}
 \int_{E(\fp)}|g|(y)M_{\mathcal{B}', p'}g(y) \, dy\,.
         \end{equation}
Using \Cref{tile-disjointness},
the last display is observed to be
\begin{equation}\label{eqtts4a}
= 2^{255a^3+110a + 1}
 {\dens_1(\mathfrak{A})^{\frac 1p}} \int |g|(y)(M_{\mathcal{B}', p'}g)(y) \, dy\,.
         \end{equation}
Applying Cauchy-Schwarz and using \Cref{Hardy-Littlewood}estimates the last display by
\begin{equation}
    2^{255a^3+110a + 1} \dens_1(\mathfrak{A})^{\frac 1p}
    \|g\|_2 \|M_{\mathcal{B}', p'} g\|_2
\end{equation}
\begin{equation}
    \le 2^{255a^3+110a + 3}\frac{2}{2-p'}
    \dens_1(\mathfrak{A})^{\frac 1p}\|g\|_2 ^2\,.
\end{equation}
Using $p>4$ and thus $1<p'<\frac 43$, we estimate the last display by
\begin{equation}
    \le 2^{255a^3+110a + 5} \dens_1(\mathfrak{A})^{\frac 1p}
    \|g\|_2^2\,.
\end{equation}
Now \Cref{dens1-antichain} follows by applying Cauchy-Schwarz on the left-hand side and using
$a\ge 4$.
\end{proof}
We have
\begin{equation}
    \left (\frac 1{\tilde{q}} -\frac 12\right) (2-q)= \frac 1q -\frac 12\,.
\end{equation}
Multiplying the $(2-q)$-th power of \eqref{eqttt9} and the $(q-1)$-th power of \eqref{eqttt3}
and estimating gives after simplification of some factors
\begin{equation}\label{eqttt8}
    \Big|\int \overline{g(x)} \sum_{\fp \in \mathfrak{A}} T_{\fp} f(x)\, d\mu(x)\Big|
\end{equation}
 \begin{equation}
    \le 2^{150a^3}({q}-1)^{-1} \dens_1(\mathfrak{A})^{\frac {q-1}{2p}}\dens_2(\mathfrak{A})^{\frac 1{q}-\frac 12} \|f\|_2\|g\|_2\, .
\end{equation}
With the definition of $p$, this implies
\Cref{antichain-operator}.


\section{Proof of the Tile Correlation Lemma}\label{sec-tile-operator}

The next lemma prepares an application of
\Cref{Holder-van-der-Corput}.
\begin{lemma}[correlation kernel bound]\label{correlation-kernel-bound}
Let $-S\le s_1\le s_2\le S$ and let $x_1,x_2\in X$.
Define \begin{equation}
 \varphi(y) := \overline{K_{s_1}(x_1, y)}
 K_{s_2}(x_2, y) \, .
\end{equation}
If $\varphi(y)\neq 0$, then
\begin{equation}\label{eqt10}
    y\in B(x_1, D^{s_1})\, .
\end{equation}
Moreover, we have with $\tau = 1/a$
\begin{equation}\label{eqt11}
  \|\varphi\|_{C^\tau(B(x_1, D^{s_1}))}\le
\frac{2^{254 a^3}}{\mu(B(x_1, D^{s_1}))\mu(B(x_2, D^{s_2}))}
      \, .
\end{equation}

\end{lemma}
\begin{proof}

If $\varphi(y)$ is not zero, then $K_{s_1}(x_1, y)$ is not zero and thus
\eqref{supp-Ks} gives \eqref{eqt10}.

We next have for $y$ with \eqref{eq-Ks-size}
\begin{equation}\label{suppart}
    |\varphi(y)|\le
    \frac{2^{204 a^3}}{\mu(B(x_1, D^{s_1}))\mu(B(x_2, D^{s_2}))}
\end{equation}
and for $y'\neq y$ additionally with \eqref{eq-Ks-smooth}
\begin{equation}
    |\varphi(y)-\varphi(y')|
 \end{equation}
 \begin{equation}
 \le
 |K_{s_1}(x_1,y)-K_{s_1}(x_1,y'))||
 K_{s_2}(x_2, y)|
\end{equation}
 \begin{equation}+|K_{s_1}(x_1, y')|
 |K_{s_2}(x_2, y) - K_{s_2}(x_2, y'))|
\end{equation}
\begin{equation}
      \le \frac{2^{252 a^3}}{\mu(B(x_1, D^{s_1}))\mu(B(x_2, D^{s_2}))}
       \left(\left(\frac{ \rho(y,y')}{D^{s_1}}\right)^{1/a}+
       \left(\frac{ \rho(y,y')}{D^{s_2}}\right)^{1/a}\right)
\end{equation}
\begin{equation}\label{holderpart}
      \le \frac{2^{253 a^3}}{\mu(B(x_1, D^{s_1}))\mu(B(x_2, D^{s_2}))}
       \left(\frac{ \rho(y,y')}{D^{s_1}}\right)^{1/a}\,.
\end{equation}
Adding the estimates \eqref{suppart} and \eqref{holderpart} gives \eqref{eqt11}.
This proves the lemma.
\end{proof}
The next lemma is a geometric estimate for two tiles.
\begin{lemma}\label{tile-uncertainty}
\uses{monotone-cube-metrics}
    Let $\fp_1, \fp_2\in \fP$ with
$\ps({\fp_1})\leq \ps({\fp_2})$. For each $x_1\in E(\fp_1)$ and
$x_2\in E(\fp_2)$ we have
\begin{equation}\label{tgeo}
  1+d_{\fp_1}(\fcc(\fp_1), \fcc(\fp_2))\le
    2^{3a}(1 + d_{B(x_1, D^{\ps(\fp_1)})}(\tQ(x_1),\tQ(x_2)))\, .
\end{equation}
\end{lemma}
\begin{proof}
Let $i\in \{1,2\}$.
By Definition \eqref{defineep} of $E$,
we have $\tQ(x_i)\in \fc(\fp_i)$
With \eqref{eq-freq-comp-ball} we then conclude
\begin{equation}\label{dponetwo}
    d_{\fp_i}(\tQ(x_i),\fcc(\fp_i))\le 1\, .
\end{equation}
We have $\scI(\fp_1)\subset \scI(\fp_2)$ by \eqref{dyadicproperty}. Using \Cref{monotone-cube-metrics} it follows that
\begin{equation}\label{tgeo0.5}
    d_{\fp_1}(\tQ(x_2), \fcc(\fp_2)) \le 1\,.
\end{equation}
By the triangle inequality, we obtain from \eqref{dponetwo} and
\eqref{tgeo0.5}
\begin{equation}\label{tgeo1}
     1+d_{\fp_1}(\fcc(\fp_1), \fcc(\fp_2))\le 3 +d_{\fp_1}(\tQ(x_1), \tQ(x_2))\, .
\end{equation}
As $x_1\in \scI(\fp_1)$ by Definition \eqref{defineep} of $E$, we have by the squeezing property \eqref{eq-vol-sp-cube}
\begin{equation}
    d(x_1,\pc(\fp_1))\le 4D^{\ps(\fp_1)}
\end{equation}
and thus by \eqref{eq-vol-sp-cube} again and the triangle inequality
\begin{equation}
    \scI(\fp_1)\subset B(x_1,8D^{\ps(\fp_1)})\, .
\end{equation}
We thus estimate the right-hand side of \eqref{tgeo1} with monotonicity \eqref{monotonedb} of the metrics $d_B$ by
\begin{equation}\label{tgeo1.5}
    \le 3+d_{B(x_1,8D^{\ps(\fp_1)})}(\tQ(x_1), \tQ(x_2))\, .
\end{equation}
This is further estimated by applying the doubling property \eqref{firstdb} three times by
\begin{equation}\label{tgeo2}
    \le 3+2^{3a}d_{B_1(x_1, D^{\ps(\fp_1)})}(\tQ(x_1), \tQ(x_2))\, .
\end{equation}
Now \eqref{tgeo} follows with $a\ge 1$.
\end{proof}




\begin{lemma}[tile range support]\label{tile-range-support}
    For each $\fp\in \fP$, and each $y\in X$, we have that
\begin{equation}\label{tstargnot0}
     T_{\fp}^* g(y)\neq 0
\end{equation}
   implies
\begin{equation}\label{ynotfar}
    y\in B(\pc(\fp),5D^{\ps(\fp)})\, .
\end{equation}
\end{lemma}
\begin{proof}
Fix $\fp$ and $y$ with \eqref{tstargnot0}.
Then there exists $x\in E(\fp)$ with
\begin{equation}
   \overline{K_{\ps(\fp)}(x,y)}e(-\tQ(x)(y)
    +\tQ(x)(x))g(x) \neq 0\, .
\end{equation}
As $E(\fp)\subset \scI(\fp)$ and by the squeezing property
\eqref{eq-vol-sp-cube}, we have
\begin{equation}
    \rho(x,\pc(\fp))\le 4D^{\ps(\fp)}\, .
\end{equation}
As $K_{\ps(\fp)}(x,y)\neq 0$, we have by \eqref{supp-Ks}
that
\begin{equation}
\rho(x,y)\le \frac 12 D^{\ps(\fp)}\, .
\end{equation}
Now \eqref{ynotfar} follows by the triangle inequality.
\end{proof}


We now prove \Cref{tile-correlation}. We begin with \eqref{eq-basic-TT*-est}.

We expand the left-hand side of \eqref{eq-basic-TT*-est} as
\begin{equation}\label{tstartstar}
\left|\int \int_{E(\fp_1)} \overline{K_{\ps(\fp_1)}(x_1,y)}e(-\tQ(x_1)(y)+
    \tQ(x_1)(x_1))g(x_1)\, d\mu(x_1) \right.
\end{equation}
\begin{equation}
 \times \left.\int_{E(\fp_2)} {K_{\ps(\fp_2)}(x_2,y)}e(\tQ(x_2)(y)
    -\tQ(x_2)(x_2))\overline{g(x_2)}\, d\mu(x_2)\, d\mu(y)\right|\, .
\end{equation}
By Fubini and the triangle inequality and
the fact $|e(\tQ(x_i)(x_i))|=1$ for $i=1,2$, we can estimate
\eqref{tstartstar} from above by
\begin{equation}\label{eqa1}
    \int_{E(\fp_1)} \int_{E(\fp_2)} {\bf I}(x_1, x_2)\, d\mu(x_1)d\mu(x_2)\,.
\end{equation}
with
\begin{equation}
    {\bf I}(x_1, x_2):=
    \left|\int
    e(-\tQ(x_1)(y)+\tQ(x_2)(y))\varphi_{x_1,x_2}(y)
    d\mu(y) \, g(x_1)g(x_2)\right|
\end{equation}


We estimate for fixed $x_1\in E(\fp_1)$ and
$x_2\in E(\fp_2)$ the inner integral of \eqref{eqa1} with
\Cref{Holder-van-der-Corput}. The function
$\varphi:=\varphi_{x_1,x_2}$ satisfies the assumptions of
\Cref{Holder-van-der-Corput} with $z = x_1$ and $R = D^{s_1}$ by \Cref{correlation-kernel-bound}.
We obtain with $B':= B(x_1, D^{\ps(\fp_1)})$,
\begin{equation*}
 {\bf I}(x_1, x_2) \le 2^{8a} \mu(B') \|{\varphi}\|_{C^\tau(B')}
       (1 + d_{B'}(\tQ(x_1),\tQ(x_2)))^{-1/(2a^2+a^3)}
\end{equation*}
\begin{equation}
\label{eqa1.5}
 \le \frac{2^{254a^3+8a}}
 {\mu(B(x_2, D^{\ps(\fp_2)}))}
       (1 + d_{B'}(\tQ(x_1),\tQ(x_2)))^{-1/(2a^2+a^3)}\,.
\end{equation}
Using \Cref{tile-uncertainty} and $a\ge 1$ estimates \eqref{eqa1.5} by
\begin{equation}\label{eqa2}
 \le \frac{2^{254a^3 + 8a + 1}}
 {\mu(B(x_2, D^{\ps(\fp_2)}))}
       (1+d_{\fp_1}(\fcc(\fp_1), \fcc(\fp_2)))^{-1/(2a^2+a^3)}\,.
\end{equation}
As $x_2\in \scI(\fp_2)$ by Definition \eqref{defineep} of $E$, we have by \eqref{eq-vol-sp-cube}
\begin{equation}
    \rho(x_2,\pc(\fp_2))\le 4D^{\ps(\fp_2)}
\end{equation}
and thus by \eqref{eq-vol-sp-cube} again and the triangle inequality
\begin{equation}
    \scI(\fp_2)\subset B(x_2,8D^{\ps(\fp_2)})\, .
\end{equation}
Using three iterations of the doubling property \eqref{doublingx} give
\begin{equation}
    \mu(\scI(\fp_2))\le 2^{3a}\mu(B(x_2,D^{\ps(\fp_2)}))\, .
\end{equation}
With $a\ge 1$ and \eqref{eqa2} we conclude \eqref{eq-basic-TT*-est}.


Now assume the left-hand side of \eqref{eq-basic-TT*-est} is not zero.
There is a $y\in X$ with
\begin{equation}
    T^*_{\fp}g(y)\overline{T^*_{\fp'}g(y)}\neq 0
\end{equation}
By the triangle inequality and \Cref{tile-range-support}, we conclude
\begin{equation}
   \rho(\pc(\fp),\pc(\fp'))\le \rho(\pc(\fp),y) +\rho(\pc(\fp'),y)
   \le 5D^{\ps(\fp)}+5D^{\ps(\fp')}\le 10 D^{\ps(\fp)}\, .
\end{equation}
By the squeezing property \eqref{eq-vol-sp-cube} and the triangle inequality,
we conclude
\begin{equation}
    \scI(\fp') \subset B(\pc(\fp), 15D^{\ps(\fp)})\, .
\end{equation}
   This completes the proof of Lemma \ref{tile-correlation}.





\section{Proof of the Antichain Tile Count Lemma}
\label{subsec-geolem}


\begin{lemma}[tile reach]\label{tile-reach}
\uses{monotone-cube-metrics}
Let $\mfa\in \Mf$ and $N\ge0$ be an integer.
Let $\fp, \fp'\in \fP$ with
\begin{equation}\label{eqassumedismfa}
    d_{\fp}(\fcc(\fp), \mfa))\le 2^N\,
\end{equation}
\begin{equation}\label{eqassumedismfap}
    d_{\fp'}(\fcc(\fp'), \mfa))\le 2^N\, .
\end{equation}
Assume $\scI(\fp)\subset \scI(\fp')$ and $\ps(\fp)<\ps(\fp')$.
Then
\begin{equation}\label{lp'lp''}2^{N+2}\fp\lesssim 2^{N+2} \fp'\, .
\end{equation}
\end{lemma}

\begin{proof}
By \Cref{monotone-cube-metrics}, we have
\begin{equation}
     d_{\fp}(\fcc(\fp'),\mfa)
     \le d_{\fp'}(\fcc(\fp'),\mfa)
     \le 2^{N} \, .
\end{equation}
Together with \eqref{eqassumedismfa} and the triangle inequality, we obtain
\begin{equation}\label{eqdistqpqp}
    d_{\fp'}(\fcc(\fp'),\fcc(\fp))\le 2^{N+1} \, .
\end{equation}
Now assume
\begin{equation}
    \mfa'\in B_{\fp'}(\fcc(\fp'),2^{N+2}).
\end{equation}
By the doubling property \eqref{firstdb}, applied five times, we have
\begin{equation}\label{ageo1} d_{B(\pc(\fp'),8D^{\ps(\fp')})}(\fcc(\fp'),\mfa') < 2^{5a+N+2}\, .
\end{equation}
We have by the squeezing property \eqref{eq-vol-sp-cube}
\begin{equation}
 \pc(\fp)\in
B(\pc(\fp'),4D^{\ps(\fp')})\, .
\end{equation}
Hence by the triangle inequality
\begin{equation}
 B(\pc(\fp), 4D^{\ps(\fp')})
 \subset
B(\pc(\fp'),8D^{\ps(\fp')})\, .
\end{equation}
Together with \eqref{ageo1} and monotonicity \eqref{monotonedb} of $d$
\begin{equation}
    d_{B(\pc(\fp),4D^{\ps(\fp')})}(\fcc(\fp'),\mfa') < 2^{5a+N+2}\, .
\end{equation}
Using the doubling property \eqref{seconddb} $5a+2$ times gives
\begin{equation}
    d_{B(\pc(\fp),2^{2-5a^2-2a}D^{\ps(\fp')})}(\fcc(\fp'),\mfa') < 2^{N}\, .
\end{equation}
Using $\ps(\fp)<\ps(\fp')$ and $D=2^{100a^2}$ and $a\ge 4$ gives
\begin{equation}
    d_{\fp}(\fcc(\fp'),\mfa') < 2^{N}\, .
\end{equation}
With the triangle inequality and \eqref{eqdistqpqp},
\begin{equation}
    d_{\fp}(\fcc(\fp),\mfa') < 2^{N+2}\, .
\end{equation}
This shows
\begin{equation}
B_{\fp'}(\fcc(\fp'),2^{N+2})\subset B_{\fp}(\fcc(\fp),2^{N+2})\, .
\end{equation}
This implies \eqref{lp'lp''} and completes the proof of the lemma.
\end{proof}

For $\mfa \in \Mf$ and $N\ge 0$ define
\begin{equation}\label{eqantidefap}
    \mathfrak{A}_{\mfa,N}:=\{\fp\in\mathfrak{A}: 2^{N}\le 1+d_{\fp}(\fcc(\fp), \mfa)\le 2^{N+1}\} \, .
\end{equation}


\begin{lemma}[stack density]
\label{stack-density}
Let $\mfa \in \Mf$, $N\ge 0$ and
$L\in \mathcal{D}$. Then
\begin{equation}\label{eqanti-1}
    \sum_{\fp\in\mathfrak{A}_{\mfa,N}:\scI(\fp)=L}\mu(E(\fp)\cap G)\le 2^{a(N+5)}\dens_1(\mathfrak{A})\mu(L)\, .
\end{equation}
\end{lemma}
\begin{proof}
Let $\mfa,N,L$ be given and set
\begin{equation}
\mathfrak{A}':=\{\fp\in\mathfrak{A}_{\mfa,N}:\scI(\fp)=L\}\, .
\end{equation}
Let
$\fp\in\mathfrak{A}'$.
We have
by Definition \eqref{definedens1}
using $\lambda=2$ and the squeezing property \eqref{eq-freq-comp-ball}
\begin{equation}\label{eqanti-3}
\mu(E(\fp)\cap G)\le \mu(E_2(2, \fp))\le 2^{a}\dens_1(\mathfrak{A}')\mu(L)\, .
\end{equation}
By the covering property \eqref{thirddb}, applied $N+4$ times, there is a collection $\Mf'$ of at most $2^{a(N+4)}$
elements such that
\begin{equation}\label{eqanti-4}
    B_{\fp}(\mfa, 2^{N+1})\subset \bigcup_{\mfa'\in \Mf'}
    B_{\fp}(\mfa', 0.2)\, .
\end{equation}
As each $\fcc(\fp)$ with $\fp\in \mathfrak{A}_{\mfa,N}$
is contained in the left-hand-side
of \eqref{eqanti-4}
by definition, it is in at least one $B_{\fp}(\mfa', 0.2)$
with $\mfa'\in \Mf'$.


For two different $\fp,\fp'\in \mathfrak{A}'$, we have by
\eqref{eq-dis-freq-cover} that
$\fc(\fp)$ and $\fc(\fp')$ are disjoint and thus by the squeezing property \eqref{eq-freq-comp-ball} we have for every $\mfa'\in \Mf'$
\begin{equation}
    \mfa'\not\in B_{\fp}(\fcc(\fp), 0.2)\cap
B_{\fp}(\fcc(\fp'), 0.2)\, .
\end{equation}
Hence at most one of $\fcc(\fp)$
and $\fcc(\fp)$ is in
$B_{\fp}(\mfa', 0.2)$.
It follows that there are at most $2^{a(N+4)}$ elements in
$\mathfrak{A}'$. Adding \eqref{eqanti-3} over $\mathfrak{A}'$ proves
\eqref{eqanti-1}.


\end{proof}


\begin{lemma}[local antichain density]\label{local-antichain-density}
\uses{tile-disjointness,tile-reach}
Let $\mfa\in\Mf$ and {$N$} be
an integer. Let $\fp_{\mfa}$ be a tile with $\mfa\in \fc(\fp_{\mfa})$.
Then we have
\begin{equation}\label{eqanti-0.5}
    \sum_{\fp\in\mathfrak{A}_{\mfa,N}: \ps(\fp_{\mfa})<\ps(\fp)}\mu(E(\fp)\cap G \cap \scI(\fp_{\mfa}))
    \le \mu (E_2(2^{N+3},\fp_{\mfa}))
 \, .
\end{equation}



\end{lemma}

\begin{proof}


Let $\fp$ be any tile in $\mathfrak{A}_{\mfa,N}$ with $\ps(\fp_{\mfa})<\ps(\fp)$. By definition of
$E$, the tile contributes zero to the sum on the left-hand side of \eqref{eqanti-0.5} unless
 $\scI(\fp)\cap \scI(\fp_{\mfa}) \neq \emptyset$, which we may assume. With $\ps(\fp_{\mfa})<\ps(\fp)$
and the dyadic property
\eqref{dyadicproperty} we conclude $\scI(\fp_{\mfa})\subset \scI(\fp)$.
By the squeezing property
\eqref{eq-freq-comp-ball},
we conclude from
$\mfa\in \fc(\fp_{\mfa})$
that
\begin{equation}
    \mfa\in B(\fcc(\fp_{\mfa}), 1)\, .
\end{equation}
We conclude from $\fp \in \mathfrak{A}_{\mfa,N}$ that
\begin{equation}
    \mfa \in B(\fcc(\fp), 2^{N+1})\, .
\end{equation}
With \Cref{tile-reach}, we conclude
\begin{equation}
    2^{N+3}\fp_{\mfa} \lesssim 2^{N+3}\fp \, .
\end{equation}
By Definition \eqref{definee2} of $E_2$, we conclude
\begin{equation}
    E(\fp)\cap G \subset E_2(2^{N+3},\fp_{\mfa})\, .
\end{equation}
Using disjointedness of the various $E(\fp)$ with $\fp\in \mathfrak{A}$ by \Cref{tile-disjointness}, we obtain \eqref{eqanti-0.5}.
This proves the lemma.
\end{proof}
\begin{lemma}[global antichain density]
\label{global-antichain-density}
\uses{stack-density,local-antichain-density}
Let $\mfa\in\Mf$ and let $N\ge 0$ be
an integer. Then we have
\begin{equation}\label{eqanti00}
    \sum_{\fp\in\mathfrak{A}_{\mfa,N}}\mu(E(\fp)\cap G)
    \le
 2^{101a^3+Na}\dens_1(\mathfrak{A})\mu\left(\cup_{\fp\in\mathfrak{A}}I_{\fp}\right)\, .
\end{equation}
\end{lemma}



\begin{proof}
{Fix $\mfa$ and $N$. Let
$\mathfrak{A}'$ be the set of $\fp\in\mathfrak{A}_{\mfa,N}$ such that $\scI(\fp)\cap G$ is not empty.}


   Let $\mathcal{L}$ be the collection of dyadic cubes $I\in\mathcal{D}$ such that $I\subset \scI(\fp)$ for some $\fp\in\mathfrak{A}'$ and if $\scI(\fp)\subset I$ for some $\fp\in\mathfrak{A}'$, then $\ps(\fp)=-S$. By \eqref{coverdyadic}, for each $\fp \in \mathfrak{A}'$
   and each $x\in \scI(\fp)\cap G$, there is $I\in \mathcal{D}$ with $s(I)=-S$ and $x\in I$. By \eqref{dyadicproperty},
   we have $I\subset \scI(\fp)$. Hence
   \begin{equation}
       \scI(\fp)\subset \bigcup\{I\in \mathcal{D}: s(I)=-S, I\subset \scI(\fp)\}\subset \bigcup \mathcal{L}\, .
   \end{equation}
As each $I\in \mathcal{L}$ satisfies $I\subset \scI(\fp)$ for some $\fp$ in $\mathfrak{A'}$, we conclude
     \begin{equation}
\bigcup\mathcal{L}=\bigcup_{\fp \in \mathfrak{A}'}\scI(\fp)\, .
   \end{equation}
Let $\mathcal{L}^*$ be the set of maximal elements in $\mathcal{L}$ with respect to set inclusion.
By \eqref{dyadicproperty}, the elements in $\mathcal{L}^*$ are pairwise disjoint and we have
 \begin{equation}\label{eqdecAprime}
\bigcup\mathcal{L}^*=\bigcup_{\fp \in \mathfrak{A}'}\scI(\fp)\, .
   \end{equation}
Using the partition \eqref{eqdecAprime} into elements of $\mathcal{L}$ in \eqref{eqanti0}, it suffices to show for each $L\in \mathcal{L}^*$
\begin{equation}\label{eqanti0}
    \sum_{\fp\in\mathfrak{A}'}\mu(E(\fp)\cap G \cap L)
    \le
    2^{101a^3+aN}
    \dens_1(\mathfrak{A})\mu(L)\,.
\end{equation}
Fix $L\in \mathcal{L}^*$.
By definition of $L$, there exists an element $\fp'\in \mathfrak{A}'$ such that $L\subset \scI(\fp')$. Pick such an element $\fp
'$
in $\mathfrak{A}$ with minimal $\ps(\fp')$. As $\scI(\fp')\not \subset L$ or $s(L) = -S$ by definition of $L$, we have
with \eqref{dyadicproperty} that $s(L)< \ps(\fp')$ or $s(L) = -S$. In particular $s(L)<S$.

If there exists no cube $J \in \mathcal{D}$ with $L \subsetneq J$, then by the definition of $\mathcal{L}$ we must have $s(L) = -S$. By \eqref{dyadicproperty}, this implies that each tile $\fp$ with $E(\fp) \cap L \ne \emptyset$ must satisfy $\scI(\fp) = L$. Thus the left hand side of \eqref{eqanti0} equals
\begin{equation}
    \sum_{\fp\in\mathfrak{A}':\scI(\fp)=L}\mu(E(\fp)\cap G\cap L)\,.
\end{equation}
Thus in this case \eqref{eqanti0} is a direct consequence of \Cref{stack-density} and $a \ge 4$.

We now assume that there exists a cube $J \in \mathcal{D}$ with $L \subsetneq J$.
By \eqref{coverdyadic}, there is an
$L'\in \mathcal{D}$ with $s(L')=s(L)+1$ and $c(L)\in L'$. By \eqref{dyadicproperty}, we have
$L\subset L'$.

We split the left-hand side of \eqref{eqanti0} as
\begin{equation}\label{eqanti1}
    \sum_{\fp\in\mathfrak{A}':\scI(\fp)=L'}\mu(E(\fp)\cap G\cap L)
\end{equation}
\begin{equation}\label{eqanti2}
    +
     \sum_{\fp\in\mathfrak{A}':\scI(\fp)\neq L'}\mu(E(\fp)\cap G\cap L)\, ,
\end{equation}

We first estimate \eqref{eqanti1}
with \Cref{stack-density} by
\begin{equation}\label{equanti1.5}
    \le \sum_{\fp\in\mathfrak{A}':\scI(\fp)=L'}\mu(E(\fp)\cap G\cap L')\le 2^{a(N+5)}\dens_1(\mathfrak{A})\mu(L')\, .
\end{equation}



We turn to \eqref{eqanti2}.
Consider the element $\fp'\in \mathfrak{A}'$ as above
with $L\subset \scI(\fp')$ and $s(L)<\ps(\fp')$.
As $L\subset L'$ and $s(L')=s(L)+1$, we conclude with the dyadic property that $L'\subset \scI(\fp')$.
By maximality of $L$, we have
$L'\not\in \mathcal{L}$.
This together with the existence of the given $\fp'\in \mathfrak{A}$
with $L'\subset \scI(\fp')$
shows by definition of $\mathcal{L}$ that there exists $\fp''\in \mathfrak{A}'$ with
$\scI(\fp'')\subset L'$.




By the covering property \eqref{eq-dis-freq-cover}, there exists a unique $\fp_{\mfa}$ with
\begin{equation*}
    \scI(\fp_{\mfa})=L'
\end{equation*}
such that $\mfa\in \fc(\fp_{\mfa})$.
Note that
\begin{equation}
    \mfa\in B(\fcc(\fp_{\mfa}), 1)
\end{equation}
and as $\fp'' \in \mathfrak{A}_{\mfa,N}$ that
\begin{equation}
    \mfa \in B(\fcc(\fp''), 2^{N+1})\, .
\end{equation}
By \Cref{tile-reach}, we conclude
\begin{equation}
    2^{N+3}\fp'' \lesssim 2^{N+3}\fp_{\mfa} \, .
\end{equation}
As $\fp''\in \mathfrak{A}'$, we have by Definition
\eqref{definedens1} of $\dens_1$ that
\begin{equation}\label{pmfadens}
   \mu(E_2(2^{N+3}, \fp_{\mfa}))\le 2^{Na+3a}\dens_1(\mathfrak{A}) {\mu(L')}\, .
\end{equation}
Now let $\fp$ be any tile in the summation set in \eqref{eqanti2}, that is, $\fp\in \mathfrak{A}'$ and $\scI(\fp)\neq L'$.
Then $\scI(\fp)\cap L\neq \emptyset$. It follows by the dyadic property \eqref{dyadicproperty}
and the definition of $L$ that
$L\subset \scI(\fp)$ and $L\neq \scI(\fp)$. By the dyadic property \eqref{dyadicproperty}, we have
$s(L)<\ps(\fp)$ and thus $s(L')\le \ps(\fp)$. By the dyadic property
   \eqref{dyadicproperty} again, we have $L'\subset \scI(\fp)$.
As $L'\neq \scI(\fp)$, we conclude $s(L)<\ps(\fp)$.
By \Cref{local-antichain-density}, we can thus estimate \eqref{eqanti2} by
\begin{equation}\label{eqanti0.5}
    \sum_{\fp\in\mathfrak{A}':\scI(\fp)\neq L'}\mu(E(\fp)\cap G\cap L')
    \le \mu (E_2(2^{N+3},\fp_{\mfa}))\, .
\end{equation}
Using the decomposition
into \eqref{eqanti1} and
\eqref{eqanti2} and the estimates
\eqref{equanti1.5},
\eqref{eqanti-0.5},
\eqref{pmfadens} we obtain the estimate
\begin{equation}\label{eqanti3.14}
\sum_{\fp\in\mathfrak{A}'}\mu(E(\fp)\cap G \cap L)
    \le (2^{a(N+5)}+2^{Na+3a})\dens_1(\mathfrak{A})\mu(L')\,.
\end{equation}

Using $s(L')=s(L)+1$ and $D=2^{100a^2}$ and the
squeezing property \eqref{eq-vol-sp-cube}
and the doubling property \eqref{doublingx} $100a^2+4$ times , we obtain
\begin{equation}
    \mu(L')\le 2^{100a^3+4a}\mu(L)\, .
\end{equation}
Inserting in \eqref{eqanti3.14} and using $a\ge 4$ gives \eqref{eqanti0}.
This completes the proof of the lemma.
\end{proof}



We turn to the proof of \Cref{antichain-tile-count}.

\begin{proof}[Proof of \Cref{antichain-tile-count}]
\proves{antichain-tile-count}

Using that $\mathfrak{A}$ is the union of the
$\mathfrak{A}_{\mfa,N}$ with $N\ge 0$,
we estimate the left-hand side \eqref{eq-antichain-Lp}
with the triangle inequality by
\begin{equation}\label{eqanti23}
\le \sum_{N\ge 0} \left\|\sum_{\fp\in \mathfrak{A}_{\mfa,N}} 2^{-N/(2a^2+a^3)}\mathbf{1}_{E(\fp)} \mathbf{1}_G\right\|_{p}
\end{equation}
We consider each individual term in this sum and estimate it's $p$-th power.
   Using that for each $x\in X$ by \Cref{global-antichain-density} there is at most one $\fp\in \mathfrak{A}$ with $x\in E(\fp)$,
 we have
 \begin{equation}
     \left\|\sum_{\fp\in \mathfrak{A}_{\mfa,N}} 2^{-N/(2a^2+a^3)}\mathbf{1}_{E(\fp)} \mathbf{1}_G\right\|_{p}^p
 \end{equation}
\begin{equation}
    =\int_G\Big(\sum_{\fp\in \mathfrak{A}_{\mfa,N}}2^{-N/(2a^2+a^3)}\mathbf{1}_{E(\fp)}(x)\Big)^p\, d\mu(x)
\end{equation}
\begin{equation}
  = \int _G\sum_{\fp\in\mathfrak{A}_{\mfa,N}}2^{-pN/(2a^2+a^3)}\mathbf{1}_{E(\fp)}(x)\, d\mu(x)
\end{equation}
\begin{equation}
  = 2^{-pN/(2a^2+a^3)} \sum_{\fp\in\mathfrak{A}_{\mfa,N}}\mu(E(\fp)\cap G)
\end{equation}

Using \Cref{global-antichain-density}, we estimate the last display by
\begin{equation}\label{eqanti21}
    \le 2^{-pN/(2a^2+a^3)+101a^3+Na}\dens_1(\mathfrak{A})\mu\left(\cup_{\fp\in\mathfrak{A}}\scI(\fp)\right)
\end{equation}
Using that with $a\ge 4$ and since $p = 4a^4$, we have
\begin{equation}
    pN/(2a^2+a^3)\ge
    4a^4N/(3a^3) \ge Na+N\, .
\end{equation}
Hence we have for \eqref{eqanti21} the upper bound
\begin{equation}\label{eqanti22}
\le 2^{101a^3-N}\dens_1(\mathfrak{A})\mu\left(\cup_{\fp\in\mathfrak{A}}\scI(\fp)\right)\, .
\end{equation}
Taking th $p$-th root and summing over $N\ge 0$ gives for \eqref{eqanti23} the upper bound

\begin{equation}
\le \left(\sum_{N\ge 0} 2^{-N/p}\right)2^{101a^3/p}\dens_1(\mathfrak{A})^{{\frac{1}{p}}}\mu\left(\cup_{\fp\in\mathfrak{A}}\scI(\fp)\right)^{{\frac{1}{p}}}
\end{equation}
\begin{equation}
\le \left(1-2^{-1/p}\right)^{-1}
2^{101a^3/p}
\dens_1(\mathfrak{A})^{\frac 1p}\mu\left(\cup_{\fp\in\mathfrak{A}}\scI(\fp)\right)^{\frac 1p}\, .
\end{equation}
Using that $p = 4a^4$ and $a \ge 4$, this proves the lemma.
\end{proof}
