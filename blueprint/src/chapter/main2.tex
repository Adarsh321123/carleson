\chapter{Proof of the Forest Operator Proposition}

\label{treesection}

\section{The pointwise tree estimate}
Fix a forest $(\fU, \fT)$. The main result of this subsection is \Cref{pointwise-tree-estimate}, we begin this section with some definitions necessary to state the lemma.

For $\fu \in \fU$ and $x\in X$, we define
$$
    \sigma (\fu, x):=\{\ps(\fp):\fp\in \fT(\fu), x\in E(\fp)\}\,.
$$
This is a subset of $\mathbb{Z} \cap [-S, S]$, so has a minimum and a maximum. We set
$$
    \overline{\sigma} (\fu, x) := \max \sigma(\fT(\fu), x)
$$
$$
    \underline{\sigma} (\fu, x) := \min\sigma(\fT(\fu), x)\,.
$$
\begin{lemma}[convex scales]
\label{convex-scales}
    For each $\fu \in \fU$, we have
    $$
        \sigma(\fu, x) = \mathbb{Z} \cap [\underline{\sigma} (\fu, x), \overline{\sigma} (\fu, x)]\,.
    $$
\end{lemma}

\begin{proof}
    Let $s \in \mathbb{Z}$ with $\underline{\sigma} (\fu, x) \le s \le \overline{\sigma} (\fu, x)$. By definition of $\sigma$, there exists $\fp \in \fT(\fu)$ with $\ps(\fp) = \underline{\sigma} (\fu, x)$ and $x \in E(\fp)$, and there exists $\fp'' \in \fT(\fu)$ with $\ps(\fp'') = \overline{\sigma} (\fu, x)$ and $x \in E(\fp'') \subset \scI(\fp'')$. By property \eqref{coverdyadic} of the dyadic grid, there exists a cube $I \in \mathcal{D}$ of scale $s$ with $x \in I$. By property \eqref{eq-dis-freq-cover}, there exists a tile $\fp' \in \fP(I)$ with $\tQ(x) \in \fc(\fp')$. By the dyadic property \eqref{dyadicproperty} we have $\scI(\fp) \subset \scI(\fp') \subset \scI(\fp'')$, and by \eqref{eq-freq-dyadic}, we have $\fc(\fp'') \subset \fc(\fp') \subset \fc(\fp)$. Thus $\fp \le \fp' \le\fp''$, which gives with the convexity property \eqref{forest2} of $\fT(\fu)$ that $\fp' \in \fT(\fu)$, so $s \in \sigma(\fu, x)$.
\end{proof}

For a nonempty collection of tiles $\mathfrak{S} \subset \fP$ we define
$$
    \mathcal{J}_0(\mathfrak{S})
$$
to be the collection of all dyadic cubes $J \in \mathcal{D}$ such that $s(J) = -S$ or
$$
    \scI(\fp) \not\subset B(c(J), 100D^{s(J) + 1})
$$
for all $\fp \in \mathfrak{S}$. We define $\mathcal{J}(\mathfrak{S})$ to be the collection of inclusion maximal cubes in $\mathcal{J}_0(\mathfrak{S})$.

We further define
$$
    \mathcal{L}_0(\mathfrak{S})
$$
to be the collection of dyadic cubes $L \in \mathcal{D}$ such that $s(L) = -S$, or there exists $\fp \in \mathfrak{S}$ with $L \subset \scI(\fp)$ and there exists no $\fp \in \mathfrak{S}$ with $\scI(\fp) \subset L$. We define $\mathcal{L}(\mathfrak{S})$ to be the collection of inclusion maximal cubes in $\mathcal{L}_0(\mathfrak{S})$.

\begin{lemma}[dyadic partitions]
    \label{dyadic-partitions}
    For each $\mathfrak{S} \subset \fP$, we have
    \begin{equation}
        \label{eq-J-partition}
        \bigcup_{I \in \mathcal{D}} I = \dot{\bigcup_{J \in \mathcal{J}(\mathfrak{S})}} J
    \end{equation}
    and
    \begin{equation}
        \label{eq-L-partition}
        \bigcup_{\fp\in \mathfrak{S}} \scI(\fp) = \dot{\bigcup_{L \in \mathcal{L}(\mathfrak{S})}} L\,.
    \end{equation}
\end{lemma}

\begin{proof}
    Since $\mathcal{J}(\mathfrak{S})$ is the set of inclusion maximal cubes in $\mathcal{J}_0(\mathfrak{S})$, cubes in $\mathcal{J}(\mathfrak{S})$ are pairwise disjoint by \eqref{dyadicproperty}. The same applies to $\mathcal{L}(\mathfrak{S})$.

    If $x \in \bigcup_{I \in \mathcal{D}} I$, then there exists by \eqref{coverdyadic} a cube $I \in \mathcal{D}$ with $x \in I$ and $s(I) = -S$. Then $I \in \mathcal{J}_0(\mathfrak{S})$. There exists an inclusion maximal cube in $\mathcal{J}_0(\mathfrak{S})$ containing $I$. This cube contains $x$ and is contained in $\mathcal{J}(\mathfrak{S})$. This shows one inclusion in \eqref{eq-J-partition}, the other one follows from $\mathcal{J}(\mathfrak{S}) \subset \mathcal{D}$.

    The proof of the two inclusions in \eqref{eq-L-partition} is similar.
\end{proof}

For a finite collection of pairwise disjoint cubes $\mathcal{C}$, define the projection operator
$$
    P_{\mathcal{C}}f(x) :=\sum_{J\in\mathcal{C}}\mathbf{1}_J(x) \frac{1}{\mu(J)}\int_J f(y) \, \mathrm{d}\mu(y)\,.
$$
Given a scale $-S \le s\le S$ and a point $x \in \bigcup_{I\in \mathcal{D}, s(I) = s} I$, there exists a unique cube in $\mathcal{D}$ of scale $s$ containing $x$ by \eqref{coverdyadic}. We denote it by $I_s(x)$. Define the nontangential maximal operator
\begin{equation}
    \label{eq-TN-def}
    T_{\mathcal{N}} f(x) := \sup_{-S \le s_1 < s_2 \le S} \sup_{x' \in I_{s_1}(x)} \left| \sum_{s = s_1}^{s_2} \int K_s(x',y) f(y) \, \mathrm{d}\mu(y) \right|\,.
\end{equation}
Define for each $\fu \in \fU$ the auxiliary operator
$$
    S_{1,\fu}f(x)
$$
\begin{equation}
    \label{eq-def-S-op}
    :=\sum_{I\in\mathcal{D}} \mathbf{1}_{I}(x) \sum_{\substack{J\in \mathcal{J}(\fT(\fu))\\
    J\subset B(c(I), 16 D^{s(I)})}} \frac{D^{(s(J) - s(I))/a}}{\mu(B(c(I), 16D^{s(I)}))}\int_J |f(y)| \, \mathrm{d}\mu(y)\,.
\end{equation}
Define also the collection of balls
$$
    \mathcal{B} = \{B(c(I), D^{s + s(I)}) \ : \ I \in \mathcal{D}\,, 0 \le s \le S + 5\}\,.
$$

The following pointwise estimate for operators associated to sets $\fT(\fu)$ is the main result of this subsection.

\begin{lemma}[pointwise tree estimate]
    \label{pointwise-tree-estimate}
    \uses{first-tree-pointwise,second-tree-pointwise,third-tree-pointwise}
    Let $\fu \in \fU$ and $L \in \mathcal{L}(\fT(\fu))$. Let $x, x' \in L$.
    Then for all bounded functions $f$ with bounded support
    $$
        \left|\sum_{\fp \in \fT(\fu)} T_{\fp}[ e(\fcc(\fu))f](x)\right|
    $$
    \begin{equation}
        \label{eq-LJ-ptwise}
        \leq 2^{151a^3}(M_{\mathcal{B},1}+S_{1,\fu})P_{\mathcal{J}(\fT(\fu))}|f|(x')+|T_{\mathcal{N}}P_{\mathcal{J}(\fT(\fu))}f(x')|,
    \end{equation}
\end{lemma}


\begin{proof}
    By \eqref{definetp}, if $T_{\fp}[ e(-\fcc(\fu))f](x) \ne 0$, then $x \in E(\fp)$. Combining this with $|e(\fcc(\fu)(x)-\tQ(x)(x))| = 1$, we obtain
    $$
        |\sum_{\fp \in \fT(\fu)} T_{\fp}[ e(\fcc(\fu))f](x)|
    $$
    \begin{multline*}
        = \Bigg| \sum_{s \in \sigma(\fu, x)} \int e(-\fcc(\fu)(y) + \tQ(x)(y) + \fcc(\fu)(x) -\tQ(x)(x))\times\\
        K_s(x,y)f(y) \, \mathrm{d}\mu(y) \Bigg|\,.
    \end{multline*}
    Using the triangle inequality, we bound this by the sum of three terms:
    \begin{multline}
        \label{eq-term-A}
        \le \Bigg| \sum_{s \in \sigma(\fu, x)} \int (e(-\fcc(\fu)(y) + \tQ(x)(y) + \fcc(\fu)(x) -\tQ(x)(x))-1)\times\\
        K_s(x,y)f(y) \, \mathrm{d}\mu(y) \Bigg|
    \end{multline}
    \begin{equation}
        \label{eq-term-B}
        + \Bigg| \sum_{s \in \sigma(\fu, x)} \int K_s(x,y) P_{\mathcal{J}(\fT(\fu))} f(y) \, \mathrm{d}\mu(y) \Bigg|
    \end{equation}
    \begin{equation}
        \label{eq-term-C}
        + \Bigg| \sum_{s \in \sigma(\fu, x)} \int K_s(x,y) (f(y) - P_{\mathcal{J}(\fT(\fu))} f(y)) \, \mathrm{d}\mu(y) \Bigg|\,.
    \end{equation}
    The proof is completed using the bounds for these three terms proven in \Cref{first-tree-pointwise}, \Cref{second-tree-pointwise} and \Cref{third-tree-pointwise}.
\end{proof}

\begin{lemma}[first tree pointwise]
    \label{first-tree-pointwise}
    \uses{convex-scales}
    For all $\fu \in \fU$, all $L \in \mathcal{L}(\fT(\fu))$, all $x, x' \in L$ and all bounded $f$ with bounded support, we have
    $$
        \eqref{eq-term-A} \le 10 \cdot 2^{105a^3} M_{\mathcal{B}, 1}P_{\mathcal{J}(\fT(\fu))}|f|(x')\,.
    $$
\end{lemma}

\begin{proof}
    Let $s \in \sigma(\fu,x)$.
    If $x, y \in X$ are such that $K_s(x,y)\neq 0$, then, by \eqref{supp-Ks}, we have $\rho(x,y)\leq 1/2 D^s$. By $1$-Lipschitz continuity of the function $t \mapsto \exp(it) = e(t)$ and the property \eqref{osccontrol} of the metrics $d_B$, it follows that
    \begin{multline*}
        |e(-\fcc(\fu)(y)+\tQ(x)(y)+\fcc(\fu)(x)-\tQ(x)(x))-1|\\
        \leq d_{B(x, 1/2 D^{s})}(\fcc(\fu), \tQ(x))\,.
    \end{multline*}
    Let $\fp_s \in \fT(\fu)$ be a tile with $\ps(\fp_s) = s$ and $x \in E(\fp_s)$, and let $\fp'$ be a tile with $\ps(\fp') = \overline{\sigma}(\fu, x)$ and $x \in E(\fp')$.
    Using the doubling property \eqref{firstdb}, the definition of $d_{\fp}$ and \Cref{monotone-cube-metrics}, we can bound the previous display by
    $$
        2^a d_{\fp_s}(\fcc(\fu), \tQ(x)) \le 2^{a} 2^{s - \overline{\sigma}(\fu, x)} d_{\fp'}(\fcc(\fu), \tQ(x))\,.
    $$
    Since $\fcc(\fu) \in B_{\fp'}(\fcc(\fp'), 4)$ by \eqref{forest1} and $\tQ(x) \in \Omega(\fp') \subset B_{\fp'}(\fcc(\fp'), 1)$ by \eqref{eq-freq-comp-ball}, this is estimated by
    $$
        \le 5 \cdot 2^{a} 2^{s - \overline{\sigma}(\fu, x)} \,.
    $$
    Using \eqref{eq-Ks-size}, it follows that
    $$
        \eqref{eq-term-A} \le 5\cdot 2^{103a^3} \sum_{s\in\sigma(x)}2^{s - \overline{\sigma}(\fu, x)} \frac{1}{\mu(B(x,D^s))}\int_{B(x,0.5D^{s})}|f(y)|\,\mathrm{d}\mu(y)\,.
    $$
    By \eqref{eq-J-partition}, the collection $\mathcal{J}$ is a partition of $X$, so this is estimated by
    $$
         5\cdot 2^{103a^3} \sum_{s\in\sigma(x)}2^{s - \overline{\sigma}(\fu, x)} \frac{1}{\mu(B(x,D^s))}\sum_{\substack{J \in \mathcal{J}(\fT(\fu))\\J \cap B(x, 0.5D^s) \ne \emptyset} }\int_{J}|f(y)|\,\mathrm{d}\mu(y)\,.
    $$
    This expression does not change if we replace $|f|$ by $P_{\mathcal{J}(\fT(\fu))}|f|$.

    Let $J \in \mathcal{J}(\fT(\fu))$ with $B(x, 0.5 D^s) \cap J \ne \emptyset$. By the triangle inequality and since $x \in E(\fp_s) \subset B(\pc(\fp_s), 4D^{s})$, it follows that $B(\pc(\fp_s), 4.5D^s) \cap J \ne \emptyset$. If $s(J) \ge s$ and $s(J) > -S$, then it follows from the triangle inequality, \eqref{eq-vol-sp-cube} and \eqref{defineD} that $\scI(\fp_s) \subset B(c(J), 100 D^{s(J)+1})$, contradicting $J \in \mathcal{J}(\mathfrak{T}(\fu))$. Thus $s(J) \le s - 1$ or $s(J) = -S$. If $s(J) = -S$ and $s(J) > s - 1$, then $s = -S$. Thus we always have $s(J) \le s$. It then follows from the triangle inequality and \eqref{eq-vol-sp-cube} that $J \subset B(\pc(\fp_s), 16 D^s)$.

    Thus we can continue our chain of estimates with
    $$
        5\cdot 2^{103a^3} \sum_{s\in\sigma(x)}2^{s - \overline{\sigma}(\fu, x)} \frac{1}{\mu(B(x,D^s))}\int_{B(\pc(\fp_s),16 D^s)}P_{\mathcal{J}(\fT(\fu))}|f(y)|\,\mathrm{d}\mu(y)\,.
    $$
    We have $B(\pc(\fp_s), 16D^s)) \subset B(x, 32D^s)$, by \eqref{eq-vol-sp-cube} and the triangle inequality, since $x \in \scI(\fp)$. Combining this with the doubling property \eqref{doublingx}, we obtain
    $$
        \mu(B(\pc(\fp_s), 16D^s)) \le 2^{5a} \mu(B(x, D^s))\,.
    $$
    Since $a \ge 4$, it follows that \eqref{eq-term-A} is bounded by
    $$
        2^{104a^3} \sum_{s\in\sigma(x)}2^{s - \overline{\sigma}(\fu, x)} \frac{1}{\mu(B(\pc(\fp_s),16D^s))}\int_{B(\pc(\fp_s),16D^s)}P_{\mathcal{J}(\fT(\fu))}|f(y)|\,\mathrm{d}\mu(y)\,.
    $$
    Since $L \in \mathcal{L}(\fT(\fu))$, we have $s(L) \le \ps(\fp)$ for all $\fp \in \fT(\fu)$. Since $x\in L \cap \scI(\fp_s)$, it follows by \eqref{dyadicproperty} that $L \subset \scI(\fp_s)$, in particular $x' \in \scI(\fp_s) \subset B(\pc(\fp_s), 16D^s)$. Thus
    $$
        \le 2^{104a^3} \sum_{s\in\sigma(x)}2^{s - \overline{\sigma}(\fu, x)} M_{\mathcal{B}, 1}P_{\mathcal{J}(\fT(\fu))}|f|(x')
    $$
    $$
        \le 2^{105a^3} M_{\mathcal{B}, 1}P_{\mathcal{J}(\fT(\fu))}|f|(x')\,.
    $$
    This completes the estimate for term \eqref{eq-term-A}.
\end{proof}

\begin{lemma}[second tree pointwise]
    \label{second-tree-pointwise}
    For all $\fu \in \fU$, all $L \in \mathcal{L}(\fT(\fu))$, all $x, x' \in L$ and all bounded $f$ with bounded support, we have
    $$
         \Bigg| \sum_{s \in \sigma(\fu, x)} \int K_s(x,y) P_{\mathcal{J}(\fT(\fu))} f(y) \, \mathrm{d}\mu(y) \Bigg| \le T_{\mathcal{N}} P_{\mathcal{J}(\fT(\fu))} f(x')\,.
    $$
\end{lemma}

\begin{proof}
    Let $s = \underline{\sigma}(\fu, x)$. By definition, there exists a tile $\fp \in \fT(\fu)$ with $\ps(\fp) = s$ and $x \in E(\fp)$. Then $x \in \scI(\fp) \cap L$. By \eqref{dyadicproperty} and the definition of $\mathcal{L}(\fT(\fu))$, it follows that $L \subset \scI(\fp)$, in particular $x' \in \scI(\fp)$, so $x \in I_s(x')$.
    The lemma now follows from the definition of $T_{\mathcal{N}}$.
\end{proof}

\begin{lemma}[third tree pointwise]
    \label{third-tree-pointwise}
    For all $\fu \in \fU$, all $L \in \mathcal{L}(\fT(\fu))$, all $x, x' \in L$ and all bounded $f$ with bounded support, we have
    \begin{equation*}
        \Bigg| \sum_{s \in \sigma(\fu, x)} \int K_s(x,y) (f(y) - P_{\mathcal{J}(\fT(\fu))} f(y)) \, \mathrm{d}\mu(y) \Bigg|
    \end{equation*}
    \begin{equation*}
          \le 2^{151a^3} S_{1,\fu} P_{\mathcal{J}(\fT(\fu))}|f|(x')\,.
    \end{equation*}
\end{lemma}

\begin{proof}
    We have for $J \in \mathcal{J}(\fT(\fu))$:
    $$
        \int_J K_{s}(x,y)(1 - P_{\mathcal{J}(\fT(\fu))})f(y) \, \mathrm{d}\mu(y)
    $$
    \begin{equation}
    \label{eq-canc-comp}
        = \int_J \frac{1}{\mu(J)} \int_J K_s(x,y) - K_s(x,z) \, \mathrm{d}\mu(z) \,f(y) \, \mathrm{d}\mu(y)\,.
    \end{equation}
    By \eqref{eq-Ks-smooth} and \eqref{eq-vol-sp-cube}, we have for $y, z \in J$
    $$
        |K_s(x,y) - K_s(x,z)| \le \frac{2^{150a^3}}{\mu(B(x, D^s))} \left(\frac{8 D^{s(J)}}{D^s}\right)^{1/a}\,.
    $$
    Suppose that $s \in \sigma(\fu, x)$.
    If $K_s(x,y) \ne 0$ for some $y \in J \in \mathcal{J}(\fT(\fu))$ then, by \eqref{supp-Ks}, $y \in B(x, 0.5 D^s) \cap J \ne \emptyset$. Let $\fp \in \fT(\fu)$ with $\ps(\fp) = s$ and $x \in E(\fp)$. Then $B(\pc(\fp_s), 4.5D^s) \cap J \ne \emptyset$ by the triangle inequality. If $s(J) \ge s$ and $s(J) > -S$, then it follows from the triangle inequality, \eqref{eq-vol-sp-cube} and \eqref{defineD} that $\scI(\fp) \subset B(c(J), 100 D^{s(J)+1})$, contradicting $J \in \mathcal{J}(\mathfrak{T}(\fu))$. Thus $s(J) \le s - 1$ or $s(J) = -S$. If $s(J) = -S$ and $s(J) > s - 1$, then $s = -S$. So in both cases, $s(J) \le s$. It then follows from the triangle inequality and \eqref{eq-vol-sp-cube} that $J \subset B(x, 16 D^s)$.

    Thus, we can estimate \eqref{eq-term-C} by
    $$
        2^{150a^3 + 3/a}\sum_{\fp\in \mathfrak{T}}\frac{\mathbf{1}_{E(\fp)}(x)}{\mu(B(x,D^{\ps(\fp)}))}\sum_{\substack{J\in \mathcal{J}(\fT(\fu))\\J\subset B(x, 16D^{\ps(\fp)})}} D^{(s(J) - \ps(\fp))/a} \int_J |f|\,.
    $$
    $$
        = 2^{150a^3 + 3/a}\sum_{I \in \mathcal{D}} \sum_{\substack{\fp\in \mathfrak{T}\\ \scI(\fp) = I}}\frac{\mathbf{1}_{E(\fp)}(x)}{\mu(B(x, D^{s(I)}))}\sum_{\substack{J\in \mathcal{J}(\fT(\fu))\\J\subset B(x, 16 D^{\ps(\fp)})}} D^{(s(J) - s(I))/a} \int_J |f|\,.
    $$
    By \eqref{eq-dis-freq-cover} and \eqref{defineep}, the sets $E(\fp)$ for tiles $\fp$ with $\scI(\fp) = I$ are pairwise disjoint.
    If $x \in E(\fp)$ then in particular $x \in \scI(\fp)$, so by \eqref{eq-vol-sp-cube} $B(c(I),16D^{s(I)}) \subset B(x, 32D^{s(I)})$. By the doubling property \eqref{doublingx}
    $$
        \mu(B(c(I), 16D^{s(I)})) \le 2^{5a} \mu(B(x, D^{s(I)}))\,.
    $$
    Since $a \ge 4$ we can continue our estimate with
    $$
        \le 2^{151a^3}\sum_{I \in \mathcal{D}} \frac{\mathbf{1}_{I}(x)}{\mu(B(c(I), 16D^{s(I)}))}\sum_{\substack{J\in \mathcal{J}(\fT(\fu))\\J\subset B(x, 16 D^{\ps(\fp)})}} D^{(s(J) - \ps(\fp))/a} \int_J |f|\,.
    $$
    Finally, it follows from the definition of $\mathcal{L}(\fT(\fu))$ that $x \in \scI(\fp)$ if and only if $x' \in \scI(\fp)$, thus this equals
    $$
         2^{151a^3} S_{1,\fu} P_{\mathcal{J}(\fT(\fu))}|f|(x')\,.
    $$
    This completes the proof.
\end{proof}

\section{An auxiliary \texorpdfstring{$L^2$}{L2} tree estimate}

In this subsection we prove the following estimate on $L^2$ for operators associated to trees.

\begin{lemma}[tree projection estimate]
    \label{tree-projection-estimate}
    \uses{dyadic-partitions,pointwise-tree-estimate,nontangential-operator-bound,boundary-operator-bound}
    Let $\fu \in \fU$.
    Then we have for all $f, g$ bounded with bounded support
    $$
        \Bigg|\int_X \sum_{\fp \in \fT(\fu)} \bar g(y) T_{\fp}f(y) \, \mathrm{d}\mu(y) \Bigg|
    $$
    \begin{equation}
        \label{eq-tree-est}
         \le 2^{104a^3}\|P_{\mathcal{J}(\fT(\fu))}|f|\|_{2}\|P_{\mathcal{L}(\fT(\fu))}|g|\|_{2}.
    \end{equation}
\end{lemma}

Below, we deduce \Cref{tree-projection-estimate} from \Cref{pointwise-tree-estimate} and the following estimates for the operators in \Cref{pointwise-tree-estimate}.

\begin{lemma}[nontangential operator bound]
    \label{nontangential-operator-bound}
  \uses{Hardy-Littlewood}
    For all bounded $f$ with bounded support
    $$
        \|T_{\mathcal{N}} f\|_2 \le 2^{103a^3} \|f\|_2\,.
    $$
\end{lemma}

\begin{lemma}[boundary operator bound]
    \label{boundary-operator-bound}
    \uses{Hardy-Littlewood,boundary-overlap}
    For all $\fu \in \fU$ and all bounded functions $f$ with bounded support
    \begin{equation}
        \label{eq-S-bound}
        \|S_{1,\fu}f\|_2 \le 2^{12a} \|f\|_2\,.
    \end{equation}
\end{lemma}

\begin{proof}[Proof of \Cref{tree-projection-estimate}]
    \proves{tree-projection-estimate}
    For each $L \in \mathcal{L}(\fT(\fu))$, choose a point $x'(L) \in L$ such that for all $y \in L$
    $$
        (M_{\mathcal{B},1}+S_{1,\fu})P_{\mathcal{J}(\fT(\fu))}|f|(x')+|T_{\mathcal{N}}P_{\mathcal{J}(\fT(\fu))}f(x')|
    $$
    \begin{equation}
        \label{eq-x'L-almost-inf}
        \le 2 ((M_{\mathcal{B},1}+S_{1,\fu})P_{\mathcal{J}(\fT(\fu))}|f|(y)+|T_{\mathcal{N}}P_{\mathcal{J}(\fT(\fu))}f(y)|)\,.
    \end{equation}
    This point exists since \eqref{eq-x'L-almost-inf} is non-negative for each $y$.
    Then we have by \Cref{pointwise-tree-estimate} for each $L \in \mathcal{L}(\fT(\fu))$
    $$
        \int_L |g(y)| \Bigg| \sum_{\fp \in \fT(\fu)} T_{\fp} f(y) \Bigg| \, \mathrm{d}\mu(y)
    $$
    $$
        \le 2^{151a^3} \int_L |g(y)| ((M_{\mathcal{B},1}+S_{1,\fu})P_{\mathcal{J}(\fT(\fu))}|f|(x')+|T_{\mathcal{N}}P_{\mathcal{J}(\fT(\fu))}f(x')| ) \, \mathrm{d}\mu(y)
    $$
    \begin{multline*}
        \le 2^{151a^3} \int_L |g(y)| \, \mathrm{d}\mu(y)\times\\
         \int_L 2((M_{\mathcal{B},1}+S_{1,\fu})P_{\mathcal{J}(\fT(\fu))}|f|(y)+|T_{\mathcal{N}}P_{\mathcal{J}(\fT(\fu))}f(y)| ) \, \mathrm{d}\mu(y)
    \end{multline*}
    \begin{multline*}
        = 2^{151a^3 + 1} \int_L P_{\mathcal{L}(\fT(\fu))}|g|(y)\times \\((M_{\mathcal{B},1}+S_{1,\fu})P_{\mathcal{J}(\fT(\fu))}|f|(y)+|T_{\mathcal{N}}P_{\mathcal{J}(\fT(\fu))}f(y)| ) \, \mathrm{d}\mu(y)\,.
    \end{multline*}
    By \eqref{definetp}, we have $T_{\fp} f = \mathbf{1}_{\scI(\fp)} T_{\fp} f$ for all $\fp \in \fP$, so
    $$
        \Bigg| \int \bar g(y) \sum_{\fp \in \fT(\fu)} T_{\fp} f(y) \, \mathrm{d}\mu(y) \Bigg| = \Bigg| \int_{\bigcup_{\fp \in \fT(\fu)} \scI(\fp)} \bar g(y) \sum_{\fp \in \fT(\fu)} T_{\fp} f(y) \, \mathrm{d}\mu(y) \Bigg|\,.
    $$
    Since $\mathcal{L}(\fT(\fu))$ partitions $\bigcup_{\fp \in \fT(\fu)} \scI(\fp)$ by \Cref{dyadic-partitions},
    we get from the triangle inequality
    $$
        \le \sum_{L \in \mathcal{L}(\fT(\fu))} \int_L |g(y)| \Bigg| \sum_{\fp \in \fT(\fu)} T_{\fp} f(y) \Bigg| \, \mathrm{d}\mu(y)
    $$
    which by the above computation is bounded by
    \begin{multline*}
        2^{151a^3 + 1} \sum_{L \in \mathcal{L}(\fT(\fu))} \int_L P_{\mathcal{L}(\fT(\fu))}|g|(y) \times \\((M_{\mathcal{B},1}+S_{1,\fu})P_{\mathcal{J}(\fT(\fu))}|f|(y)+|T_{\mathcal{N}}P_{\mathcal{J}(\fT(\fu))}f(y)| ) \, \mathrm{d}\mu(y)
    \end{multline*}
    \begin{multline*}
        = 2^{151a^3 + 1} \int_X P_{\mathcal{L}(\fT(\fu))}|g|(y)\times \\((M_{\mathcal{B},1}+S_{1,\fu})P_{\mathcal{J}(\fT(\fu))}|f|(y)+|T_{\mathcal{N}}P_{\mathcal{J}(\fT(\fu))}f(y)| ) \, \mathrm{d}\mu(y)\,.
    \end{multline*}
    Applying Cauchy-Schwarz and Minkowski's inequality, this is bounded by
    \begin{multline*}
        2^{151a^3 + 1} \|P_{\mathcal{L}(\fT(\fu))}|g|\|_2 \times \\(\|M_{\mathcal{B},1}P_{\mathcal{J}(\fT(\fu))}|f|\|_2 + \|S_{1,\fu}P_{\mathcal{J}(\fT(\fu))}|f|\|_2 + \|T_{\mathcal{N}}P_{\mathcal{J}(\fT(\fu))}f(y)|\|_2)\,.
    \end{multline*}
    By \Cref{Hardy-Littlewood}, \Cref{nontangential-operator-bound} and \Cref{boundary-operator-bound}, the second factor is at most
    $$
        (2^{2a+1} + 2^{12a})\|P_{\mathcal{J}(\fT(\fu))}|f|\|_2 + 2^{103a^3} \|P_{\mathcal{J}(\fT(\fu))}f\|_2\,.
    $$
    By the triangle inequality we have for all $x \in X$ that $|P_{\mathcal{J}(\fT(\fu))}f|(x) \le P_{\mathcal{J}(\fT(\fu))}|f|(x)$, thus we can further estimate the above by
    $$
        (2^{2a+1} + 2^{12a} + 2^{103a^3}) \|P_{\mathcal{J}(\fT(\fu))}|f|\|_2\,.
    $$
    This completes the proof since $a \ge 4$.
\end{proof}

Now we prove the two auxiliary lemmas. We begin with the nontangential maximal operator $T_{\mathcal{N}}$.

\begin{proof}[Proof of \Cref{nontangential-operator-bound}]
    \proves{nontangential-operator-bound}
    Fix $s_1, s_2$. By \eqref{eq-psisum} we have for all $x \in (0, \infty)$
    $$
        \sum_{s = s_1}^{s_2} \psi(D^{-s}x) = 1 - \sum_{s < s_1} \psi(D^{-s}x) - \sum_{s > s_1} \psi(D^{-s}x)\,.
    $$
    Since $\psi$ is supported in $[\frac{1}{4D}, \frac{1}{2}]$, the two sums on the right hand side are zero for all $x \in [\frac{1}{2}D^{s_1-1}, \frac{1}{4} D^{s_2 - 1}]$, hence
    $$
        x \in [\frac{1}{2}D^{s_1-1}, \frac{1}{4} D^{s_2}] \implies \sum_{s = s_1}^{s_2} \psi(D^{-s}x) = 1\,.
    $$
    Since $\psi$ is supported in $[\frac{1}{4D}, \frac{1}{2}]$, we further have
    $$
        x \notin [\frac{1}{4}D^{s_1 - 1}, \frac{1}{2}D^{s_2}] \implies \sum_{s = s_1}^{s_2} \psi(D^{-s}x) = 0\,.
    $$
    Finally, since $\psi \ge 0$ and $\sum_{s \in \mathbb{Z}} \psi(D^{-s}x) = 1$, we have for all $x$
    $$
        0 \le \sum_{s = s_1}^{s_2} \psi(D^{-s}x) \le 1\,.
    $$
    Let $x' \in I_{s_1}(x)$. By the triangle inequality and \eqref{eq-vol-sp-cube}, it holds that $\rho(x,x') \le 8D^{s_1}$. We have
    $$
        \Bigg|\sum_{s = s_1}^{s_2} \int K_s(x',y) f(y) \, \mathrm{d}\mu(y)\Bigg|
    $$
    $$
        = \Bigg|\int \sum_{s = s_1}^{s_2} \psi(D^{-s}\rho(x',y)) K(x',y) f(y) \, \mathrm{d}\mu(y)\Bigg|
    $$
    \begin{equation}
        \label{eq-sharp-trunc-term}
        \le \Bigg| \int_{8D^{s_1} \le \rho(x',y) \le \frac{1}{4}D^{s_2}} K(x',y) f(y) \, \mathrm{d}\mu(y) \Bigg|
    \end{equation}
    \begin{equation}
        \label{eq-lower-bound-term}
        + \int_{\frac{1}{4}D^{s_1-1} \le \rho(x',y) \le 8D^{s_1}} |K(x', y)| |f(y)| \, \mathrm{d}\mu(y)
    \end{equation}
    \begin{equation}
        \label{eq-upper-bound-term}
        + \int_{\frac{1}{4}D^{s_2} \le \rho(x',y) \le \frac{1}{2}D^{s_2}} |K(x', y)| |f(y)| \, \mathrm{d}\mu(y)\,.
    \end{equation}
    The first term \eqref{eq-sharp-trunc-term} is at most $T_* f(x)$.

    The other two terms will be estimated by the finitary maximal function from \Cref{Hardy-Littlewood}.
    For the second term \eqref{eq-lower-bound-term} we use \eqref{eqkernel-size} which implies that for all $y$ with $\rho(x', y) \ge \frac{1}{4}D^{s_1 - 1}$, we have
    $$
        |K(x', y)| \le \frac{2^{a^3}}{\mu(B(x', \frac{1}{4}D^{s_1 - 1}))}\,.
    $$
    Using $D=2^{100a^2}$
    and the doubling property \eqref{doublingx} $6 +100a^2$ times estimates
    the last display by
    \begin{equation}
    \label{pf-nontangential-operator-bound-imeq}
        \le \frac{2^{6a+101a^3}}{\mu(B(x', 16D^{s_1}))}\, .
    \end{equation}
    By the triangle inequality and \eqref{eq-vol-sp-cube}, we have
    $$
        B(x', 8D^{s_1}) \subset B(c(I_{s_1}(x)), 16D^{s(I_{s_1}(x))})\,.
    $$
    Combining this with \eqref{pf-nontangential-operator-bound-imeq}, we conclude that \eqref{eq-lower-bound-term} is at most
    $$
        2^{6a + 101a^3} M_{\mathcal{B},1} f(x)\,.
    $$

    For \eqref{eq-upper-bound-term} we argue similarly. We have for all $y$ with $\rho(x', y) \ge \frac{1}{4}D^{s_2}$
    $$
        |K(x', y)| \le \frac{2^{a^3}}{\mu(B(x', \frac{1}{4}D^{s_2}))}\,.
    $$
    Using the doubling property \eqref{doublingx} $6$ times estimates
    the last display by
    \begin{equation}
        \le \frac{2^{6a + a^3}}{\mu(B(x', 16 D^{s_2}))}\, .
    \end{equation}
    Note that by \eqref{dyadicproperty} we have $I_{s_1}(x) \subset I_{s_2}(x)$, in particular $x' \in I_{s_2}(x)$.
    By the triangle inequality and \eqref{eq-vol-sp-cube}, we have
    $$
        B(x', 8D^{s_2}) \subset B(c(I_{s_2}(x)), 16D^{s(I_{s_2}(x))})\,.
    $$
    Combining this, \eqref{eq-upper-bound-term} is at most
    $$
        2^{6a+a^3} M_{\mathcal{B},1} f(x)\,.
    $$

    Using $a \ge 4$, taking a supremum over all $x' \in I_{s_1}(x)$ and then a supremum over all $-S \le s_1 < s_2 \le S$, we obtain
    $$
        T_{\mathcal{N}} f(x) \le T_*f(x) + 2^{102a^3} M_{\mathcal{B},1} f(x)\,.
    $$
    The lemma now follows from assumption \eqref{nontanbound}, \Cref{Hardy-Littlewood}and $a \ge 4$.
\end{proof}

We need the following lemma to prepare the $L^2$-estimate for the auxiliary operators $S_{1, \fu}$.

\begin{lemma}[boundary overlap]
    \label{boundary-overlap}
    For every cube $I \in \mathcal{D}$, there exist at most $2^{8a}$ cubes $J \in \mathcal{D}$ with $s(J) = s(I)$ and $B(c(I), 16D^{s(I)}) \cap B(c(J), 16 D^{s(J)}) \ne \emptyset$.
\end{lemma}

\begin{proof}
    Suppose that $B(c(I), 16 D^{s(I)}) \cap B(c(J), 16 D^{s(J)}) \ne \emptyset$ and $s(I) = s(J)$. Then $B(c(I), 32 D^{s(I)}) \subset B(c(J), 64 D^{s(J)})$. Hence by the doubling property \eqref{doublingx}
    $$
        2^{8a}\mu(B(c(J), \frac{1}{4}D^{s(J)})) \ge \mu(B(c(I), 32 D^{s(I)}))\,,
    $$
    and by the triangle inequality, the ball $B(c(J), \frac{1}{4}D^{s(J)})$ is contained in $B(c(I), 32 D^{s(I)})$.

    If $\mathcal{C}$ is any finite collection of cubes $J \in \mathcal{D}$ satisfying $s(J) = s(I)$ and
    \begin{equation*}
        B(c(I), 16 D^{s(I)}) \cap B(c(J), 16 D^{s(J)}) \ne\emptyset\ ,
    \end{equation*} then it follows from \eqref{eq-vol-sp-cube} and pairwise disjointedness of cubes of the same scale \eqref{dyadicproperty} that the balls $B(c(J), \frac{1}{4} D^{s(J)})$ are pairwise disjoint. Hence
    \begin{align*}
        \mu(B(c(I), 32 D^{s(I)})) &\ge \sum_{J \in \mathcal{C}} \mu(B(c(J), \frac{1}{4}D^{s(J)}))\\
        &\ge |\mathcal{C}| 2^{-8a} \mu(B(c(I), 32 D^{s(I)}))\,.
    \end{align*}
    Since $\mu$ is doubling and $\mu \ne 0$, we have $\mu(B(c(I), 32D^{s(I)})) > 0$. The lemma follows after dividing by $2^{-8a}\mu(B(c(I), 32D^{s(I)}))$.
\end{proof}

Now we can bound the operators $S_{1, \fu}$.

\begin{proof}[Proof of \Cref{boundary-operator-bound}]
    \proves{boundary-operator-bound}
    Note that by definition, $S_{1,\fu}f$ is a finite sum of indicator functions of cubes $I \in \mathcal{D}$ for each locally integrable $f$, and hence is bounded, has bounded support and is integrable. Let $g$ be another function with the same three properties. Then $\bar g S_{1,\fu}f$ is integrable, and we have
    $$
        \Bigg|\int \bar g(y) S_{1,\fu}f(y) \, \mathrm{d}\mu(y)\Bigg|
    $$
    \begin{multline*}
        = \Bigg|\sum_{I\in\mathcal{D}} \frac{1}{\mu(B(c(I), 16 D^{s(I)}))} \int_I \bar g(y) \, \mathrm{d}\mu(y)\\
        \times \sum_{J\in \mathcal{J}\,:\,J\subseteq B(c(I), 16 D^{s(I)})} D^{(s(J)-s(I))/a}\int_J |f(y)| \,\mathrm{d}\mu(y)\Bigg|
    \end{multline*}
    \begin{multline*}
        \le \sum_{I\in\mathcal{D}} \frac{1}{\mu(B(c(I), 16D^{s(I)}))} \int_{B(c(I), 16D^{s(I)})} | g(y)| \, \mathrm{d}\mu(y)\\ \times \sum_{J\in \mathcal{J}\,:\,J\subseteq B(c(I), 16 D^{s(I)})} D^{(s(J)-s(I))/a}\int_J |f(y)| \,\mathrm{d}\mu(y)\,.
    \end{multline*}
    Changing the order of summation and using $J \subset B(c(I), 16 D^{s(I)})$ to bound the first average integral by $M_{\mathcal{B},1}|g|(y)$ for any $y \in J$, we obtain
    \begin{align}
    \label{eq-boundary-operator-bound-1}
        \le \sum_{J\in\mathcal{J}}\int_J|f(y)| M_{\mathcal{B},1}|g|(y) \, \mathrm{d}\mu(y) \sum_{I \in \mathcal{D} \, : \, J\subset B(c(I),16 D^{s(I)})} D^{(s(J)-s(I))/a}.
    \end{align}
    By \eqref{eq-vol-sp-cube} and \eqref{defineD} the condition $J \subset B(c(I), 16 D^{s(I)})$ implies $s(I) \ge s(J)$. By \Cref{boundary-overlap}, there are at most $2^{8a}$ cubes $I$ at each scale with $J \subset B(c(I), D^{s(I)})$.
    By convexity of $t \mapsto D^t$ and since $D \ge 2$, we have for all $-1 \le t \le 0$
    $$
        D^t \le 1 + t\left(1 - \frac{1}{D}\right) \le 1 + \frac{1}{2}t\,,
    $$
    so $(1 - D^{-1/a})^{-1} \le 2a \le 2^a$.
    Using this estimate for the sum of the geometric series,
    we conclude that \eqref{eq-boundary-operator-bound-1} is at most
    $$
        2^{9a} \sum_{J\in\mathcal{J}}\int_J|f(y)| M_{\mathcal{B},1}|g|(y) \, \mathrm{d}\mu(y)\,.
    $$
    The collection $\mathcal{J}$ is a partition of $X$, so this equals
    $$
        2^{9a} \int_X|f(y)| M_{\mathcal{B},1}|g|(y) \, \mathrm{d}\mu(y)\,.
    $$
    Using Cauchy-Schwarz and \Cref{Hardy-Littlewood}we conclude
    $$
        \left|\int \bar g S_{1,\fu}f \, \mathrm{d}\mu \right| \le 2^{11a+1} \|g\|_2\|f\|_2\,.
    $$
    The lemma now follows by choosing $g = S_{1,\fu}f$ and dividing on both sides by the finite $\|S_{1,\fu}f\|_2$.
\end{proof}

\section{The quantitative \texorpdfstring{$L^2$}{L2} tree estimate}

The main result of this subsection is the following quantitative bound for operators associated to trees, with decay in the densities $\dens_1$ and $\dens_2$.

\begin{lemma}[densities tree bound]
    \label{densities-tree-bound}
    \uses{tree-projection-estimate,local-dens1-tree-bound,local-dens2-tree-bound}
    Let $\fu \in \fU$. Then for all $f,g$ bounded with bounded support
    \begin{equation}
        \label{eq-cor-tree-est}
        \left|\int_X \bar g \sum_{\fp \in \fT(\fu)} T_{\fp }f \, \mathrm{d}\mu \right| \le 2^{155a^3} \dens_1(\fT(\fu))^{1/2} \|f\|_2\|g\|_2\,.
    \end{equation}
    If $|f| \le \mathbf{1}_F$, then we have
    \begin{equation}
        \label{eq-cor-tree-est-F}
        \left| \int_X \bar g \sum_{\fp \in \fT(\fu)} T_{\fp }f\, \mathrm{d}\mu \right| \le 2^{256a^3} \dens_1(\fT(\fu))^{1/2} \dens_2(\fT(\fu))^{1/2} \|f\|_2\|g\|_2\,.
    \end{equation}
\end{lemma}

Below, we deduce this lemma from \Cref{tree-projection-estimate} and the following two estimates controlling the size of support of the operator and its adjoint.

\begin{lemma}[local dens1 tree bound]
    \label{local-dens1-tree-bound}
    \uses{monotone-cube-metrics}
    Let $\fu \in \fU$ and $L \in \mathcal{L}(\fT(\fu))$. Then
    \begin{equation}
    \label{eq-1density-estimate-tree}
        \mu(L \cap \bigcup_{\fp \in \fT(\fu)} E(\fp)) \le 2^{101a^3} \dens_1(\fT(\fu)) \mu(L)\,.
    \end{equation}
\end{lemma}

\begin{lemma}[local dens2 tree bound]
    \label{local-dens2-tree-bound}
    Let $J \in \mathcal{J}(\fT(\fu))$ be such that there exist $\fq \in \fT(\fu)$ with $J \cap \scI(\fq) \ne \emptyset$. Then
    $$
        \mu(F \cap J) \le 2^{200a^3 + 19} \dens_2(\fT(\fu))\,.
    $$
\end{lemma}

\begin{proof}[Proof of \Cref{densities-tree-bound}]
    \proves{densities-tree-bound}
    Denote
    $$
        \mathcal{E}(\fu) = \bigcup_{\fp \in \fT(\fu)} E(\fp)\,.
    $$
    Then we have
    $$
        \left| \int_X \bar g \sum_{\fp \in \fT(\fu)} T_{\fp} f \, \mathrm{d}\mu \right| = \left| \int_X \overline{ g\mathbf{1}_{\mathcal{E}(\fu)}} \sum_{\fp \in \fT(\fu)} T_{\fp} f \, \mathrm{d}\mu \right|\,.
    $$
    By \Cref{tree-projection-estimate}, this is bounded by
    \begin{equation}
        \label{eq-both-factors-tree}
        \le 2^{104a^3}\|P_{\mathcal{J}(\fT(\fu))}|f|\|_2 \|P_{\mathcal{L}(\fT(\fu))} |\mathbf{1}_{\mathcal{E}(\fu)}g|\|_2\,.
    \end{equation}
    We bound the two factors separately.
    We have
    $$
        \|P_{\mathcal{L}(\fT(\fu))} |\mathbf{1}_{\mathcal{E}(\fu)}g|\|_2 = \left( \sum_{L \in \mathcal{L}(\fT(\fu))} \frac{1}{\mu(L)} \left(\int_{L \cap \mathcal{E}(\fu)} |g(y)| \, \mathrm{d}\mu(y)\right)^2 \right)^{1/2}\,.
    $$
    By Cauchy-Schwarz and \Cref{local-dens1-tree-bound} this is at most
    $$
        \le \left( \sum_{L \in \mathcal{L}(\fT(\fu))} 2^{101a^3} \dens_1(\fT(\fu)) \int_{L \cap \mathcal{E}(\fu)} |g(y)|^2 \, \mathrm{d}\mu(y) \right)^{1/2}\,.
    $$
    Since cubes $L \in \mathcal{L}(\fT(\fu))$ are pairwise disjoint by \Cref{dyadic-partitions}, this is
    \begin{equation}
        \label{eq-factor-L-tree}
         \le 2^{51 a^3} \dens_1(\fT(\fu))^{1/2} \|g\|_2\,.
    \end{equation}
    Similarly, we have
    \begin{equation}
        \label{eq-cor-tree-proof}
        \|P_{\mathcal{J}(\fT(\fu))}|f|\|_2 = \left( \sum_{J \in \mathcal{J}(\fT(\fu))} \frac{1}{\mu(J)} \left(\int_J |f(y)| \, \mathrm{d}\mu(y)\right)^2 \right)^{1/2}\,.
    \end{equation}
    By Cauchy-Schwarz, this is
    $$
        \le \left( \sum_{J \in \mathcal{J}(\fT(\fu))} \int_J |f(y)|^2 \, \mathrm{d}\mu(y) \right)^{1/2}\,.
    $$
    Since cubes in $\mathcal{J}(\fT(\fu))$ are pairwise disjoint by \Cref{dyadic-partitions}, this at most
    \begin{equation}
        \label{eq-factor-J-tree}
        \|f\|_2\,.
    \end{equation}
    Combining \eqref{eq-both-factors-tree}, \eqref{eq-factor-L-tree} and \eqref{eq-factor-J-tree} and using $a \ge 4$ gives \eqref{eq-cor-tree-est}.

    If $f \le \mathbf{1}_F$ then $f = f\mathbf{1}_F$, so
    $$
        \left( \sum_{J \in \mathcal{J}(\fT(\fu))} \int_J |f(y)|^2 \, \mathrm{d}\mu(y) \right)^{1/2} = \left( \sum_{J \in \mathcal{J}(\fT(\fu))} \int_{J \cap F} |f(y)|^2 \, \mathrm{d}\mu(y) \right)^{1/2}\,.
    $$
    We estimate as before, using now \Cref{local-dens2-tree-bound} and Cauchy-Schwarz, and obtain that this is
    $$
        \le 2^{100a^3 + 10} \dens_2(\fT(\fu))^{1/2} \|f\|_2\,.
    $$
    Combining this with \eqref{eq-both-factors-tree}, \eqref{eq-factor-L-tree} and $a \ge 4$ gives \eqref{eq-cor-tree-est-F}.
\end{proof}

Now we prove the two auxiliary estimates.

\begin{proof}[Proof of \Cref{local-dens1-tree-bound}]
    \proves{local-dens1-tree-bound}
    If the set on the right hand side is empty, then \eqref{eq-1density-estimate-tree} holds. If not, then there exists $\fp \in \fT(\fu)$ with $L \cap \scI(\fp) \ne \emptyset$.

    Suppose first that there exists such $\fp$ with $\ps(\fp) \le s(L)$. Then by \eqref{dyadicproperty} $\scI(\fp) \subset L$, which gives by the definition of $\mathcal{L}(\fT(\fu))$ that $s(L) = -S$ and hence $L = \scI(\fp)$. Let $\fq \in \fT(\fu)$ with $E(\fq) \cap L \ne \emptyset$. Since $s(L) = -S \le \ps(\fq)$ it follows from \eqref{dyadicproperty} that $\scI(\fp) = L \subset \scI(\fq)$. We have then by \Cref{monotone-cube-metrics}
    \begin{align*}
        d_{\fp}(\fcc(\fp), \fcc(\fq)) &\le d_{\fp}(\fcc(\fp), \fcc(\fu)) + d_{\fp}(\fcc(\fq), \fcc(\fu))\\
        &\le d_{\fp}(\fcc(\fp), \fcc(\fu)) + d_{\fq}(\fcc(\fq), \fcc(\fu))\,.
    \end{align*}
    Using that $\fp, \fq \in \fT(\fu)$ and \eqref{forest1}, this is at most $8$. Using again the triangle inequality and \Cref{monotone-cube-metrics}, we obtain that for each $q \in B_{\fq}(\fcc(\fq), 1)$
    $$
        d_{\fp}(\fcc(\fp), q) \le d_{\fp}(\fcc(\fp), \fcc(\fq)) + d_{\fq}(\fcc(\fq), q) \le 9\,.
    $$
    Thus $L \cap E(\fq) \subset E_2(9, \fp)$. We obtain
    $$
        \mu(L \cap \bigcup_{\fq \in \fT(\fu)} E(\fq)) \le \mu(E_2(9, \fp))\,.
    $$
    By the definition of $\dens_1$, this is bounded by
    $$
        9^a \dens_1(\fT(\fu)) \mu(\scI(\fp)) =9^a \dens_1(\fT(\fu)) \mu(L)\,.
    $$
    Since $a \ge 4$, \eqref{eq-1density-estimate-tree} follows in this case.

    Now suppose that for each $\fp \in \fT(\fu)$ with $L \cap E(\fp) \ne \emptyset$, we have $\ps(\fp) > s(L)$. Since there exists at least one such $\fp$, there exists in particular at least one cube $L'' \in \mathcal{D}$ with $L \subset L''$ and $s(L'') > s(L)$. By \eqref{coverdyadic}, there exists $L' \in \mathcal{D}$ with $L \subset L'$ and $s(L') = s(L) + 1$. By the definition of $\mathcal{L}(\fT(\fu))$ there exists a tile $\fp'' \in \fT(\fu)$ with $\scI(\fp'') \subset L'$. Let $\fp'$ be the unique tile such that $\scI(\fp') = L'$ and such that $\Omega(\fu) \cap \Omega(\fp') \ne \emptyset$. Since by \eqref{forest1} $\ps(\fp') = s(L') \le \ps(\fp) < \ps(\fu)$, we have by \eqref{dyadicproperty} and \eqref{eq-freq-dyadic} that $\Omega(\fu) \subset \Omega(\fp')$. Let $\fq \in \fT(\fu)$ with $L \cap E(\fq) \ne \emptyset$. As shown above, this implies $\ps(\fq) \ge s(L')$, so by \eqref{dyadicproperty} $L' \subset \scI(\fq)$. If $q \in B_{\fq}(\fcc(\fq), 1)$, then by a similar calculation as above, using the triangle inequality, \Cref{monotone-cube-metrics} and \eqref{forest1}, we obtain
    $$
        d_{\fp'}(\fcc(\fp'), q) \le d_{\fp'}(\fcc(\fp'), \fcc(\fq)) + d_{\fq}(\fcc(\fq), q) \le 6\,.
    $$
    Thus $L \cap E(\fq) \subset E_2(6, \fp')$. Since $\scI(\fp'') \subset \scI(\fp') \subset \scI(\fp)$ and $\fp'', \fp \in \fT(\fu)$, we have $\fp' \in \fP(\fT(\fu))$. We deduce using the definition \eqref{definedens1} of $\dens_1$
    $$
        \mu(L \cap \bigcup_{\fq \in \fT(\fu)} E(\fq)) \le \mu(E_2(6, \fq')) \le 6^a \dens_1(\fT(\fu)) \mu(L')\,.
    $$
    Using the doubling property \eqref{doublingx}, \eqref{eq-vol-sp-cube}, and $a \ge 4$ this is estimated by
    $$
        6^a 2^{100a^3 + 5}\dens_1(\fT(\fu)) \mu(L) \le 2^{101 a^3} \dens_1(\fT(\fu))\mu(L)\,.
    $$
    This completes the proof.
\end{proof}

\begin{proof}[Proof of \Cref{local-dens2-tree-bound}]
    \proves{local-dens2-tree-bound}
    Suppose first that there exists a tile $\fp \in \fT(\fu)$ with $\scI(\fp) \subset B(c(J), 100 D^{s(J) + 1})$. By the definition of $\mathcal{J}(\fT(\fu))$, this implies that $s(J) = -S$, and in particular $\ps(\fp) \ge s(J)$. Using the triangle inequality and \eqref{eq-vol-sp-cube} it follows that $J \subset B(\pc(\fp), 200 D^{\ps(\fp) + 1})$. From the doubling property \eqref{doublingx}, $D=2^{100a^2}$ and \eqref{eq-vol-sp-cube}, we obtain
    $$
        \mu(\scI(\fp)) \le 2^{100a^3 + 9} \mu(J)
    $$
    and hence
    $$
        \mu( B(\pc(\fp), 200 D^{\ps(\fp) + 1})) \le 2^{200a^3 +19} \mu(J)\,.
    $$
    With the definition \eqref{definedens2} of $\dens_2$ it follows that
    $$
        \mu(J \cap F) \le \mu( B(\pc(\fp), 200 D^{\ps(\fp) + 1}) \cap F)
    $$
    $$
        \le \dens_2(\fT(\fu)) \mu( B(\pc(\fp), 200 D^{\ps(\fp) + 1})) \le 2^{200a^3 +19} \dens_2(\fT(\fu))\mu(J)\,,
    $$
    completing the proof in this case.

    Now suppose that there does not exist a tile $\fp \in \fT(\fu)$ with $\scI(\fp) \subset B(c(J), 100 D^{s(J) + 1})$. If we had $\ps(\fq) \le s(J)$, then by \eqref{dyadicproperty} and \eqref{eq-vol-sp-cube} $\scI(\fq) \subset J \subset B(c(J), 100 D^{s(J) + 1})$, contradicting our assumption. Thus $\ps(\fq) > s(J)$. Then, by \eqref{coverdyadic} and \eqref{dyadicproperty}, there exists some cube $J' \in \mathcal{D}$ with $s(J') = s(J) + 1$ and $J \subset J'$. By definition of $\mathcal{J}(\fT(\fu))$ there exists some $\fp \in \fT(\fu)$ such that $\scI(\fp) \subset B(c(J'), 100 D^{s(J') + 1})$. From the doubling property \eqref{doublingx}, $D=2^{100a^2}$ and \eqref{eq-vol-sp-cube}, we obtain
    \begin{equation}
        \label{eq-measure-comparison-1}
        \mu(B(\pc(\fp), 4D^{\ps(\fp)})) \le 2^{4a} \mu(\scI(\fp)) \le 2^{200a^3 + 14} \mu(J)\,.
    \end{equation}
    If $J \subset B(\pc(\fp), 4 D^{\ps(\fp)})$, then we bound
    $$
        \mu(J \cap F) \le \mu(B(\pc(\fp), 4D^{\ps(\fp)}) \cap F)
    $$
    and use the definition \eqref{definedens2}
    $$
        \le \dens_2(\fT(\fu)) \mu(B(\pc(\fp), 4D^{\ps(\fp)})) \le 2^{200a^3 + 14} \mu(J)\,.
    $$
    From now on we assume $J \not \subset B(\pc(\fp), 4 D^{\ps(\fp)})$.
    Since
    \begin{equation*}
        \pc(\fp) \in \scI(\fp) \subset B(c(J'), 100 D^{s(J') + 1})\, ,
    \end{equation*}
    we have by \eqref{eq-vol-sp-cube} and the triangle inequality
    $$
        J \subset J' \subset B(c(J'), 4D^{s(J')}) \subset B(\pc(\fp), 104 D^{s(J') + 1})\,.
    $$
    In particular this implies $104 D^{s(J') + 1} > 4D^{\ps(\fp)}$. By the triangle inequality we also have
    $$
        B(\pc(\fp), 104 D^{s(J') + 1}) \subset B(c(J), 204 D^{s(J') + 1})\,,
    $$
    so from the doubling property \eqref{doublingx}
    $$
        \mu( B(\pc(\fp), 104 D^{s(J') + 1})) \le 2^{200a^3 + 10} \mu(J)\,.
    $$
    From here one completes the proof as in the other cases.
\end{proof}

\section{Almost orthogonality of separated trees}

The main result of this subsection is the almost orthogonality estimate for operators associated to distinct trees in a forest in \Cref{correlation-separated-trees} below. We will deduce it from Lemmas \ref{correlation-distant-tree-parts} and \ref{correlation-near-tree-parts}, which are proven in Subsections \ref{subsec-big-tiles} and \ref{subsec-rest-tiles}, respectively. Before stating it, we introduce some relevant notation.

The adjoint of the operator $T_{\fp}$ defined in \eqref{definetp} is given by
\begin{equation}
    \label{definetp*}
    T_{\fp}^* g(x) = \int_{E(\fp)} \overline{K_{\ps(\fp)}(y,x)} e(-\tQ(y)(x)+ \tQ(y)(y)) g(y) \, \mathrm{d}\mu(y)\,.
\end{equation}

\begin{lemma}[adjoint tile support]
    \label{adjoint-tile-support}
    For each $\fp \in \fP$, we have
    $$
        T_{\fp}^* g = \mathbf{1}_{B(\pc(\fp), 5D^{\ps(\fp)})} T_{\fp}^* \mathbf{1}_{\scI(\fp)} g\,.
    $$
    For each $\fu \in \fU$ and each $\fp \in \fT(\fu)$, we have
    $$
        T_{\fp}^* g = \mathbf{1}_{\scI(\fu)} T_{\fp}^* \mathbf{1}_{\scI(\fu)} g\,.
    $$
\end{lemma}

\begin{proof}
    By \eqref{forest1}, $E(\fp) \subset \scI(\fp) \subset \scI(\fu)$. Thus by \eqref{definetp*}
    $$
         T_{\fp}^* g(x) = T_{\fp}^* (\mathbf{1}_{\scI(\fp)} g)(x)
    $$
    $$
        = \int_{E(\fp)} \overline{K_{\ps(\fp)}(y,x)} e(-\tQ(y)(x) + \tQ(y)(y)) \mathbf{1}_{\scI(\fp)}(y) g(y) \, \mathrm{d}\mu(y)\,.
    $$
    If this integral is not $0$, then there exists $y \in \scI(\fp)$ such that $K_{\ps(\fp)}(y,x) \ne 0$. By \eqref{supp-Ks}, \eqref{eq-vol-sp-cube} and the triangle inequality, it follows that
    \begin{equation*}
        x \in B(\pc(\fp), 5 D^{\ps(\fp)})\, .
    \end{equation*}
    Thus
    $$
        T_{\fp}^* g(x) = \mathbf{1}_{B(\pc(\fp), 5D^{\ps(\fp)})}(x) T_{\fp}^* (\mathbf{1}_{\scI(\fp)} g)(x)\,.
    $$
    The second claimed equation follows now since $\scI(\fp) \subset \scI(\fu)$ and by \eqref{forest6} $B(\pc(\fp), 5D^{\ps(\fp)}) \subset \scI(\fu)$.
\end{proof}

\begin{lemma}[adjoint tree estimate]
    \label{adjoint-tree-estimate}
    \uses{densities-tree-bound}
    For all bounded $g$ with bounded support, we have that
    $$
        \left\| \sum_{\fp \in \fT(\fu)} T_{\fp}^* g\right\|_2 \le 2^{155a^3} \dens_1(\fT(\fu))^{1/2} \|g\|_2\,.
    $$
\end{lemma}

\begin{proof}
    By Cauchy-Schwarz and \Cref{densities-tree-bound}, we have for all bounded $f,g$ with bounded support that
    $$
        \left| \int_X \overline{\sum_{\fp\in \fT(\fu)} T_{\fp}^* g} f \,\mathrm{d}\mu \right| = \left| \int_X g \sum_{\fp \in \fT(\fu)} T_{\fp} f \,\mathrm{d}\mu \right|
    $$
    \begin{equation}
        \label{eq-adjoint-bound}
        \le 2^{155a^3} \dens_1(\fT(\fu))^{1/2} \|g\|_2 \|f\|_2\,.
    \end{equation}
    Let $f = \sum_{\fp \in \fT(\fu)} T_{\fp}^* g$. If $g$ is bounded and has bounded support, then the same is true for $f$. In particular $\|f\|_2 < \infty$. Dividing \eqref{eq-adjoint-bound} by $\|f\|_2$ completes the proof.
\end{proof}

We define
$$
    S_{2,\fu}g := \left|\sum_{\fp \in \fT(\fu)} T_{\fp}^*g \right| + M_{\mathcal{B},1}g + |g|\,.
$$
\begin{lemma}[adjoint tree control]
    \label{adjoint-tree-control}
    \uses{adjoint-tree-estimate}
    We have for all bounded $g$ with bounded support
    $$
        \|S_{2, \fu} g\|_2 \le 2^{156a^3} \|g\|_2\,.
    $$
\end{lemma}

\begin{proof}
    This follows immediately from Minkowski's inequality, \Cref{Hardy-Littlewood}and \Cref{adjoint-tree-estimate}, using that $a \ge 4$.
\end{proof}


Now we are ready to state the main result of this subsection.

\begin{lemma}[correlation separated trees]
    \label{correlation-separated-trees}
    \uses{correlation-distant-tree-parts,correlation-near-tree-parts}
    For any $\fu_1 \ne \fu_2 \in \fU$ and all bounded $g_1, g_2$ with bounded support, we have
    \begin{equation}
        \label{eq-lhs-sep-tree}
        \left| \int_X \sum_{\fp_1 \in \fT(\fu_1)} \sum_{\fp_2 \in \fT(\fu_2)} T^*_{\fp_1}g_1 \overline{T^*_{\fp_2}g_2 }\,\mathrm{d}\mu \right|
    \end{equation}
    \begin{equation}
        \label{eq-rhs-sep-tree}
        \le 2^{550a^3-3n} \prod_{j =1}^2 \| S_{2, \fu_j} g_j\|_{L^2(\scI(\fu_1) \cap \scI(\fu_2))}\,.
    \end{equation}
\end{lemma}

\begin{proof}[Proof of \Cref{correlation-separated-trees}]
    \proves{correlation-separated-trees}
    By \Cref{adjoint-tile-support} and \eqref{dyadicproperty}, the left hand side \eqref{eq-lhs-sep-tree} is $0$ unless $\scI(\fu_1) \subset \scI(\fu_2)$ or $\scI(\fu_2) \subset \scI(\fu_1)$. Without loss of generality we assume that $\scI(\fu_1) \subset \scI(\fu_2)$.

    Define
    \begin{equation}
        \label{def-Tree-S-set}
         \mathfrak{S} := \{\fp \in \fT(\fu_1) \cup \fT(\fu_2) \ : \ d_{\fp}(\fcc(\fu_1), \fcc(\fu_2)) \ge 2^{Zn/2}\,\}.
    \end{equation}
    \Cref{correlation-separated-trees} follows by combining the definition \eqref{defineZ} of $Z$ with the following two lemmas.
    \begin{lemma}[correlation distant tree parts]
        \label{correlation-distant-tree-parts}
        \uses{Holder-van-der-Corput,Lipschitz-partition-unity,Holder-correlation-tree,lower-oscillation-bound}
        We have for all $\fu_1 \ne \fu_2 \in \fU$ with $\scI(\fu_1) \subset \scI(\fu_2)$ and all bounded $g_1, g_2$ with bounded support
        \begin{equation}
            \label{eq-lhs-big-sep-tree}
            \left| \int_X \sum_{\fp_1 \in \fT(\fu_1)} \sum_{\fp_2 \in \fT(\fu_2) \cap \mathfrak{S}} T^*_{\fp_1}g_1 \overline{T^*_{\fp_2}g_2 }\,\mathrm{d}\mu \right|
        \end{equation}
        \begin{equation}
            \label{eq-rhs-big-sep-tree}
            \le 2^{541a^3} 2^{-Zn/(4a^2 + 2a^3)} \prod_{j =1}^2 \| S_{2, \fu_j} g_j\|_{L^2(\scI(\fu_1))}\,.
        \end{equation}
    \end{lemma}
    \begin{lemma}[correlation near tree parts]
        \label{correlation-near-tree-parts}
        \uses{tree-projection-estimate,dyadic-partition-2,bound-for-tree-projection}
        We have for all $\fu_1 \ne \fu_2 \in \fU$ with $\scI(\fu_1) \subset \scI(\fu_2)$ and all bounded $g_1, g_2$ with bounded support
        \begin{equation}
            \label{eq-lhs-small-sep-tree}
            \left| \int_X \sum_{\fp_1 \in \fT(\fu_1)} \sum_{\fp_2 \in \fT(\fu_2) \setminus \mathfrak{S}} T^*_{\fp_1}g_1 \overline{T^*_{\fp_2}g_2 }\,\mathrm{d}\mu \right|
        \end{equation}
        \begin{equation}
            \label{eq-rhs-small-sep-tree}
            \le 2^{222a^3} 2^{-Zn 2^{-10a}} \prod_{j =1}^2 \| S_{2, \fu_j} g_j\|_{L^2(\scI(\fu_1)}\,.
        \end{equation}
    \end{lemma}
\end{proof}

In the proofs of both lemmas, we will need the following observation.

\begin{lemma}[overlap implies distance]
    \label{overlap-implies-distance}
    Let $\fu_1 \ne \fu_2 \in \fU$ with $\scI(\fu_1) \subset \scI(\fu_2)$. If $\fp \in \fT(\fu_1) \cup \fT(\fu_2)$ with $\scI(\fp) \cap \scI(\fu_1) \ne \emptyset$, then $\fp \in \mathfrak{S}$. In particular, we have $\fT(\fu_1) \subset \mathfrak{S}$.
\end{lemma}

\begin{proof}
    Suppose first that $\fp \in \fT(\fu_1)$. Then $\scI(\fp) \subset \scI(\fu_1) \subset \scI(\fu_2)$, by \eqref{forest1}. Thus we have by the separation condition \eqref{forest5}, \eqref{eq-freq-comp-ball}, \eqref{forest1} and the triangle inequality
    \begin{align*}
        d_{\fp}(\fcc(\fu_1), \fcc(\fu_2)) &\ge d_{\fp}(\fcc(\fp), \fcc(\fu_2)) - d_{\fp}(\fcc(\fp), \fcc(\fu_1))\\
        &\ge 2^{Z(n+1)} - 4\\
        &\ge 2^{Zn}\,,
    \end{align*}
    using that $Z= 2^{12a}\ge 4$. Hence $\fp \in \mathfrak{S}$.

    Suppose now that $\fp \in \fT(\fu_2)$. If $\scI(\fp) \subset \scI(\fu_1)$, then the same argument as above with $\fu_1$ and $\fu_2$ swapped shows $\fp \in \mathfrak{S}$. If $\scI(\fp) \not \subset \scI(\fu_1)$ then, by \eqref{dyadicproperty}, $\scI(\fu_1) \subset \scI(\fp)$. Pick $\fp' \in \fT(\fu_1)$, we have $\scI(\fp') \subset \scI(\fu_1) \subset \scI(\fp)$. Hence, by \Cref{monotone-cube-metrics} and the first paragraph
    $$
        d_{\fp}(\fcc(\fu_1), \fcc(\fu_2)) \ge d_{\fp'}(\fcc(\fu_1), \fcc(\fu_2)) \ge 2^{Zn}\,,
    $$
    so $\fp \in \mathfrak{S}$.
\end{proof}

To simplify the notation, we will write at various places throughout the proof of Lemmas \ref{correlation-distant-tree-parts} and \ref{correlation-near-tree-parts} for a subset $\fC \subset \fP$
$$
    T_{\fC} f := \sum_{\fp \in \fC} T_{\fp} f\,, \quad\quad T_{\fC}^* g := \sum_{\fp\in\fC} T_{\fp}^* g\,.
$$

\section{Proof of the Tiles with large separation Lemma}
    \label{subsec-big-tiles}

\Cref{correlation-distant-tree-parts} follows from the van der Corput estimate in \Cref{Holder-van-der-Corput}. We apply this proposition in \Cref{subsubsec-van-der-corput}. To prepare this application, we first, in \Cref{subsubsec-pao}, construct a suitable partition of unity, and show then, in \Cref{subsubsec-holder-estimates} the H\"older estimates needed to apply \Cref{Holder-van-der-Corput}.

\subsection{A partition of unity}
\label{subsubsec-pao}
    Define
    $$
        \mathcal{J}' = \{J \in \mathcal{J}(\mathfrak{S}) \ : \ J \subset \scI(\fu_1)\}\,.
    $$

    \begin{lemma}[dyadic partition 1]
        \label{dyadic-partition-1}
        \uses{dyadic-partitions}
        We have that
        $$
            \scI(\fu_1) = \dot{\bigcup_{J \in \mathcal{J}'}} J\,.
        $$
    \end{lemma}

    \begin{proof}
        By \Cref{dyadic-partitions}, it remains only to show that each $J \in \mathcal{J}(\mathfrak{S})$ with $J \cap \scI(\fu_1) \ne \emptyset$ is in $\mathcal{J}'$. But if $J \notin \mathcal{J}'$, then by \eqref{dyadicproperty} $\scI(\fu_1) \subsetneq J$. Pick $\fp \in \fT(\fu_1) \subset \mathfrak{S}$. Then $\scI(\fp) \subsetneq J$. This contradicts the definition of $\mathcal{J}(\mathfrak{S})$.
    \end{proof}

    For cubes $J \in \mathcal{D}$, denote
    \begin{equation}
        \label{def-BJ}
        B(J) := B(c(J), 8D^{s(J)}).
    \end{equation}
    The main result of this subsubsection is the following.

    \begin{lemma}[Lipschitz partition unity]
        \label{Lipschitz-partition-unity}
           \uses{dyadic-partition-1,moderate-scale-change}
        There exists a family of functions $\chi_J$, $J \in \mathcal{J}'$ such that \begin{equation}
            \label{eq-pao-1}
            \mathbf{1}_{\scI(\fu_1)} = \sum_{J \in \mathcal{J}'} \chi_J\,,
        \end{equation}
        and for all $J \in \mathcal{J}'$ and all $y,y' \in \scI(\fu_1)$
        \begin{equation}
            \label{eq-pao-2}
            0 \leq \chi_J(y) \leq \mathbf{1}_{B(J)}(y)\,,
        \end{equation}
        \begin{equation}
            \label{eq-pao-3}
            |\chi_J(y) - \chi_J(y')| \le 2^{226a^3} \frac{\rho(y,y')}{D^{s(J)}}\,.
        \end{equation}
    \end{lemma}

    In the proof, we will use the following auxiliary lemma.

    \begin{lemma}[moderate scale change]
        \label{moderate-scale-change}

        If $J, J' \in \mathcal{J'}$ with
        $$
            B(J) \cap B(J') \ne \emptyset\,,
        $$
        then $|s(J) - s(J')| \le 1$.
    \end{lemma}

    \begin{proof}[Proof of \Cref{Lipschitz-partition-unity}]
        \proves{Lipschitz-partition-unity}
        For each cube $J \in \mathcal{J}$ let
        $$
            \tilde\chi_J(y) = \max\{0, 8 - D^{-s(J)} \rho(y, c(J))\}\,,
        $$
        and set
        $$
            a(y) = \sum_{J \in \mathcal{J}'} \tilde \chi_J(y)\,.
        $$
        We define
        \[
            \chi_J(y) := \frac{\tilde \chi_J(y)}{a(y)}\,.
        \]
        Then, due to \eqref{forest6} and \eqref{def-BJ}, the properties \eqref{eq-pao-1} and \eqref{eq-pao-2} are clearly true. Estimate \eqref{eq-pao-3} follows from \eqref{eq-pao-2} if $y, y' \notin B(J)$. Thus we can assume that $y \in B(J)$. We have by the triangle inequality
        $$
            |\chi_J(y) - \chi_J(y')| \le \frac{|\tilde \chi_J(y) - \tilde \chi_J(y')|}{a(y)} + \frac{\tilde \chi_J(y')|a(y) - a(y')|}{a(y)a(y')}
        $$
        Since $\tilde \chi_J(z) \ge 4$ for all $z \in B(c(J), 4) \supset J$ and by \Cref{dyadic-partition-1}, we have that $a(z) \ge 4$ for all $z \in \scI(\fu_1)$. So we can estimate the above further by
        $$
            \le 2^{-2}(|\tilde \chi_J(y) - \tilde \chi_J(y')| + \tilde \chi_J(y')|a(y) - a(y')|)\,.
        $$
        If $y' \notin B(\pc(\fp), 8D^{\ps(\fp)})$ then the second summand vanishes. Else, we can estimate the above, using also that $|\tilde \chi_J(y')| \le 8$, by
        $$
            \le 2^{-2} |\tilde \chi_J(y) - \tilde \chi_J(y')| + 2 \sum_{\substack{J' \in \mathcal{J}'\\ B(c(J'), 8D^{s(J')}) \cap B(c(J), 8D^{s(J)}) \ne \emptyset}}|\tilde \chi_{J'}(y) - \tilde \chi_{J'} (y')|\,.
        $$
        By the triangle inequality, we have for all dyadic cubes $I \in \mathcal{J}'$
        $$
            |\tilde \chi_I(y) - \tilde \chi_I(y')| \le \rho(y, y') D^{-s(I)}\,.
        $$
        Using this above, we obtain
        $$
            |\chi_J(y) - \chi_J(y')| \le \rho(y,y') \Big( \frac{1}{4} D^{-s(J)} + 2 \sum_{\substack{J' \in \mathcal{J}'\\ B(J') \cap B(J) \ne \emptyset}} D^{-s(J')}\Big)\,.
        $$
        By \Cref{moderate-scale-change}, this is at most
        $$
             \frac{\rho(y,y')}{D^{s(J)}} \left( \frac{1}{4} + 2D |\{J' \in \mathcal{J}' \ : \ B(J') \cap B(J) \ne \emptyset\}|\right)\,.
        $$
        By \eqref{eq-vol-sp-cube} and \Cref{dyadic-partition-1}, the balls $B(c(J'), \frac{1}{4} D^{s(J')})$ are pairwise disjoint, so by \Cref{moderate-scale-change} the balls $B(c(J'), \frac{1}{4} D^{s(J) - 1})$ are also disjoint. By the triangle inequality and \Cref{moderate-scale-change}, each such ball for $J'$ in the set of the last display is contained in
        $$
            B(c(J), 9 D^{s(J) + 1})\,.
        $$
        By the doubling property \eqref{doublingx}, we further have
        $$
            \mu\Big(B(c(J'), \frac{1}{4}D^{s(J')})\Big) \ge 2^{-200a^3 - 6} \mu(B(c(J), 9 D^{s(J) + 1}))
        $$
        for each such ball.
        Thus
        $$
            |\{J' \in \mathcal{J}' \ : \ B(J') \cap B(J) \ne \emptyset\}| \le 2^{200a^3 + 6}\,.
        $$
        Recalling that $D=2^{100a^2}$, we obtain
        $$\frac{1}{4} + 2D |\{J' \in \mathcal{J}' \ : \ B(J') \cap B(J) \ne \emptyset\}|\leq 2^{200a^3 + 100a^2+ 8}.$$
        Since $a\ge 4$, \eqref{eq-pao-3} follows.
    \end{proof}

    \begin{proof}[Proof of \Cref{moderate-scale-change}]
        \proves{moderate-scale-change}
        Suppose that $s(J') < s(J) - 1$. Then $s(J) > -S$. Thus, by the definition of $\mathcal{J}'$ there exists no $\fp \in \mathfrak{S}$ with
        \begin{equation}
            \label{eq-tile-incl-1}
            \scI(\fp) \subset B(c(J), 100D^{s(J) + 1})\,.
        \end{equation}
        Since $s(J') < s(J)$ and $J', J \subset \scI(\fu_1)$, we have $J' \subsetneq \scI(\fu_1)$.
        By \eqref{coverdyadic}, \eqref{dyadicproperty} there exists a cube $J'' \in \mathcal{D}$ with $J \subset J''$ and $s(J'') = s(J') + 1$. By the definition of $\mathcal{J}'$, there exists a tile $\fp \in \mathfrak{S}$ with
        \begin{equation}
            \label{tile-incl-2}
            \scI(\fp) \subset B(c(J''), 100 D^{s(J')+2})\,.
        \end{equation}
        But by the triangle inequality and \eqref{defineD}, we have
        $$
            B(c(J''), 100 D^{s(J')+2}) \subset B(c(J), 100D^{s(J) + 1})\,,
        $$
        which contradicts \eqref{eq-tile-incl-1} and \eqref{tile-incl-2}.
    \end{proof}


\subsection{H\"older estimates for adjoint tree operators}
\label{subsubsec-holder-estimates}
    Let $g_1, g_2:X \to \mathbb{C}$ be bounded with bounded support.
    Define for $J \in \mathcal{J}'$
    \begin{equation}
        \label{def-hj}
        h_J(y) := \chi_J(y)\cdot(e(\fcc(\fu_1)(y)) T_{\fT(\fu_1)}^* g_1(y)) \cdot \overline{(e(\fcc(\fu_2)(y)) T_{\fT(\fu_2) \cap \mathfrak{S}}^* g_2(y))}\,.
    \end{equation}
    The main result of this subsubsection is the following $\tau$-H\"older estimate for $h_J$, where $\tau = 1/a$.

    \begin{lemma}[Holder correlation tree]
        \label{Holder-correlation-tree}
        \uses{global-tree-control-2}
        We have for all $J \in \mathcal{J}'$ that
        \begin{equation}
            \label{hHolder}
            \|h_J\|_{C^{\tau}(B(c(J), 8D^{s(J)}))} \le 2^{535a^3} \prod_{j = 1,2} (\inf_{B(c(J), \frac{1}{8}D^{s(J)})} |T_{\fT(\fu_j)}^* g_j| + \inf_J M_{\mathcal{B}, 1} |g_j|)\,.
        \end{equation}
    \end{lemma}

    We will prove this lemma at the end of this section, after establishing several auxiliary results.

    We begin with the following H\"older continuity estimate for adjoints of operators associated to tiles.
    \begin{lemma}[Holder correlation tile]
        \label{Holder-correlation-tile}
        \uses{adjoint-tile-support}
        Let $\fu \in \fU$ and $\fp \in \fT(\fu)$. Then for all $y, y' \in X$ and all bounded $g$ with bounded support, we have
        $$
            |e(\fcc(\fu)(y)) T_{\fp}^* g(y) - e(\fcc(\fu)(y')) T_{\fp}^* g(y')|
        $$
        \begin{equation}
            \label{T*Holder2}
            \le \frac{2^{151a^3}}{\mu(B(\pc(\fp), 4D^{\ps(\fp)}))} \left(\frac{\rho(y, y')}{D^{\ps(\fp)}}\right)^{1/a} \int_{E(\fp)} |g(x)| \, \mathrm{d}\mu(x)\,.
        \end{equation}
    \end{lemma}

    \begin{proof}
        By \eqref{definetp*}, we have
        $$
            |e(\fcc(\fu)(y)) T_{\fp}^* g(y) - e(\fcc(\fu)(y')) T_{\fp}^* g(y')|
        $$
        \begin{multline*}
            =\bigg| \int_{E(\fp)} e(\tQ(x)(x) - \tQ(x)(y) + \fcc(\fu)(y)) \overline{K_{\ps(\fp)}(x, y)} g(x) \\
            - e(\tQ(x)(x) - \tQ(x)(y') + \fcc(\fu)(y')) \overline{K_{\ps(\fp)}(x, y')} g(x) \, \mathrm{d}\mu(x)\bigg|
        \end{multline*}
        \begin{multline*}
            \leq\int_{E(\fp)} |g(x)| |e(\tQ(x)(y) - \tQ(x)(y') - \fcc(\fu)(y) + \fcc(\fu)(y'))\overline{K_{\ps(\fp)}(x, y)}\\
            - \overline{K_{\ps(\fp)}(x, y')}| \, \mathrm{d}\mu(x)
        \end{multline*}
        \begin{multline}
            \leq\int_{E(\fp)} |g(x)| |e(-\tQ(x)(y) + \tQ(x)(y') + \fcc(\fu)(y) - \fcc(\fu)(y')) - 1| \\
            \times |\overline{K_{\ps(\fp)}(x, y)}|\, \mathrm{d}\mu(x) \label{T*Holder1b}
        \end{multline}
        \begin{equation}
            + \int_{E(\fp)} |g(x)| |\overline{K_{\ps(\fp)}(x, y)} - \overline{K_{\ps(\fp)}(x, y')} |\, \mathrm{d}\mu(x)\,.\label{T*Holder1}
        \end{equation}
        By the oscillation estimate \eqref{osccontrol}, we have
        $$
            |-\tQ(x)(y) + \tQ(x)(y') + \fcc(\fu)(y) - \fcc(\fu)(y')|
        $$
        \begin{equation}
            \label{eq-lem-tile-Holder-comp}
            \le d_{B(y, \rho(y,y'))}(\tQ(x), \fcc(\fu))\,.
        \end{equation}
        Suppose that $y, y' \in B(\pc(\fp), 5D^{\ps(\fp)})$, so that $\rho(y,y') \le 10D^{\ps(\fp)}$. Let $k \in \mathbb{Z}$ be such that $2^{ak}\rho(y,y') \le 10D^{\ps(\fp)}$ but $2^{a(k+1)} \rho(y,y') > 10D^{\ps(\fp)}$. In particular, $k \ge 0$. Then, using \eqref{firstdb}, we can bound \eqref{eq-lem-tile-Holder-comp} from above by
        $$
            2^{-k} d_{B(\pc(\fp), 10 D^{\ps(\fp)})}(\tQ(x), \fcc(\fu)) \le 2^{6a - k} d_{\fp}(\tQ(x), \fcc(\fu))\,.
        $$
        Since $x \in E(\fp)$ we have $\tQ(x) \in \Omega(\fp) \subset B_{\fp}(\fcc(\fp), 1)$, and since $\fp \in \fT(\fu)$ we have $\fcc(\fu) \in B_{\fp}(\fcc(\fp), 4)$, so this is estimated by
        $$
            \le 5 \cdot 2^{6a - k}\,.
        $$
        By definition of $k$, we have
        $$
            k \le \frac{1}{a} \log_2\left(\frac{10 D^{\ps(\fp)}}{\rho(y,y')}\right)\,,
        $$
        which gives
        \begin{equation}
            \label{eq-lem-Tile-holder-im1}
             |-\tQ(x)(y) + \tQ(x)(y') + \fcc(\fu)(y) - \fcc(\fu)(y')| \le 5 \cdot 2^{6a} \left(\frac{\rho(y,y')}{10 D^{\ps(\fp)}}\right)^{1/a}\,.
        \end{equation}
        For all $x \in \scI(\fp)$, we have by \eqref{doublingx} that
        $$
            \mu(B(x, D^{\ps(\fp)})) \ge 2^{-3a} \mu(B(\pc(\fp), 4D^{\ps(\fp)}))\,.
        $$
        Combining the above with \eqref{eq-Ks-size}, \eqref{eq-Ks-smooth} and \eqref{eq-lem-Tile-holder-im1},
        we obtain
        $$
            \eqref{T*Holder1b}+\eqref{T*Holder1} \le \frac{2^{3a}}{\mu(B(\pc(\fp), 4D^{\ps(\fp)}))} \int_{E(\fp)}|g(x)| \, \mathrm{d}\mu(x) \times
        $$
        $$
            (2^{102a^3} \cdot 5 \cdot 2^{6a} \left(\frac{\rho(y,y')}{ D^{\ps(\fp)}}\right)^{1/a} + 2^{150a^3} \left(\frac{\rho(y,y')}{D^{\ps(\fp)}}\right)^{1/a})
        $$
        Since $\rho(y,y') \le 10 D^{\ps(\fp)}$, we conclude
        $$
            \eqref{T*Holder1b}+\eqref{T*Holder1} \le \frac{2^{151a^3}}{\mu(B(\pc(\fp), 4D^{\ps(\fp)}))} \left(\frac{\rho(y,y')}{D^{\ps(\fp)}}\right)^{1/a} \int_{E(\fp)}|g(x)| \, \mathrm{d}\mu(x)\,.
        $$

        Next, if $y,y' \notin B(\pc(\fp), 5D^{\ps(\fp)})$, then $T_{\fp}^*g(y) = T_{\fp}^*g(y') = 0$, by \Cref{adjoint-tile-support}. Then \eqref{T*Holder2} holds.

        Finally, if $y \in B(\pc(\fp), 5D^{\ps(\fp)})$ and $y' \notin B(\pc(\fp), 5D^{\ps(\fp)})$, then
        $$
            |e(\fcc(\fu)(y)) T_{\fp}^* g(y) - e(\fcc(\fu)(y')) T_{\fp}^* g(y')| = |T_{\fp}^* g(y)|
        $$
        $$
            \le \int_{E(\fp)} |K_{\ps(\fp)}(x,y)| |g(x)| \, \mathrm{d}\mu(x)\,.
        $$
        By the same argument used to prove \eqref{eq-Ks-aux}, this is bounded by
        \begin{equation}
            \label{eq-lem-Tile-holder-im2}
            \le 2^{102a^3} \int_{E(\fp)} \frac{1}{\mu(B(x, D^s))} \psi(D^{-s} \rho(x,y)) |g(x)| \, \mathrm{d}\mu(x)\,.
        \end{equation}
        It follows from the definition of $\psi$ that
        $$
            \psi(x) \le \max\{0, (2 - 4x)^{1/a}\}\,.
        $$
        Now for all $x\in E(\fp)$, it follows by the triangle inequality and \eqref{eq-vol-sp-cube} that
        \begin{multline*}
        2 - 4D^{-\ps(\fp)}\rho(x,y)\leq 2 - 4D^{-\ps(\fp)}\rho(y, \pc(\fp)) + 4 D^{-\ps(\fp)}\rho(x, \pc(\fp))\\\leq 18 - 4 D^{-\ps(\fp)} \rho(y, \pc(\fp)) \leq 4 D^{-\ps(\fp)}\rho(y,y') - 2 .
        \end{multline*}
        Combining the above with the previous estimate on $\psi$, we get
        $$
            \psi(D^{-\ps(\fp)}\rho(x,y)) \le 4 (D^{-\ps(\fp)}\rho(y,y'))^{1/a}.
        $$
        Further, we obtain from the doubling property \eqref{doublingx} and \eqref{eq-vol-sp-cube} that
        $$
            \mu(B(x, D^s)) \ge 2^{-2a} \mu(\pc(\fp), 4D^s)\,.
        $$
        Plugging this into \eqref{eq-lem-Tile-holder-im2} and using $a \ge 4$, we get
        $$
            |T_{\fp}^* g(y)| \le \frac{2^{103a^3}}{\mu(B(\pc(\fp), 4D^{\ps(\fp)}))} \left(\frac{\rho(y,y')}{D^{\ps(\fp)}}\right)^{1/a} \int_{E(\fp)} |g(x)| \, \mathrm{d}\mu(y)\,,
        $$
        which completes the proof of the lemma.
    \end{proof}

    Recall that
    \begin{equation*}
        B(J) := B(c(J), 8D^{s(J)}).
    \end{equation*}
    We also denote
    \begin{equation*}
     B^\circ{}(J) := B(c(J), \frac{1}{8}D^{s(J)})\, .
    \end{equation*}


    \begin{lemma}[limited scale impact]
        \label{limited-scale-impact}
        \uses{overlap-implies-distance}
        Let $\fp \in \fT_2 \setminus \mathfrak{S}$, $J \in \mathcal{J}'$ and suppose that
        $$
           B(\scI(\fp)) \cap B^\circ(J) \ne \emptyset\,.
        $$
        Then
        $$
            s(J) \le \ps(\fp) \le s(J) + 10a^2 + 2\,.
        $$
    \end{lemma}

    \begin{proof}
        For the first estimate, assume that $\ps(\fp) < s(J)$, then in particular $\ps(\fp) \le \ps(\fu_1)$. Since $\fp \notin \mathfrak{S}$, we have by \Cref{overlap-implies-distance} that $\scI(\fp) \cap \scI(\fu_1) = \emptyset$.
        Since $B\Big(c(J), \frac{1}{4} D^{s(J)}\Big) \subset \scI(J) \subset \scI(\fu_1)$, this implies
        $$
            \rho(c(J), \pc(\fp)) \ge \frac{1}{4}D^{s(J)}\,.
        $$
        On the other hand
        $$
            \rho(c(J), \pc(\fp)) \le \frac{1}{8} D^{s(J)} + 8 D^{\ps(\fp)}\,,
        $$
        by our assumption. Thus $D^{\ps(\fp)} \ge 64 D^{s(J)}$, which contradicts \eqref{defineD} and $a \ge 4$.

        For the second estimate, assume that $\ps(\fp) > s(J) +10a^2 + 2$. Since $J \in \mathcal{J}'$, we have $J \subsetneq \scI(\fu_1)$. Thus there exists $J' \in \mathcal{D}$ with $J \subset J'$ and $s(J') = s(J) + 1$, by \eqref{coverdyadic} and \eqref{dyadicproperty}. By definition of $\mathcal{J}'$, there exists some $\fp' \in \mathfrak{S}$ such that $\scI(\fp') \subset B(c(J'), 100 D^{s(J) + 2})$. On the other hand, since $B(\scI(\fp)) \cap B^\circ(J) \ne \emptyset$, by the triangle inequality it holds that
        $$
            B(c(J'), 100 D^{s(J) + 10a^2 + 2}) \subset B(\pc(\fp), 10 D^{\ps(\fp)})\,.
        $$
        Using the definition of $\mathfrak{S}$, we have
        $$
            2^{Zn/2} \le d_{\fp'}(\fcc(\fu_1), \fcc(\fu_2)) \le d_{B(c(J'), 100 D^{s(J) + 2})}(\fcc(\fu_1), \fcc(\fu_2))\,.
        $$
        By \eqref{seconddb}, this is
        $$
            \le 2^{-10a} d_{B(c(J'), 100 D^{s(J) + 10a^2 + 2})}(\fcc(\fu_1), \fcc(\fu_2))
        $$
        $$
            \le 2^{-10a} d_{B(\pc(\fp), 10 D^{\ps(\fp)})}(\fcc(\fu_1), \fcc(\fu_2))\,,
        $$
        and by \eqref{firstdb} and the definition of $\mathfrak{S}$
        $$
            \le 2^{-4a} d_{\fp}(\fcc(\fu_1), \fcc(\fu_2)) \le 2^{-4a} 2^{Zn/2}\,.
        $$
        This is a contradiction, the second estimate follows.
    \end{proof}


    \begin{lemma}[local tree control]
        \label{local-tree-control}
        \uses{limited-scale-impact}
        For all $J \in \mathcal{J}'$ and all bounded $g$ with bounded support
        $$
            \sup_{B^\circ{}(J)} |T_{\mathfrak{T}(\mathfrak{u}_2)\setminus\mathfrak{S}}^* g| \le 2^{104a^3} \inf_J M_{\mathcal{B},1}|g|
        $$
    \end{lemma}

    \begin{proof}
        By the triangle inequality and since $T_{\fp}^* g = \mathbf{1}_{B(\pc(\fp), 5D^{\ps(\fp)})} T_{\fp}^* g$, we have
        $$
            \sup_{B^\circ{}(J)} |T_{\fT(\fu_2) \setminus\mathfrak{S}}^* g|
            \leq \sup_{B^\circ{}(J)} \sum_{\substack{\fp \in \fT(\fu_2) \setminus \mathfrak{S}\\ B(\scI(\fp)) \cap B^\circ(J) \ne \emptyset}} |T_{\fp}^*g|\,.
        $$
        By \Cref{limited-scale-impact}, this is at most
        \begin{equation}
            \label{eq-sep-tree-aux-3}
            \sum_{s = s(J)}^{s(J) + 10a^2 + 2} \sum_{\substack{\fp \in \fP, \ps(\fp) = s\\ B(\scI(\fp)) \cap B^\circ(J) \ne \emptyset}} \sup_{B^\circ{}(J)} |T_{\fp}^* g|\,.
        \end{equation}
        If $x \in E(\fp)$ and $B(\scI(\fp)) \cap B^\circ(J) \ne \emptyset$, then
        $$
            B(c(J), 16D^{\ps(\fp)}) \subset B(x, 32 D^{\ps(\fp)})\,,
        $$
        by \eqref{eq-vol-sp-cube} and the triangle inequality. Using the doubling property \eqref{doublingx}, it follows that
        $$
            \mu(B(x, D^{\ps(\fp)})) \ge 2^{-5a} \mu(B(c(J), 16D^{\ps(\fp)}))\,.
        $$
        Using \eqref{definetp*}, \eqref{eq-Ks-size} and that $a \ge 4$, we bound \eqref{eq-sep-tree-aux-3} by
        $$
            2^{103a^3}\sum_{s = s(J)}^{s(J) + 10a^2 + 2} \sum_{\substack{\fp \in \fP, \ps(\fp) = s\\B(\scI(\fp)) \cap B^\circ(J) \ne \emptyset}} \frac{1}{\mu(B(c(J), 16 D^s)} \int_{E(\fp)} |g| \, \mathrm{d}\mu\,.
        $$
        For each $I \in \mathcal{D}$, the sets $E(\fp)$ for $\fp \in \fP$ with $\scI(\fp) = I$ are pairwise disjoint by \eqref{defineep} and \eqref{eq-dis-freq-cover}. Further, if $B(\scI(\fp)) \cap B^\circ(J) \ne \emptyset$ and $\ps(\fp) \ge s(J)$, then $E(\fp) \subset B(c(J), 32 D^{\ps(\fp)})$. Thus the last display is bounded by
        $$
            2^{103a^3}\sum_{s = s(J)}^{s(J) + 10a^2 + 2} \frac{1}{\mu(B(c(J), 32 D^s))} \int_{B(c(J), 16 D^s)} |g| \, \mathrm{d}\mu\,.
        $$
        $$
            \le \inf_{x' \in J} 2^{103a^3}(10a^2 + 3) M_{\mathcal{B}, 1} |g|\,.
        $$
        The lemma follows since for $a \ge 4$ it holds that $10a^2 + 3 \le 2^{a^3}$.
    \end{proof}

    \begin{lemma}[scales impacting interval]
        \label{scales-impacting-interval}
        \uses{overlap-implies-distance}
        Let $\fC = \fT(\fu_1)$ or $\fC = \fT(\fu_2) \cap \mathfrak{S}$. Then for each $J \in \mathcal{J}'$ and $\fp \in \fC$ with $B(\scI(\fp)) \cap B(J) \neq \emptyset$, we have $\ps(\fp) \ge s(J)$.
    \end{lemma}

    \begin{proof}
        By \Cref{overlap-implies-distance}, we have that in both cases, $\fC \subset \mathfrak{S}$. If $\fp \in \fC$ with $B(\scI(\fp)) \cap B(J) \neq \emptyset$ and $\ps(\fp) < s(J)$, then $\scI(\fp) \subset B(c(J), 100 D^{s(J) + 1})$. Since $\fp \in \mathfrak{S}$, it follows from the definition of $\mathcal{J}'$ that $s(J) = -S$, which contradicts $\ps(\fp) < s(J)$.
    \end{proof}

    \begin{lemma}[global tree control 1]
        \label{global-tree-control-1}
        \uses{Holder-correlation-tile,scales-impacting-interval}
        Let $\fC = \fT(\fu_1)$ or $\fC = \fT(\fu_2) \cap \mathfrak{S}$. Then for each $J \in \mathcal{J}'$ and all bounded $g$ with bounded support, we have
        \begin{align}
            \label{TreeUB}
            \sup_{B(J)} |T_{\fC}^*g| \leq \inf_{B^\circ{}(J)} |T^*_{\fC} g| + 2^{154a^3} \inf_{J} M_{\mathcal{B}, 1} |g|
        \end{align}
        and for all $y,y' \in B(J)$
        $$
            |e(\fcc(\fu)(y)) T_{\fC}^* g(y) - e(\fcc(\fu)(y')) T_{\fC}^* g(y')|
        $$
        \begin{equation}
            \label{TreeHolder}
             \le 2^{153a^3} \left(\frac{\rho(y,y')}{D^{s(J)}}\right)^{1/a} \inf_J M_{\mathcal{B},1} |g|\,.
        \end{equation}
    \end{lemma}

    \begin{proof}
        Note that \eqref{TreeUB} follows from \eqref{TreeHolder}, since for $y'\in B^\circ{}(J)$, by the triangle inequality,
        $$\left(\frac{\rho(y,y')}{D^{s(J)}}\right)^{1/a}\le \Big(8 + \frac{1}8\Big)^{1/a}\le 2^{a^3}.$$

        By the triangle inequality, \Cref{adjoint-tile-support} and \Cref{Holder-correlation-tile}, we have for all $y, y' \in B(J)$
        \begin{equation}
            \label{eq-C-Lip}
            |e(\fcc(\fu)(y)) T_{\fC}^* g(y) - e(\fcc(\fu)(y')) T_{\fC}^* g(y')|
        \end{equation}
        $$
            \leq \sum_{\substack{\fp \in \fC\\ B(\scI(\fp)) \cap B(J) \neq \emptyset}} |e(\fcc(\fu)(y)) T_{\fp}^* g(y) - e(\fcc(\fu)(y')) T_{\fp}^* g(y')|
        $$
        $$
            \le 2^{151a^3}\rho(y,y')^{1/a} \sum_{\substack{\fp \in \fC\\ B(\scI(\fp)) \cap B(J) \neq \emptyset}} \frac{D^{- \ps(\fp)/a}}{\mu(B(\pc(\fp), 4D^{\ps(\fp)}))} \int_{E(\fp)} |g| \, \mathrm{d}\mu\,.
        $$
        By \Cref{scales-impacting-interval}, we have $\ps(\fp) \ge s(J)$ for all $\fp$ occurring in the sum. Further, for each $s \ge s(J)$, the sets $E(\fp)$ for $\fp \in \fP$ with $\ps(\fp) = s$ are pairwise disjoint by \eqref{defineep} and \eqref{eq-dis-freq-cover}, and contained in $B(c(J), 32D^{s})$ by \eqref{eq-vol-sp-cube} and the triangle inequality. Using also the doubling estimate \eqref{doublingx}, we obtain that the expression in the last display can be estimated by
        $$
            2^{151a^3}\rho(y,y')^{1/a} \sum_{S \ge s \ge s(J)} D^{-s/a} \frac{2^{3a}}{\mu(B(c(J), 32D^{s}))} \int_{B(c(J), 32D^{s})} |g| \, \mathrm{d}\mu
        $$
        $$
            \le 2^{152a^3} \left(\frac{\rho(y,y')}{D^s}\right)^{1/a} \sum_{S \ge s \ge s(J)} D^{(s(J) - s)/a} \inf_J M_{\mathcal{B},1} |g|\,.
        $$
        By convexity of $t \mapsto D^t$ and since $D \ge 2$, we have for all $-1 \le t \le 0$
        $$
            D^t \le 1 + t(1 - \frac{1}{D}) \le 1 + \frac{1}{2}t\,.
        $$
        Since $-1 \le -1/a <0$, it follows that
        $$
            \sum_{S \ge s \ge s(J)} D^{(s(J) - s)/a} \le \frac{1}{1 - D^{-1/a}} \le 2a \le 2^a\,.
        $$
        Estimate \eqref{TreeHolder}, and therefore the lemma, follow.
    \end{proof}

    \begin{lemma}[global tree control 2]
        \label{global-tree-control-2}
        \uses{global-tree-control-1}
        We have for all $J \in \mathcal{J}'$ and all bounded $g$ with bounded support
        $$
            \sup_{B(J)} |T^*_{\mathfrak{T}_2 \cap \mathfrak{S}} g| \le \inf_{B^\circ{}(J)} |T^*_{\mathfrak{T}_2} g| + 2^{155a^3} \inf_{J} M_{\mathcal{B},1}|g|\,.
        $$
    \end{lemma}

    \begin{proof}
        By \Cref{global-tree-control-1}
        $$
            \sup_{B(J)} |T^*_{\fT(\fu_2) \cap \mathfrak{S}} g| \le \inf_{B^\circ{}(J)} |T_{\fT(\fu_2) \cap \mathfrak{S}}^* g| + 2^{154a^3} \inf_{J} M_{\mathcal{B}, 1} |g|
        $$
        $$
            \le \inf_{B^\circ{}(J)} |T_{\fT(\fu_2)}^* g| + \sup_{B^\circ{}(J)} |T_{\fT(\fu_2) \setminus \mathfrak{S}}^* g| + 2^{154a^3} \inf_{J} M_{\mathcal{B}, 1} |g|\,,
        $$
        and by \Cref{local-tree-control}
        $$
            \le \inf_{B^\circ{}(J)} |T_{\fT(\fu_2)}^* g| + (2^{104a^3} + 2^{154a^3}) \inf_{J} M_{\mathcal{B}, 1} |g|\,.
        $$
        This completes the proof.
    \end{proof}



    \begin{proof}[Proof of \Cref{Holder-correlation-tree}]
        \proves{Holder-correlation-tree}
        Let $P$ be the product on the right hand side of \eqref{hHolder}, and $h_J$ as defined in \eqref{def-hj}.

        By \eqref{eq-pao-2} and \Cref{adjoint-tile-support}, the function $h_J$ is supported in $B(J) \cap \scI(\fu_1)$.
        By \eqref{eq-pao-2} and \Cref{global-tree-control-1}, we have for all $y \in B(J)$:
        $$
            |h_J(y)| \le 2^{308a^3} P\,.
        $$
        We have by the triangle inequality
        \begin{align}
            &|h_J(y) - h_J(y')|\nonumber\\
            \label{eq-h-Lip-1}
            &\le |\chi_J(y) - \chi_J(y')| |T_{\fT(\fu_1)}^* g_1(y)| |T_{\fT(\fu_2) \cap \mathfrak{S}}^* g_2(y)|\\
            \label{eq-h-Lip-2}
            & + |\chi_J(y')| |T_{\fT(\fu_1)}^* g_1(y) - T_{\fT(\fu_1)}^* g_1(y')| |T_{\fT(\fu_2) \cap \mathfrak{S}}^* g_2(y)|\\
            \label{eq-h-Lip-3}
            & + |\chi_J(y')| |T_{\fT(\fu_1)}^* g_1(y')| |T_{\fT(\fu_2) \cap \mathfrak{S}}^* g_2(y) - T_{\fT(\fu_2) \cap \mathfrak{S}}^* g_2(y')|\,.
        \end{align}

        As $h_J$ is supported in $\scI(\fu_1)$, we can assume without loss of generality that $y' \in \scI(\fu_1)$.
        If $y \notin \scI(\fu_1)$, then \eqref{eq-h-Lip-1} vanishes. If $y \in \scI(\fu_1)$ then we have by \eqref{eq-pao-3}, \Cref{global-tree-control-1} and \Cref{global-tree-control-2}
        $$
            \eqref{eq-h-Lip-1} \le 2^{534a^3} \frac{\rho(y,y')}{D^{s(J)}} P\,,
        $$
        where $P$ denotes the product on the right hand side of \eqref{hHolder}.

        By \eqref{eq-pao-2}, \Cref{global-tree-control-1} and \Cref{global-tree-control-2}, we have
        $$
            \eqref{eq-h-Lip-2} \le 2^{310a^3} P\,.
         $$

        By \eqref{eq-pao-2}, and twice \Cref{global-tree-control-1}, we have
        $$
            \eqref{eq-h-Lip-3} \le 2^{308a^3} P\,.
        $$
        Using that $\rho(y,y') \le 16D^{s(J)}$ and $a \ge 4$, the lemma follows.
    \end{proof}

\subsection{The van der Corput estimate}
\label{subsubsec-van-der-corput}
    \begin{lemma}[lower oscillation bound]
        \label{lower-oscillation-bound}
        \uses{overlap-implies-distance}
        For all $J \in \mathcal{J}'$, we have that
        $$
            d_{B(J)}(\fcc(\fu_1), \fcc(\fu_2)) \ge 2^{-201a^3} 2^{Zn/2}\,.
        $$
    \end{lemma}

    \begin{proof}
    Since $\emptyset \ne \fT(\fu_1) \subset \mathfrak{S}$ by \Cref{overlap-implies-distance}, there exists at least one tile $\fp \in \mathcal{S}$ with $\scI(\fp) \subsetneq \scI(\fu_1)$. Thus $\scI(\fu_1) \notin \mathcal{J}'$, so $J \subsetneq \scI(\fu_1)$. Thus there exists a cube $J' \in \mathcal{D}$ with $J \subset J'$ and $s(J') = s(J) + 1$, by \eqref{coverdyadic} and \eqref{dyadicproperty}. By definition of $\mathcal{J'}$ and the triangle inequality, there exists $\fp \in \mathfrak{S}$ such that
    $$
        \scI(\fp) \subset B(c(J'), 100 D^{s(J') + 1}) \subset B(c(J), 128 D^{s(J)+2})\,.
    $$
    Thus, by definition of $\mathfrak{S}$:
    \begin{align*}
        2^{Zn/2} \le d_{\fp}(\fcc(\fu_1), \fcc(\fu_2)) \le d_{B(c(J), 128 D^{s(J)+2})}(\fcc(\fu_1), \fcc(\fu_2))\,.
    \end{align*}
    By the doubling property \eqref{firstdb}, this is
    $$
        \le 2^{200a^3 + 4a} d_{B(J)}(\fcc(\fu_1), \fcc(\fu_2))\,,
    $$
    which gives the lemma using $a \ge 4$.
    \end{proof}

    Now we are ready to prove \Cref{correlation-distant-tree-parts}.
    \begin{proof}[Proof of \Cref{correlation-distant-tree-parts}]
    \proves{correlation-distant-tree-parts}
    We have
    $$
        \eqref{eq-lhs-big-sep-tree} = \left| \int_{X} T_{\fT(\fu_1)}^* g_1 \overline{T_{\fT(\fu_2) \cap \mathfrak{S}}^* g_2 }\right|\,.
    $$
    By \Cref{adjoint-tile-support}, the right hand side is supported in $\scI(\fu_1)$. Using \eqref{eq-pao-1} of \Cref{Lipschitz-partition-unity} and the definition \eqref{def-hj} of $h_J$, we thus have
    $$
        \le \sum_{J \in \mathcal{J}'} \left|\int_{B(J)} e(\fcc(\fu_2)(y) - \fcc(\fu_1)(y)) h_J(y) \, \mathrm{d}\mu(y) \right|\,.
    $$
    Using \Cref{Holder-van-der-Corput} with the ball $B(J)$, we bound this by
    $$
        \le 2^{8a} \sum_{J \in \mathcal{J}'} \mu(B(J)) \|h_J\|_{C^{\tau}(B(J))} (1 + d_{B(J)}(\fcc(\fu_1), \fcc(\fu_1)))^{-1/(2a^2+a^3)}\,.
    $$
    Using \Cref{Holder-correlation-tree}, \Cref{lower-oscillation-bound} and $a \ge 4$, we have that the above is bounded from above by
    \begin{multline}
        \label{eq-big-sep-1}
        \le 2^{540a^3} 2^{-Zn/(4a^2 + 2a^3)} \sum_{J \in \mathcal{J}'} \mu(B(J)) \\
        \times \prod_{j=1}^2 (\inf_{B^\circ{}(J)} |T_{\fT(\fu_j)}^* g_j| + \inf_J M_{\mathcal{B},1} g_j)\,.
    \end{multline}
    By the doubling property \eqref{doublingx}
    $$
        \mu(B(J)) \le 2^{6a} \mu(B^\circ{}(J))\,,
    $$
    thus
    $$
        \mu(B(J)) \prod_{j=1}^2 (\inf_{B^\circ{}(J)} |T_{\fT(\fu_j)}^* g_j| + \inf_J M_{\mathcal{B},1} g_j)
    $$
    $$
        \le 2^{6a} \int_{B^\circ{}(J)} \prod_{j=1}^2 ( |T_{\fT(\fu_j)}^* g_j|(x) + M_{\mathcal{B},1} g_j(x)) \, \mathrm{d}\mu(x)
    $$
    $$
        \le 2^{6a} \int_J \prod_{j=1}^2 ( |T_{\fT(\fu_j)}^* g_j|(x) + M_{\mathcal{B},1} g_j(x)) \, \mathrm{d}\mu(x)\,.
    $$
    Summing over $J \in \mathcal{J}'$, we obtain
    $$
        \eqref{eq-big-sep-1} \le 2^{541a^3} 2^{-Zn/(4a^2 + 2a^3)} \int_X \prod_{j=1}^2 ( |T_{\fT(\fu_j)}^* g_j|(x) + M_{\mathcal{B},1} g_j(x)) \, \mathrm{d}\mu(x)\,.
    $$
    Applying the Cauchy-Schwarz inequality, \Cref{correlation-distant-tree-parts} follows.
    \end{proof}

\section{Proof of The Remaining Tiles Lemma}
    \label{subsec-rest-tiles}
    We define
    $$
        \mathcal{J}' = \{J \in \mathcal{J}(\fT(\fu_1)) \, : \, J \subset \scI(\fu_1)\}\,,
    $$
    note that this is different from the $\mathcal{J}'$ defined in the previous subsection.
    \begin{lemma}[dyadic partition 2]
        \label{dyadic-partition-2}
        \uses{dyadic-partitions}
        We have
        $$
            \scI(\fu_1) = \dot{\bigcup_{J \in \mathcal{J}'}} J\,.
        $$
    \end{lemma}

    \begin{proof}
        By \Cref{dyadic-partitions}, it remains only to show that each $J \in \mathcal{J}(\fT(\fu_1))$ with $J \cap \scI(\fu_1) \ne \emptyset$ is in $\mathcal{J}'$. But if $J \notin \mathcal{J}'$, then by \eqref{dyadicproperty} $\scI(\fu_1) \subsetneq J$. Pick $\fp \in \fT(\fu_1)$. Then $\scI(\fp) \subsetneq J$. This contradicts the definition of $\mathcal{J}(\fT(\fu_1))$.
    \end{proof}

    \Cref{correlation-near-tree-parts} follows from the following key estimate.

    \begin{lemma}[bound for tree projection]
        \label{bound-for-tree-projection}
        \uses{adjoint-tile-support,overlap-implies-distance,dyadic-partition-2,thin-scale-impact,square-function-count}
        We have
        $$
            \|P_{\mathcal{J}'}|T_{\mathfrak{T}_2 \setminus \mathfrak{S}}^* g_2|\|_2
            \le 2^{118a^3} 2^{-\frac{100}{202a}Zn\kappa} \|\mathbf{1}_{\scI(\fu_1)} M_{\mathcal{B},1} |g_2|\|_2
        $$
    \end{lemma}

    We prove this lemma below. First, we deduce \Cref{correlation-near-tree-parts}.

    \begin{proof}[Proof of \Cref{correlation-near-tree-parts}]
        \proves{correlation-near-tree-parts}
        By \Cref{tree-projection-estimate} and \Cref{adjoint-tile-support}, we have
        \begin{align*}
            \eqref{eq-lhs-small-sep-tree} \le 2^{104a^3} \|P_{\mathcal{L}(\fT(\fu_1))} |g_1\mathbf{1}_{\scI(\fu_1)}| \|_2 \|\mathbf{1}_{\scI(\fu_1)} P_{\mathcal{J}(\fT(\fu_1) )}|T_{\fT(\fu_2) \setminus \mathfrak{S}}^* g_2|\|_2\,.
        \end{align*}
        It follows from the definition of the projection operator $P$ and Jensen's inequality that
        $$
            \|P_{\mathcal{L}(\fT(\fu_1))} |g_1\mathbf{1}_{\scI(\fu_1)}| \|_2 \le \|g_1 \mathbf{1}_{\scI(\fu_1)}\|_2\,.
        $$
        Since cubes in $\mathcal{J}'$ are pairwise disjoint and by \Cref{dyadic-partition-2}, a cube $J \in \mathcal{J}'$ intersect $\scI(\fu_1)$ if and only if $J \in \mathcal{J}'$. Thus
        $$
            \mathbf{1}_{\scI(\fu_1)} P_{\mathcal{J}(\fT(\fu_1) )}|T_{\fT(\fu_2) \setminus \mathfrak{S}}^* g_2| = P_{\mathcal{J}'}|T_{\fT(\fu_2) \setminus \mathfrak{S}}^* g_2|\,.
        $$
        Combining this with \Cref{bound-for-tree-projection}, the definition \eqref{definekappa} and $a \ge 4$ proves the lemma.
    \end{proof}

    We need two more auxiliary lemmas before we prove \Cref{bound-for-tree-projection}.

    \begin{lemma}[thin scale impact]
        \label{thin-scale-impact}
        If $\fp \in \fT_2 \setminus \mathfrak{S}$ and $J \in \mathcal{J'}$ with $B(\scI(\fp)) \cap B(J) \ne \emptyset$, then
        $$
            \ps(\fp) \le s(J) + 2 - \frac{Zn}{202a^3}\,.
        $$
    \end{lemma}

    \begin{proof}
        Suppose that $\ps(\fp) > s(J) + 2 -\frac{Zn}{202a^3} =: s(J) - s_1$. Then, we have
        $$
            \rho(\pc(\fp), c(J)) \le 8D^{s(J)}+8D^{\ps(\fp)} \le 16 D^{\ps(\fp) + s_1}\,.
        $$
        There exists a tile $\fq \in \fT(\fu_1)$. By \eqref{forest1}, it satisfies $\scI(\fq) \subsetneq \scI(\fu_1)$. Thus $\scI(\fu_1) \notin \mathcal{J}'$. It follows that $J \subsetneq \scI(\fu_1)$. By \eqref{coverdyadic} and \eqref{dyadicproperty}, there exists a cube $J' \in \mathcal{D}$ with $J \subset J'$ and $s(J') = s(J) + 1$. By definition of $\mathcal{J}'$, there exists a tile $\fp' \in \fT(\fu_1)$ with
        $$
            \scI(\fp') \subset B(c(J'), 100 D^{s(J') + 1})\,.
        $$
        By the triangle inequality, the definition \eqref{defineD} and $a \ge 4$, we have
        $$
            B(c(J'), 100 D^{s(J')+1}) \subset B(\pc(\fp), 128 D^{\ps(\fp) + s_1 + 1})\,.
        $$
        Since $\fp' \in \fT(\fu_1)$ and $\scI(\fu_1) \subset \scI(\fu_2)$, we have by \eqref{forest5}
        $$
            d_{\fp'}(\fcc(\fp'), \fcc(\fu_2)) > 2^{Z(n+1)}\,.
        $$
        Hence, by \eqref{forest1}, the triangle inequality and using that by \eqref{defineZ} $Z \ge 2$
        $$
            d_{\fp'}(\fcc(\fu_1), \fcc(\fu_2)) > 2^{Z(n+1)} - 4 \ge 2^{Zn}\,.
        $$
        It follows that
        $$
            2^{Zn} \le d_{\fp'}(\fcc(\fu_1), \fcc(\fu_2)) \le d_{B(\pc(\fp), 128 D^{\ps(\fp) + s_1+ 1})}(\fcc(\fu_1), \fcc(\fu_2))\,.
        $$
        Using \eqref{firstdb}, we obtain
        $$
            \le 2^{9a + 100a^3 (s_1+1)} d_{\fp}(\fcc(\fu_1), \fcc(\fu_2))\,.
        $$
        Since $\fp' \notin \mathfrak{S}$ this is bounded by
        $$
            \le 2^{9a + 100a^3 (s_1+1)} 2^{Zn/2}\,.
        $$
        Thus
        $$
            Z n/2 \le 9a + 100a^3(s_1 + 1)\,,
        $$
        contradicting the definition of $s_1$.
    \end{proof}

    \begin{lemma}[square function count]
        \label{square-function-count}
        For each $J \in \mathcal{J}'$, we have
        $$
            \frac{1}{\mu(J)} \int_J \Bigg(\sum_{\substack{I \in \mathcal{D}, s(I) = s(J) - s\\ I \cap \scI(\fu_1) = \emptyset\\
        J \cap B(I) \ne \emptyset}} \mathbf{1}_{B(I)}\bigg)^2 \, \mathrm{d}\mu \le 2^{104a^2} (8 D^{-s})^\kappa\,.
        $$
    \end{lemma}

    \begin{proof}[Proof of \Cref{square-function-count}]
        \proves{square-function-count}
        Since $J \in \mathcal{J}'$ we have $J \subset \scI(\fu_1)$. Thus, if $B(I) \cap J \ne \emptyset$ then
    \begin{equation}
        \label{eq-sep-small-incl}
        B(I) \cap J \subset \{x \in J \ : \ \rho(x, X \setminus J) \le 8D^{s(I)}\}\,.
    \end{equation}
    Furthermore, for each $s$ the balls $B(I)$ with $s(I) = s$ have bounded overlap: Consider the collection $\mathcal{D}_{s,x}$ of all $I \in \mathcal{D}$ with $x \in B(I)$ and $s(I) = s$. By \eqref{eq-vol-sp-cube} and \eqref{dyadicproperty}, the balls $B(c(I), \frac{1}{4} D^{s(I)})$, $I \in \mathcal{D}_{s,x}$ are disjoint, and by the triangle inequality, they are contained in $B(x, 9 D^{s})$. By the doubling property \eqref{doublingx}, we have
    $$
        \mu(B(x, 9D^{s})) \le 2^{8a} \mu(B(c(I), \frac{1}{4} D^{s(I)}))
    $$
    for each $I \in \mathcal{D}_{s,x}$.
    Thus
    $$
        \mu(B(x, 9D^{s})) \ge \sum_{I \in \mathcal{D}_{s,x}} \mu(B(c(I), \frac{1}{4} D^{s(I)})) \ge 2^{-8a} |\mathcal{D}_{s,x}| \mu(B(x, 9D^{s}))\,.
    $$
    Dividing by the positive $\mu(B(x, 9D^{s}))$, we obtain that for each $x$
    \begin{equation}
        \label{eq-sep-small-bound}
        \Bigg(\sum_{\substack{I \in \mathcal{D}, s(I) = s(J) - s\\ I \cap \scI(\fu_1) = \emptyset\\
        J \cap B(I) \ne \emptyset}} \mathbf{1}_{B(I)}(x) \bigg)^2 = |\mathcal{D}_{s(J) - s,x}|^2 \le 2^{16a} \,.
    \end{equation}
    Combining \eqref{eq-sep-small-incl}, \eqref{eq-sep-small-bound} and the small boundary property \eqref{eq-small-boundary}, noting that $8D^{s(I)}=8D^{-s}D^{s(J)}$, the lemma follows.
\end{proof}


\begin{proof}[Proof of \Cref{bound-for-tree-projection}]
    \proves{bound-for-tree-projection}
    Expanding the definition of $P_{\mathcal{J}'}$, we have
    $$
        \|P_{\mathcal{J}'}|T_{\mathfrak{T}_2 \setminus \mathfrak{S}}^* g_2|\|_2
    $$
    $$
        = \left(\sum_{J \in \mathcal{J}'} \frac{1}{\mu(J)} \left|\int_J \sum_{\fp \in \fT(\fu_2) \setminus \mathfrak{S}} T_{\fp}^* g_2 \, \mathrm{d}\mu(y) \right|^2 \right)^{1/2}\,.
    $$
    We split the innermost sum according to the scale of the tile $\fp$, and then apply the triangle inequality and Minkowski's inequality:
    $$
        \le \sum_{s = -S}^S \left( \sum_{J \in \mathcal{J}'} \frac{1}{\mu(J)} \left|\int_J \sum_{\substack{\fp \in \fT(\fu_2) \setminus \mathfrak{S}\\ \ps(\fp) = s}} T_{\fp}^* g_2 \, \mathrm{d}\mu(y) \right|^2\right)^{1/2}\,.
    $$
    By \Cref{adjoint-tile-support}, the integral in the last display is $0$ if $J \cap B(\scI(\fp)) = \emptyset$. By \Cref{thin-scale-impact}, it follows with $s_1 := \frac{Zn}{202a^3} - 2$:
    \begin{equation}
    \label{eq-sep-tree-small-1}
        = \sum_{s = s_1}^{s_1 + 2S} \Bigg( \sum_{J \in \mathcal{J}'} \frac{1}{\mu(J)} \Bigg|\int_J \sum_{\substack{\fp \in \fT(\fu_2) \setminus \mathfrak{S}\\ \ps(\fp) = s(J) - s\\
        J \cap B(\scI(\fp)) \ne \emptyset}} T_{\fp}^* g_2 \, \mathrm{d}\mu(y) \Bigg|^2\Bigg)^{1/2}\,.
    \end{equation}
    We have by \Cref{adjoint-tile-support} and \eqref{eq-Ks-size}
    $$
        \int |T_{\fp}^* g_2|(y) \, \mathrm{d}\mu(y)
    $$
    $$
        \le 2^{102a^3} \int_{B(\scI(\fp))} \int \frac{1}{\mu(B(x, D^{\ps(\fp)}))} \mathbf{1}_{E(\fp)}(x) |g_2|(x) \, \mathrm{d}\mu(x) \, \mathrm{d}\mu(y)\,.
    $$
    If $x \in E(\fp) \subset \scI(\fp)$, then we have by \eqref{eq-vol-sp-cube} that
    $$
        B(\pc(\fp), 4D^{\ps(\fp)}) \subset B(\scI(\fp))\,.
    $$
    Using the doubling property \eqref{doublingx}, it follows that
    $$
        \mu(B(\pc(\fp), 4D^{\ps(\fp)})) \le 2^{3a} \mu(B(x, D^s))\,.
    $$
    Thus, using also $a \ge 4$
    $$
        \int |T_{\fp}^* g_2|(y) \, \mathrm{d}\mu(y)
    $$
    $$
        \le 2^{103 a^3} \int_{B(\scI(\fp))} \int \frac{1}{\mu(B(\pc(\fp), 4D^{\ps(\fp)}))} \mathbf{1}_{E(\fp)}(x) |g_2|(x) \, \mathrm{d}\mu(x) \, \mathrm{d}\mu(y)\,.
    $$
    Since for each $I \in \mathcal{D}$ the sets $E(\fp), \fp \in \fP(I)$ are disjoint, it follows that
    $$
        \bigg| \int_J \sum_{\substack{\fp \in \fT(\fu_2) \setminus \mathfrak{S}\\ \scI(\fp) = I\\
        J \cap B(\scI(\fp)) \ne \emptyset}} T_{\fp}^* g_2 \, \mathrm{d}\mu \bigg|
    $$
    $$
        \le 2^{103a^3} \int_J \mathbf{1}_{B(I)} \frac{1}{\mu(B(\pc(\fp), 4D^{\ps(\fp)}))} \int_{B(\pc(\fp), 4D^{\ps(\fp)})} |g_2|(x) \, \mathrm{d}\mu(x)
    $$
    $$
        \le 2^{103a^3} \int_J M_{\mathcal{B},1} |g_2|(y) \mathbf{1}_{B(I)}(y) \, \mathrm{d}\mu(y)\,.
    $$
    By \Cref{overlap-implies-distance}, we have $\scI(\fp) \cap \scI(\fu_1) = \emptyset$ for all $\fp \in \fT(\fu_2) \setminus \mathfrak{S}$.
    Thus we can estimate \eqref{eq-sep-tree-small-1} by
    $$
        2^{103a^3} \sum_{s = s_1}^{s_1 + 2S} \Bigg( \sum_{J \in \mathcal{J}'} \frac{1}{\mu(J)} \Bigg|\int_J \sum_{\substack{I \in \mathcal{D}, s(I) = s(J) - s\\ I \cap \scI(\fu_1) = \emptyset\\
        J \cap B(I) \ne \emptyset}} M_{\mathcal{B},1} |g_2| \mathbf{1}_{B(I)} \, \mathrm{d}\mu \Bigg|^2\Bigg)^{\frac 1 2}\,,
    $$
    which is by Cauchy-Schwarz at most
    \begin{equation}
    \label{eq-sep-tree-small-2}
        2^{103a^3} \sum_{s = s_1}^{s_1 + 2S} \Bigg( \sum_{J \in \mathcal{J}'} \int_J ( M_{\mathcal{B},1} |g_2|)^2 \frac{1}{\mu(J)} \int_J \Bigg(\sum_{\substack{I \in \mathcal{D}, s(I) = s(J) - s\\ I \cap \scI(\fu_1) = \emptyset\\
        J \cap B(I) \ne \emptyset}} \mathbf{1}_{B(I)}\bigg)^2 \, \mathrm{d}\mu \Bigg)^{\frac 12}\,.
    \end{equation}
    Using \Cref{square-function-count}, we bound \eqref{eq-sep-tree-small-2} by
    $$
        2^{103a^3} \sum_{s = s_1}^{s_1 + 2S} \left(\sum_{J \in \mathcal{J}'} \int_J (M_{\mathcal{B},1} |g_2|)^2 2^{104a^2} (8 D^{-s})^\kappa\right)^{\frac 1 2}\,,
    $$
    and, since dyadic cubes in $\mathcal{J}'$ form a partition of $\scI(\fu_1)$ by \Cref{dyadic-partition-2}, $\kappa \le 1$ by \eqref{definekappa}, and $a \ge 4$
    $$
        \le 2^{116a^3} \sum_{s = s_1}^{s_1 + 2S} D^{-s\kappa/2} \|\mathbf{1}_{\scI(\fu_1)} M_{\mathcal{B},1} |g_2|\|_2
    $$
    $$
        \le 2^{116a^3} D^{-s_1 \kappa /2} \frac{1}{1 - D^{-\kappa/2}} \|\mathbf{1}_{\scI(\fu_1)} M_{\mathcal{B},1} |g_2|\|_2\,.
    $$
    By convexity of $t \mapsto D^t$ and since $D \ge 2$, we have for all $-1 \le t \le 0$
    $$
        D^t \le 1 + t(1 - \frac{1}{D}) \le 1 + \frac{1}{2}t\,.
    $$
    Using this for $t = -\kappa/2$ and using that $s_1 = \frac{Zn}{202a^3} - 2$ and the definitions \eqref{defineD} and \eqref{definekappa} of $\kappa$ and $D$
    $$
        \le 2^{116a^3} 2^{-100a^2(\frac{Zn}{202a^3} - 2) \frac{\kappa}{2}} \frac{2}{\kappa} \|\mathbf{1}_{\scI(\fu_1)} M_{\mathcal{B},1} |g_2|\|_2
    $$
    $$
        \le \frac{2^{117a^3 + 1}}{\kappa} 2^{-\frac{100}{202a}Zn\kappa} \|\mathbf{1}_{\scI(\fu_1)} M_{\mathcal{B},1} |g_2|\|_2\,.
    $$
    Using the definition \eqref{definekappa} of $\kappa$ and $a \ge 4$, the lemma follows.
\end{proof}



\section{Forests}
In this subsection, we complete the proof of \Cref{forest-operator} from the results of the previous subsections.

Define an $n$-row to be an $n$-forest $(\fU, \fT)$, i.e. satisfying conditions \eqref{forest1} - \eqref{forest6}, such that in addition the sets $\scI(\fu), \fu \in \fU$ are pairwise disjoint.

\begin{lemma}[forest row decomposition]
    \label{forest-row-decomposition}
    Let $(\fU, \fT)$ be an $n$-forest. Then there exists a decomposition
    $$
        \fU = \dot{\bigcup_{1 \le j \le 2^n}} \fU_j
    $$
    such that for all $j = 1, \dotsc, 2^n$ the pair $(\fU_j, \fT|_{\fU_j})$ is an $n$-row.
\end{lemma}

\begin{proof}
    Define recursively $\fU_j$ to be a maximal disjoint set of tiles $\fu$ in
    $$
        \fU \setminus \bigcup_{j' < j} \fU_{j'}
    $$
    with inclusion maximal $\scI(\fu)$. Properties \eqref{forest1}, -\eqref{forest6} for $(\fU_j, \fT|_{\fU_k})$ follow immediately from the corresponding properties for $(\fU, \fT)$, and the cubes $\scI(\fu), \fu \in \fU_j$ are disjoint by definition. The collections $\fU_j$ are also disjoint by definition.

    Now we show by induction on $j$ that each point is contained in at most $2^n - j$ cubes $\scI(\fu)$ with $\fu \in \fU \setminus \bigcup_{j' \le j} \fU_{j'}$. This implies that $\bigcup_{j = 1}^{2^n} \fU_j = \fU$, which completes the proof of the Lemma. For $j = 0$ each point is contained in at most $2^n$ cubes by \eqref{forest3}. For larger $j$, if $x$ is contained in any cube $\scI(\fu)$ with $\fu \in \fU \setminus \bigcup_{j' < j} \fU_{j'}$, then it is contained in a maximal such cube. Thus it is contained in a cube in $\scI(\fu)$ with $\fu \in \fU_j$. Thus the number $\fu \in \fU \setminus \bigcup_{j' \le j} \fU_{j'}$ with $x\in \scI(\fu)$ is zero, or is less than the number of $\fu \in \fU \setminus \bigcup_{j' \le j-1} \fU_{j'}$ with $x \in \scI(\fu)$ by at least one.
\end{proof}

We pick a decomposition of the forest $(\fU, \fT)$ into $2^n$ $n$-rows
\begin{equation*}
(\fU_j, \fT_j) := (\fU_j, \fT|_{\fU_j})
\end{equation*}
as in \Cref{forest-row-decomposition}.

\begin{lemma}[row bound]
    \label{row-bound}
    \uses{densities-tree-bound,adjoint-tile-support}
    For each $1 \le j \le 2^n$ and each bounded $g$ with bounded support, we have
    \begin{equation}
        \label{eq-row-bound-1}
        \left\| \sum_{\fu \in \fU_j} \sum_{\fp \in \fT(\fu)} T_{\fp}^* g\right\|_2 \le 2^{156a^3} 2^{-n/2} \|g\|_2
    \end{equation}
    and
    \begin{equation}
        \label{eq-row-bound-2}
        \left\| \sum_{\fu \in \fU_j} \sum_{\fp \in \fT(\fu)} \mathbf{1}_F T_{\fp}^* g\right\|_2 \le 2^{257a^3} 2^{-n/2} \dens_2(\bigcup_{\fu\in \fU}\fT(\fu))^{1/2} \|g\|_2\,.
    \end{equation}
\end{lemma}

\begin{proof}
    By \Cref{densities-tree-bound} and the density assumption \eqref{forest4}, we have for each $\fu \in \fU$ and all bounded $f$ of bounded support that
    \begin{equation}
        \label{eq-explicit-tree-bound-1}
        \left\|\sum_{\fp \in \fT(\fu)} T_{\fp} f \right\|_{2} \le 2^{155a^3} 2^{(4a+1-n)/2} \|f\|_2\,
    \end{equation}
    and
    \begin{equation}
        \label{eq-explicit-tree-bound-2}
        \left\|\sum_{\fp \in \fT(\fu)} T_{\fp} \mathbf{1}_F f \right\|_{2} \le 2^{256a^3} 2^{(4a + 1-n)/2} \dens_2(\fT(\fu))^{1/2} \|f\|_2\,.
    \end{equation}
    Since for each $j$ the top cubes $\scI(\fu)$, $\fu \in \fU_j$ are disjoint, we further have for all bounded $g$ of bounded support by \Cref{adjoint-tile-support}
    $$
        \left\|\mathbf{1}_F \sum_{\fu \in \fU_j} \sum_{\fp \in \fT(\fu)} T_{\fp}^* g\right\|_2^2 = \left\|\mathbf{1}_F \sum_{\fu \in \fU_j} \sum_{\fp \in \fT(\fu)} \mathbf{1}_{\scI(\fu)} T_{\fp}^* \mathbf{1}_{\scI(\fu)} g\right\|_2^2
    $$
    $$
        = \sum_{\fu \in \fU_j} \int_{\scI(\fu)} \left| \mathbf{1}_F \sum_{\fp \in \fT(\fu)} T_{\fp}^* \mathbf{1}_{\scI(\fu)} g\right|^2 \, \mathrm{d}\mu
        \le \sum_{\fu \in \fU_j} \left\|\sum_{\fp \in \fT(\fu)} \mathbf{1}_F T_{\fp}^* \mathbf{1}_{\scI(\fu)} g\right\|_2^2\,.
    $$
    Applying the estimate for the adjoint operator following from equation \eqref{eq-explicit-tree-bound-2}, we obtain
    $$
        \le 2^{256a^3} 2^{(4a+1-n)/2} \max_{\fu \in \fU_j}\dens_2(\fT(\fu))^{1/2} \sum_{\fu \in \fU_j} \left\| \mathbf{1}_{\scI(\fu)} g\right\|_2^2\,.
    $$
    Again by disjointedness of the cubes $\scI(\fu)$, this is estimated by
    $$
        2^{256a^3} 2^{(4a+1-n)/2} \max_{\fu \in \fU_j}\dens_2(\fT(\fu))^{1/2} \|g\|_2^2\,.
    $$
    Thus, \eqref{eq-row-bound-2} follows, since $a \ge 4$.
    The proof of \eqref{eq-row-bound-1} from \eqref{eq-explicit-tree-bound-1} is the same up to replacing $F$ by $X$.
\end{proof}

\begin{lemma}[row correlation]
    \label{row-correlation}
    \uses{adjoint-tree-control,correlation-separated-trees}
    For all $1 \le j < j' \le 2^n$ and for all bounded $g_1, g_2$ with bounded support, it holds that
    $$
        \left| \int \sum_{\fu \in \fU_j} \sum_{\fu' \in \fU_{j'}} \sum_{\fp \in \fT_j(\fu)} \sum_{\fp' \in \fT_{j'}(\fu')} T^*_{\fp} g_1 \overline{T^*_{\fp'} g_2} \, \mathrm{d}\mu \right| \le
        2^{862a^3-3n}\|g_1\|_2 \|g_2\|_2\,.
    $$
\end{lemma}

\begin{proof}
    To save some space we will write for subsets $\fC \subset \fP$
    $$
        T_{\fC}^* = \sum_{\fp \in \fC} T_{\fp}^*\,.
    $$
    We have by \Cref{adjoint-tile-support} and the triangle inequality that
    $$
        \left| \int \sum_{\fu \in \fU_j} \sum_{\fu' \in \fU_{j'}} \sum_{\fp \in \fT_j(\fu)} \sum_{\fp' \in \fT_{j'}(\fu')} T^*_{\fp} g_1 \overline{T^*_{\fp'} g_2} \, \mathrm{d}\mu \right|
    $$
    $$
        \le \sum_{\fu \in \fU_j} \sum_{\fu' \in \fU_{j'}} \left| \int T^*_{\fT_j(\fu)} (\mathbf{1}_{\scI(\fu)} g_1) \overline{T^*_{\fT_{j'}(\fu')} (\mathbf{1}_{\scI(\fu')} g_2)} \, \mathrm{d}\mu \right|\,.
    $$
    By \Cref{correlation-separated-trees}, this is bounded by
    \begin{equation}
        \label{eq-S2uu'}
         2^{550a^3-3n} \sum_{\fu \in \fU_j} \sum_{\fu' \in \fU_{j'}} \|S_{2,\fu} g_1\|_{L^2(\scI(\fu')\cap \scI(\fu)} \|S_{2, \fu'} g_2\|_{L^2(\scI(\fu')\cap\scI(\fu))}\,.
    \end{equation}
    We apply the Cauchy-Schwarz inequality in the form
    \begin{equation*}
        \sum_{i \in M} a_i b_i \le (\sum_{i \in M} a_i^2 )^{1/2}(\sum_{i \in M} b_i^2 )^{1/2}
    \end{equation*} to the outer two sums:
    $$
        \le 2^{550a^3-3n} \left(\sum_{\fu \in \fU_j} \sum_{\fu' \in \fU_{j'}} \|S_{2,\fu} g_1\|_{L^2(\scI(\fu')\cap \scI(\fu))}^2 \right)^{1/2}
    $$
    $$
        \left(\sum_{\fu \in \fU_j} \sum_{\fu' \in \fU_{j'}} \|S_{2,\fu'} g_2\|_{L^2(\scI(\fu')\cap\scI(\fu))}^2 \right)^{1/2}\,.
    $$
    By pairwise disjointedness of the sets $\scI(\fu)$ for $\fu \in \fU_j$ and of the sets $\scI(\fu')$ for $\fu' \in \fU_{j'}$, we have
    $$
        \sum_{\fu \in \fU_j}\sum_{\fu' \in \fU_{j'}} \|S_{2,\fu} g_1\|_{L^2(\scI(\fu')\cap \scI(\fu))}^2
        = \sum_{\fu \in \fU_j}\sum_{\fu' \in \fU_{j'}} \int_{\scI(\fu) \cap \scI(\fu')} |S_{2,\fu} g_1(y)|^2 \, \mathrm{d}\mu(y)
    $$
    $$
        \le \int_X |S_{2,\fu}g_1(y)|^2 \, \mathrm{d}\mu(y) = \|S_{2,\fu} g_1\|_2^2\,.
    $$
    Arguing similar for $g_2$, we can estimate \eqref{eq-S2uu'} to be
    $$
        \le 2^{550a^3-3n} \|S_{2,\fu}g_1\|_2 \|S_{2,\fu'} g_2\|_2\,.
    $$
    The lemma now follows from \Cref{adjoint-tree-control}.
\end{proof}

Define for $1 \le j \le 2^n$
$$
    E_j := \bigcup_{\fu \in \fU_j} \bigcup_{\fp \in \fT(\fu)} E(\fp)\,.
$$

\begin{lemma}[disjoint row support]
    \label{disjoint-row-support}
    The sets $E_j$, $1 \le j \le 2^n$ are pairwise disjoint.
\end{lemma}

\begin{proof}
    Suppose that $\fp \in \fT(\fu)$ and $\fp' \in \fT(\fu')$ with $\fu \ne \fu'$ and $x \in E(\fp) \cap E(\fp')$. Suppose without loss of generality that $\ps(\fp) \le \ps(\fp')$. Then $x \in \scI(\fp) \cap \scI(\fp') \subset \scI(\fu')$. By \eqref{dyadicproperty} it follows that $\scI(\fp) \subset \scI(\fu')$. By \eqref{forest5}, it follows that
    $$
        d_{\fp}(\fcc(\fp), \fcc(\fu')) > 2^{Z(n+1)}\,.
    $$
    By the triangle inequality. \Cref{monotone-cube-metrics} and \eqref{forest1} it follows that
    \begin{align*}
        d_{\fp}(\fcc(\fp), \fcc(\fp')) &\ge d_{\fp}(\fcc(\fp), \fcc(\fu')) - d_{\fp}(\fcc(\fp'), \fcc(\fu'))\\
        &> 2^{Z(n+1)} - d_{\fp'}(\fcc(\fp'), \fcc(\fu'))\\
        &\ge 2^{Z(n+1)} - 4\,.
    \end{align*}
    Since $Z \ge 3$ by \eqref{defineZ}, it follows that $\fcc(\fp') \notin B_{\fp}(\fcc(\fp), 1)$, so $\Omega(\fp') \not\subset \Omega(\fp)$ by \eqref{eq-freq-comp-ball}. Hence, by \eqref{eq-freq-dyadic}, $\Omega(\fp) \cap \Omega(\fp') = \emptyset$. But if $x \in E(\fp) \cap E(\fp')$ then $Q(x) \in \Omega(\fp) \cap \Omega(\fp')$. This is a contradiction, and the lemma follows.
\end{proof}

Now we prove \Cref{forest-operator}.

\begin{proof}[Proof of \Cref{forest-operator}]
    \proves{forest-operator}
    To save some space, we will write
    $$
        T_{\mathfrak{R}_j}^* = \sum_{\fu \in \fU_j} \sum_{\fp \in \fT(\fu)} T_{\fp}^*\,.
    $$
    By \eqref{definetp*}, we have for each $j$
    $$
        T_{\mathfrak{R}_j}^*g = \sum_{\fu \in \fU_j} \sum_{\fp \in \fT(\fu)} T_{\fp}^* g = \sum_{\fu \in \fU_j} \sum_{\fp \in \fT(\fu)} T_{\fp}^* \mathbf{1}_{E_j} g = T_{\mathfrak{R}_j}^* \mathbf{1}_{E_j} g\,.
    $$
    Hence, by \Cref{forest-row-decomposition},
    $$
        \left\|\sum_{\fu \in \fU} \sum_{\fp \in \fT(\fu)} T^*_{\fp} g\right\|_2^2 = \left\|\sum_{j = 1}^{2^n} T^*_{\mathfrak{R}_{j}} g\right\|_2^2 = \left\|\sum_{j=1}^{2^n} T^*_{\mathfrak{R}_{j}} \mathbf{1}_{E_j} g\right\|_2^2
    $$
    $$
        = \int_X \left|\sum_{j=1}^{2^n} T^*_{\mathfrak{R}_{j}} \mathbf{1}_{E_j} g\right|^2 \, \mathrm{d}\mu
    $$
    $$
        = \sum_{j=1}^{2^n} \int_X |T_{\mathfrak{R}_j}^* \mathbf{1}_{E_j} g|^2 + \sum_{j =1}^{2^n} \sum_{\substack{j' = 1\\j' \ne j}}^{2^n} \int_X \overline{ T_{\mathfrak{R}_j}^* \mathbf{1}_{E_j} g} T_{\mathfrak{R}_{j'}}^* \mathbf{1}_{E_{j'}} g \, \mathrm{d}\mu\,.
    $$
    We use \Cref{row-bound} to estimate each term in the first sum, and \Cref{row-correlation} to bound each term in the second sum:
    $$
        \le 2^{312a^3-n} \sum_{j = 1}^{2^n} \|\mathbf{1}_{E_j} g\|_2^2 + 2^{862a^3-3 n}\sum_{j=1}^{2^n}\sum_{j' = 1}^{2^n} \|\mathbf{1}_{E_j} g\|_2 \|\mathbf{1}_{E_{j'}}g\|_2\,.
    $$
    By Cauchy-Schwarz in the second two sums, this is at most
    $$
        2^{862a^3} (2^{-n} + 2^{n}2^{-3 n}) \sum_{j = 1}^n \|\mathbf{1}_{E_j} g\|_2^2\,,
    $$
    and by disjointedness of the sets $E_j$, this is at most
    $$
        2^{863a^3 - n} \|g\|_2^2\,.
    $$
    Taking adjoints and square roots, it follows that for all $f$
    \begin{equation}
        \label{eq-forest-bound-1}
        \left\|\sum_{\fu \in \fU} \sum_{\fp \in \fT(\fu)} T_{\fp} f\right\|_2 \le 2^{432a^3-\frac{n}{2}} \|f\|_2\,.
    \end{equation}
    On the other hand, we have by disjointedness of the sets $E_j$
    from \Cref{disjoint-row-support}
    $$
        \left\|\sum_{\fu \in \fU} \sum_{\fp \in \fT(\fu)} T_{\fp} f\right\|_2^2 = \left\|\sum_{j=1}^{2^n} \mathbf{1}_{E_j} T_{\mathfrak{R}_{j}} f\right\|_2^2 = \sum_{j = 1}^{2^n} \|\mathbf{1}_{E_j} T_{\mathfrak{R}_{j}} f\|_2^2\,.
    $$
    If $|f| \le \mathbf{1}_F$ then we obtain from \Cref{row-bound} and taking square roots that
    $$
        \le 2^{257a^3} \dens_2(\bigcup_{\fu\in \fU}\fT(\fu))^{\frac{1}{2}} 2^{-\frac{n}{2}} (\sum_{j = 1}^{2^n} \|f\|_2^2)^{\frac{1}{2}}
    $$
    \begin{equation}
        \label{eq-forest-bound-2}
        = 2^{257a^3} \dens_2(\bigcup_{\fu\in \fU}\fT(\fu))^{\frac{1}{2}} \|f\|_2\,.
    \end{equation}
    \Cref{forest-operator} follows by taking the product of the $(2 - \frac{2}{q})$-th power of \eqref{eq-forest-bound-1} and the $(\frac{2}{q} - 1)$-st power of \eqref{eq-forest-bound-2}.
\end{proof}

\chapter{Proof of the H\"older cancellative condition}
\label{liphoel}

We need the following auxiliary lemma.
Recall that $\tau = 1/a$.

\begin{lemma}[Lipschitz Holder approximation]
    \label{Lipschitz-Holder-approximation}
    Let $z\in X$ and $R>0$. Let $\varphi: X \to \mathbb{C}$ be a function supported in the ball
    $B:=B(z,R)$ with finite norm $\|\varphi\|_{C^\tau(B)}$. Let $0<t \leq 1$. There exists a function $\tilde \varphi : X \to \mathbb{C}$, supported in $B(z,2R)$, such that for every $x\in X$
    \begin{equation}\label{eq-firstt}
        |\varphi(x) - \tilde \varphi(x)| \leq t^{\tau} \|\varphi\|_{C^\tau(B)}
    \end{equation}and
   \begin{equation}\label{eq-secondt}
       \|\tilde \varphi\|_{\Lip(B(z,2R))} \leq 2^{4a}t^{-1-a} \|\varphi\|_{C^{\tau}(B)}\, .
   \end{equation}
\end{lemma}


\begin{proof}
   Define for $x,y\in X$ the Lipschitz and thus measurable function
     \begin{equation}
           L(x,y) := \max\{0, 1 - \frac{\rho(x,y)}{tR}\}\, .
    \end{equation}
We have that $L(x,y)\neq 0$ implies
\begin{equation}\label{eql01}
    y\in B(x, tR)\, .
\end{equation}
We have for $y\in B(x, 2^{-1}tR)$ that
\begin{equation}\label{eql30}
           |L(x,y)|\ge 2^{-1} \ .
    \end{equation}
Hence
\begin{equation}
        \int L(x,y) \, \mathrm{d}\mu(y)\ge 2^{-1}\mu(B(x, 2^{-1}Rt))\, .
    \end{equation}
 Let $n$ be the smallest integer so that
 \begin{equation}\label{2nt1}
     2^nt\ge 1\, .
 \end{equation} Iterating $n+2$ times the doubling condition \eqref{doublingx}, we obtain
      \begin{equation}\label{eql32}
        \int L(x,y) \, \mathrm{d}\mu(y)\ge 2^{-1-a(n+2)}\mu(B(x, 2R))\, .
    \end{equation}

Now define
    $$
        \tilde \varphi(x) := \left(\int L(x,y) \, \mathrm{d}\mu(y)\right)^{-1}\int L(x,y) \varphi(y) \, \mathrm{d}\mu(y)\, .
    $$
Using that $\varphi$ is supported in $B(z,R)$ and
\eqref{eql01}, we have that $\tilde{\varphi}$ is supported in $B(z,2R)$.

We prove \eqref{eq-firstt}.
 For any $x\in X$, using
    that $L$ is nonnegative,
    \begin{equation}\label{eql1}
    \left(\int L(x,y) \, \mathrm{d}\mu(y)\right)
        |\varphi(x) - \tilde \varphi(x)|
    \end{equation}
 \begin{equation}\label{eql2}
 = \left| \int L(x,y)(\varphi(x) - \varphi(y)) \, \mathrm{d}\mu(y)\right|\, .
    \end{equation}
Using \eqref{eql01}, we estimate the last display by
 \begin{equation}\label{eql3}
         \le \int_{B(x, tR)} L(x,y)|\varphi(x) - \varphi(y)| \, \mathrm{d}\mu(y)\, .\end{equation}
  Using the definition of $\|\varphi\|_{C^\tau(B)}$, we estimate the last display further by
        \begin{equation}\label{eql4}
         \le \left(\int_{B(x, tR)} L(x,y)
          \rho(x,y)^\tau \, \mathrm{d}\mu(y) \right)\|\varphi\|_{C^\tau(B)}R^{-\tau}\, .
    \end{equation}
  Using the condition on the domain of integration to estimate $\rho(x,y)$ by $tR$ and then expanding the domain by positivity of the integrand, we estimate this further by

    \begin{equation}\label{eql5}
         \le \left(\int L(x,y) \, \mathrm{d}\mu(y)\right)
         \|\varphi\|_{C^\tau(B)} t^{\tau} \, .
    \end{equation}
 Dividing the string of inequalities from \eqref{eql1} to
\eqref{eql5} by the positive integral of $L$ proves \eqref{eq-firstt}.


We turn to \eqref{eq-secondt}. For every $x\in X$, we have
\begin{equation}
    \left|\int L(x,y) \, \mathrm{d}\mu(y)\right||\tilde{\varphi}(x)|
    =\left|\int L(x,y) {\varphi}(y)\, \mathrm{d}\mu(y)\right|
\end{equation}
 \begin{equation}
    \le \left|\int L(x,y) \, \mathrm{d}\mu(y)\right| \sup_{x'\in X}
    |{\varphi}(x')|\ .
\end{equation}
As $\varphi$ is supported on $B$, dividing by the integral of $L$, we obtain
\begin{equation}\label{eql42}
 |\tilde{\varphi}(x)|\le \sup_{x'\in B}
    |{\varphi}(x')|\le \|\varphi\|_{C^\tau(B)}\ .
\end{equation}
If $\rho(x,x')\ge R$, then we have by the triangle inequality
  \begin{equation}\label{eql52}
 R\frac{|\tilde{\varphi}(x') - \tilde \varphi(x)|}{\rho(x,x')} \le
 2\sup_{x''\in X} |\tilde{\varphi}(x'')|\le 2\|\varphi\|_{C^\tau(B)}\, .
\end{equation}
Now assume $\rho(x,x')< R$. For $y\in X$ we have by the triangle inequality and a two fold case distinction
for the maximum in the definition of $L$,
\begin{equation}\label{eql10}
   |L(x,y) - L(x',y)| \le \frac{\rho(x,x')}{tR}.
\end{equation}
We compute with \eqref{eql10}, first adding and subtracting a term in the integral,
\begin{equation}
    \left(\int L(x,y) \, \mathrm{d}\mu(y)\right)
    |\tilde{\varphi}(x') - \tilde \varphi(x)|=
\end{equation}
\begin{equation}
    \left|\int L(x,y) \tilde{\varphi}(x')
    -L(x,y) \tilde{\varphi}(x)
    +L(x',y) \tilde{\varphi}(x')-
     L(x',y) \tilde{\varphi}(x')
    \, \mathrm{d}\mu(y)\right|\,.
\end{equation}
Grouping the second and third and the first and fourth term, we obtain using the definition of $\tilde \varphi$ and Fubini,
\begin{equation}\label{eql21}
    \le \left| \int (L(x',y)-L(x,y)) \varphi(y) \, \mathrm{d}\mu(y)\right|
\end{equation}
\begin{equation}\label{eql22}
    + \left| \int L(x,y) \, \mathrm{d}\mu(y)-\int L(x',y) \, \mathrm{d}\mu(y)\right||\tilde{\varphi}(x')|
\end{equation}
\begin{equation}\label{eql23}
    \le 2 \int |L(x,y) -L(x',y)| \, \mathrm{d}\mu(y)
    \|\varphi\|_{C^\tau(B)}\, ,
\end{equation}
where in the last inequality we have used \eqref{eql42}.
Using further \eqref{eql10} and the support of $L$, we estimate the last display by
\begin{equation}\label{eql224}\le 2 \frac{\rho(x,x')} {tR}\mu(B(x,tR)\cup B(x',tR))
\|\varphi\|_{C^\tau(B)}\, .
    \end{equation}
  Using $\rho(x,x')<R$ and the triangle inequality, we estimate the last display by
\begin{equation}\label{eql225}\le 2
\frac{\rho(x,x')} {tR}
\mu(B(x,2R))
\|\varphi\|_{C^\tau(B)}\, .
    \end{equation}
Dividing by the integral over $L$ and using \eqref{eql32} and \eqref{2nt1}, we obtain
\begin{equation}\label{eql226}
 \frac {R |\tilde{\varphi}(x') - \tilde \varphi(x)|}{\rho(x,x')}
 \le 2^{2+a(n+2)}t^{-1}\|\varphi\|_{C^\tau(B)} \le
 2^{2+3a} t^{-1-a} \|\varphi\|_{C^\tau(B)}\, .
\end{equation}
Combining \eqref{eql52} and \eqref{eql226} using $a\ge 4$ and $t\le 1$ and
adding \eqref{eql42} proves \eqref{eq-secondt} and completes the proof
of \Cref{Lipschitz-Holder-approximation}.
\end{proof}


We turn to the proof of \Cref{Holder-van-der-Corput}.
Let $z\in X$ and $R>0$ and set $B=B(z,R)$. Let $\varphi$
be given as in \Cref{Holder-van-der-Corput}.
Set
\begin{equation}\label{eql69}
    t:=(1+d_B(\mfa,\mfb))^{-\frac{\tau}{2+a}}
\end{equation}
and define $\tilde{\varphi}$ as in \Cref{Lipschitz-Holder-approximation}. Let $\mfa$ and $\mfb$ be in $\Mf$.
Then
   \begin{equation}\label{eql60}
       \left|\int e(\mfa(x)-{\mfb(x)}) \varphi (x)\, \mathrm{d}\mu(x)\right|
   \end{equation}
      \begin{equation}\label{eql61}
   \le \left|\int e(\mfa(x)-{\mfb(x)}) \tilde{\varphi} (x)\, \mathrm{d}\mu(x)\right|
   \end{equation}
           \begin{equation}\label{eql62}
     + \left|\int e(\mfa(x)-{\mfb(x)}) (\varphi (x)-\tilde{\varphi}(x))\, \mathrm{d}\mu(x)\right|
   \end{equation}
Using the cancellative condition \eqref{eq-vdc-cond} of $\Mf$ on the ball $B(z,2R)$, the term \eqref{eql61} is bounded above by
 \begin{equation}\label{eql63}
       2^a \mu(B(z,2R)) \|\tilde{\varphi}\|_{\Lip(B(z,2R))} (1 + d_{B(z,2R)}(\mfa,\mfb))^{-\tau} \, .
 \end{equation}



Using the doubling condition \eqref{doublingx},
the inequality \eqref{eq-secondt}, and the estimate
$d_B\le d_{B(z,2R)}$ from the definition,
we estimate \eqref{eql63} from above by
\begin{equation}\label{eql64}
       2^{6a}t^{-1-a} \mu(B) \|{\varphi}\|_{C^\tau(B)}
       (1 + d_{B}(\mfa,\mfb))^{-\tau} \, .
 \end{equation}

The term \eqref{eql62} we estimate using
\eqref{eq-firstt} and that
$\mfa$ and $\mfb$ are real and thus $e(\mfa)$ and
$e(\mfb)$ bounded in absolute value by $1$.
We obtain for \eqref{eql62} with \eqref{doublingx}
the upper bound
  \begin{equation}\label{eql65}
      \mu(B(z,2R)) t^{\tau} \|\varphi\|_{C^\tau(B)}
      \le 2^a \mu(B) t^{\tau} \|\varphi\|_{C^\tau(B)}
      \,.
 \end{equation}
Using the definition \eqref{eql69} of $t$ and adding
\eqref{eql64} and \eqref{eql65} estimates
\eqref{eql60} from above by
\begin{equation}
       2^{6a} \mu(B) \|{\varphi}\|_{C^\tau(B)}
       (1 + d_{B}(\mfa,\mfb))^{-\frac{\tau}{2+a}}
       \end{equation}
\begin{equation} +
        2^a \mu(B) \|{\varphi}\|_{C^\tau(B)}
       (1 + d_{B}(\mfa,\mfb))^{-\frac{\tau^2}{2+a}}\, .
 \end{equation}
\begin{equation}\label{eql66}
      \le 2^{1+6a} \mu(B) \|{\varphi}\|_{C^\tau(B)}
       (1 + d_{B}(\mfa,\mfb))^{-\frac{\tau^2}{2+a}} \, ,
 \end{equation}
where we used $\tau\le 1$.
This completes the proof of \Cref{Holder-van-der-Corput}.


\chapter{Proof of Vitali covering and Hardy--Littlewood}
\label{sec-hlm}


We begin with a classical representation of the Lebesgue norm.
\begin{lemma}[layer cake representation]\label{layer-cake-representation}
\leanok
\lean{snorm_pow_eq_distribution}
Let $1\le p< \infty$. Then for any measurable function $u:X\to [0,\infty)$ on the measure space $X$
relative to the measure $\mu$
we have
\begin{equation}\label{eq-layercake}
    \|u\|_p^p=p\int_0^\infty \lambda^{p-1}\mu(\{x: u(x)\ge \lambda\})\, d\lambda\, .
\end{equation}
\end{lemma}
\begin{proof}
    The left-hand side of \eqref{eq-layercake} is by definition
\begin{equation}
    \int_X u(x)^p \, d\mu(x)\, .\end{equation}
    Writing $u(x)$ as an elementary integral in $\lambda$ and then using Fubini, we write for the last display
    \begin{equation}
    =\int_X \int _0^{u(x)}
    p \lambda^{p-1} d\lambda\, d\mu(x)
\end{equation}
\begin{equation}
 =p\int _0^{\infty}
    \lambda^{p-1} \mu(\{x: u(x)\ge \lambda\}) d\lambda\, .
\end{equation}
This proves the lemma.
\end{proof}

We turn to the proof of \Cref{Hardy-Littlewood}.
Let the collection $\mathcal{B}$ be given.
We first show \eqref{eq-besico}.



We recursively choose a finite sequence $B_i\in \mathcal{B}$
for $i\ge 0$ as follows. Assume $B_{i'}$
is already chosen for $0\le i'<i$.
If there exists a ball $B_{i}\in \mathcal{B}$ so that $B_{i}$
is disjoint from all $B_{i'}$
with $0\le i'<i$, then choose
such a ball $B_i=B(x_i,r_i)$ with maximal $r_i$.

If there is no such ball, stop the selection and set
$i'':=i$.

By disjointedness of the chosen balls and since $0 \le u$, we have
\begin{equation}
\sum_{0\le i<i''}\int_{B_i} u(x)\, d\mu(x) \le \int_X u(x)\, d\mu(x)\, .
\end{equation}
By \eqref{eq-ball-assumption}, we conclude
\begin{equation}\label{eqbes1}
\lambda \sum_{0\le i<i''}\mu(B_i)
\le \int_X u(x)\, d\mu(x)\, .
\end{equation}
Let $x\in \bigcup \mathcal{B}$.
Choose a ball $B'=B(x',r')\in \mathcal{B}$
such that $x\in B'$.
If $B'$ is one of the selected balls, then
\begin{equation}\label{3rone}
    x\in \bigcup _{0\le i< i''}B_i\subset \bigcup _{0\le i< i''}B(x_i,3r_i)\, .
\end{equation}
If $B'$ is not one of the selected balls, then as it is not selected at time $i''$, there is a selected ball $B_i$ with
$B'\cap B_i\neq \emptyset$.
Choose such $B_i$ with minimal index $i$. As $B'$ is therefore disjoint from all
balls $B_{i'}$ with $i'<i$ and
as it was not selected in place of $B_i$, we have $r_i\ge r'$.

Using a point $y$ in the intersection of $B_i$ and $B'$,
we conclude by the triangle inequality
\begin{equation}
   \rho(x_i,x')\le \rho(x_i,y)+\rho(x',y)\le r_i+r'\le 2r_i \, .
\end{equation}
By the triangle inequality again, we further conclude
\begin{equation}
   \rho(x_i,x)\le \rho(x_i,x')+\rho(x',x)\le 2r_i+r'\le 3r_i \, .
\end{equation}
It follows that
\begin{equation}\label{3rtwo}
    x\in \bigcup _{0\le i< i''}B(x_i,3r_i)\, .
\end{equation}
With \eqref{3rone} and \eqref{3rtwo}, we conclude
\begin{equation}
\bigcup \mathcal{B}\subset
\bigcup _{0\le i< i''}B(x_i,3r_i)\, .
\end{equation}
With the doubling property
\eqref{doublingx} applied twice, we conclude
\begin{equation}\label{eqbes2}
    \mu(\bigcup{\mathcal{B}})
    \le \sum _{0\le i< i''}\mu (B(x_i,3r_i))
    \le 2^{2a}\sum _{0\le i< i''}\mu (B_i)\, .
\end{equation}
With \eqref{eqbes1} and \eqref{eqbes2} we conclude
\eqref{eq-besico}.


We turn to the proof of \eqref{eq-hlm}. We first consider the case $p_1=1$ and recall $M_{\mathcal{B}}=M_{\mathcal{B},1}$.
We write for the $p_2$-th power of left-hand side of \eqref{eq-hlm}
with \Cref{layer-cake-representation}
and a change of variables
\begin{equation}
    \|M_{\mathcal{B}}u(x)\|_{p_2}^{p_2}
   =p_2\int _0^{\infty}
    \lambda^{p_2-1} \mu(\{x: M_{\mathcal{B}}u(x)\ge \lambda\}) d\lambda\,
\end{equation}
\begin{equation} \label{eqbesi11}
   =2^{p_2} p_2\int _0^{\infty}
    \lambda^{p_2-1} \mu(\{x: M_{\mathcal{B}}u(x)\ge 2\lambda\}) d\lambda\, .
\end{equation}
Fix $\lambda\ge 0$ and let $x\in X$ satisfy $M_{\mathcal{B}}u(x)\ge 2\lambda$. By definition of $M_{\mathcal{B}}$, there is a ball
$B'\in \mathcal{B}$ such that
$x\in B'$ and
\begin{equation}\label{eqbesi10}
\int_{B'} u(y)\, d\mu(y)\ge 2\lambda \mu({B'}) \, .
\end{equation}
Define
$u_\lambda(y):=0$ if $|u(y)|<\lambda$ and $u_\lambda(y):=u(y)$ if $|u(y)|\ge \lambda$.
Then with \eqref{eqbesi10}
\begin{equation}
\int_{B'} u_\lambda (y)\, d\mu(y)
=\int_{B'} u (y)\, d\mu(y)-
\int_{B'} (u-u_\lambda) (y) d\mu(y)\,
\end{equation}
\begin{equation}
\ge 2\lambda \mu({B'})-
\int_{B'} (u-u_\lambda) (y) d\mu(y)\, .
\end{equation}
As $(u-u_\lambda)(y)\le \lambda$
by definition, we can estimate the last display by
\begin{equation}
\ge 2\lambda \mu({B'})-
\int_{B'} \lambda \, d\mu(y)
=\lambda \mu({B'})\, .
\end{equation}
Hence $x$ is contained in
$\bigcup(\mathcal{B}_\lambda)$,
where $\mathcal{B}_\lambda$
is the collection of balls $B''$ in $\mathcal{B}$ such that
\begin{equation}
    \int_{B''} u_\lambda (y)\, d\mu(y)\ge \lambda \mu(B'')\, .
\end{equation}
We have thus seen
\begin{equation}
    \{x: M_{\mathcal{B}}u(x)\ge 2\lambda\}\subset
    \bigcup \mathcal{B}_\lambda
\, .
\end{equation}
Applying \eqref{eq-besico} to the collection $\mathcal{B}_\lambda$
gives
\begin{equation}
    \lambda \mu(\{x: M_{\mathcal{B}}u(x)\ge 2\lambda\})\le
   2^{2a}
    \int u_\lambda (x)\, dx\, .
\end{equation}
With \Cref{layer-cake-representation},
\begin{equation}\label{eqbesi12}
    \lambda \mu(\{x: M_{\mathcal{B}}u(x)\ge 2\lambda\})\le
   2^{2a}
    \int_0^\infty \mu (\{x: |u_\lambda (x)|\ge \lambda'\})\, d\lambda'\, .
\end{equation}
By definition of $h_\lambda$, making a case distinction between $\lambda\ge \lambda'$ and $\lambda <\lambda'$, we see that
\begin{equation}\label{eqbesi13}
   \mu (\{x: |u_\lambda (x)|\ge \lambda'\})
   \le
   \mu (\{x: |u (x)|\ge \max(\lambda,\lambda')\})\, .
\end{equation}
We obtain with \eqref{eqbesi11},
\eqref{eqbesi12}, and \eqref{eqbesi13}
\begin{equation}
    \|M_{\mathcal{B}}u(x)\|_{p_2}^{p_2}
 \end{equation}
 \begin{equation}
   \le 2^{p_2+2a} p_2
   \int_0^\infty \lambda^{p_2-2}
   \int_0^\infty
   \mu (\{x: |u (x)|\ge \max(\lambda,\lambda')\})
   \, d\lambda'd\lambda\, .
\end{equation}
We split the integral into $\lambda\ge \lambda'$ and $\lambda<\lambda'$ and resolve the
maximum correspondingly.
We have for $\lambda\ge \lambda'$
with \Cref{layer-cake-representation}
\begin{equation}
    \int_0^\infty \lambda^{p_2-2}
   \int_0^\lambda
   \mu (\{x: |u (x)|\ge \lambda\})
   \, d\lambda'd\lambda
\end{equation}
\begin{equation}
   =\int_0^\infty \lambda^{p_2-1}
     \mu (\{x: |u (x)|\ge \lambda\})
d\lambda.
\end{equation}
\begin{equation}\label{eqbesi14}
   =p_2^{-1} \|u\|_{p_2}^{p_2}\, .
\end{equation}
We have for $\lambda< \lambda'$
with Fubini and \Cref{layer-cake-representation}
\begin{equation}
    \int_0^\infty \lambda^{p_2-2}
   \int_\lambda^\infty
   \mu (\{x: |u(x)|\ge \lambda'\})
   \, d\lambda'd\lambda.
\end{equation}
\begin{equation}
   =\int_0^\infty \int_0^{\lambda'}\lambda^{p_2-2}
     \mu (\{x: |u (x)|\ge \lambda'\})
d\lambda d\lambda'.
\end{equation}
\begin{equation}
   =(p_2-1)^{-1}\int_0^\infty (\lambda')^{p_2-1}
     \mu (\{x: |u(x)|\ge \lambda'\})
d\lambda'.
\end{equation}
\begin{equation}\label{eqbesi15}
   =(p_2-1)^{-1} p_2^{-1}\|u\|_{p_2}^{p_2}\, .
\end{equation}
Adding the two estimates
\eqref{eqbesi14} and \eqref{eqbesi15} gives
\begin{equation}
    \|M_{\mathcal{B}}u(x)\|_{p_2}^{p_2}
   \le 2^{p_2+2a} (1+(p_2-1)^{-1})\|u\|_{p_2}^{p_2}
   = 2^{p_2+2a} p_2(p_2-1)^{-1}\|u\|_{p_2}^{p_2}
   \, .
   \end{equation}
With $a\ge 1$ and $p_2>1$, taking the $p_2$-th root, we obtain \eqref{eq-hlm}.
We turn to the case of general
$1\le p_1<p_2$.
We have
\begin{equation}
    M_{\mathcal{B},p_1}u=(M_{\mathcal{B}} (|u|^{p_1}))^{\frac 1{p_1}}\, .
\end{equation}
Applying the special case of \eqref{eq-hlm} for $M_{\mathcal{B}}$ gives
\begin{equation}
    \|M_{\mathcal{B},p_1}u\|_{p_2}=
    \|M_{\mathcal{B}} (|u|^{p_1})\|_{p_2/p_1}^{\frac 1{p_1}}
\end{equation}
\begin{equation}
    \le 2^{2a} (p_2/p_1) (p_2/p_1-1)^{-1}
    \|(|u|^{p_1})\|_{p_2/p_1}^{\frac 1{p_1}}
    =2^{2a} p_2(p_2-p_1)^{-1}\|u\|_{p_2}\, .
\end{equation}
This proves \eqref{eq-hlm} in general.

Now we construct the operator $M$ satisfying \eqref{eq-ball-av} and \eqref{eq-hlm-2}.

\begin{lemma}[covering separable space]
    \label{covering-separable-space}
    For each $r > 0$, there exists a countable collection $C(r) \subset X$ of points such that
    $$
        X \subset \bigcup_{c \in C(r)} B(c, r)\,.
    $$
\end{lemma}

\begin{proof}
    It clearly suffices to construct finite collections $C(r,k)$ such that
    $$
        B(o, r2^k) \subset \bigcup_{c \in C(r,k)} B(c,r)\,,
    $$
    since then the collection $C(r) = \bigcup_{k \in \mathbb{N}} C(r,k)$ has the desired property.

    Suppose that $Y \subset B(o, r2^k)$ is a collection of points such that for all $y, y' \in Y$ with $y \ne y'$, we have $\rho(y,y') \ge r$. Then the balls $B(y, r/2)$ are pairwise disjoint and contained in $B(o, r2^{k+1})$. If $y \in B(o, r)$, then $B(o, r2^{k+1}) \subset B(y, r2^{k+2})$. Thus, by the doubling property \eqref{doublingx},
    $$
        \mu(B(y, \frac{r}{2})) \ge 2^{-(k+2)a} \mu(B(o, r2^{k+1}))\,.
    $$
    Thus, we have
    $$
        \mu(B(o, r2^{k+1})) \ge \sum_{y \in Y} \mu(B(y, \frac{r}{2})) \ge |Y| 2^{-(k+2)a} \mu(B(o, r2^{k+1}))\,.
    $$
    We conclude that $|Y| \le 2^{(k+2)a}$. In particular, there exists a set $Y$ of maximal cardinality. Define $C(r,k)$ to be such a set.

    If $x \in B(o, r2^k)$ and $x \notin C(r,k)$, then there must exist $y \in C(r,k)$ with $\rho(x,y) < r$. Thus $C(r,k)$ has the desired property.
\end{proof}

For each $k \in \mathbb{N}$ we choose a countable set $C(2^k)$ as in the lemma.
Define
$$
    \mathcal{B}_\infty = \{B(c, 2^k) \ : \ c \in C(2^k), k \in \mathbb{N}\}\,.
$$
By \Cref{covering-separable-space}, this is a countable collection of balls. We choose an enumeration $\mathcal{B}_\infty = \{B_1, \dotsc\}$ and define
$$
    \mathcal{B}_n = \{B_1, \dotsc, B_n\}\,.
$$
We define
$$
    Mw := 2^{2a}\sup_{n \in \mathbb{N}} M_{\mathcal{B}_n}w\,.
$$
This function is measurable for each measurable $w$, since it is a countable supremum of measurable functions. Estimate \eqref{eq-hlm-2} follows immediately from \eqref{eq-hlm} and the monotone convergence theorem.

It remains to show \eqref{eq-ball-av}. Let $B = B(x, r) \subset X$. Let $k$ be the smallest integer such that $2^k \ge r$, in particular we have $2^k < 2r$. By definition of $C(2^k)$, there exists $c \in C(2^k)$ with $x \in B(c, 2^k)$. By the triangle inequality, we have $B(c, 2^k) \subset B(x, 4r)$, and hence by the doubling property \eqref{doublingx}
$$
    \mu(B(c, 2^k)) \le 2^{2a} \mu(B(x,r))\,.
$$
It follows that for each $z \in B(x,r)$
\begin{align*}
    \frac{1}{\mu(B(x,r))}\int_{B(x,r)} |w(y)| \, \mathrm{d}\mu(y) &\le \frac{2^{2a}}{\mu(B(c,2^k))}\int_{B(c,2^k)} |w(y)| \, \mathrm{d}\mu(y) \\
    &\le Mw(z)\,.
\end{align*}
This completes the proof.


\chapter{Proof of The Classical Carleson Theorem}



The convergence of partial Fourier sums is proved in
\Cref{10classical} in two steps. In the first step,
we establish convergence on a suitable dense subclass of functions. We choose smooth functions as subclass, the convergence is stated in \Cref{convergence-for-smooth} and proved in \Cref{10smooth}.
In the second step, one controls the relevant error of approximating a general function by a function in
the subclass. This is stated in \Cref{control-approximation-effect} and proved
in \Cref{10difference}.
The proof relies on a bound on the real Carleson maximal operator stated in \Cref{real-Carleson} and proved in \Cref{10carleson}.
This latter proof refers to the main Carleson \Cref{metric-space-Carleson}. Two assumptions in \Cref{metric-space-Carleson} require more work. The boundedness of the nontangential maximal operator $T^*$ defined in \eqref{def-tang-unm-op} is established in \Cref{nontangential-Hilbert} using $L^2$ and weak $L^1$ bounds for the Hilbert transform, Lemmas
\ref{Hilbert-strong-2-2} and \ref{Hilbert-weak-1-1}. These lemmas are proved in Subsections \ref{subsec-cotlar},
\ref{10hilbert} and \ref{subsec-CZD}.
The cancellative property is verified by \Cref{van-der-Corput}, which is proved in \Cref{10vandercorput}.
Several further auxiliary lemmas are stated
and proved in Subsection
\ref{10classical}, the proof of one of these auxiliary lemmas, \Cref{spectral-projection-bound}, is done in \Cref{10projection}.

All subsections past \Cref{10classical} are mutually independent.

 Subsections \ref{subsec-cotlar} uses bounds for the Hardy--Littlewood maximal function on the real line. There may be a better path through Lean than we choose here.
 We refer to \Cref{Hardy-Littlewood}, the assumptions of it,
 namely that the real line fits into the setting of \Cref{overviewsection},
 is done in \Cref{10carleson}.



\section{The classical Carleson theorem}
\label{10classical}

Let a uniformly continuous $2\pi$-periodic function $f:\R\to \mathbb{C}$ and $0<\epsilon<1$ be given.
Let
\begin{equation}
    C_{a,q} := \frac{2^{450a^3}}{(q-1)^5}
\end{equation}
denote the constant from \Cref{metric-space-Carleson}.
Define
\begin{equation}
    \epsilon' := \frac \epsilon {4 (C_{4,2} \cdot (8 / (\pi\epsilon))^{\frac 1 2} + \pi)}.
\end{equation}
By uniform continuity of $f$, there is a $0<\delta<\pi$
such that for all $x,x' \in \R$ with $|x-x'|\le \delta$
we have
\begin{equation}\label{uniconbound}
|f(x)-f(x')|\le \epsilon' \, .
\end{equation}
Define
\begin{equation}\label{def-fzero}
f_0:=f \ast \phi_\delta,
\end{equation}
where $\phi_\delta$ is a nonnegative smooth bump function with $\supp (\phi_\delta) \subset (-\delta, \delta)$ and $\int _\R \phi_\delta (x) \, dx = 1$.

\begin{lemma}[smooth approximation]
    \label{smooth-approximation}
    \leanok
    \lean{}
    The function $f_0$ is $2\pi$-periodic.
    The function $f_0$ is smooth (and therefore measurable).
    The function $f_0$ satisfies for all $x\in \R$:
    \begin{equation}\label{eq-ffzero}
    |f(x)-f_0(x)|\le \epsilon' \, ,
    \end{equation}
\end{lemma}

\begin{proof}
    \leanok
    Most of this is part of the Lean library.
\end{proof}

We prove in \Cref{10smooth}:
\begin{lemma}[convergence for smooth]
    \label{convergence-for-smooth}
    \leanok
    \lean{}
    \uses{convergence-for-twice-contdiff}
    There exists some $N_0 \in \N$ such that for all $N>N_0$ and $x\in [0,2\pi]$ we have
    \begin{equation}
        |S_N f_0 (x)- f_0(x)|\le \frac \epsilon 4\, .
    \end{equation}
\end{lemma}

We prove in \Cref{10difference}:
\begin{lemma}[control approximation effect]
    \label{control-approximation-effect}
    \leanok
    \lean{}
    \uses{real-Carleson}
    There is a set $E \subset \R$ with Lebesgue measure
    $|E|\le \epsilon$ such that for all
    \begin{equation}
        x\in [0,2\pi)\setminus E
    \end{equation}
   we have
   \begin{equation}
    \label{eq-max-partial-sum-diff}
    \sup_{N\ge 0} |S_Nf(x)-S_Nf_0(x)| \le \frac \epsilon 4\,.
    \end{equation}
\end{lemma}

We are now ready to prove classical Carleson:
\begin{proof} [Proof of \Cref{classical-Carleson}]
\proves{classical-Carleson}
\leanok
Let $N_0$ be as in \Cref{convergence-for-smooth}.
For every
\begin{equation}
x\in [0, 2\pi) \setminus E\, ,
\end{equation}
and every $N>N_0$ we have by the triangle inequality
\begin{equation*}
    |f(x)-S_Nf(x)|
    \end{equation*}
    \begin{equation}\label{epsilonthird}
    \le |f(x)-f_0(x)|+ |f_0(x)-S_Nf_0(x)|+|S_Nf_0(x)-S_N f(x)|\, .
\end{equation}
Using \Cref{smooth-approximation,convergence-for-smooth,control-approximation-effect}, we estimate \eqref{epsilonthird} by
\begin{equation}
    \le \epsilon' +\frac \epsilon 4 +\frac \epsilon 4\le \epsilon\, .
\end{equation}
This shows \eqref{aeconv} for the given $E$ and $N_0$.
\end{proof}

Let $\kappa:\R\to \R$ be the function defined by
$\kappa(0)=0$ and for $0<|x|<1$
\begin{equation}\label{eq-hilker}
\kappa(x)=\frac { 1-|x|}{1-e^{ix}}\,
\end{equation}
and for $|x|\ge 1$
\begin{equation}\label{eq-hilker1}
\kappa(x)=0\, .
\end{equation}
Note that this function is continuous at every point $x$ with $|x|>0$.

The proof of \Cref{control-approximation-effect} will
use the following \Cref{real-Carleson}, which itself is proven
 in \Cref{10carleson} as an application of
 \Cref{metric-space-Carleson}.
\begin{lemma}[real Carleson]\label{real-Carleson}
\leanok
\lean{}
\uses{metric-space-Carleson,real-line-metric,real-line-measure,real-line-doubling,oscillation-control,frequency-monotone,frequency-ball-doubling,frequency-ball-growth,integer-ball-cover,real-van-der-Corput,nontangential-Hilbert}
    Let $F,G$ be Borel subsets of $\R$ with finite measure. Let $f$ be a bounded measurable function on $\R$ with $|f|\le \mathbf{1}_F$. Then
\begin{equation}
    \left|\int _G Tf(x) \, dx\right| \le C_{4,2} |F|^{\frac 12} |G|^{\frac 12} \, ,
\end{equation}
where
\begin{equation}
    \label{define-T-carleson}
    T f(x)=\sup_{n\in \mathbb{Z}}
    \sup_{r>0}\left|\int_{r<|x-y|<1} f(y)\kappa(x-y) e^{iny}\, dy\right|\, .
\end{equation}
\end{lemma}


One of the main assumption of \Cref{metric-space-Carleson}, concerning the operator $T_*$ defined in \eqref{def-tang-unm-op}, is verified by the following lemma, which is proved in \Cref{subsec-cotlar}.


\begin{lemma}[nontangential Hilbert]\label{nontangential-Hilbert}
\uses{simple-nontangential-Hilbert}
    For every bounded measurable function $g$ with bounded support we have
\begin{equation}\label{concretetstarbound}
    \|T_*g\|_2\le 2^{43}\|g\|_2,
\end{equation}
where
\begin{equation}\label{concretetstar}
    T_* g(x):=\sup_{0<r_1<r_2<1}\sup_{|x-x'|<r_1}\frac 1{2\pi} \left|\int_{r_1<|x'-y|<r_2}
g(y) \kappa(x'-y)\, dy\right|\, .
\end{equation}
\end{lemma}
The proof of \Cref{nontangential-Hilbert} relies on the next two auxiliary Lemmas.

For $r\in (0,1)$, $x\in \mathbb{R}$ and a bounded, measurable function $g$ on $\mathbb{R}$ with bounded support, we define
\begin{equation}
\label{def-H_r}
H_r g(x):= \int_{r<|x-y|<1}
g(y) \kappa(x-y)\, dy.
\end{equation}

The following Lemma is proved in \Cref{10hilbert}


\begin{lemma}[Hilbert strong 2 2]
    \label{Hilbert-strong-2-2}
    \uses{modulated-averaged-projection,integrable-bump-convolution,dirichlet-approximation}
    Let $0<r<1$. Let $f$ be a bounded, measurable function on $\mathbb{R}$ with bounded support. Then
    \begin{equation}
        \label{eq-Hr-L2-bound}
        \|H_rf\|_{2}\leq 2^{13} \|f\|_2.
    \end{equation}
\end{lemma}

The following Lemma is proved in \Cref{subsec-CZD}

\begin{lemma}[Hilbert weak 1 1]
    \label{Hilbert-weak-1-1}
    \uses{Hilbert-strong-2-2,Hilbert-kernel-bound,Hilbert-kernel-regularity,Calderon-Zygmund-decomposition}
    Let $f$ be a bounded measurable function on $\mathbb{R}$ with bounded support. Let $\alpha>0$. Then for all $r\in (0, 1)$, we have
    \begin{equation}
        \label{eq-weak-1-1}
        \mu\left(\{x\in \mathbb{R}: |H_r f(x)|>\alpha\}\right)\leq \frac{2^{19}}{\alpha} \int |f(y)|\, dy.
    \end{equation}
\end{lemma}

The next lemma will be used to verify that the collection $\Mf$ of modulation functions in our application of \Cref{metric-space-Carleson} satisfies the condition \eqref{eq-vdc-cond}.
It is proved in \Cref{10vandercorput}.

\begin{lemma}[van der Corput]
\label{van-der-Corput}
    Let $\alpha<\beta$ be real numbers. Let $g:\R\to \C$ be a measurable function and assume
    \begin{equation}
        \|g\|_{Lip(\alpha,\beta)}:=\sup_{\alpha\le x\le \beta}|g(x)|+|\beta-\alpha|
        \sup_{\alpha\le x<y\le \beta} \frac {|g(y)-g(x)|}{|y-x|}<\infty\, .
    \end{equation}
    Then for any $0\le \alpha<\beta\le 2\pi$ we have
    \begin{equation}
        \int _{\alpha}^{\beta} g(x) e^{-inx}\, dx\le 2\pi |\beta-\alpha|\|g\|_{Lip(\alpha,\beta)}(1+|n||\beta-\alpha|)^{-1}\, .
    \end{equation}

\end{lemma}


We close this section with six lemmas that are used
across the following subsections.

\begin{lemma}[mean zero oscillation]
\label{mean-zero-oscillation}
Let $n\in \mathbb{Z}$ with $n\neq 0$, then
\begin{equation}
\int_0^{2\pi} e^{inx}\, dx=0\,.
\end{equation}
\end{lemma}
\begin{proof}
We have
\begin{equation*}
\int_0^{2\pi} e^{inx}\, dx=\left[ \frac 1{in}e^{inx}\right]_0^{2\pi}=\frac 1{in}(e^{2\pi i n}-e^{2\pi i 0})=\frac 1{in}(1-1)=0\, . \qedhere
\end{equation*}

\end{proof}

\begin{lemma}[Dirichlet kernel]
\label{dirichlet-kernel}
\leanok
\lean{dirichletKernel_eq, partialFourierSum_eq_conv_dirichletKernel}
We have for every $2\pi$-periodic bounded measurable $f$ and every $N\ge 0$
\begin{equation}
    S_Nf(x)=\frac 1{2\pi}\int_{0}^{2\pi}f(y) K_N(x-y)\, dy
\end{equation}
where $K_N$ is the $2\pi$-periodic continuous function of
$\R$ given by
\begin{equation}\label{eqksumexp}
\sum_{n=-N}^N e^{in x'}\, .
\end{equation}
We have for $e^{ix'}\neq 1$ that
\begin{equation}\label{eqksumhil}
    K_N(x')=\frac{e^{iNx'}}{1-e^{-ix'}}
      +\frac {e^{-iNx'}}{1-e^{ix'}} \, .
\end{equation}


\end{lemma}


\begin{proof}
\leanok
We have by definitions and interchanging sum and integral
   \begin{equation*}
        S_Nf(x)=\sum_{n=-N}^N \widehat{f}_n e^{inx}
    \end{equation*}
       \begin{equation*}
    =\sum_{n=-N}^N \frac 1{2\pi}\int_{0}^{2\pi}
    f(x) e^{in(x-y)}\, dy
    \end{equation*}
 \begin{equation}\label{eq-expsum}
     =\frac 1{2\pi}\int_{0}^{2\pi}
    f(y) \sum_{n=-N}^N e^{in(x-y)}\, dy\, .
 \end{equation}
This proves the first statement of the lemma.
By a telescoping sum, we have for every $x'\in \R$
\begin{equation}
    \left( e^{\frac 12 ix'}-e^{-\frac 12 ix'}\right) \sum_{n=-N}^N e^{inx'}= e^{(N+\frac 12) ix'}-e^{-(N+\frac 12) ix'}\, .
\end{equation}
If $e^{ix'}\neq 1$, the first factor on the left-hand side is not $0$ and we may divide by this factor to obtain
\begin{equation}
      \sum_{n=-N}^N e^{inx'}= \frac{e^{i(N+\frac 1 2)x'}}{e^{\frac 12 ix'}-e^{-\frac 12ix'}}
      -\frac{e^{-i(N+\frac 1 2)x'}}{e^{\frac 12 ix'}-e^{-\frac 12ix'}}
       =\frac{e^{iNx'}}{1-e^{-ix'}}
      +\frac {e^{-iNx'}}{1-e^{ix'}}\, .
\end{equation}
This proves the second part of the lemma.
\end{proof}

\begin{lemma}[lower secant bound]
    \label{lower-secant-bound}
    \leanok
    \lean{lower_secant_bound'}
    Let $\eta>0$ and $-2\pi +\eta \le x\le 2\pi-\eta$ with $|x|\ge \eta$. Then
    \begin{equation}
        |1-e^{ix}|\ge \frac{2}{\pi} \eta
    \end{equation}
\end{lemma}
\begin{proof}
    \leanok
    We have
    $$
        |1 - e^{ix}| = \sqrt{(1 - \cos(x))^2 + \sin^2(x)} \ge |\sin(x)|\,.
    $$
    If $0 \le x \le \frac{\pi}{2}$, then we have from concavity of $\sin$ on $[0, \pi]$ and $\sin(0) = 0$ and $\sin(\frac{\pi}{2}) = 1$
    $$
        |\sin(x)| \ge \frac{2}{\pi} x \ge \frac{2}{\pi} \eta\,.
    $$
    When $x\in \frac{m\pi}{2} + [0, \frac{\pi}{2}]$ for $m \in \{-4, -3, -2, -1, 1, 2, 3\}$ one can argue similarly.
\end{proof}

The following lemma will be proved in \Cref{10projection}.

\begin{lemma}[spectral projection bound]
\label{spectral-projection-bound}
\uses{partial-sum-projection,partial-sum-selfadjoint}
    Let $f$ be a bounded $2\pi$-periodic measurable function. Then, for all $N\ge 0$
   \begin{equation}\label{snbound}
   \|S_Nf\|_{L^2[-\pi, \pi]} \le \|f\|_{L^2[-\pi, \pi]}.
   \end{equation}
\end{lemma}

\begin{lemma}[Hilbert kernel bound]
\label{Hilbert-kernel-bound}
\leanok
\lean{}
\uses{lower-secant-bound}
    For $x,y\in \R$ with $x\neq y$ we have
    \begin{equation}\label{eqcarl30}
        |\kappa(x-y)|\le 2^2(2|x-y|)^{-1}\, .
    \end{equation}
\end{lemma}
\begin{proof}
\leanok
    Fix $x\neq y$. If $\kappa(x-y)$ is zero, then \eqref{eqcarl30} is evident. Assume $\kappa(x-y)$ is not zero, then $0<|x-y|<1$.
    We have
\begin{equation}\label{eqcarl31}
|\kappa(x-y)|=\left|\frac {1-|x-y|}{1-e^{i(x-y)}}\right|\, .
\end{equation}
We estimate
with \Cref{lower-secant-bound}
\begin{equation}\label{eqcarl311}
|\kappa(x-y)|\le \frac {1}{|1-e^{i(x-y)}|}\le \frac 2{|x-y|}\, .
\end{equation}
This proves \eqref{eqcarl30} in the given case and completes the proof of the lemma.
\end{proof}

\begin{lemma}[Hilbert kernel regularity]
\label{Hilbert-kernel-regularity}
\leanok
\lean{}
\uses{lower-secant-bound}
    For $x,y,y'\in \R$ with $x\neq y,y'$ and
    \begin{equation}
        \label{eq-close-hoelder}
        2|y-y'|\le |x-y|\, ,
    \end{equation}
    we have
    \begin{equation}\label{eqcarl301}
        |\kappa(x-y) - \kappa(x-y')|\le 2^{8}\frac{1}{|x-y|} \frac{|y-y'|}{|x-y|}\, .
    \end{equation}
\end{lemma}

\begin{proof}
\leanok
    Upon replacing $y$ by $y-x$ and $y'$ by $y'-x$ on the left-hand side of \eqref{eq-close-hoelder}, we can assume that $x = 0$. Then the assumption \eqref{eq-close-hoelder} implies that $y$ and $y'$ have the same sign. Since $\kappa(y) = \bar \kappa(-y)$ we can assume that they are both positive. Then it follows from \eqref{eq-close-hoelder} that
    $$
        \frac{y}{2} \le y' \,.
    $$
    We distinguish four cases. If $y, y' \le 1$, then we have
    $$
        |\kappa(-y) - \kappa(-y')| = \left| \frac{1 - y}{1- e^{-iy}} - \frac{1 - y'}{1- e^{-iy'}}\right|
    $$
    and by the fundamental theorem of calculus
    $$
        = \left| \int_{y'}^{y} \frac{-1 + e^{-it} + i(1-t)e^{it}}{(1 - e^{-it})^2} \,dt \right|\,.
    $$
    Using $y' \ge \frac{y}{2}$ and \Cref{lower-secant-bound}, we bound this by
    $$
        \le |y - y'| \sup_{\frac{y}{2} \le t \le 1} \frac{3}{|1 - e^{-it}|^2} \le 3 |y-y'| (2 \frac{2}{y})^2 \le 2^{6} \frac{|y-y'|}{|y|^2}\,.
    $$
    If $y \le 1$ and $y' > 1$, then $\kappa(-y') = 0$ and we have from the first case
    $$
        |\kappa(-y) - \kappa(-y')| = |\kappa(-y) - \kappa(-1)| \le 2^{6} \frac{|y-1|}{|y|^2} \le 2^{6} \frac{|y-y'|}{|y|^2}\,.
    $$
    Similarly, if $y > 1$ and $y' \le 1$, then $\kappa(-y) = 0$ and we have from the first case
    $$
        |\kappa(-y) - \kappa(-y')| = |\kappa(-y') - \kappa(-1)| \le 2^{6} \frac{|y'-1|}{|y'|^2} \le 2^{6} \frac{|y-y'|}{|y'|^2}\,.
    $$
    Using again $y' \ge \frac{y}{2}$, we bound this by
    $$
        \le 2^{6} \frac{|y-y'|}{|y / 2|^2} = 2^{8} \frac{|y-y'|}{|y|^2}
    $$
    Finally, if $y, y' > 1$ then
    $$
        |\kappa(-y) - \kappa(-y')| = 0 \le 2^{8} \frac{|y-y'|}{|y|^2}\,.
    $$
\end{proof}






\section{Smooth functions.}
\label{10smooth}
\begin{lemma}
\label{fourier-coeff-derivative}
\leanok
\lean{}
Let $f:\R \to \C$ be $2\pi$-periodic and differentiable, and let $n \in \Z \setminus \{0\}$. Then
\begin{equation}
    \widehat{f}_n = \frac{1}{i n} \widehat{f'}_n.
\end{equation}
\end{lemma}
\begin{proof}
\leanok
This is part of the Lean library.
\end{proof}

\begin{lemma}
\label{convergence-of-coeffs-summable}
\leanok
\lean{}
Let $f:\R \to \C$ such that
\begin{equation}
    \sum_{n\in \Z} |\widehat{f}_n| < \infty.
\end{equation}
Then
\begin{equation}
    \sup_{x\in [0,2\pi]} |f(x) - S_Nf(x)| \rightarrow 0
\end{equation}
as $N \rightarrow \infty$.
\end{lemma}

\begin{proof}
\leanok
    This is part of the Lean library.
\end{proof}

\begin{lemma}
    \label{convergence-for-twice-contdiff}
    \uses{fourier-coeff-derivative,convergence-of-coeffs-summable}
    \leanok
    \lean{}
    Let $f:\R \to \C$ be $2\pi$-periodic and twice continuously differentiable. Then
    \begin{equation}
        \sup_{x\in [0,2\pi]} |f(x) - S_Nf(x)| \rightarrow 0
    \end{equation}
    as $N \rightarrow \infty$.
\end{lemma}
\begin{proof}
\leanok
By \Cref{convergence-of-coeffs-summable}, it suffices to show that the Fourier coefficients $\widehat{f}_n$ are summable.
Applying \Cref{fourier-coeff-derivative} twice and using the fact that $f''$ is continuous and thus bounded on $[0,2\pi]$ , we compute
\begin{equation*}
    \sum_{n\in \Z} |\widehat{f}_n| = |\widehat{f}_0| + \sum_{n\in \Z \setminus \{0\}} \frac {1}{n^2} |\widehat{f''}_n|
    \le |\widehat{f}_0| + \left(\sup_{x\in [0,2\pi]} |f(x)| \right) \cdot \sum_{n\in \Z \setminus \{0\}} \frac {1}{n^2}
    < \infty.
\end{equation*}
\end{proof}

\begin{proof}
\leanok
\proves{convergence-for-smooth}
    \Cref{convergence-for-smooth} now follows directly from the previous \Cref{convergence-for-twice-contdiff}.
\end{proof}

\section{Proof of Cotlar's Inequality}
\label{subsec-cotlar}




\begin{lemma}[estimate x shift]
\label{estimate-x-shift}
\uses{Hilbert-kernel-bound,Hilbert-kernel-regularity}
Let $0<r<1$ and $x\in \mathbb{R}$. Let $g$ be a bounded measurable function with bounded support on $\mathbb{R}$. Let $Mg$ be the Hardy--Littlewood function
defined in Proposition
\ref{Hardy-Littlewood}.
Then for all $x'$ with $|x-x'|<r$.
\begin{equation}\label{xx'difference}
\left|\int_{r<|x-y|<1}
g(y) \kappa(x-y)\, dy
-\int_{r<|x'-y|<1}
g(y) \kappa(x'-y)\, dy
\right|\le 2^{13}Mg(x)\, .
\end{equation}
\end{lemma}
\begin{proof}
\proves{estimate-x-shift}

First note that the
conditions $|x-y|<1$
and $|x'-y|<1$ may be removed as the factor containing $\kappa$
vanishes without these conditions.

We split the first integral in \eqref{xx'difference}
into the domains $r<|x-y|\le 2r$
and $2r<|x-y|$. The integral over the first domain we estimate
by $\eqref{firstxx'}$ below.
For the second domain, we
observe with $|x-x'|<r$ and the triangle inequality that $r<|x'-y|$. We therefore combine on this domain with the
corresponding part of the second integral in \eqref{xx'difference} and estimate that by $\eqref{secondxx'}$
below. The remaining part of the second integral in
\eqref{xx'difference} we estimate by $\eqref{thirdxx'}$.
Overall, we have estimated \eqref{xx'difference}
by


\begin{equation}
\label{firstxx'}
\int_{r<|x-y|\le 2r}
|g(y)| |\kappa(x-y)|\, dy
\end{equation}
\begin{equation}\label{secondxx'}
  + \left|\int_{2r<|x-y|}
g(y) (\kappa(x'-y)-\kappa(x-y))\, dy
\right|
\end{equation}
\begin{equation}
\label{thirdxx'}
  +\left|\int_{r<|x'-y|, r<|x-y|<2r}
|g(y) \kappa(x'-y)|\, dy
\right|
\end{equation}
Using the bound on $\kappa$ in \Cref{Hilbert-kernel-bound}, we estimate
\eqref{firstxx'} by
\begin{equation}
\label{firstxx'b}
 \frac 8 r\int_{|x-y|<2r}
|g(y)|\, dy\, .
\end{equation}
Using the definition of $Mg$, we estimate
\eqref{firstxx'b} by
\begin{equation}
\label{firstxx'c}
 \le 32{Mg(x)}\, .
\end{equation}
Similarly, in the domain of \eqref{thirdxx'}
we note by the triangle inequality
and assumption on $x'$ that $|x'-y|\le 3r$ and thus we estimate
\eqref{thirdxx'}
by
\begin{equation}
\label{thirdxx'b}
 \frac 8{r}\int_{|x'-y|<3r}
|g(y)| \, dy\le 48 Mg(x)
\end{equation}

We turn to the term \eqref{secondxx'}.
Let $\nu$ be the smallest integer
such that $2^\nu> 2/r$.
Using that the kernel vanishes unless $|x-y|<2$, we decompose and estimate
\eqref{secondxx'}
with the triangle inequality by
\begin{equation}\label{secondxx'b}
 \sum_{j=1}^\nu \int_{2^jr<|x-y|\le 2^{j+1}r}
|g(y)| |\kappa(x'-y)-\kappa(x-y)|\, dy\,.
\end{equation}
Using \Cref{Hilbert-kernel-regularity}, we estimate \eqref{secondxx'b}
by
\begin{equation*}
 2^{10}\sum_{j=1}^\nu \int_{2^jr<|x-y|\le 2^{j+1}r}
|g(y)| \frac{|x-x'|}{|x-y|^2}\, dy
\end{equation*}
\begin{equation*}
\le 2^{10}\sum_{j=1}^\nu \frac 1{2^{2j}r}\int_{2^jr<|x-y|\le 2^{j+1}r}
|g(y)| \, dy
\end{equation*}
\begin{equation}\label{secondxx'c}
\le 2^{10}\sum_{j=1}^\nu {2^{1-j}} Mg(x)\,.
\end{equation}
Using a geometric series, we estimate
\eqref{secondxx'c} by
\begin{equation}\label{secondxx'd}
\le 2^{11} Mg(x)\,.
\end{equation}
Summing the estimates
for \eqref{firstxx'},
\eqref{secondxx'}, and
\eqref{thirdxx'}
proves the lemma.
\end{proof}

Recall that
\begin{equation*}
    H_rg(x)=\int_{r<|x-y|<1} g(y) \kappa(x-y)\, dy\, .
\end{equation*}


\begin{lemma}[Cotlar control]
\label{Cotlar-control}
\uses{estimate-x-shift}
Let $0<r<r_1<1$ and $x\in \mathbb{R}$. Let $g$ be a bounded measurable function with bounded support on $\mathbb{R}$. Let $Mg$ be the Hardy--Littlewood function
defined in Proposition
\ref{Hardy-Littlewood}.
Then for all $x'\in \R$ with $|x'-x|<\frac {r_1}4$ we have
\begin{equation}\label{eq-cotlar-control}
\left|H_{r_1} g(x)
\right|\le
|H_{r}(g-g\mathbf{1}_{[x-\frac {r_1} 2,x+\frac {r_1}2]})(x')|+
2^{15}Mg(x)\, .
\end{equation}
\end{lemma}

\begin{proof}
Let $x$ and $x'$ be given with $|x'-x|<\frac {r_1}4$.
By an application of \Cref{estimate-x-shift}, we estimate
the left-hand-side of
\eqref{eq-cotlar-control} by
\begin{equation}\label{eqcotlar0}
|H_{r_1}(g)(x')|+
2^{13}Mg(x)\,.
\end{equation}
We have
\begin{equation}
\label{eqcotlar-1}
H_{r_1}(g)(x')=
\int_{r_1<|x'-y|<1} g(y) \kappa(x'-y)\, dy\, .
\end{equation}
On the domain $r_1<|x'-y|$, we have $\frac {r_1}2<|x-y|$. Hence we may write
for \eqref{eqcotlar-1}
\begin{equation*}
H_{r_1}(g)(x')=\int_{r_1<|x'-y|<1} (g-g\mathbf{1}_{[x-\frac {r_1} 2,x+\frac {r_1}2]})(y) \kappa(x'-y)\, dy
\end{equation*}
\begin{equation}\label{eqcotlar1}
=H_{r_1}(g-g\mathbf{1}_{[x-\frac {r_1} 2,x+\frac {r_1}2]})(x')\, .
\end{equation}
Combining the estimate \eqref{eqcotlar0} with the identification \eqref{eqcotlar1}, we obtain
\begin{equation}\label{eqcotlar5}
\left|H_{r_1} g(x)
\right|\le
|H_{r_1}(g-g\mathbf{1}_{[x-\frac {r_1} 2,x+\frac {r_1}2]})(x')|+
2^{13}Mg(x)\, .
\end{equation}
We have

\begin{equation*}
(H_{r}-H_{r_1})(g-g\mathbf{1}_{[x-\frac {r_1} 2,x+\frac {r_1}2]})(x')
\end{equation*}
\begin{equation}\label{eqcotlar2}
= \int_{ {r}<|x'-y|<r_1}
(g-g\mathbf{1}_{[x-\frac {r_1} 2,x+\frac {r_1}2]})(y)k(x'-y)\, dy\, .
\end{equation}
Assume first $r\ge \frac {r_1} 4$.
Then we estimate \eqref{eqcotlar2} with
\Cref{Hilbert-kernel-bound} by
\begin{equation*}
\int_{\frac {r_1}4<|x'-y|<r_1}
|g(y)k(x'-y)|\, dy
\end{equation*}
\begin{equation}\label{eqcotlar3}
\le \frac {32}{r_1}\int_{|x'-y|<r_1}
|g(y)|\, dy \le 64 Mg(x')\, .
\end{equation}
Assume now $r\le \frac {r_1} 4$.
As $|x'-y|\le \frac{r_1}4$ implies
$|x-y|\le \frac {r_1} 2$, we see that
\eqref{eqcotlar2} equals
\begin{equation*}
 \int_{ {\frac {r_1} 4}<|x'-y|<r_1}
(g-g\mathbf{1}_{[x-\frac {r_1} 2,x+\frac {r_1}2]})(y)k(x'-y)\, dy\, ,
\end{equation*}
which we again estimate as above by \eqref{eqcotlar3}.
In both cases, \eqref{eq-cotlar-control} follows by the triangle inequality from \eqref{eqcotlar5} and the estimate for
\eqref{eqcotlar2}.
\end{proof}

\begin{lemma}[Cotlar sets]
\label{Cotlar-sets}
\uses{Hilbert-weak-1-1}
Let $0<r<r_1<1$ and $x\in \mathbb{R}$. Let $g$ be a bounded measurable function with bounded support on $\mathbb{R}$. Let $Mg$ be the Hardy--Littlewood maximal function
defined in Proposition
\ref{Hardy-Littlewood}.
Let $x\in \R$.
Then the measure $|F_1|$ of the set $F_1$ of all $x'\in
[x-\frac{r_1}4,x+\frac{r_1}4]$ such that
\begin{equation}
\label{first-cotlar-exception}
    |H_{r}(g)(x')|>4M(H_{r}g)(x)
\end{equation}
is less than or equal to $r_1/8$.
Moreover, the measure $|F_2|$ of the set $F_2$ of all $x'\in
[x-\frac{r_1}4,x+\frac{r_1}4]$ such that
\begin{equation}
\label{second-cotlar-exception}
    |H_{r}(g\mathbf{1}_{[x-\frac {r_1} 2,x+\frac {r_1}2]})|>2^{22}Mg(x)
\end{equation}
is less than $r_1/8$.
\end{lemma}

\begin{proof}
Let $r$, $r_1$, $x$ and $g$ be given.
If $M(H_{r}g)(x)=0$, then $H_{r}g$ is constant zero
and $F_1$ is empty and the estimate on $F_1$ trivial.
Assume $M(H_{r}g)(x)>0$.
We have with \eqref{first-cotlar-exception}
\begin{equation}
    M(H_{r}g)(x)\ge
    \frac 2{r_1}\int_{|x'-x|<\frac {r_1}4}|H_{r}g(x')|\, dx'
\end{equation}
\begin{equation}
    \ge
    \frac 2{r_1}\int_{F_1} 4 M(H_{r}g)(x)\, dx'
\end{equation}
Dividing by $M(H_{r}g)(x)$ gives
\begin{equation}
    1\ge \frac 8{r_1} |F_1|\, .
\end{equation}
This gives the desired bound for the measure of $F_1$.

We turn to the set $F_2$. Similarly as above we may assume $Mg(x)>0$.
The set $F_2$ is then estimated with \Cref{Hilbert-weak-1-1}
by
\begin{equation}
   \frac {2^{19}}{2^{22}Mg(x)}\int |g\mathbf{1}_{[x-\frac {r_1}2, x+\frac {r_1}2]}|(y)\, dy
\end{equation}
\begin{equation}
   \le \frac {2^{19}}{2^{22}Mg(x)}r_1Mg(x)= \frac {r_1}8\,.
\end{equation}
This gives the desired bound for the measure of $F_2$.
\end{proof}

\begin{lemma}[Cotlar estimate]
\label{Cotlar-estimate}
\uses{Cotlar-control, Cotlar-sets}
Let $0<r<r_1<1$ and $x\in \mathbb{R}$. Let $g$ be a bounded measurable function with bounded support on $\mathbb{R}$. Let $Mg$ and $M(H_r g)$ be the respective Hardy--Littlewood maximal functions
defined in Proposition
\ref{Hardy-Littlewood}.
Then for all $x\in \R$
\begin{equation}\label{eq-cotlar-estimate}
\left|
\int_{r_1<|x-y|<1}
g(y) \kappa(x-y)\, dy
\right|
\end{equation}
\begin{equation}\le
2^{2}M(H_rg)(x)+ 2^{23} Mg(x)
\, .
\end{equation}
\end{lemma}


\begin{proof}
By \Cref{Cotlar-control}, the measure of the set
of all $x'\in
[x-\frac{r_1}4,x+\frac{r_1}4]$
such that at least one of the conditions
\eqref{first-cotlar-exception} and
\eqref{second-cotlar-exception} is satisfied is at most $r_1/4$ and hence not all of
 $x'\in
[x-\frac{r_1}4,x+\frac{r_1}4]$. Pick an
$x'\in
[x-\frac{r_1}4,x+\frac{r_1}4]$ such that both conditions are not satisfied.
Applying \Cref{Cotlar-control}
for this $x'$ and using the triangle inequality
estimates the left-hand side of \eqref{eq-cotlar-estimate}
by
\begin{equation}
    4M(H_rg)(x)+2^{22}Mg(x)+2^{15}Mg(x)\, .
\end{equation}
This proves the lemma.
\end{proof}

\begin{lemma}[simple nontangential Hilbert]\label{simple-nontangential-Hilbert}
\uses{Hardy-Littlewood,Hilbert-strong-2-2,Cotlar-estimate}
    For every $0<r<1$ and every bounded measurable function $g$ with bounded support we have
\begin{equation}\label{trzerobound}
    \|T_{r}g\|_2\le 2^{42}\|g\|_2,
\end{equation}
where
\begin{equation}\label{eq-simple--nontangential}
    T_{r} g(x):=\sup_{r<r_1<1}\sup_{|x-x'|<r_1}\frac 1{2\pi} \left|\int_{r_1<|x'-y|<1}
g(y) \kappa(x'-y)\, dy\right|\, .
\end{equation}
\end{lemma}
\begin{proof}
With \Cref{estimate-x-shift}
and the triangle inequality, we estimate
for every $x\in \R$
\begin{equation}
     T_{r} g(x)
     \le 2^{13} Mg(x)+\sup_{r<r_1<1}
     \frac 1{2\pi} \left|\int_{r_1<|x-y|<1}
g(y) \kappa(x-y)\, dy\right|\, .
\end{equation}
Using further \Cref{Cotlar-estimate},
we estimate
\begin{equation}
      T_{r} g(x)
     \le 2^{13} Mg(x)+2^{23}
     Mg(x)+2^{2}M(H_{r}g)(x)\, .
\end{equation}
Taking the $L^2$ norm and using \Cref{Hardy-Littlewood}with $a=4$ and $p_2=2$ and $p_1=1$ , we obtain
\begin{equation}
      \|T_{r} g\|_2
     \le 2^{24} \|Mg\|_2+2^{2}\|M(H_{r}g)\|_2
\end{equation}
\begin{equation}
     \le 2^{41} \|g\|_2+2^{19}\|H_{r}(g)\|_2\, .
\end{equation}
Applying \Cref{Hilbert-strong-2-2}, gives
\begin{equation}
      \|T_{r} g\|_2\le 2^{41} \|g\|_2+2^{32}\|g\|_2\, .
\end{equation}
This shows \eqref{trzerobound} and completes the proof of the lemma.
\end{proof}

\begin{proof}[Proof of \Cref{nontangential-Hilbert}]
    \proves{nontangential-Hilbert}
Fix $g$ as in the Lemma.
Applying \Cref{simple-nontangential-Hilbert} with a
sequence of $r$ tending to $0$ and using Lebesgue monotone convergence shows
\begin{equation}\label{tzerobound}
    \|T_{0}g\|_2\le 2^{42}\|g\|_2,
\end{equation}
where
\begin{equation}\label{eq-simpler--nontangential}
    T_{0} g(x):=\sup_{0<r_1<1}\sup_{|x-x'|<r_1}\frac 1{2\pi} \left|\int_{r_1<|x'-y|<1}
g(y) \kappa(x'-y)\, dy\right|\, .
\end{equation}
We now write by the triangle inequality
\begin{equation}\label{concretetstartriangle}
    T_* g(x)\le \sup_{0<r_1<r_2<1}\sup_{|x-x'|<r_1}\frac 1{2\pi} \left|\int_{r_1<|x'-y|<1}g(y) \kappa(x'-y)\, dy\right|
\end{equation}
    \begin{equation*}
+\sup_{0<r_1< r_2<1}\sup_{|x-x'|<r_1}\frac 1{2\pi} \left|\int_{r_2<|x'-y|<1} g(y) \kappa(x'-y)\, dy\right|\, .
\end{equation*}
Noting that the first integral does not depend on $r_2$ and
estimating the second integral by the larger supremum over all
$|x-x'|<r_2$, at which time the integral does not depend on $r_1$, we estimate \eqref{concretetstartriangle} by
\begin{equation}\label{concretetstartriangle2}
   \sup_{0<r_1<1}\sup_{|x-x'|<r_1}\frac 1{2\pi} \left|\int_{r_1<|x'-y|<1}g(y) \kappa(x'-y)\, dy\right|
\end{equation}
    \begin{equation*}
+\sup_{0< r_2<1}\sup_{|x-x'|<r_2}\frac 1{2\pi} \left|\int_{r_2<|x'-y|<1} g(y) \kappa(x'-y)\, dy\right|\, .
\end{equation*}
    Applying the triangle inequality on the left-hand side
    of \eqref{concretetstarbound} and applying
     \eqref{tzerobound} twice
    proves \eqref{concretetstarbound}.
    This completes the proof of \Cref{nontangential-Hilbert}.
\end{proof}





\section{The truncated Hilbert transform}
\label{10hilbert}





Let $M_n$ be the modulation operator
acting on measurable $2\pi$-periodic functions
defined by
\begin{equation}
    M_ng(x)=g(x) e^{inx}\, .
\end{equation}
Define the approximate Hilbert transform by
\begin{equation}
    L_N g=\frac 1N\sum_{n=0}^{N-1}
       M_{-n-N} S_{N+n}M_{N+n}g\, .
\end{equation}


\begin{lemma}[modulated averaged projection]
\label{modulated-averaged-projection}
\uses{spectral-projection-bound}
We have for every bounded measurable $2\pi$-periodic function $g$
\begin{equation}\label{lnbound}
    \|L_Ng\|_{L^2[-\pi, \pi]}\le \|g\|_{L^2[-\pi, \pi]}\,.
\end{equation}
\end{lemma}
\begin{proof}
    We have
    \begin{equation}\label{mnbound}
        \|M_ng\|_{L^2[-\pi, \pi]}^2=\int_{-\pi}^{\pi} |e^{inx}g(x)|^2\, dx
        =\int_{-\pi}^{\pi} |g(x)|^2\, dx=\|g\|_{L^2[-\pi, \pi]}^2\, .
    \end{equation}
     We have by the triangle inequality, the square root of the identity in \eqref{mnbound}, and \Cref{spectral-projection-bound}
    \begin{equation*}
        \|L_ng\|_{L^2[-\pi, \pi]}=\|\frac 1N\sum_{n=0}^{N-1}
       M_{-n-N} S_{N+n}M_{N+n}g\|_{L^2[-\pi, \pi]}
    \end{equation*}
    \begin{equation*}
        \le \frac 1N\sum_{n=0}^{N-1} \|
       M_{-n-N} S_{N+n}M_{N+n}g\|_{L^2[-\pi, \pi]}
         = \frac 1N\sum_{n=0}^{N-1} \|
    S_{N+n}M_{N+n}g\|_{L^2[-\pi, \pi]}
    \end{equation*}
       \begin{equation}
     \le \frac 1N\sum_{n=0}^{N-1} \|
 M_{N+n}g\|_{L^2[-\pi, \pi]} = \frac 1N\sum_{n=0}^{N-1} \|
g\|_{L^2[-\pi, \pi]} =\|g\|_{L^2[-\pi, \pi]}\, .
    \end{equation}
This proves \eqref{mnbound} and completes the proof of the lemma.
\end{proof}

\begin{lemma}[periodic domain shift]\label{periodic-domain-shift}
\lean{Function.Periodic.intervalIntegral_add_eq, intervalIntegral.integral_comp_sub_right}
\leanok
Let $f$ be a bounded $2\pi$-periodic function. We have for any
$0 \le x\le 2\pi$ that
\begin{equation}
 \int_0^{2\pi} f(y)\, dy= \int_{-x}^{2\pi -x} f(y)\, dy
 =\int_{-\pi}^{\pi} f(y-x)\, dy\,.
\end{equation}
\end{lemma}
\begin{proof}
\leanok
    We have by periodicity and change of variables
    \begin{equation}\label{eqhil9}
 \int_{-x}^{0} f(y)\, dy=\int_{-x}^{0} f(y+2\pi)\, dy= \int_{2\pi -x}^{2\pi} f(y)\, dy\, .
\end{equation}
We then have by breaking up the domain of integration
and using \eqref{eqhil9}
\begin{equation*}
 \int_0^{2\pi} f(y)\, dy= \int_0^{2\pi -x} f(y)\, dy+
 \int_{2\pi -x}^{2\pi} f(y)\, dy
 \end{equation*}
\begin{equation}
= \int_0^{2\pi -x} f(y)\, dy+
 \int_{ -x}^{0} f(y)\, dy
 = \int_{-x}^{2\pi-x} f(y)\, dy\, .
 \end{equation}
This proves the first identity of the lemma. The second identity follows by substitution of $y$ by $y-x$.
\end{proof}



\begin{lemma}[Young convolution]\label{Young-convolution}
\uses{periodic-domain-shift}
    Let $f$ and $g$ be two bounded non-negative measurable $2\pi$-periodic functions on $\R$. Then
    \begin{equation}\label{eqyoung}
        \left(\int_{-\pi}^{\pi} \left(\int_{-\pi}^{\pi}
        f(y)g(x-y)\, dy\right)^2\, dx\right)^{\frac 12}\le \|f\|_{L^2[-\pi, \pi]} \|g\|_{L^1[-\pi, \pi]}\, .
    \end{equation}
    \end{lemma}
\begin{proof}
Using Fubini and \Cref{periodic-domain-shift}, we observe
\begin{equation*}
  \int_{-\pi}^{\pi}\int_{-\pi}^{\pi}f(y)^2g(x-y)\, dy
    \, dx=\int_{-\pi}^{\pi}f(y)^2\int_{-\pi}^{\pi}g(x-y)\, dx
    \, dy
\end{equation*}
\begin{equation}\label{eqhil4}
=\int_{-\pi}^{\pi}f(y)^2\int_{-\pi}^{\pi}g(x) \, dx
     dy
=\|f\|_{L^2[-\pi, \pi]}^2\|g\|_{L^1[-\pi, \pi]}\, .
\end{equation}
   Let $h$ be the nonnegative square root of $g$, then
   $h$ is bounded and $2\pi$-periodic with $h^2=g$.
   We estimate the square of the left-hand side of
   \eqref{eqyoung} with Cauchy-Schwarz and then with
   \eqref{eqhil4} by
       \begin{equation*}
         \int_{-\pi}^{\pi} (\int_{-\pi}^{\pi}f(y)h(x-y)h(x-y)\, dy)^2\, dx
   \end{equation*}
\begin{equation*}
    \le \int_{-\pi}^{\pi}\left(\int_{-\pi}^{\pi}f(y)^2g(x-y)\, dy\right)
    \left(\int_{-\pi}^{\pi}g(x-y)\, dy\right)\, dx
\end{equation*}
\begin{equation*}
    = \|f\|_{L^2[-\pi, \pi]}^2\|g\|_{L^1[-\pi, \pi]}^2\, .
\end{equation*}
Taking square roots, this proves the lemma.
\end{proof}

For $0<r<1$, Define the kernel $k_r$ to be the $2\pi$-periodic function
\begin{equation}
    |k_r(x)|:=\min \left(r^{-1}, 1+\frac r{|1-e^{ix}|^2}\right)\, ,
\end{equation}
where the minimum is understood to be $r^{-1}$ in case $1=e^{ix}$.
\begin{lemma}[integrable bump convolution]
\label{integrable-bump-convolution}
\uses{Young-convolution}
Let $g,f$ be bounded measurable $2\pi$-periodic functions. Let $0<r<\pi$.
Assume we have for all $0\le x\le 2\pi$
\begin{equation}\label{ebump1}
    |g(x)|\le k_r(x)\, .
\end{equation}
Let
\begin{equation}
   h(x)= \int_{-\pi}^{\pi} f(y)g(x-y)\, dy \, .
\end{equation}
Then
\begin{equation}
   \|h\|_{L^2[-\pi, \pi]}\le 2^{5}\|f\|_{L^2[-\pi, \pi]} \, .
\end{equation}

\end{lemma}

\begin{proof}
From monotonicity of the integral and \eqref{ebump1},
\begin{equation}
    \|g\|_{L^1[-\pi, \pi]} \le \int_{-\pi}^{\pi}k_r(x)\, dx\,.
\end{equation}
Using the symmetry
$k_r(x)=k_r(-x)$, the assumption, and \Cref{lower-secant-bound}, the last display
is equal to
\begin{equation*}
    = 2 \int_0^\pi \min\left(\frac 1r, 1+\frac r{|1-e^{ix}|^2}\right)\, dx
\end{equation*}
\begin{equation*}
    \le 2\int_0^{r} \frac 1r \, dx+2\int_r^{\pi}1+\frac {64r}{x^2}\, dx
\end{equation*}
\begin{equation}
    \le 2+2\pi + 2\left(\frac {64r}r-\frac {64r}{\pi}\right)
    \le 2^{5}\, .
\end{equation}
    Together with \Cref{Young-convolution}, this proves the lemma.
\end{proof}

\begin{lemma}[dirichlet approximation]\label{dirichlet-approximation}
\uses{dirichlet-kernel,lower-secant-bound}
Let $0<r<1$. Let $N$ be the smallest
integer larger than $\frac 1r$.
There is a $2\pi$-periodic continuous function
 ${L'}$ on $\R$ that satisfies for all $-\pi\le x\le \pi$
and all $2\pi$-periodic bounded measurable functions $f$ on $\R$
\begin{equation}\label{lthroughlprime}
    L_Nf(x)=\frac 1{2\pi}\int_{-\pi}^{\pi}f(y) {L'}(x-y)\, dy
\end{equation}
and
\begin{equation}\label{eqdifflhil}
    \left|L'(x)-\mathbf{1}_{\{y:\, r<|y|<1\}} \kappa(x)\right|\le 2^{5}k_r(x)\, .
\end{equation}
\end{lemma}


\begin{proof}
We have by definition and \Cref{dirichlet-kernel}
\begin{equation}
    L_Ng(x)=
    \frac 1N\sum_{n=0}^{N-1}
       \int_{-\pi}^{\pi} e^{-i(N+n)x} K_{N+n}(x-y) e^{i(N+n)y}g(y)
\, dy \, .\end{equation}
This is of the form \eqref{lthroughlprime} with
the continuous function
\begin{equation}
    {L'}(x)= \frac 1N\sum_{n=0}^{N-1}
      K_{N+n}(x) e^{-i(N+n)x}\, .
\end{equation}
With \eqref{eqksumexp} of \Cref{dirichlet-kernel}
we have $|K_N(x)|\le N$ for every $x$ and thus
\begin{equation}\label{eqhil13}
    |{L'}(x)|\le \frac 1N\sum_{n=0}^{N-1}
      (N+n) \le 2N\le 2^2 r^{-1}\, .
\end{equation}
Therefore, for $|x|\in [0, r)\cup (1, \pi]$, we have
\begin{equation}
    \label{eqdiffzero}
    \left|L'(x)-\mathbf{1}_{\{y:\, r<|y|<1\}}(x)\kappa(x)\right|=|L'(x)|\leq 2^{2} r^{-1}.
\end{equation}
This proves \eqref{eqdifflhil} for $|x|\in [0, r)$ since $k_r(x)=r^{-1}$ in this case.

For $e^{ix'}\neq 1$
and may use the expression
\eqref{eqksumhil} for $K_N$
in \Cref{dirichlet-kernel} to obtain
\begin{equation*}
    {L'}(x)= \frac 1N\sum_{n=0}^{N-1}
     \left(\frac{e^{i(N+n)x}}{1-e^{-ix}}
      +\frac {e^{-i(N+n)x}}{1-e^{ix}}\right) e^{-i(N+n)x}
\end{equation*}
\begin{equation*}
    = \frac 1N\sum_{n=0}^{N-1}
    \left(\frac{1}{1-e^{-ix}}
      +\frac {e^{-i2(N+n)x}}{1-e^{ix}}\right)
\end{equation*}
\begin{equation}\label{eqhil3}
    = \frac{1}{1-e^{-ix}} +
     \frac 1N \frac {e^{-i2Nx}}{1-e^{ix}}
     \sum_{n=0}^{N-1}
    {e^{-i2nx}}
\end{equation}
and thus
\begin{equation}
\label{eq-L'L''}
  {L'}(x) -\mathbf{1}_{\{y:\, r<|y|<1\}}\kappa(x)=L''(x)+ \frac{1-\mathbf{1}_{{\{y:\, r<|y|<1\}}}(x)(1-|x|)}{1-e^{-ix}},
\end{equation}
where
$$L''(x):=\frac 1N \frac {e^{-i2Nx}}{1-e^{ix}}
     \sum_{n=0}^{N-1}
    {e^{-i2nx}}.$$
For $x\in [-\pi, r]\cup [r, \pi]$, we have using \Cref{lower-secant-bound} that
\begin{equation*}
   \left|\frac{1-\mathbf{1}_{{\{y:\, r<|y|<1\}}}(x)(1-|x|)}{1-e^{-ix}} \right|=\left|\frac{\min(|x|, 1)}{1-e^{-ix}} \right|\leq \frac{8\min(|x|, 1)}{|x|}
\end{equation*}
\begin{equation}
 \label{eq-diffzero2}
    \leq 2^3\cdot 1\leq 2^{3} k_r(x).
\end{equation}
Next, we need to estimate $L''(x)$. If the real part of
$e^{ix}$ is negative, we have
\begin{equation}
  1\le |1-e^{ix}|\le 2\, .
\end{equation}
and hence
\begin{equation}\label{eqhil12}
    |L''(x)|\le
     \frac 1N
     \sum_{n=0}^{N-1}
    1=1\le 1+\frac r{|1-e^{ix}|^2}\, .
\end{equation}
If the real part of $e^{ix}$ is positive and in particular while still $e^{ix}\neq \pm 1$, then we have by telescoping
\begin{equation}
 (1-e^{-2ix})
     \sum_{n=0}^{N-1}
    {e^{-i2nx}}=1-e^{-i2Nx}\, .
\end{equation}
As $e^{-2ix}\neq 1$, we may divide by $1-e^{-2ix}$ and insert this into
\eqref{eqhil3} to obtain
\begin{equation}
 L''(x)=
           \frac 1N \frac {e^{-i2Nx}}{1-e^{ix}}
     \frac{1-e^{-i2Nx}}{1-e^{-2ix}}\, .
\end{equation}
Hence, with \Cref{lower-secant-bound} and nonnegativity of the real part of $e^{ix}$
\begin{equation*}
    |L''(x)|
 \le \frac 2 N \frac {1}{|1-e^{ix}|}
     \frac{1}{|1-e^{-2ix}|}
 \end{equation*}
\begin{equation}\label{eqhil11}
    = \frac 2 N \frac {1}{|1-e^{ix}|^2}
     \frac{1}{|1+e^{ix}|}\le
 \frac {4r}{|1-e^{ix}|^2}\le 2^{2} \left (1+\frac {r}{|1-e^{ix}|^2}\right)
\end{equation}
Inequalities \eqref{eqhil13}, \eqref{eqdiffzero}, \eqref{eq-L'L''}, \eqref{eq-diffzero2}, \eqref{eqhil12}, and \eqref{eqhil11} prove \eqref{eqdifflhil}. This completes the proof of the lemma.
\end{proof}


We now prove \Cref{Hilbert-strong-2-2}.

\begin{proof}[Proof of \Cref{Hilbert-strong-2-2}]
    \proves{Hilbert-strong-2-2}
    We first show that if $f$ is supported in $[-3/2, 3/2]$, then
    \begin{equation}
        \label{eq-Hr-short-support}
        \|H_r f\|_{L^2(\R)} \le 2^{16} \|f\|_{L^2(\R)}\,.
    \end{equation}
    Let $\tilde{f}$ be the $2\pi$-periodic extension of $f$ to $\mathbb{R}$. Let $N$ be the smallest
    integer larger than $\frac 1r$. Then, by \Cref{dirichlet-approximation} and the triangle inequality, for $x\in [-\pi, \pi]$ we have
    \begin{equation*}
        |H_r \tilde{f}(x)|\leq 2\pi |L_N \tilde{f}(x)|+2^{5}\left|\int_{-\pi}^{\pi}k_r(x-y)\tilde{f}(y)\, dy\right|.
    \end{equation*}
    Taking $L^2$ norm over the interval $[-\pi, \pi]$ and using its sub-additivity, we get
    $$
         \|H_r \tilde{f}\|_{L^2([-\pi, \pi])}
    $$
    \begin{equation*}
       \leq 2\pi \|L_N \tilde{f}\|_{L^2([-\pi, \pi])}\, + 2^{5}\left(\int_{-\pi}^{\pi} \left|\int_{-\pi}^{\pi}k_r(x-y)\tilde{f}(y)\, dy\right|^2\, dx\right)^{\frac{1}{2}}.
    \end{equation*}
    Since $k_r$ is supported in $[-1,1]$, we have that $H_rf$ is supported in $[-5/2, 5/2]$ and agrees there with $H_r \tilde f(x)$.
    Using \Cref{modulated-averaged-projection} and \Cref{integrable-bump-convolution}, we conclude
    \begin{equation}
        \|H_r f\|_{L^2(\R)} \le \|H_r \tilde f\|_{L^2([-\pi, \pi])} \leq 2\pi \|f\|_{L^2(\R)} + 2^{10}\|f\|_{L^2(\R)}\,,
    \end{equation}
    which gives \eqref{eq-Hr-short-support}.

    Suppose now that $f$ is supported in $[c, c+3]$ for some $c \in \R$. Then the function $g(x) = f(c+ \frac{3}{2} +x)$ is supported in $[-3/2,3/2]$. By a change of variables in \eqref{def-H_r}, we have $H_r g(x ) = H_r f(c+ 3/2+x)$. Thus, by \eqref{eq-Hr-short-support}
    \begin{equation}
        \label{eq-Hr-short-support-2}
        \|H_rg\|_2 = \|H_r f\|_2 \le 2^{11} \|f\|_2 = \|g\|_2\,.
    \end{equation}

    Let now $f$ be arbitrary.
    Since $\kappa(x) = 0$ for $|x| > 1$, we have for all $x \in [c+1, c+2]$
    $$
        H_rf(x) = H_r(f \mathbf{1}_{[c, c+3]})(x)\,.
    $$
    Thus
    $$
        \int_{c+1}^{c+2} |H_r f(x)|^2 \, \mathrm{d}x \le \int_{\R} |H_r(f \mathbf{1}_{[c, c+3]})(x)|^2 \, \mathrm{d}x\,.
    $$
    Applying the bound \eqref{eq-Hr-short-support-2}, this is
    $$
        \le 2^{11} \int_{c}^{c+3} |f(x)|^2 \, \mathrm{d}x\,.
    $$
    Summing over all $c \in \mathbb{Z}$, we obtain
    $$
        \int_{\R} |H_rf(x)|^2 \, \mathrm{d}x \le 3 \cdot 2^{11} \int_{\R} |f(x)|^2 \, \mathrm{d}x\,.
    $$
    This completes the proof.
\end{proof}







\section{Calder\'on-Zygmund Decomposition}
\label{subsec-CZD}

For $I=[s, t)\subset \mathbb{R}$, we define
\begin{equation}
    \label{eq-bisec}
    \textrm{bi} (I):= \left\{\left[s, \frac{s+t}{2}\right), \left[\frac{s+t}{2}, t\right)\right\}.
\end{equation}

In what follows, we write $\mu$ for the Lebesgue measure on $\R$.

\begin{lemma}[interval bisection]
\label{interval-bisection}
    Let $I=[s, t)\subset \mathbb{R}$ be a bounded, right-open interval. Let $I_1, I_2\in \textrm{bi} (I)$ with $I_1\neq I_2$. Then
    \begin{equation}
        \label{eq-bi-size}
        \mu (I_1)=\mu(I_2)=\frac{\mu(I)}{2}.
    \end{equation}
    Further,
    \begin{equation}
        \label{eq-bi-union}
        I=I_1\cup I_2,
    \end{equation}
    and the intervals $I_1$ and $I_2$ are disjoint.
\end{lemma}
\begin{proof}
    This in some form can be taken from the Lean library.
\end{proof}
For a bounded interval $I=[a, b)\subset \mathbb{R}$ and a non-negative integer $n$, we inductively define
\begin{equation}
    \label{eq-ch-I}
    \textrm{ch}_0(I):=\{I\}, \qquad \textrm{ch}_n(I)=\cup_{J\in \textrm{ch}_{n-1}(J)} \textrm{bi} (J).
\end{equation}

\begin{lemma}[bisection children]
    \label{bisection-children}
    \uses{interval-bisection}
    Let $I=[s, t)$ and let $n$ be a non-negative integer. Then the intervals in $\textrm{ch}_n(I)$ are pairwise disjoint. For any $J\in \textrm{ch}_n(I)$,
    \begin{equation}
        \label{eq-ch-size}
        \mu(J)=\frac{\mu(I)}{2^{n}}=\frac{t-s}{2^n}.
    \end{equation}
    Further,
    \begin{equation}
        \label{eq-ch-cover}
        I= \cup_{J\in \textrm{ch}_n(I)} J
    \end{equation}
    and
    \begin{equation}
        \label{eq-ch-card}
        |\textrm{ch}_n(I)|= 2^{n+1}.
    \end{equation}
\end{lemma}
\begin{proof}
This follows by induction on $n$, using \Cref{interval-bisection}.
\end{proof}


\begin{lemma}[Lebesgue differentiation]
    \label{Lebesgue-differentiation}
    Let $f$ be a bounded measurable function with bounded support. Then for $\mu$ almost every $x$, we have
    $$\frac{1}{\mu(I_n)}\int_{I_n} f(y)\, dy= f(x),$$
    where $\{I_n\}_{n\geq 1}$ is a sequence of intervals such that $x\in I_n$ for each $n\geq 1$ and $$\lim_{n\to \infty} \mu(I_n)=0.$$
\end{lemma}
\begin{proof}
    This follows from the Lebesgue differentiation theorem, which is already formalized in Lean.
\end{proof}

\begin{lemma}[stopping time]
   \label{stopping-time}
   \uses{bisection-children,Lebesgue-differentiation}
   Let $f$ be a bounded, measurable function with bounded support on $\mathbb{R}$. Let $\alpha>0$. Then there exists $A\subset \mathbb{R}$ such that the following properties \eqref{eq-CZ-good-set}, \eqref{eq-CZ-good-set-comp}, \eqref{eq-CZ-bad-sets-1}, \eqref{eq-CZ-bad-sets-2}, and \eqref{eq-CZ-meas-b-set} are satisfied. For all $x\in A$
    \begin{equation}
    \label{eq-CZ-good-set}
        |f(x)|\leq \alpha.
    \end{equation}
    The set $\mathbb{R}\setminus A$ can be decomposed into a countable union of intervals
    \begin{equation}
        \label{eq-CZ-good-set-comp}
        \mathbb{R}\setminus A= \cup_{j} I_j= \cup_{j} [s_j, t_j),
    \end{equation}
    such that
   \begin{equation}
   \label{eq-CZ-bad-sets-1}
    [s_j, t_j)\cap [s_{j'}, t_{j'})=\emptyset \text{ for } j\neq j' \,.
   \end{equation}
   For each $j$,
   \begin{equation}
       \label{eq-CZ-bad-sets-2}
       \frac{\alpha}{2}\leq \frac{1}{t_j-s_j}\int_{s_j}^{t_j} |f(y)|\, dy\leq \alpha.
   \end{equation}
   Further,
   \begin{equation}
       \label{eq-CZ-meas-b-set}
        \sum_{j}(t_j-s_j)\leq \frac{2}{\alpha}\int |f(y)|\, dy\,.
   \end{equation}
\end{lemma}
\begin{proof}
    Since $f$ is bounded with bounded support, there exists a non-negative integer $\ell$ such that
    \begin{equation}
        \label{eq-f-supp}
        f(x)=0 \textrm{ for } x\not\in [-2^{\ell-1}, 2^{\ell-1})\,,
    \end{equation}
    and
    $$2^{-\ell}\int_{-2^{\ell-1}}^{2^{\ell-1}} |f(y)|\, dy< \alpha\,.$$
    Let $$I_0:=[-2^{\ell}, 2^{\ell}),\qquad \mathcal{Q}_0:=\{I_0\}\,,$$
    and
    $$\Tilde{\mathcal{Q}}_1:=\textrm{bi}(I_0).$$
    For $n\geq 1$, we inductively define
    \begin{equation*}
        {\mathcal{Q}}_n:=\left\{I\in \Tilde{\mathcal{Q}}_n: \frac{1}{\mu(I)}\int_{I} |f(y)|\, dy> \frac{\alpha}{2}\right\}\,,
    \end{equation*}
    and
    \begin{equation*}
     \Tilde{\mathcal{Q}}_{n+1}:=\cup_{I\in {\Tilde{\mathcal{Q}}}_n\setminus \mathcal{Q}_n} \textrm{bi} (I)\,.
    \end{equation*}
Finally, let
\begin{equation*}
    \mathcal{Q}:=\cup_{n\geq 1} \mathcal{Q}_n\,.
\end{equation*}

For each $n$ we have
$$\mathcal{Q}_n\subseteq \textrm{ch}_n(I_0).$$
Therefore, by \Cref{bisection-children},
$$|\mathcal{Q}_n|\leq \frac{\mu(I_0)}{2^n}= 2^{\ell+1-n}.$$
It follows that $\mathcal{Q}$ is a countable union of finite sets and hence, is countable itself. Let $\{I_j\}_{j\geq 1}$ be an enumeration of this set, with
$$I_j:=[s_j, t_j),$$
for $s_j, t_j\in I_0$ and $s_j<t_j$.
We set
\begin{equation*}
    A:=\mathbb{R}\setminus \cup_{j\geq 1} [s_j, t_j).
\end{equation*}
Then \eqref{eq-CZ-good-set-comp} holds by definition. We next show \eqref{eq-CZ-bad-sets-1}. Let $I_j, I_{j'}\in \mathcal{Q}$. Then, there exist $1\leq n, n'$ such that $I_j\in \mathcal{Q}_n$ and $I_{j'}\in \mathcal{Q}_{n'}$. If $n=n'$, \eqref{eq-CZ-bad-sets-1} follows from \Cref{bisection-children} since $\mathcal{Q}_n\subseteq \textrm{ch}_n(I_0).$ Otherwise, assume without loss of generality that $n<n'$. Then by construction, there exists $J\in \Tilde{\mathcal{Q}}_{n}\setminus {\mathcal{Q}_n}$ such that $I_{j'} \subset J$. Since $I_{j}\in \mathcal{Q}_n$, it follows that $I_j\neq J$. Since $I_j, J\in \Tilde{\mathcal{Q}}_{n}\subset \textrm{ch}_n(I_0)$, by \Cref{bisection-children}, we deduce that they are disjoint. Since $I_{j'}\subset J$, we conclude that $I_j$ and
$I_{j'}$ are disjoint as well. This proves \eqref{eq-CZ-bad-sets-1}.

To see \eqref{eq-CZ-bad-sets-2}, let $I_j=(s_j, t_j)\in \mathcal{Q}$. Then $I_j\in \mathcal{Q}_n$ for some $n\geq 1$. By definition of $\mathcal{Q}_n$, we have
$$\frac{\alpha}{2}<\frac{1}{\mu(I)}\int_{I} |f(y)|\, dy=\frac{1}{t_j-s_j}\int_{s_j}^{t_j} |f(y)|\, dy.$$
As $I_j\in \mathcal{Q}_n\subseteq \Tilde{\mathcal{Q}_{n}}$, by definition of the latter set, there exists $J\in \Tilde{\mathcal{Q}}_{n-1}\setminus \mathcal{Q}_{n-1}$ with $I\in \textrm{bi} (J)$. Since $J\not\in \mathcal{Q}_{n-1}$, we conclude that
$$\frac{1}{\mu(J)}\int_J |f(y)|\, dy\leq \frac{\alpha}{2}.$$
Using \Cref{interval-bisection} and the above, we get
$$\frac{1}{t_j-s_j}\int_{s_j}^{t_j} |f(y)|\, dy=\frac{1}{\mu(I)}\int_I |f(y)|\, dy\leq \frac{2}{\mu(J)} \int_J |f(y)|\, dy\leq \alpha.$$
This establishes \eqref{eq-CZ-bad-sets-2}. Using this, we see that for each $j$, we have
$$t_j-s_j\leq \frac{2}{\alpha}\int_{s_j}^{t_j} |f(y)|\, dy.$$
Summing up in $j$ and using the disjointedness property \eqref{eq-CZ-bad-sets-1}, we get
$$\sum_{j} (t_j-s_j)\leq \frac{2}{\alpha}\int_{s_j}^{t_j} |f(y)|\, dy= \frac{2}{\alpha}\int |f(y)|\, dy.$$

Finally, we show $\eqref{eq-CZ-good-set}$. Let $x\in A$. If $x\not \in [-2^{\ell}, 2^{\ell}]$, then by \eqref{eq-f-supp} that $f(x)=0$. Thus \eqref{eq-CZ-good-set} is true in this case. Alternately, let $x\in I_0\cap A$. Then for each $n$, there exists $I(n)\in \Tilde{\mathcal{Q}}_n$ such that $x\in I(n)$ and
\begin{equation}
    \label{eq-avg-In}
    \frac{1}{\mu(I(n))}\int_{I(n)} |f(y)| dy\leq {\alpha}.
\end{equation}
Since $\mu(I(n))=2^{\ell+1-n}$, we have
$$\lim_{n\to \infty} \mu(I(n))=0.$$
By \Cref{Lebesgue-differentiation}, we also have
\begin{equation*}
 \lim_{n\to \infty}\frac{1}{\mu(I(n))}\int_{I(n)} |f(y)| dy= |f(x)|
\end{equation*}
for almost every $x\in A$.
Combining the above with \eqref{eq-avg-In}, we conclude that
$$|f(x)|\leq \alpha.$$
This finishes the proof of \eqref{eq-CZ-good-set}, and hence the lemma.
\end{proof}

Calder\'on-Zygmund decomposition is a tool to extend $L^2$ bounds to $L^p$ bounds with $p<2$ or to the so-called weak $(1, 1)$ type endpoint bound.
It is classical and can be found in \cite{stein-book}.

\begin{lemma}[Calderon Zygmund decomposition]
    \label{Calderon-Zygmund-decomposition}
    \uses{stopping-time}
    Let $f$ be a bounded, measurable function with bounded support. Let $\alpha>0$ and $\gamma\in (0, 1)$. Then there exists a measurable functions $g$, a countable family of disjoint intervals $I_j = [s_j, t_j)$, and a countable family of measurable functions $\{g_j\}_{j\geq 1}$ such that for almost every $x \in \R$
    \begin{equation}
       \label{eq-gb-dec}
       f(x)= g(x)+ \sum_{j\geq 1} b_j(x)
    \end{equation}
    and such that the following holds. For almost every $x\in \mathbb{R}$,
    \begin{equation}
        \label{eq-g-max}
       |g(x)|\leq \gamma\alpha\,.
    \end{equation}
    We have
    \begin{equation}
        \label{eq-g-L1-norm}
        \int |g(y)|\, dy\leq \int |f(y)|\, dy.
    \end{equation}
    For every $j$
    \begin{equation}
        \label{eq-supp-bj}
        \operatorname{supp} b_j \subset I_j\,.
    \end{equation}
    For every $j$
    \begin{equation}
        \label{eq-bad-mean-zero}
        \int_{I_j} b_j(x)\, dx=0,
    \end{equation}
    and
     \begin{equation}
        \label{eq-bj-L1}
        \int_{I_j} |b_j(x)|\, dx \leq 2\gamma\alpha.
    \end{equation}
    We have
    \begin{equation}
        \label{eq-bset-length-sum}
        \sum_j (t_j-s_j)\leq \frac{2}{\gamma\alpha}\int |f(y)|\, dy
    \end{equation}
    and
    \begin{equation}
    \label{eq-b-L1}
    \sum_{j}\int_{I_j} |b_j(y)|\, dy\leq 2 \int |f(y)|\, dy\,.
    \end{equation}
\end{lemma}

\begin{proof}
    Applying \Cref{stopping-time} to $f$ and $\gamma\alpha$, we obtain a collection $I_j$ of intervals such that the conditions \eqref{eq-CZ-good-set}-\eqref{eq-CZ-meas-b-set} are satisfied. We set $A = \R \setminus \cup_j I_j$ and
\begin{equation}
    \label{eq-g-def}
    g(x):=\begin{cases}
     f(x), & x\in A,\\
     \frac{1}{\mu(I_j)}\int_{I_j} f(y)\, dy, &x\in (s_j, t_j),\\
     0, & x\in [s_j, t_j]\setminus (s_j, t_j),
    \end{cases}
\end{equation}
and, for each $j$,
\begin{equation}
    b_j(x):=\begin{cases}
        f(x)-\frac{1}{\mu(I_j)}\int_{I_j} f(y)\, dy, &x\in (s_j, t_j),\\
        0, & x\not\in (s_j, t_j).
    \end{cases}
\end{equation}
Then \eqref{eq-supp-bj} and \eqref{eq-bset-length-sum} are true by construction and \Cref{stopping-time}.
Further, let $b(x)=\sum_{j} b_j(x).$
Then
\begin{equation*}
    f(x)=g(x)+b(x)=g(x)+\sum_{j} b_j(x) ,
\end{equation*}
for all $x$ not in the measure zero set $\cup_j \{s_j, t_j\}.$

For almost every $x\in A$, we get using \eqref{eq-CZ-good-set}
$$|g(x)|\leq \gamma\alpha\,.$$ In the case when neither of the above is true, we have $g(x)=0$ by definition.
Thus, we obtain \eqref{eq-g-max}.

To prove \eqref{eq-g-L1-norm}, we estimate
$$\int |g(y)|\, dy\leq \int_A |f(y)|\, dy+ \sum_{j} \int_{I_j}\frac{1}{\mu(I_j)}\int_{I_j}|f(y)|\, dy.$$
Since the intervals $I_j$ are disjoint, and $A=\mathbb{R}\setminus \cup_j I_j$, we conclude
$$\int |g(y)|\, dy\leq \int_A |f(y)|\, dy+\sum_{j} \int_{I_j}|f(y)|\, dy= \int |f(y)|\, dy.$$
This establishes \eqref{eq-g-L1-norm}.
If $x\in I_j$ for some $j$, \eqref{eq-CZ-bad-sets-2} yields
$$|g(x)|\leq \frac{1}{\mu(I_j)}\int_{I_j}|f(y)|\, dy\leq \gamma\alpha.$$


Further, for each $j$, it follows from the definition of $t_j$ that
$$ \int_{I_j} b_j(x)\, dx= \int_{I_j} f(x)\, dx-\int_{I_j} \frac{1}{\mu(I_j)}\int_{I_j}f(y)\, dy\, dx$$
\begin{equation*}
    =\int_{I_j} f(x)\, dx- \int_{I_j}f(y)\, dy=0.
\end{equation*}
This establishes \eqref{eq-bad-mean-zero}.

Using the triangle inequality, we have that
\begin{equation*}
    \int_{I_j} |b_j(y)|\, dy\leq \int_{I_j} |f(y)|\, dy + \int_{I_j} \frac{1}{\mu(I_j)}\int_{I_j} |f(x)|\, dx\, dy.
\end{equation*}
\begin{equation}
    \label{eq-bj-int}
    =2 \int_{I_j} |f(y)|\, dy.
\end{equation}
Dividing both sides by $\mu(I_j)$ and using \eqref{eq-CZ-bad-sets-2}, we obtain \eqref{eq-bj-L1}.

Further, summing up \eqref{eq-bj-int} in $j$ yields
\begin{equation*}
    \sum_{j}\int_{I_j} |b_j(y)|\, dy\leq \sum_{j}\int_{I_j} |f(y)|\, dy + \sum_{j}\int_{I_j} \frac{1}{\mu(I_j)}\int_{I_j} |f(x)|\, dx\, dy.
\end{equation*}
Using the disjointedness of $(s_j, t_j)$, we get
\begin{equation*}
    \sum_{j}\int_{I_j} |b_j(y)|\, dy\leq 2 \int |f(y)|\, dy.
\end{equation*}
This proves \eqref{eq-b-L1}, and completes the proof.
\end{proof}

\begin{proof}[Proof of \Cref{Hilbert-weak-1-1}]
\proves{Hilbert-weak-1-1}
Using \Cref{Calderon-Zygmund-decomposition} for $f$ and $2^{-10}\alpha$, we obtain the decomposition
$$f=g+b=g+\sum_j b_j$$
such that the properties \eqref{eq-gb-dec}-\eqref{eq-b-L1} are satisfied with $\gamma=2^{-10}$. For each $j$, let
\begin{equation}
    \label{eq-Ij-cj}
    I_j=[s_j, t_j), \qquad c_j:=\frac{s_j+t_j}{2},
\end{equation}
and
\begin{equation}
    \label{eq-Ij*}
    I_j{^*}=\left[s_j-{2(t_j-s_j)}, t_j+2(t_j-s_j)\right].
\end{equation}
Then $I_j^*$ is an interval with the same center as $I_j$ but with
\begin{equation}
    \label{eq-Ij*-dim}
    \mu(I_j^*)=5(t_j-s_j)=5\mu(I_j).
\end{equation}
Let
\begin{equation}
    \label{eq-omega}
    \Omega:=\cup_j I_{j}^*.
\end{equation}
By definition, for each $x\in \mathbb{R}\setminus\Omega$ and $y\in I_j$,
\begin{equation}
    \label{eq-Om-cj}
    |x-c_j|>\frac{5(t_j-s_j)}{2}\geq 5|y-c_j|,
\end{equation}
and
\begin{equation}
    \label{eq-Om-y}
    |x-y|>{2(t_j-s_j)}.
\end{equation}
It follows by the triangle inequality and subadditivity of $\mu$ that
$$ \mu\left(\{x\in \mathbb{R}: |H_r f(x)|>\alpha\}\right) $$
\begin{equation}
\label{eq-set-dec-1}
 \le \mu\left(\{x\in \mathbb{R}: |H_r g(x)|>{\alpha}/2\}\right)+ \mu\left(\{x\in \mathbb{R}: |H_r b(x)|>{\alpha}/2\}\right).
\end{equation}
We estimate using monotonicity of the integral
$$ \mu\left(\{x\in \mathbb{R}: |H_r g(x)|>{\alpha}/2\}\right)\leq
\frac{4}{\alpha^2} \int |H_r g(y)|^2\, dy. $$
Using \Cref{Hilbert-strong-2-2} followed by \eqref{eq-g-max} and \eqref{eq-g-L1-norm}, we estimate the right hand side above by
$$\leq 2^{26}\frac{4}{\alpha^2} \int |g(y)|^2\, dy\leq \frac{2^{18}}{\alpha} \int |g(y)|\, dy\leq \frac{2^{18}}{\alpha} \int |f(y)|\, dy.$$
Thus, we conclude
\begin{equation}
    \label{eq-g-func-bd}
    \mu\left(\{x\in \mathbb{R}: |H_r g(x)|>{\alpha}/2\}\right)\leq
    \frac{2^{18}}{\alpha} \int |f(y)|\, dy.
\end{equation}
Next, we estimate
$$\mu\left(\{x\in \mathbb{R}: |H_r b(x)|>\alpha/2\}\right) $$
$$
    \le \mu (\Omega) + \mu\left(\{x\in \mathbb{R}\setminus\Omega: |H_r b(x)|>{\alpha}/2\}\right)\,.
$$
Using \eqref{eq-Ij*-dim} and \eqref{eq-bset-length-sum}, we conclude that
\begin{equation}
    \label{eq-omega-bd}
    \mu(\Omega) \le \sum_{j} \mu (I_j^*)
    = 5 \sum_{j} (t_j-s_j)\leq \frac{2^{13}}{\alpha} \int |f(y)|\, dy\,.
\end{equation}
We now focus on estimating the remaining term
$$\mu\left(\{x\in \mathbb{R}\setminus\Omega: |H_r b(x)|>{\alpha}/2\}\right)\,.$$
For $x\in \mathbb{R}\setminus\Omega$, define
\begin{align*}
    \mathcal{J}_1(x)&:=\{j\,:[s_j, t_j)\cap [x-r, x+r]=\emptyset \},\\
    \mathcal{J}_2(x)&:=\{j\,: |y-x|=r \text{ for some } y \in [s_j, t_j)\},\\
    \mathcal{J}_3(x)&:=\{j\,: [s_j, t_j)\subset [x-r, x+r]\}.
\end{align*}
Since $H_rb_j(x)=0$ for all $j\in \mathcal{J}_3(x)$, using the triangle inequality and the decomposition above, we get
\begin{equation}
    \label{eq-b-dec}
    |H_r b(x)|\leq \sum_{j\in \mathcal{J}_1(x)} |H_rb_j(x)|+\sum_{j\in \mathcal{J}_2(x)} |H_rb_j(x)|.
\end{equation}
Further, for $j\in \mathcal{J}_1(x)$, we have
$$H_rb_j(x)=\int_{I_j} \kappa (x-y)\mathbf{1}_{\{z:\, r<|z|<1\}}(x-y) b_j(y)\,dy=\int_{I_j} \kappa (x-y) b_j(y)\,dy\,.$$
Using \eqref{eq-bad-mean-zero}, the above is equal to
$$\int_{I_j} (\kappa (x-y)-\kappa(x-c_j)) b_j(y).$$
Thus, using the triangle inequality, \eqref{eq-Om-cj}, \eqref{eq-Om-y} and \Cref{Hilbert-kernel-regularity}, we can estimate
$$\sum_{j\in \mathcal{J}_1(x)} |H_rb_j(x)|\leq \sum_{j\in \mathcal{J}_1(x)}2^{10}\int_{I_j}\frac{|y-c_j|}{|x-c_j|^2} |b_j(y)|\, dy$$
\begin{equation}
    \label{eq-J1-diff-est}
    \leq 2^{10}\sum_{j} \frac{(t_j-s_j)}{|x-c_j|^2}\int_{I_j} |b_j(y)|\, dy:=F_1(x).
\end{equation}
Next, we estimate the second sum in \eqref{eq-b-dec}. For each $j\in \mathcal{J}_2(x)$, set
$$d_j:=\frac{1}{t_j-s_j}\int_{I_j} \mathbf{1}_{\{z:\, r<|z|<1\}}(x-y) b_j(y)\, dy.$$
Then by \eqref{eq-bj-L1}
\begin{equation}
    \label{eq-dj-est}
    |d_j|\leq 2\cdot 2^{-10} \alpha=2^{-9}\alpha.
\end{equation}
For each $j\in \mathcal{J}_2(x)$, we have
$$
H_r b_j(x)=\int_{I_j} \kappa(x-y) (\mathbf{1}_{\{z:\, r<|z|<1\}}(x-y)b_j(y)-d_j)\, dy$$
$$+ \int_{I_j} d_j \kappa(x-y) \, dy.
$$
$$=\int_{I_j} (\kappa(x-y)-\kappa(x-c_j)) (\mathbf{1}_{\{z:\, r<|z|<1\}}(x-y)b_j(y)-d_j)\, dy$$
$$+ \int_{I_j} d_j \kappa(x-y) \, dy.$$
Thus, using the triangle inequality, the estimate above and \eqref{eq-dj-est}, we obtain
$$|H_r b_j(x)|\leq $$
\begin{equation}
    \label{eq-J2-diff-est}
    \int_{I_j} |(\kappa(x-y)-\kappa(x-c_j)) \left(|b_j(y)|+2^{-9}\alpha\right)\, dy +2^{-9}\alpha \int_{I_j} |\kappa(x-y)| \, dy.
\end{equation}
Using \eqref{eq-Om-cj}, \eqref{eq-Om-y} and \Cref{Hilbert-kernel-regularity}, and arguing as in \eqref{eq-J1-diff-est}, we get the the first term above can be estimated by
\begin{equation}
    \label{eq-J2-diff-est-2}
    2^{10}\sum_{j} \frac{(t_j-s_j)}{|x-c_j|^2}\left(\int_{I_j} |b_j(y)|+2^{-9} \alpha (t_j-s_j)\right)=F_1(x)+F_2(x),
\end{equation}
with $F_1$ as in \eqref{eq-J1-diff-est} and
\begin{equation}
    \label{eq-def-F2}
    F_2(x):= 2\sum_{j} \frac{(t_j-s_j)}{|x-c_j|^2} \alpha (t_j-s_j).
\end{equation}
For each $j\in \mathcal{J}_2(x)$, let $y_j\in [s_j, t_j)$ be such that
$$|x-y_j|=r.$$ Using \eqref{eq-Om-y}, we have
$$r=|x-y_j|>2(t_j-s_j).$$
Further, using the triangle inequality, for each $y\in I_j$, we obtain
$$|x-y|\leq |x-y_j|+|y-y_j|\leq r+(t_j-s_j)<\frac{3r}{2},$$
and
$$|x-y|\geq |x-y_j|-|y-y_j|\geq r-(t_j-s_j)>\frac{r}{2}.$$
We conclude that
$$[s_j, t_j)\subset \left\{y\,: \frac{r}{2}<|x-y|<\frac{3r}{2}\right\}.$$
Using this and \Cref{Hilbert-kernel-bound}, we get
\begin{equation}
    \label{eq-J2-diff-est-4}
    \int_{I_j}|\kappa(x-y)|\,dy \leq \int_{\{y\,: \frac{r}{2}<|x-y|<\frac{3r}{2}\}}\frac{8}{|x-y|}\, dy\leq 32\,.
\end{equation}

Combining \eqref{eq-b-dec}, \eqref{eq-J1-diff-est}, \eqref{eq-J2-diff-est}, \eqref{eq-J2-diff-est-2} and \eqref{eq-J2-diff-est-4}, we get
\begin{equation*}
    |H_rb(x)|\leq 2F_1(x)+F_2(x)+2^{-4}\alpha.
\end{equation*}
Using the triangle inequality, we deduce that
$$\mu({\{x\in \Omega: |H_r b(x)|>\alpha/4\}})\leq $$
\begin{equation}
    \label{eq-set-dec-3}
    \mu(\{x\in \Omega: |F_1(x)|> 2^{-4} \alpha\})+\mu(\{x\in \Omega: | F_2(x)|> 2^{-4}\alpha\}).
\end{equation}

We estimate
\begin{equation}
    \label{eq-F1-est-1}
    \mu(\{x\in \Omega: |F_1(x)|> 2^{-4}\alpha\})\leq \frac{2^{14}}{\alpha}\sum_{j} \int_{I_j} |b_j(y)|\, dy\int_{\Omega} \frac{(t_j-s_j)}{|x-c_j|^2}\, dx.
\end{equation}
Using \eqref{eq-Om-cj}, we can bound
\begin{equation}
    \label{eq-Om-int}
    \int_{\Omega} \frac{(t_j-s_j)}{|x-c_j|^2}\, dx \le 2(t_j - s_j) \int_{5|t_j - s_j|}^\infty \frac{1}{t^2} \, \mathrm{d}t = \frac{2}{5}\,.
\end{equation}
Plugging \eqref{eq-Om-int} into \eqref{eq-F1-est-1} and using \eqref{eq-b-L1}, we conclude that
\begin{equation}
    \label{eq-F1-est-2}
    \mu(\{x\in \Omega: |F_1(x)|>2^{-4}\alpha\})\leq \frac{2^{14}}{\alpha}\int |f(y)|\,dy\,.
\end{equation}

Similarly, we estimate
$$\mu(\{x\in \Omega: |F_2(x)|>2^{-4} \alpha\})\leq \frac{2^{5}}{\alpha}\sum_{j} \alpha (t_j-s_j)\, \int_{\Omega} \frac{(t_j-s_j)}{|x-c_j|^2}\, dx.$$
Using \eqref{eq-Om-int} and \eqref{eq-bset-length-sum}, this is bounded by
\begin{equation}
\label{eq-F2-est}
2^{4}\sum_{j} (t_j-s_j)\leq \frac{2^{15}}{\alpha}\int |f(y)|\, dy.
\end{equation}

Combining estimates \eqref{eq-set-dec-1}, \eqref{eq-g-func-bd}, \eqref{eq-omega-bd}, \eqref{eq-set-dec-3}, \eqref{eq-F1-est-2} and \eqref{eq-F2-est} yields \eqref{eq-weak-1-1}.
\end{proof}
















































\section{The proof of the van der Corput Lemma}
\label{10vandercorput}

\begin{proof}[Proof of \Cref{van-der-Corput}]
\proves{van-der-Corput}
Let $g$ be a Lipschitz continuous function as in the lemma. We have
$$
    e^{in(x + \pi/n)} = -e^{inx}\,.
$$
Using this, we write
$$
    \int_\alpha^\beta g(x) e^{-inx} \, \mathrm{d}x
    = \frac{1}{2} \int_\alpha^\beta g(x) e^{inx} \, \mathrm{d}x - \frac{1}{2} \int_\alpha^\beta g(x) e^{in(x + \pi/n)}) \, \mathrm{d}x\,.
$$
We split the the first integral at $\alpha + \frac{\pi}{n}$ and the second one at $\beta - \frac{\pi}{n}$, and make a change of variables in the second part of the first integral to obtain
$$
    = \frac{1}{2} \int_{\alpha}^{\alpha + \frac{\pi}{n}} g(x) e^{inx} \, \mathrm{d}x - \frac{1}{2} \int_{\beta - \frac{\pi}{n}}^{\beta} g(x) e^{in(x + \pi/n)} \, \mathrm{d}x
$$
$$
    + \frac{1}{2} \int_{\alpha + \frac{\pi}{n}}^{\beta} (g(x) - g(x - \frac{\pi}{n})) e^{inx} \, \mathrm{d}x\,.
$$
The sum of the first two terms is by the triangle inequality bounded by
$$
    \frac{\pi}{n} \sup_{x \in [\alpha,\beta]} |g(x)|\,.
$$
The third term is by the triangle inequality at most
$$
    \frac{1}{2} \int_{\alpha + \frac{\pi}{n}}^\beta |g(x) - g(x - \frac{\pi}{n})| \, \mathrm{d}x
$$
$$
    \le \frac{\pi}{2n} \sup_{\alpha \le x < y \le \beta} \frac{|g(x) - g(y)|}{|x-y|} |\beta-\alpha|\,.
$$
Adding the two terms, we obtain
$$
    \left|\int_\alpha^\beta g(x) e^{-inx} \, \mathrm{d}x\right| \le \frac{\pi}{n} \|g\|_{\mathrm{Lip}(\alpha,\beta)}\,.
$$
By the triangle inequality, we also have
$$
    \left|\int_\alpha^\beta g(x) e^{-inx} \, \mathrm{d}x\right| \le |\beta -\alpha| \sup_{x \in [\alpha,\beta]} |g(x)| \le |\beta-\alpha| \|g\|_{\mathrm{Lip}(\alpha,\beta)}\,.
$$
This completes the proof of the lemma, using that
$$
    \min\{|\beta-\alpha|, \frac{\pi}{n}\} \le 2 \pi |\beta-\alpha|(1 + n|\beta-\alpha|)^{-1}\,.
$$
\end{proof}




\section{Partial sums as orthogonal projections}
\label{10projection}

This subsection proves \Cref{spectral-projection-bound}






\begin{lemma}[partial sum projection]
\label{partial-sum-projection}
\uses{mean-zero-oscillation}
  Let $f$ be a bounded $2\pi$-periodic measurable function. Then, for all $N\ge 0$
   \begin{equation}\label{projection}
   S_N(S_N f)=S_Nf\, .
   \end{equation}
   \end{lemma}
\begin{proof}
Let $N>0$ be given. With $K_N$ as in \Cref{dirichlet-kernel},
\begin{equation*}
S_N (S_Nf) (x)=
\frac{1}{2\pi} \int_0^{2\pi} S_Nf(y)K_N(x-y)\, dy
\end{equation*}
\begin{equation}\label{eqhil1}
=
\frac{1}{(2\pi)^2}\int_0^{2\pi} \int_0^{2\pi} f(y')K_N(y-y') K_N(x-y)\, \, dy' dy\, .
\end{equation}
We have by \Cref{dirichlet-kernel}
\begin{equation*}
\frac{1}{2\pi}\int_0^{2\pi} K_N(y-y') K_N(x-y)\, dy
\end{equation*}
\begin{equation*}
=\frac{1}{2\pi}\sum_{n=-N}^N\sum_{n'=-N}^N
\int_0^{2\pi} e^{in(y-y')}e^{in'(x-y)}\, dy
\end{equation*}
\begin{equation}\label{eqhil6}
=\frac{1}{2\pi}\sum_{n=-N}^N\sum_{n'=-N}^N
e^{i(n'x-ny')}\int_0^{2\pi} e^{i(n-n')y}\, dy\, .
\end{equation}
By \Cref{mean-zero-oscillation}, the summands for $n\neq n'$ vanish.
We obtain for \eqref{eqhil6}
\begin{equation}\label{eqhil2}
=\frac{1}{2\pi}\sum_{n=-N}^N
e^{in(x-y')}\int_0^{2\pi} \, dy=K_N(x-y')\, .
\end{equation}
Applying Fubini in \eqref{eqhil1} and using
\eqref{eqhil2} gives
\begin{equation}
S_N(S_Nf)(x)=
\frac{1}{2\pi} \int_0^{2\pi} f(y')K(x-y') \, dy'=S_N f(x)
\end{equation}
This proves the lemma.
\end{proof}
\begin{lemma}[partial sum selfadjoint]
\label{partial-sum-selfadjoint}
\uses{dirichlet-kernel}
    We have for any $2\pi$-periodic bounded measurable $g,f$ that
    \begin{equation}
       \int_0^{2\pi} \overline{S_Nf(x)} g(x)=\int_0^{2\pi} \overline{f(x)} S_Ng(x)\, dx\, .
    \end{equation}
\end{lemma}
\begin{proof}
  We have with $K_N$ as in \Cref{dirichlet-kernel} for every $x$
  \begin{equation}
      \overline{K_N(x)}=\sum_{n=-N}^N\overline{ e^{in x}}=
      {\sum_{n=-N}^N e^{-in x}}=K_N(-x)\, .
  \end{equation}
 Further, with \Cref{dirichlet-kernel} and Fubini
\begin{equation*}
\int_0^{2\pi} \overline{S_Nf(x)} g(x)
= \frac 1{2\pi} \int_0^{2\pi} \int_{-\pi}^{\pi}\overline{f(y) K_N(x-y)} g(x)\, dy dx
 \end{equation*}
 \begin{equation}
=
\frac 1{2\pi} \int_0^{2\pi} \int_{-\pi}^{\pi}\overline{f(y)} K_N(y-x)
g(x)\, dx dy
=\int_0^{2\pi} \overline{f(x)} S_Ng(x)\, dx
\, .
\end{equation}
 This proves the lemma.
\end{proof}


We turn to the proof of Lemma
\ref{spectral-projection-bound}.

We have with \Cref{partial-sum-selfadjoint}, then \Cref{partial-sum-projection} and the \Cref{partial-sum-selfadjoint} again
\begin{equation*}
 \int_0^{2\pi} S_Nf(x)\overline{S_Nf(x)}\, dx
 \int_0^{2\pi} f(x)\overline{S_N(S_Nf)(x)}\, dx
\end{equation*}
\begin{equation}\label{eqhil7}
 =\int_0^{2\pi} f(x)\overline{S_Nf(x)}\, dx=
 \int_0^{2\pi} S_N f(x)\overline{f(x)}\, dx\, .
\end{equation}

We have by the distributive law
\begin{equation}\label{diffnorm}
    \int_0^{2\pi} (f(x)-S_Nf(x))(\overline{f(x)-S_Nf(x)})\, dx=
\end{equation}
\begin{equation*}
 \int_0^{2\pi} f(x)\overline{f(x)}
    -S_Nf(x)\overline{f(x)}
   -f(x)\overline{S_Nf(x)}
     + S_Nf(x)\overline{S_Nf(x)}\, dx
\end{equation*}
Using the various identities expressed in \eqref{eqhil7}, this becomes
\begin{equation}
 =\int_0^{2\pi} f(x)\overline{f(x)}\, dx
    -
   \int_0^{2\pi} S_Nf(x)\overline{S_Nf(x)}\, dx\, .
\end{equation}
As \eqref{diffnorm} has nonnegative integrand and is thus nonnegative, we conclude
\begin{equation}
  \int_0^{2\pi} S_Nf(x)\overline{S_Nf(x)}\, dx\le
 \int_0^{2\pi} f(x)\overline{f(x)})\, dx\, .
\end{equation}
As both sides are positive, we may take the square root of this inequality.
This completes the proof of the lemma.





\section{The error bound}
\label{10difference}

%\begin{lemma}
%\label{control-approximation-effect}
%Let $h : \R \to \C$ be measurable, $2\pi$-periodic and
%\end{lemma}

\begin{proof}[Proof of \Cref{control-approximation-effect}]
\leanok
Define
$$
    E := \{x \in [0, 2\pi) \ : \ \sup_{N > 0} |S_N f(x) - S_N f_0(x)| > \frac{\epsilon}{4} \}\,.
$$
Then \eqref{eq-max-partial-sum-diff} clearly holds, and it remains to show that $|E| \le \frac{\epsilon}{2}$. This will follow from \Cref{real-Carleson}.

Let $x \in E$. Then there exists $N > 0$ with
$$
    |S_N f(x) - S_N f_0(x)| > \frac{\epsilon}{4}\,,
$$
pick such $N$. We have with \Cref{dirichlet-kernel}
$$
    \frac{\epsilon}{4} < |S_N f(x) - S_N f_0(x)| = \frac{1}{2\pi} \left| \int_0^{2\pi} (f(y) - f_0(y)) K_N(x-y) \, dy\right|\,.
$$
We make a change of variables, replacing $y$ by $x -y$. Then we use $2\pi$-periodicity of $f$, $f_0$ and $K_N$ to shift the domain of integration to obtain
$$
    = \frac{1}{2\pi} \left|\int_{-\pi}^{\pi} (f(x-y) - f_0(x-y)) K_N(y) \, dy\right|\,.
$$
Using the triangle inequality, we split this as
$$
    \le \frac{1}{2\pi} \left|\int_{-\pi}^{\pi} (f(x-y) - f_0(x-y)) \max(1 - |y|, 0) K_N(y) \, dy\right|
$$
$$
    + \frac{1}{2\pi} \left|\int_{-\pi}^{\pi} (f(x-y) - f_0(x-y)) \min(|y|, 1) K_N(y) \, dy\right|\,.
$$
Note that all integrals are well defined, since $K_N$ is by \eqref{eqksumexp} bounded by $2N+1$.
Using \eqref{eqksumhil} and the definition \eqref{eq-hilker} of $\kappa$, we rewrite the two terms and obtain
\begin{equation}
    \label{eq-diff-singular}
    \frac{\epsilon}{4} < \frac{1}{2\pi} \left| \int_{-\pi}^{\pi} (f(x-y) - f_0(x-y)) (e^{iNy} \overline{\kappa}(y) + e^{-iNy}\kappa(y)) \, dy\right|
\end{equation}
\begin{equation}
    \label{eq-diff-integrable}
    + \frac{1}{2\pi} \left|\int_{-\pi}^{\pi} (f(x-y) - f_0(x-y)) ( e^{iNy} \frac{\min\{|y|, 1\} }{1 - e^{-iy}} + e^{-iNy} \frac{\min\{|y|, 1\} }{1 - e^{iy}}) \, dy \right|\,.
\end{equation}
By \Cref{lower-secant-bound} with $\eta = |y|$, we have for $-1 \le y \le 1$
$$
    |e^{iNy}\frac{\min\{|y|, 1\} }{1 - e^{-iy}}| = \frac{|y|}{|1 - e^{iy}|} \le \frac \pi 2 \,.
$$
By \Cref{lower-secant-bound} with $\eta = 1$, we have for $1 \le |y| \le \pi$
$$
    |e^{iNy}\frac{\min\{|y|, 1\} }{1 - e^{-iy}}| = \frac{1}{|1 - e^{iy}|} \le \frac \pi 2 \,.
$$
Thus we obtain using the triangle inequality and \eqref{eq-ffzero}
$$
    \eqref{eq-diff-integrable} \le \frac{1}{2} \int_{-\pi}^{\pi} |f(x-y) - f_0(x-y)| \, dy \le \pi\epsilon'.
$$
Consequently, we have that
$$
    \frac{\epsilon}{4} - \pi\epsilon' \le \frac{1}{2\pi} \left| \int_{-\pi}^{\pi} (f(x-y) - f_0(x-y)) (e^{iNy} \overline{\kappa}(y) + e^{-iNy}\kappa(y)) \, dy\right|\,.
$$
By dominated convergence and since $\kappa(y) = 0$ for $|y| > 1$, this equals
$$
     = \frac{1}{2\pi} \lim_{r \to 0^+} \left| \int_{r < |y| < 1} (f(x-y) - f_0(x-y)) (e^{iNy} \overline{\kappa}(y) + e^{-iNy}\kappa(y)) \, dy\right|\,.
$$
Let $h = (f - f_0) \mathbf{1}_{[-\pi, 3\pi]}$. Since $x \in [0, 2\pi]$, the above integral does not change if we replace $(f - f_0)$ by $h$. We do that, apply the triangle inequality and bound the limits by suprema
$$
    \le \frac{1}{2\pi} \sup_{r > 0} \left| \int_{r < |y| < 1} h(x-y) e^{-iNy} \kappa(y) \, dy\right|
$$
$$
    + \frac{1}{2\pi} \sup_{r > 0} \left| \int_{r < |y| < 1} \overline{h}(x-y) e^{-iNy} \kappa(y) \, dy\right|\,.
$$
By the definition \eqref{define-T-carleson} of $T$, this is
$$
    \le \frac{1}{2\pi} (Th(x) + T\bar{h}(x))\,.
$$
Thus for each $x \in E$, at least one of $Th(x)$ and $T\bar h(x)$ is larger than $\frac{\pi(\epsilon - 4 \pi\epsilon')}{4}$.
Thus at least one of $Th$ and $T\bar h$ is $\ge \frac{\pi(\epsilon - 4 \pi\epsilon')}{4}$ on a subset $E'$ of $E$ with $2|E'| \ge |E|$. Without loss of generality this is $Th$. By assumption \eqref{eq-ffzero}, we have $|2^{2^{50}}\epsilon^{-2} h| \le \mathbf{1}_{[-\pi, 3\pi]}$. Applying \Cref{real-Carleson} with $F = [-\pi, 3\pi]$ and $G = E'$, it follows that
$$
    \frac{\pi(\epsilon - 4 \pi\epsilon')}{4} |E'| \le \int_{E'} Th(x) \, dx = \epsilon' \int_{E'} T(\epsilon'^{-1} h)(x) \, dx
$$
$$
    \le \epsilon' \cdot C_{4,2} |F|^{\frac{1}{2}} |E'|^{\frac{1}{2}} \le (4\pi)^\frac{1}{2} C_{4,2} \epsilon' \cdot |E'|^{\frac{1}{2}}\,.
$$
Rearranging, we obtain
$$
    |E'| \le \left(\frac{4 (4\pi)^\frac{1}{2} C_{4,2} \epsilon'}{\pi(\epsilon - 4 \pi\epsilon')}\right)^2 = \frac{\epsilon}{2}\,.
$$
This completes the proof using $|E| \le 2|E'|$.
\end{proof}




\section{Carleson on the real line}
\label{10carleson}

We prove \Cref{real-Carleson}.

Consider the standard distance function
\begin{equation}
    \rho(x,y)=|x-y|
\end{equation}
on the real line $\R$.
\begin{lemma}[real line metric]
\label{real-line-metric}
The space $(\R,\rho)$ is a complete locally compact metric space.
\end{lemma}
\begin{proof}
    This is part of the Lean library.
\end{proof}
\begin{lemma}[real line ball]
\label{real-line-ball}
    \lean{Real.ball_eq_Ioo}
    \leanok
    For $x\in R$ and $R>0$, the ball $B(x,R)$ is the interval $(x-R,x+R)$
\end{lemma}
\begin{proof}
\leanok
Let $x'\in B(x,R)$. By definition of the ball,
$|x'-x|<R$. It follows that $x'-x<R$ and $x-x'<R$.
It follows $x'<x+R$ and $x'>x-R$. This implies
$x'\in (x-R,x+R)$.
Conversely, let $x'\in (x-R,x+R)$. Then
$x'<x+R$ and $x'>x-R$. It follows that
$x'-x<R$ and $x-x'<R$. It follows that $|x'-x|<R$,
hence $x'\in B(x,R)$.
This proves the lemma.
\end{proof}
We consider the Lebesgue measure $\mu$ on $\R$.
\begin{lemma}[real line measure]
\label{real-line-measure}
    The measure $\mu$ is a sigma-finite non-zero
    Radon-Borel measure on $\R$.
\end{lemma}
\begin{proof}
    This is part of the Lean library.
\end{proof}
\begin{lemma}[real line ball measure]
\label{real-line-ball-measure}
\uses{real-line-ball}
    \lean{Real.volume_ball}
    \leanok
    We have for every $x\in \R$ and $R>0$
    \begin{equation}
        \mu(B(x,R))=2R\, .
    \end{equation}
\end{lemma}
\begin{proof}
\leanok
We have with \Cref{real-line-ball}
\begin{equation}
    \mu(B(x,R))=\mu((x-R,x+R))=2R\, .
\end{equation}
\end{proof}

\begin{lemma}[real line doubling]
\label{real-line-doubling}
\uses{real-line-ball-measure}
    We have for every $x\in \R$ and $R>0$
    \begin{equation}
        \mu(B(x,2R))=2\mu(B(x,R))\, .
    \end{equation}
\end{lemma}
\begin{proof}
    We have with \Cref{real-line-ball-measure}
\begin{equation}
    \mu(B(x,2R)=4R=2\mu(B(x,R)\, .
\end{equation}
This proves the lemma.
\end{proof}


The preceding four lemmas show that $(\R, \rho, \mu, 4)$ is a doubling metric measure space. Indeed, we even show
that $(\R, \rho, \mu, 1)$ is a doubling metric measure space, but we may relax the estimate in \Cref{real-line-doubling} to conclude that $(\R, \rho, \mu, 4)$
is a doubling metric measure space.


For each $n\in \mathbb{Z}$ define
$\mfa_n:\R\to \R$ by
\begin{equation}
    \mfa_n(x)=nx\, .
\end{equation}


Let $\Mf$ be the collection $\{\mfa_n, n\in \mathbb{Z}\}$.
Note that for every $n\in \mathbb{Z}$ we have $\mfa_n(0)=0$.
Define
\begin{equation}
    \label{eqcarl4}
    d_{B(x,R)}(\mfa_n, \mfa_m) := 2R|n-m|\,.
\end{equation}

\begin{lemma}[frequency metric]
    \label{frequency-metric}
    For every $R > 0$ and $x \in X$, the function $d_{B(x,R)}$ is a metric on $\Mf$.
\end{lemma}

\begin{proof}
    This follows immediately from the fact that the standard metric on $\mathbb{Z}$ is a metric.
\end{proof}

\begin{lemma}[oscillation control]
    \label{oscillation-control}
    For every $R > 0$ and $x \in X$, and for all $n, m \in \mathbb{Z}$, we have
    \begin{equation}\label{eqcarl2}
       \sup_{y,y'\in B(x,R)}|ny-ny'-my+my'|\le 2|n-m|R\, .
    \end{equation}
\end{lemma}

\begin{proof}
    The right hand side of \eqref{eqcarl2} equals
    $$
        \sup_{y,y'\in B(x,R)}|(n-m)(y-x)-(n-m)(y'-x)|\,.
    $$
    The lemma then follows from the triangle inequality.
\end{proof}

\begin{lemma}[frequency monotone]
\label{frequency-monotone}

    For any $x, x' \in X$ and $R, R' > 0$ with $B(x,R) \subset B(x, R')$, and for any $n, m \in \mathbb{Z}$
    $$
        d_{B(x,R)}(\mfa_n, \mfa_m) \le d_{B(x',R')}(\mfa_n, \mfa_m)\,.
    $$
\end{lemma}

\begin{proof}
    This follows immediately from the definition \eqref{eqcarl4} and $R \le R'$.
\end{proof}

\begin{lemma}[frequency ball doubling]
\label{frequency-ball-doubling}
\uses{frequency-metric}
  For any $x,x'\in \R$ and $R>0$ with
   $x\in B(x',2R)$ and any $n,m\in \mathbb{Z}$, we have
\begin{equation}\label{firstdb1}
    d_{B(x',2R)}(\mfa_n,\mfa_m)\le 2 d_{B(x,R)}(\mfa_n,\mfa_m) \, .
\end{equation}
\end{lemma}
\begin{proof}
With \eqref{eqcarl4}, both sides of \eqref{firstdb1} are equal to $4R|n-m|$. This proves the lemma.
\end{proof}

\begin{lemma}[frequency ball growth]
\label{frequency-ball-growth}
    For any $x,x'\in \R$ and $R>0$ with
   $B(x,R)\subset B(x',2R)$ and any $n,m\in \mathbb{Z}$, we have
\begin{equation}\label{seconddb1}
    2d_{B(x,R)}(\mfa_n,\mfa_m)\le d_{B(x',2R)}(\mfa_n,\mfa_m) \, .
\end{equation}
\end{lemma}
\begin{proof}
With \eqref{eqcarl4}, both sides of \eqref{firstdb1} are equal to $4R|n-m|$. This proves the lemma.
\end{proof}

\begin{lemma}[integer ball cover]
\label{integer-ball-cover}
    For every $x\in \R$ and $R>0$ and every
    $n\in \mathbb{Z}$ and $R'>0$,
    there exist $m_1, m_2, m_3\in \mathbb{Z}$
    such that
    \begin{equation}\label{eqcarl5}
        B'\subset B_1\cup B_2\cup B_3\, ,
    \end{equation}
where
\begin{equation}
B'= \{ \mfa \in \Mf: d_{B(x,R)}(\mfa, \mfa_n)<2R'\}
\end{equation}
and for $j=1,2,3$
\begin{equation}
  B_j=
     \{ \mfa \in \Mf: d_{B(x,R)}(\mfa, \mfa_{m_j})<R'\}
     \, .
\end{equation}

\end{lemma}
\begin{proof}
Let $m_1$ be the largest integer smaller than
or equal to
$n- \frac {R'} {2R}$.
Let $m_2=n$.
Let $m_3$ be the smallest integer larger than
or equal to $n+ \frac {R'} {2R}$.

Let $\mfa_{n'}\in B'$, then with \eqref{eqcarl4},
we have
\begin{equation}\label{eqcarl6}
    2R|n-n'|< 2R'\, .
\end{equation}

Assume first $n'\le m_1$. With \eqref{eqcarl6}
we have
\begin{equation*}
    R|m_1-n'|=R(m_1-n')=R(m_1-n)+R(n-n')
\end{equation*}
\begin{equation}
    < -\frac{R'}2+R'=-\frac{R'}2\, .
\end{equation}
We conclude $\mfa_{n'}\in B_1$.

Assume next $m_1<n'<m_3$. Then
$\mfa_{n'}\in B_2$.

Assume finally that $m_3\le n'$. With \eqref{eqcarl6}
we have
\begin{equation*}
    R|m_3-n'|=R(n'-m_3)=R(n'-n)+R(n-m_3)
\end{equation*}
\begin{equation}
    < R' -\frac{R'}2=-\frac{R'}2\, .
\end{equation}
We conclude $\mfa_{n'}\in B_1$.
This completes the proof of the lemma.
\end{proof}


\begin{lemma}[real van der Corput]
\label{real-van-der-Corput}
\uses{van-der-Corput}
    For any $x\in \R$ and $R>0$ and any
    function $\varphi: X\to \C$ supported on $B'=B(x,R)$
    such that
\begin{equation}
    \|\varphi\|_{\Lip(B')} = \sup_{x \in B'} |\varphi(x)| + R \sup_{x,y \in B', x \neq y} \frac{|\varphi(x) - \varphi(y)|}{\rho(x,y)}
\end{equation}
is finite and for any $n,m\in \mathbb{Z}$, we have
\begin{equation}
    \label{eq-vdc-cond1}
    \left|\int_{B'} e(\mfa_n(x)-{\mfa_m(x)}) \varphi(x) d\mu(x)\right|\le 2\pi \mu(B')\frac{\|\varphi\|_{\Lip(B')}}{1+d_{B'}(\mfa_n,\mfa_m)}
\, .
\end{equation}
\end{lemma}
\begin{proof}
Set $n'=n-m$. Then we have to prove
\begin{equation}
    \label{eq-vdc-cond2}
    \left|\int_{x-R}^{x+R} e^{in'y}\varphi(y) dy\right|\le 4\pi R\|\varphi\|_{\Lip(B')}
(1+2R|n'|)^{-1}\, .
\end{equation}
This follows from \Cref{van-der-Corput} with $\alpha = x - R$ and $\beta = x + R$.
\end{proof}

The preceding chain of lemmas establish that $\Mf$ is a cancellative, compatible collection of functions on $(\R, \rho, \mu, 4)$. Again, some of the statements in these lemmas are stronger than what is needed for $a=4$, but can be relaxed to give the desired conclusion for $a=4$.




With $\kappa$ as near \eqref{eq-hilker}, define
the function $K:\R\times \R\to \mathbb{C}$ as in Theorem
\ref{metric-space-Carleson} by
\begin{equation}
    K(x,y):=\kappa(x-y)\, .
\end{equation}
The function $K$ is continuous outside the diagonal
$x=y$ and vanishes on the diagonal. Hence it is measurable.


By the Lemmas \ref{Hilbert-kernel-bound} and \ref{Hilbert-kernel-regularity}, it follows that $K$ is a one-sided Calder\'on--Zygmund kernel on $(\R,\rho,\mu,4)$.


The operator $T^*$ defined in \eqref{def-tang-unm-op} coincides in our setting with the operator $T^*$ defined in \eqref{concretetstar}. By \Cref{nontangential-Hilbert}, this operator satisfies the bound \eqref{nontanbound}.




Thus the assumptions of \Cref{metric-space-Carleson} are all satisfied. Applying the Theorem, \Cref{real-Carleson} follows.

\printbibliography
